% This is the UK TeX FAQ
%
\documentclass{faq}
%
\setcounter{errorcontextlines}{999}
%
% read the first two definitions of faqbody.tex for the file version
% and date
\afterassignment\endinput\input gather-faqbody
%
\typeout{The UK TeX FAQ, v\faqfileversion, date \faqfiledate}
%
% let's not be too fussy while we're developing...
\hfuzz10pt
\emergencystretch10pt
%
\begin{document}
\title{The \acro{UK} \TeX{} \acro{FAQ}\\
  Your \ifpdf\ref*{lastquestion}\else\protect\ref{lastquestion}\fi
  \ Questions Answered\\
       version \faqfileversion, date \faqfiledate}
\author{} 
\maketitle

\centerline{\textsc{Note}}

\begin{quotation}\small
  This document is an updated and extended version of the \acro{FAQ} article
  that was published as the December 1994 and 1995, and March 1999
  editions of the \acro{UK}\,\acro{TUG} magazine \BV{} (which weren't
  formatted like this).

  The article is also available via the World Wide Web.
\end{quotation}

\ifsinglecolumn
  \tableofcontents
\else
  \begin{multicols}{2}
  \tableofcontents
  \end{multicols}
\fi

\Dings

% label list for later processing
\labellist

%%%%%%%%%%%%%%%%%%%%%%%%%%%%%%%%%%%%%%%%%%%%%%%%%%%%%%%%%%%%%%%%%
% load the CTAN references if necessary
% $Id: dirctan.tex,v 1.302 2013/07/24 21:43:10 rf10 Exp rf10 $
%
% protect ourself against being read twice
\csname readCTANdirs\endcsname
\let\readCTANdirs\endinput
%
% declarations of significant directories on CTAN
\CTANdirectory{2etools}{macros/latex/required/tools}[tools]
\CTANdirectory{4spell}{support/4spell}[fourspell]
\CTANdirectory*{Catalogue}{help/Catalogue}
\CTANdirectory*{MathTeX}{support/mathtex}[mathtex]
\CTANdirectory{MimeTeX}{support/mimetex}[mimetex]
\CTANdirectory{Tabbing}{macros/latex/contrib/Tabbing}[tabbing]
\CTANdirectory*{TeXtelmExtel}{systems/msdos/emtex-contrib/TeXtelmExtel}
\CTANdirectory{TftI}{info/impatient}[impatient]
\CTANdirectory{a0poster}{macros/latex/contrib/a0poster}[a0poster]
\CTANdirectory*{a2ping}{graphics/a2ping}[a2ping]
\CTANdirectory{a4}{macros/latex/contrib/a4}[a4]
\CTANdirectory*{abc2mtex}{support/abc2mtex}
\CTANdirectory{abstract}{macros/latex/contrib/abstract}[abstract]
\CTANdirectory{abstyles}{biblio/bibtex/contrib/abstyles}
\CTANdirectory{accents}{support/accents}
\CTANdirectory{acronym}{macros/latex/contrib/acronym}[acronym]
\CTANdirectory*{ada}{web/ada/aweb}
\CTANdirectory{addindex}{indexing/addindex}
\CTANdirectory{addlines}{macros/latex/contrib/addlines}[addlines]
\CTANdirectory{adjkerns}{fonts/utilities/adjkerns}
\CTANdirectory{ae}{fonts/ae}[ae]
\CTANdirectory{aeguill}{macros/latex/contrib/aeguill}[aeguill]
\CTANdirectory{afmtopl}{fonts/utilities/afmtopl}
\CTANdirectory{akletter}{macros/latex/contrib/akletter}
\CTANdirectory{aleph}{systems/aleph}
\CTANdirectory{alg}{macros/latex/contrib/alg}
\CTANdirectory{algorithm2e}{macros/latex/contrib/algorithm2e}[algorithm2e]
\CTANdirectory{algorithmicx}{macros/latex/contrib/algorithmicx}[algorithmicx]
\CTANdirectory{algorithms}{macros/latex/contrib/algorithms}[algorithms]
\CTANdirectory*{alpha}{systems/mac/support/alpha}[alpha]
\CTANdirectory{amiga}{systems/amiga}
\CTANdirectory{amscls}{macros/latex/required/amslatex/amscls}[amscls]
\CTANdirectory{amsfonts}{fonts/amsfonts}[amsfonts]
\CTANdirectory{amslatex}{macros/latex/required/amslatex}[amslatex]
\CTANdirectory{amslatex-primer}{info/amslatex/primer}[amslatex-primer]
\CTANdirectory{amspell}{support/amspell}
\CTANdirectory{amsrefs}{macros/latex/contrib/amsrefs}[amsrefs]
\CTANdirectory{amstex}{macros/amstex}[amstex]
\CTANdirectory{anonchap}{macros/latex/contrib/anonchap}[anonchap]
\CTANdirectory{answers}{macros/latex/contrib/answers}[answers]
\CTANdirectory{ant}{systems/ant}[ant]
\CTANdirectory{anyfontsize}{macros/latex/contrib/anyfontsize}[anyfontsize]
\CTANdirectory{apl}{fonts/apl}
\CTANdirectory{aplweb}{web/apl/aplweb}
\CTANdirectory{appl}{web/reduce/rweb/appl}
\CTANdirectory{appendix}{macros/latex/contrib/appendix}[appendix]
\CTANdirectory{arabtex}{language/arabic/arabtex}
\CTANdirectory{arara}{support/arara}[arara]
\CTANdirectory{aro-bend}{info/challenges/aro-bend}[aro-bend]
\CTANdirectory{asana-math}{fonts/Asana-Math}[asana-math]
\CTANdirectory{asc2tex}{systems/msdos/asc2tex}
\CTANdirectory{ascii}{fonts/ascii}
\CTANdirectory*{aspell}{support/aspell}[aspell]
\CTANdirectory{astro}{fonts/astro}
\CTANdirectory{asymptote}{graphics/asymptote}[asymptote]
\CTANdirectory{asyfig}{macros/latex/contrib/asyfig}[asyfig]
\CTANdirectory{atari}{systems/atari}
\CTANdirectory*{atari-cstex}{systems/atari/cs-tex}[atari-cstex]
\CTANdirectory{attachfile}{macros/latex/contrib/attachfile}
\CTANdirectory*{auctex}{support/auctex}[auctex]
\CTANdirectory{autolatex}{support/autolatex}
\CTANdirectory{auto-pst-pdf}{macros/latex/contrib/auto-pst-pdf}[auto-pst-pdf]
\CTANdirectory{aweb}{web/ada/aweb}
\CTANdirectory*{awk}{web/spiderweb/src/awk}
\CTANdirectory{axodraw}{graphics/axodraw}[axodraw]
\CTANdirectory{babel}{macros/latex/required/babel}[babel]
\CTANdirectory{babelbib}{biblio/bibtex/contrib/babelbib}[babelbib]
\CTANdirectory{badge}{macros/plain/contrib/badge}
\CTANdirectory{bakoma}{fonts/cm/ps-type1/bakoma}[bakoma-fonts]
\CTANdirectory*{bakoma-tex}{systems/win32/bakoma}[bakoma]
\CTANdirectory*{bakoma-texfonts}{systems/win32/bakoma/fonts}
\CTANdirectory*{bard}{fonts/bard}
\CTANdirectory{barr}{macros/generic/diagrams/barr}
\CTANdirectory{bashkirian}{fonts/cyrillic/bashkirian}
\CTANdirectory{basix}{macros/generic/basix}
\CTANdirectory{bbding}{fonts/bbding}
\CTANdirectory{bbfig}{support/bbfig}
\CTANdirectory{bbm}{fonts/cm/bbm}[bbm]
\CTANdirectory{bbm-macros}{macros/latex/contrib/bbm}[bbm-macros]
\CTANdirectory{bbold}{fonts/bbold}[bbold]
\CTANdirectory{bdfchess}{fonts/chess/bdfchess}
\CTANdirectory{beamer}{macros/latex/contrib/beamer}[beamer]
\CTANdirectory{beamerposter}{macros/latex/contrib/beamerposter}[beamerposter]
\CTANdirectory{beebe}{dviware/beebe}
\CTANdirectory{belleek}{fonts/belleek}[belleek]
\CTANdirectory{beton}{macros/latex/contrib/beton}[beton]
\CTANdirectory{bezos}{macros/latex/contrib/bezos}[bezos]
\CTANdirectory{bib-fr}{biblio/bibtex/contrib/bib-fr}
\CTANdirectory{bib2dvi}{biblio/bibtex/utils/bib2dvi}
\CTANdirectory{bib2xhtml}{biblio/bibtex/utils/bib2xhtml}
\CTANdirectory*{bibcard}{biblio/bibtex/utils/bibcard}
\CTANdirectory*{bibclean}{biblio/bibtex/utils/bibclean}
\CTANdirectory*{bibdb}{support/bibdb}
\CTANdirectory{biber}{biblio/biber}[biber]
\CTANdirectory{bibextract}{biblio/bibtex/utils/bibextract}
\CTANdirectory{bibgerm}{biblio/bibtex/contrib/germbib}
\CTANdirectory{bibindex}{biblio/bibtex/utils/bibindex}
\CTANdirectory{biblatex}{macros/latex/contrib/biblatex}[biblatex]
\CTANdirectory*{biblatex-contrib}{macros/latex/contrib/biblatex-contrib}
\CTANdirectory{biblio}{info/biblio}
\CTANdirectory{biblist}{macros/latex209/contrib/biblist}
\CTANdirectory{bibsort}{biblio/bibtex/utils/bibsort}
\CTANdirectory{bibtex}{biblio/bibtex/base}[bibtex]
\CTANdirectory*{bibtex8}{biblio/bibtex/8-bit}[bibtex8bit]
\CTANdirectory{bibtex-doc}{biblio/bibtex/contrib/doc}[bibtex]
\CTANdirectory{bibtool}{biblio/bibtex/utils/bibtool}
\CTANdirectory{bibtools}{biblio/bibtex/utils/bibtools}
\CTANdirectory{bibtopic}{macros/latex/contrib/bibtopic}[bibtopic]
\CTANdirectory{bibunits}{macros/latex/contrib/bibunits}[bibunits]
\CTANdirectory{bibview}{biblio/bibtex/utils/bibview}
\CTANdirectory{bigfoot}{macros/latex/contrib/bigfoot}[bigfoot]
\CTANdirectory{bigstrut}{macros/latex/contrib/multirow}[bigstrut]
\CTANdirectory{bit2spr}{graphics/bit2spr}
\CTANdirectory{black}{fonts/cm/utilityfonts/black}
\CTANdirectory{blackboard}{info/symbols/blackboard}[blackboard]
\CTANdirectory{blackletter}{fonts/blackletter}
\CTANdirectory{blindtext}{macros/latex/contrib/blindtext}[blindtext]
\CTANdirectory{blocks}{macros/text1/blocks}
\CTANdirectory{blu}{macros/blu}
\CTANdirectory{bm2font}{graphics/bm2font}
\CTANdirectory{boites}{macros/latex/contrib/boites}[boites]
\CTANdirectory{bold}{fonts/cm/mf-extra/bold}
\CTANdirectory{bold-extra}{macros/latex/contrib/bold-extra}[bold-extra]
\CTANdirectory{bonus}{systems/msdos/emtex-contrib/bonus}
\CTANdirectory{booktabs}{macros/latex/contrib/booktabs}[booktabs]
\CTANdirectory{boondox}{fonts/boondox}[boondox]
\CTANdirectory{borceux}{macros/generic/diagrams/borceux}
\CTANdirectory{braket}{macros/latex/contrib/braket}[braket]
\CTANdirectory{breakurl}{macros/latex/contrib/breakurl}[breakurl]
\CTANdirectory{bridge}{macros/plain/contrib/bridge}
\CTANdirectory{brief_t}{support/brief_t}
\CTANdirectory{bst}{biblio/bibtex/contrib/germbib/bst}
\CTANdirectory{btable}{macros/plain/contrib/btable}
\CTANdirectory{btex8fmt}{macros/generic/cptex/btex8fmt}
\CTANdirectory{btOOL}{biblio/bibtex/utils/btOOL}
\CTANdirectory{bundledoc}{support/bundledoc}[bundledoc]
\CTANdirectory{c}{web/spiderweb/src/c}
\CTANdirectory{c++}{web/spiderweb/src/c++}
\CTANdirectory{c++2latex}{support/C++2LaTeX-1_1pl1}
\CTANdirectory{c2cweb}{web/c_cpp/c2cweb}
\CTANdirectory{c2latex}{support/c2latex}
\CTANdirectory{c_cpp}{web/c_cpp}
\CTANdirectory{caesar-fonts-generic.dir}{macros/generic/caesarcm/caesar-fonts-generic.dir}
\CTANdirectory{caesarcm}{macros/generic/caesarcm}
\CTANdirectory{caesarcmfonts.dir}{macros/generic/caesarcm/caesarcmfonts.dir}
\CTANdirectory{caesarcmv2.dir}{macros/generic/caesarcm/caesarcmv2.dir}
\CTANdirectory{calendar}{macros/plain/contrib/calendar}
\CTANdirectory{calligra}{fonts/calligra}
\CTANdirectory{calrsfs}{macros/latex/contrib/calrsfs}
\CTANdirectory{cancel}{macros/latex/contrib/cancel}[cancel]
\CTANdirectory{capt-of}{macros/latex/contrib/capt-of}[capt-of]
\CTANdirectory{caption}{macros/latex/contrib/caption}[caption]
\CTANdirectory{carlisle}{macros/latex/contrib/carlisle}[carlisle]
\CTANdirectory{cascover}{macros/plain/contrib/cascover}
\CTANdirectory{casslbl}{macros/plain/contrib/casslbl}
\CTANdirectory{catdvi}{dviware/catdvi}[catdvi]
\CTANdirectory{ccaption}{macros/latex/contrib/ccaption}[ccaption]
\CTANdirectory{ccfonts}{macros/latex/contrib/ccfonts}[ccfonts]
\CTANdirectory{cellular}{macros/plain/contrib/cellular}
\CTANdirectory{cellspace}{macros/latex/contrib/cellspace}[cellspace]
\CTANdirectory{changebar}{macros/latex/contrib/changebar}[changebar]
\CTANdirectory{changepage}{macros/latex/contrib/changepage}[changepage]
\CTANdirectory{changes}{macros/latex/contrib/changes}[changes]
\CTANdirectory{chappg}{macros/latex/contrib/chappg}[chappg]
\CTANdirectory{chapterfolder}{macros/latex/contrib/chapterfolder}[chapterfolder]
\CTANdirectory{charconv}{support/charconv}
\CTANdirectory{charter}{fonts/charter}
\CTANdirectory{chbar}{macros/plain/contrib/chbar}
\CTANdirectory{chbars}{macros/latex209/contrib/chbars}
\CTANdirectory{check}{support/check}
\CTANdirectory{chemstruct}{macros/latex209/contrib/chemstruct}
\CTANdirectory{chemtex}{macros/latex209/contrib/chemtex}
\CTANdirectory{cheq}{fonts/chess/cheq}
\CTANdirectory{cherokee}{fonts/cherokee}
\CTANdirectory{chesstools}{support/chesstools}
\CTANdirectory{chi2tex}{support/chi2tex}
\CTANdirectory{china2e}{macros/latex/contrib/china2e}
\CTANdirectory{chinese}{language/chinese}
\CTANdirectory{chngcntr}{macros/latex/contrib/chngcntr}[chngcntr]
\CTANdirectory{circ}{macros/generic/diagrams/circ}
\CTANdirectory{circuit_macros}{graphics/circuit_macros}
\CTANdirectory{cirth}{fonts/cirth}
\CTANdirectory{cite}{macros/latex/contrib/cite}[cite]
\CTANdirectory{citeref}{macros/latex/contrib/citeref}[citeref]
\CTANdirectory{clark}{fonts/utilities/afmtopl/clark}
\CTANdirectory{classico}{fonts/urw/classico}[classico]
\CTANdirectory{clrscode}{macros/latex/contrib/clrscode}[clrscode]
\CTANdirectory{cm}{fonts/cm}
\CTANdirectory{cm-lgc}{fonts/ps-type1/cm-lgc}[cm-lgc]
\CTANdirectory{cm-super}{fonts/ps-type1/cm-super}[cm-super]
\CTANdirectory{cm-unicode}{fonts/cm-unicode}[cm-unicode]
\CTANdirectory{cmactex}{systems/mac/cmactex}[cmactex]
\CTANdirectory{cmap}{macros/latex/contrib/cmap}
\CTANdirectory{cmastro}{fonts/cmastro}
\CTANdirectory{cmbright}{fonts/cmbright}[cmbright]
\CTANdirectory{cmcyralt}{macros/latex/contrib/cmcyralt}
%[fonts/cmcyralt]
\CTANdirectory{cmfrak}{fonts/gothic/cmfrak}
\CTANdirectory{cmoefont}{fonts/cmoefont}
\CTANdirectory{cmolddig}{fonts/cmolddig}
\CTANdirectory{cmoutlines}{fonts/cm/cmoutlines}[cmoutlines]
\CTANdirectory{cmpica}{fonts/cmpica}
\CTANdirectory{cms_help_files}{macros/text1/cms_help_files}
\CTANdirectory{cmtest}{fonts/cm/cmtest}
\CTANdirectory{cnoweb}{web/c_cpp/cnoweb}
\CTANdirectory{collref}{macros/latex/contrib/collref}
\CTANdirectory{combine}{macros/latex/contrib/combine}[combine]
\CTANdirectory{commado}{macros/generic/commado}[commado]
\CTANdirectory{comment}{macros/latex/contrib/comment}[comment]
\CTANdirectory{committee}{fonts/unsupported/committee}
\CTANdirectory{comp-fonts-FAQ}{help/comp-fonts-FAQ}
\CTANdirectory{components-of-TeX}{info/components-of-TeX}
\CTANdirectory{compugraphics_8600}{macros/text1/compugraphics_8600}
\CTANdirectory{concmath}{macros/latex/contrib/concmath}[concmath]
\CTANdirectory{concmath-f}{fonts/concmath}[concmath-fonts]
\CTANdirectory{concrete}{fonts/concrete}[concrete]
\CTANdirectory{context}{macros/context/current}[context]
\CTANdirectory{context-contrib}{macros/context/contrib}
\CTANdirectory{cprotect}{macros/latex/contrib/cprotect}[cprotect]
\CTANdirectory{cptex}{macros/generic/cptex}
\CTANdirectory{crop}{macros/latex/contrib/crop}[crop]
\CTANdirectory*{crosstex}{biblio/crosstex}[crosstex]
\CTANdirectory{crosswrd}{macros/latex/contrib/crosswrd}
\CTANdirectory{crudetype}{dviware/crudetype}[crudetype]
\CTANdirectory{crw}{macros/plain/contrib/crw}
\CTANdirectory{cs-tex}{systems/atari/cs-tex}
\CTANdirectory{csvsimple}{macros/latex/contrib/csvsimple}[csvsimple]
\CTANdirectory{ctable}{macros/latex/contrib/ctable}[ctable]
\CTANdirectory{ctan}{help/ctan}
\CTANdirectory{cun}{fonts/cun}
\CTANdirectory{currfile}{macros/latex/contrib/currfile}[currfile]
\CTANdirectory{currvita}{macros/latex/contrib/currvita}[currvita]
\CTANdirectory{curve}{macros/latex/contrib/curve}[curve]
\CTANdirectory{curves}{macros/latex/contrib/curves}
\CTANdirectory{custom-bib}{macros/latex/contrib/custom-bib}[custom-bib]
\CTANdirectory{cutwin}{macros/latex/contrib/cutwin}[cutwin]
\CTANdirectory{cweb}{web/c_cpp/cweb}
\CTANdirectory{cweb-p}{web/c_cpp/cweb-p}
\CTANdirectory{cypriote}{fonts/cypriote}
\CTANdirectory{cyrillic}{language/cyrillic}
\CTANdirectory{cyrtug}{language/cyrtug}
\CTANdirectory{dante}{usergrps/dante}
\CTANdirectory{dante-faq}{help/de-tex-faq}
\CTANdirectory{databases}{biblio/bibtex/databases}
\CTANdirectory{datatool}{macros/latex/contrib/datatool}[datatool]
\CTANdirectory{datetime}{macros/latex/contrib/datetime}
\CTANdirectory{db2tex}{support/db2tex}
\CTANdirectory{dblfloatfix}{macros/latex/contrib/dblfloatfix}[dblfloatfix]
\CTANdirectory{dbtex}{support/dbtex}
\CTANdirectory{dc-latex}{language/hyphen-accent/dc-latex}
\CTANdirectory{dc-nfss}{language/hyphen-accent/dc-nfss}
\CTANdirectory{detex}{support/detex}[detex]
\CTANdirectory{devanagari}{language/devanagari}
\CTANdirectory{diagbox}{macros/latex/contrib/diagbox}[diagbox]
\CTANdirectory{diagrams}{macros/generic/diagrams}
\CTANdirectory{dijkstra}{web/spiderweb/src/dijkstra}
\CTANdirectory{dinbrief}{macros/latex/contrib/dinbrief}
\CTANdirectory{dingbat}{fonts/dingbat}
\CTANdirectory{djgpp}{systems/msdos/djgpp}
\CTANdirectory{dk-bib}{biblio/bibtex/contrib/dk-bib}
\CTANdirectory{dktools}{support/dktools}[dktools]
\CTANdirectory{dm-latex}{language/hyphen-accent/dm-latex}
\CTANdirectory{dm-plain}{language/hyphen-accent/dm-plain}
\CTANdirectory{doc2sty}{language/swedish/slatex/doc2sty}
\CTANdirectory{docmfp}{macros/latex/contrib/docmfp}[docmfp]
\CTANdirectory{docmute}{macros/latex/contrib/docmute}[docmute]
\CTANdirectory{docu}{support/makeprog/docu}
\CTANdirectory{document}{biblio/bibtex/contrib/germbib/document}
\CTANdirectory{dos-dc}{systems/msdos/dos-dc}
\CTANdirectory{dos-psfonts}{systems/msdos/emtex-fonts/psfonts}
\CTANdirectory{doublestroke}{fonts/doublestroke}[doublestroke]
\CTANdirectory{dowith}{macros/generic/dowith}[dowith]
\CTANdirectory{dpfloat}{macros/latex/contrib/dpfloat}[dpfloat]
\CTANdirectory{dpmigcc}{systems/msdos/dpmigcc}
\CTANdirectory{draftcopy}{macros/latex/contrib/draftcopy}[draftcopy]
\CTANdirectory{draftwatermark}{macros/latex/contrib/draftwatermark}[draftwatermark]
\CTANdirectory{dratex}{graphics/dratex}[dratex]
\CTANdirectory{drawing}{graphics/drawing}
\CTANdirectory{dropcaps}{macros/latex209/contrib/dropcaps}
\CTANdirectory{dropping}{macros/latex/contrib/dropping}[dropping]
\CTANdirectory{dtl}{dviware/dtl}[dtl]
\CTANdirectory{dtxgen}{support/dtxgen}[dtxgen]
\CTANdirectory{dtxtut}{info/dtxtut}[dtxtut]
\CTANdirectory{duerer}{fonts/duerer}
\CTANdirectory{dvgt}{dviware/dvgt}
\CTANdirectory{dvi-augsburg}{dviware/dvi-augsburg}
\CTANdirectory{dvi2bitmap}{dviware/dvi2bitmap}[dvi2bitmap]
\CTANdirectory{dvi2pcl}{dviware/dvi2pcl}
\CTANdirectory{dvi2tty}{dviware/dvi2tty}[dvi2tty]
\CTANdirectory{dviasm}{dviware/dviasm}[dviasm]
\CTANdirectory{dvibit}{dviware/dvibit}
\CTANdirectory{dvibook}{dviware/dvibook}
\CTANdirectory{dvichk}{dviware/dvichk}
\CTANdirectory{dvicopy}{dviware/dvicopy}
\CTANdirectory{dvidjc}{dviware/dvidjc}
\CTANdirectory{dvidvi}{dviware/dvidvi}
\CTANdirectory{dviimp}{dviware/dviimp}
\CTANdirectory{dviljk}{dviware/dviljk}
\CTANdirectory{dvimerge}{dviware/dvimerge}
\CTANdirectory{dvimfj}{systems/msdos/emtex-contrib/dvimfj}
\CTANdirectory{dvipage}{dviware/dvipage}
\CTANdirectory{dvipaste}{dviware/dvipaste}
\CTANdirectory{dvipdfm}{dviware/dvipdfm}
\CTANdirectory{dvipdfmx}{dviware/dvipdfmx}[dvipdfmx]
\CTANdirectory{dvipj}{dviware/dvipj}
\CTANdirectory{dvipng}{dviware/dvipng}[dvipng]
\CTANdirectory{dvips-pc}{systems/msdos/dviware/dvips}
\CTANdirectory{dvips}{dviware/dvips}[dvips]
\CTANdirectory{dvistd}{dviware/driv-standard}
\CTANdirectory{dvisun}{dviware/dvisun}
\CTANdirectory{dvitty}{dviware/dvitty}
\CTANdirectory{dvivga}{dviware/dvivga}
\CTANdirectory{e4t}{systems/msdos/e4t}
\CTANdirectory*{e-TeX}{systems/e-tex}
\CTANdirectory{easytex}{systems/msdos/easytex}
\CTANdirectory{ebib}{biblio/bibtex/utils/ebib}
\CTANdirectory{ec}{fonts/ec}[ec]
\CTANdirectory{ec-plain}{macros/ec-plain}[ec-plain]
\CTANdirectory{eco}{fonts/eco}[eco]
\CTANdirectory{economic}{biblio/bibtex/contrib/economic}
\CTANdirectory{edmac}{macros/plain/contrib/edmac}[edmac]
\CTANdirectory{ednotes}{macros/latex/contrib/ednotes}[ednotes]
\CTANdirectory{eepic}{macros/latex/contrib/eepic}[eepic]
\CTANdirectory{ega2mf}{fonts/utilities/ega2mf}
\CTANdirectory{egplot}{macros/latex/contrib/egplot}[egplot]
\CTANdirectory{eiad}{fonts/eiad}
\CTANdirectory{elvish}{fonts/elvish}
\CTANdirectory{elwell}{fonts/utilities/afmtopl/elwell}
\CTANdirectory{emp}{macros/latex/contrib/emp}[emp]
\CTANdirectory{emptypage}{macros/latex/contrib/emptypage}[emptypage]
\CTANdirectory{emt2tex}{systems/msdos/emtex-contrib/emt2tex}
\CTANdirectory{emtex}{systems/msdos/emtex}
\CTANdirectory{emtex-contrib}{systems/msdos/emtex-contrib}
\CTANdirectory{emtex-fonts}{systems/msdos/emtex-fonts}
\CTANdirectory{emtextds}{obsolete/systems/os2/emtex-contrib/emtexTDS}
\CTANdirectory{enctex}{systems/enctex}[enctex]
\CTANdirectory{endfloat}{macros/latex/contrib/endfloat}[endfloat]
\CTANdirectory{english}{language/english}
\CTANdirectory{engwar}{fonts/engwar}
\CTANdirectory{enumitem}{macros/latex/contrib/enumitem}[enumitem]
\CTANdirectory{environment}{support/lsedit/environment}
\CTANdirectory{epic}{macros/latex/contrib/epic}[epic]
\CTANdirectory{epigraph}{macros/latex/contrib/epigraph}[epigraph]
\CTANdirectory{eplain}{macros/eplain}[eplain]
\CTANdirectory{epmtex}{systems/os2/epmtex}
\CTANdirectory{epslatex}{info/epslatex}[epslatex]
\CTANdirectory*{epstopdf}{support/epstopdf}[epstopdf]
\CTANdirectory{eqparbox}{macros/latex/contrib/eqparbox}[eqparbox]
\CTANdirectory{ergotex}{systems/msdos/ergotex}
\CTANdirectory{errata}{systems/knuth/dist/errata}
\CTANdirectory{eso-pic}{macros/latex/contrib/eso-pic}[eso-pic]
\CTANdirectory{et}{support/et}
\CTANdirectory{etex}{systems/e-tex}[etex]
\CTANdirectory{etex-pkg}{macros/latex/contrib/etex-pkg}[etex-pkg]
\CTANdirectory{etextools}{macros/latex/contrib/etextools}[etextools]
\CTANdirectory{ethiopia}{language/ethiopia}
\CTANdirectory{ethtex}{language/ethiopia/ethtex}
\CTANdirectory{etoc}{macros/latex/contrib/etoc}[etoc]
\CTANdirectory{etoolbox}{macros/latex/contrib/etoolbox}[etoolbox]
\CTANdirectory{euler-latex}{macros/latex/contrib/euler}[euler]
\CTANdirectory{eulervm}{fonts/eulervm}[eulervm]
\CTANdirectory{euro-ce}{fonts/euro-ce}[euro-ce]
\CTANdirectory{euro-fonts}{fonts/euro}
\CTANdirectory{eurofont}{macros/latex/contrib/eurofont}[eurofont]
\CTANdirectory{europecv}{macros/latex/contrib/europecv}[europecv]
\CTANdirectory{eurosym}{fonts/eurosym}[eurosym]
\CTANdirectory{everypage}{macros/latex/contrib/everypage}[everypage]
\CTANdirectory{excalibur}{systems/mac/support/excalibur}
\CTANdirectory{excel2latex}{support/excel2latex}
\CTANdirectory{excerpt}{web/spiderweb/tools/excerpt}
\CTANdirectory{excludeonly}{macros/latex/contrib/excludeonly}[excludeonly]
\CTANdirectory{expdlist}{macros/latex/contrib/expdlist}
\CTANdirectory{extract}{macros/latex/contrib/extract}[extract]
\CTANdirectory{extsizes}{macros/latex/contrib/extsizes}[extsizes]
% cat links to here
\CTANdirectory{fancyhdr}{macros/latex/contrib/fancyhdr}[fancyhdr]
\CTANdirectory{fancyvrb}{macros/latex/contrib/fancyvrb}[fancyvrb]
\CTANdirectory{faq}{help/uk-tex-faq}[uk-tex-faq]
\CTANdirectory{fc}{fonts/jknappen/fc}
\CTANdirectory{fdsymbol}{fonts/fdsymbol}[fdsymbol]
\CTANdirectory{feyn}{fonts/feyn}
\CTANdirectory{feynman}{macros/latex209/contrib/feynman}
\CTANdirectory{feynmf}{macros/latex/contrib/feynmf}[feynmf]
\CTANdirectory{fig2eng}{graphics/fig2eng}
\CTANdirectory{fig2mf}{graphics/fig2mf}
\CTANdirectory{fig2mfpic}{graphics/fig2mfpic}
\CTANdirectory{figflow}{macros/plain/contrib/figflow}[figflow]
\CTANdirectory{filehook}{macros/latex/contrib/filehook}[filehook]
\CTANdirectory{fink}{macros/latex/contrib/fink}[fink]
\CTANdirectory{first-latex-doc}{info/first-latex-doc}
\CTANdirectory{fix2col}{macros/latex/contrib/fix2col}[fix2col]
\CTANdirectory{fixfoot}{macros/latex/contrib/fixfoot}[fixfoot]
\CTANdirectory{float}{macros/latex/contrib/float}[float]
\CTANdirectory{floatflt}{macros/latex/contrib/floatflt}[floatflt]
\CTANdirectory{flow}{support/flow}
\CTANdirectory{flowfram}{macros/latex/contrib/flowfram}[flowfram]
\CTANdirectory{fltpage}{macros/latex/contrib/fltpage}[fltpage]
\CTANdirectory{fnbreak}{macros/latex/contrib/fnbreak}[fnbreak]
\CTANdirectory{fncychap}{macros/latex/contrib/fncychap}[fncychap]
\CTANdirectory{fncylab}{macros/latex/contrib/fncylab}
\CTANdirectory{foiltex}{macros/latex/contrib/foiltex}[foiltex]
\CTANdirectory{font-change}{macros/plain/contrib/font-change}[font-change]
\CTANdirectory{fontch}{macros/plain/contrib/fontch}[fontch]
\CTANdirectory{fontinst}{fonts/utilities/fontinst}[fontinst]
\CTANdirectory{fontname}{info/fontname}[fontname]
\CTANdirectory{fontspec}{macros/latex/contrib/fontspec}[fontspec]
\CTANdirectory{font_selection}{macros/plain/contrib/font_selection}[font-selection]
\CTANdirectory{footbib}{macros/latex/contrib/footbib}[footbib]
\CTANdirectory{footmisc}{macros/latex/contrib/footmisc}[footmisc]
\CTANdirectory{footnpag}{macros/latex/contrib/footnpag}[footnpag]
\CTANdirectory{forarray}{macros/latex/contrib/forarray}[forarray]
\CTANdirectory{forloop}{macros/latex/contrib/forloop}[forloop]
\CTANdirectory{for_tex}{biblio/bibtex/contrib/germbib/for_tex}
\CTANdirectory{fourier}{fonts/fourier-GUT}[fourier]
\CTANdirectory{fouriernc}{fonts/fouriernc}[fouriernc]
\CTANdirectory{framed}{macros/latex/contrib/framed}[framed]
\CTANdirectory{frankenstein}{macros/latex/contrib/frankenstein}[frankenstein]
\CTANdirectory{french-faq}{help/LaTeX-FAQ-francaise}
\CTANdirectory{funnelweb}{web/funnelweb}
\CTANdirectory{futhark}{fonts/futhark}
\CTANdirectory{futhorc}{fonts/futhorc}
\CTANdirectory{fweb}{web/fweb}[fweb]
\CTANdirectory{garamondx}{fonts/garamondx}[garamondx]
\CTANdirectory{gellmu}{support/gellmu}[gellmu]
\CTANdirectory{genfam}{support/genfam}
\CTANdirectory{geometry}{macros/latex/contrib/geometry}[geometry]
\CTANdirectory{germbib}{biblio/bibtex/contrib/germbib}
\CTANdirectory{getoptk}{macros/plain/contrib/getoptk}[getoptk]
\CTANdirectory{gfs}{info/examples/FirstSteps} % gratzer's
\CTANdirectory{gitinfo}{macros/latex/contrib/gitinfo}[gitinfo]
\CTANdirectory{glo+idxtex}{indexing/glo+idxtex}[idxtex]
\CTANdirectory{gmp}{macros/latex/contrib/gmp}[gmp]
\CTANdirectory{gnuplot}{graphics/gnuplot}
\CTANdirectory{go}{fonts/go}
\CTANdirectory{gothic}{fonts/gothic}
\CTANdirectory{graphbase}{support/graphbase}
\CTANdirectory{graphics}{macros/latex/required/graphics}[graphics]
\CTANdirectory{graphics-plain}{macros/plain/graphics}[graphics-pln]
\CTANdirectory{graphicx-psmin}{macros/latex/contrib/graphicx-psmin}[graphicx-psmin]
\CTANdirectory{gray}{fonts/cm/utilityfonts/gray}
\CTANdirectory{greek}{fonts/greek}
\CTANdirectory{greektex}{fonts/greek/greektex}
\CTANdirectory{gsftopk}{fonts/utilities/gsftopk}[gsftopk]
\CTANdirectory{gut}{usergrps/gut}
\CTANdirectory*{gv}{support/gv}[gv]
\CTANdirectory{ha-prosper}{macros/latex/contrib/ha-prosper}[ha-prosper]
\CTANdirectory{half}{fonts/cm/utilityfonts/half}
\CTANdirectory{halftone}{fonts/halftone}
\CTANdirectory{hands}{fonts/hands}
\CTANdirectory{harvard}{macros/latex/contrib/harvard}
\CTANdirectory{harvmac}{macros/plain/contrib/harvmac}
\CTANdirectory{hebrew}{language/hebrew}
\CTANdirectory{help}{help}
\CTANdirectory{here}{macros/latex/contrib/here}[here]
\CTANdirectory{hershey}{fonts/hershey}
\CTANdirectory{hfbright}{fonts/ps-type1/hfbright}[hfbright]
\CTANdirectory{hge}{fonts/hge}
\CTANdirectory{hieroglyph}{fonts/hieroglyph}
\CTANdirectory{highlight}{support/highlight}
\CTANdirectory{histyle}{macros/plain/contrib/histyle}
\CTANdirectory{hp2pl}{support/hp2pl}
\CTANdirectory{hp2xx}{support/hp2xx}
\CTANdirectory{hpgl2ps}{graphics/hpgl2ps}
\CTANdirectory{hptex}{macros/hptex}
\CTANdirectory{hptomf}{support/hptomf}
\CTANdirectory{html2latex}{support/html2latex}[html2latex]
\CTANdirectory{htmlhelp}{info/htmlhelp}
\CTANdirectory{hvfloat}{macros/latex/contrib/hvfloat}[hvfloat]
\CTANdirectory{hvmath}{fonts/micropress/hvmath}[hvmath-fonts]
\CTANdirectory{hyacc-cm}{macros/generic/hyacc-cm}
\CTANdirectory{hyper}{macros/latex/contrib/hyper}
\CTANdirectory{hyperbibtex}{biblio/bibtex/utils/hyperbibtex}
\CTANdirectory{hypernat}{macros/latex/contrib/hypernat}[hypernat]
\CTANdirectory{hyperref}{macros/latex/contrib/hyperref}[hyperref]
\CTANdirectory{hyphen-accent}{language/hyphen-accent}
\CTANdirectory{hyphenat}{macros/latex/contrib/hyphenat}[hyphenat]
\CTANdirectory{hyphenation}{language/hyphenation}
\CTANdirectory{ibygrk}{fonts/greek/ibygrk}
\CTANdirectory{iching}{fonts/iching}
\CTANdirectory{icons}{support/icons}
\CTANdirectory{ifmtarg}{macros/latex/contrib/ifmtarg}[ifmtarg]
\CTANdirectory{ifmslide}{macros/latex/contrib/ifmslide}[ifmslide]
\CTANdirectory{ifoddpage}{macros/latex/contrib/ifoddpage}[ifoddpage]
\CTANdirectory{ifxetex}{macros/generic/ifxetex}[ifxetex]
\CTANdirectory{imakeidx}{macros/latex/contrib/imakeidx}[imakeidx]
\CTANdirectory{imaketex}{support/imaketex}
\CTANdirectory{impact}{web/systems/mac/impact}
\CTANdirectory{import}{macros/latex/contrib/import}[import]
\CTANdirectory{index}{macros/latex/contrib/index}[index]
\CTANdirectory{indian}{language/indian}
\CTANdirectory{infpic}{macros/generic/infpic}
\CTANdirectory{inlinebib}{biblio/bibtex/contrib/inlinebib}
\CTANdirectory{inrsdoc}{macros/inrstex/inrsdoc}
\CTANdirectory{inrsinputs}{macros/inrstex/inrsinputs}
\CTANdirectory{inrstex}{macros/inrstex}
\CTANdirectory{isi2bibtex}{biblio/bibtex/utils/isi2bibtex}[isi2bibtex]
\CTANdirectory{iso-tex}{support/iso-tex}
\CTANdirectory{isodoc}{macros/latex/contrib/isodoc}[isodoc]
\CTANdirectory{ispell}{support/ispell}[ispell]
\CTANdirectory{ite}{support/ite}[ite]
\CTANdirectory{ivd2dvi}{dviware/ivd2dvi}
\CTANdirectory{jadetex}{macros/jadetex}
\CTANdirectory{jemtex2}{systems/msdos/jemtex2}
\CTANdirectory{jknappen-macros}{macros/latex/contrib/jknappen}
\CTANdirectory{jpeg2ps}{support/jpeg2ps}[jpeg2ps]
\CTANdirectory{jspell}{support/jspell}[jspell]
\CTANdirectory{jurabib}{macros/latex/contrib/jurabib}[jurabib]
\CTANdirectory{kamal}{support/kamal}
\CTANdirectory{kane}{dviware/kane}
\CTANdirectory{karta}{fonts/karta}
\CTANdirectory{kd}{fonts/greek/kd}
\CTANdirectory{kelem}{web/spiderweb/src/kelem}
\CTANdirectory{kelly}{fonts/greek/kelly}
\CTANdirectory{klinz}{fonts/klinz}
\CTANdirectory{knit}{web/knit}
\CTANdirectory{knot}{fonts/knot}
\CTANdirectory{knuth}{systems/knuth}
\CTANdirectory{knuth-dist}{systems/knuth/dist}[knuth-dist]
\CTANdirectory{koma-script}{macros/latex/contrib/koma-script}[koma-script]
\CTANdirectory{korean}{fonts/korean}
\CTANdirectory{kpfonts}{fonts/kpfonts}[kpfonts]
\CTANdirectory{kyocera}{dviware/kyocera}
\CTANdirectory{l2a}{support/l2a}[l2a]
\CTANdirectory{l2sl}{language/swedish/slatex/l2sl}
\CTANdirectory*{l2tabu}{info/l2tabu}
\CTANdirectory{l2x}{support/l2x}
\CTANdirectory{la}{fonts/la}
\CTANdirectory{laan}{macros/generic/laan}
\CTANdirectory{laansort}{macros/generic/laansort}
\CTANdirectory{labelcas}{macros/latex/contrib/labelcas}[labelcas]
\CTANdirectory{labels}{macros/latex/contrib/labels}
\CTANdirectory{labtex}{macros/generic/labtex}
\CTANdirectory{lacheck}{support/lacheck}[lacheck]
\CTANdirectory{lametex}{support/lametex}
\CTANdirectory{lamstex}{macros/lamstex}
\CTANdirectory{lastpage}{macros/latex/contrib/lastpage}[lastpage]
\CTANdirectory{latex}{macros/latex/base}
\CTANdirectory{latex-course}{info/latex-course}[latex-course]
\CTANdirectory{latex-essential}{info/latex-essential}
\CTANdirectory{latex2e-help-texinfo}{info/latex2e-help-texinfo}[latex2e-help-texinfo]
\CTANdirectory{latex4jed}{support/jed}[latex4jed]
\CTANdirectory*{latex-tds}{macros/latex/contrib/latex-tds}[latex-tds]
\CTANdirectory*{latex2html}{support/latex2html}[latex2html]
\CTANdirectory{latex2rtf}{support/latex2rtf}
\CTANdirectory{latexdiff}{support/latexdiff}[latexdiff]
\CTANdirectory{latexdoc}{macros/latex/doc}[latex-doc]
\CTANdirectory{latexhlp}{systems/atari/latexhlp}
\CTANdirectory{latexmake}{support/latexmake}[latexmake]
\CTANdirectory{latex-make}{support/latex-make}[latex-make]
\CTANdirectory{latex_maker}{support/latex_maker}[mk]
\CTANdirectory{latexmk}{support/latexmk}[latexmk]
\CTANdirectory{lecturer}{macros/generic/lecturer}[lecturer]
\CTANdirectory{ledmac}{macros/latex/contrib/ledmac}[ledmac]
\CTANdirectory{lettrine}{macros/latex/contrib/lettrine}[lettrine]
\CTANdirectory{levy}{fonts/greek/levy}
\CTANdirectory{lextex}{macros/plain/contrib/lextex}
\CTANdirectory{lgc}{info/examples/lgc}
\CTANdirectory{lgrind}{support/lgrind}[lgrind]
\CTANdirectory{libertine}{fonts/libertine}[libertine]
\CTANdirectory{libgreek}{macros/latex/contrib/libgreek}[libgreek]
\CTANdirectory{lilyglyphs}{macros/luatex/latex/lilyglyphs}
\CTANdirectory{lineno}{macros/latex/contrib/lineno}[lineno]
\CTANdirectory{lipsum}{macros/latex/contrib/lipsum}[lipsum]
\CTANdirectory{listbib}{macros/latex/contrib/listbib}[listbib]
\CTANdirectory{listings}{macros/latex/contrib/listings}[listings]
\CTANdirectory{lm}{fonts/lm}[lm]
\CTANdirectory{lm-math}{fonts/lm-math}[lm-math]
\CTANdirectory{lollipop}{macros/lollipop}[lollipop]
\CTANdirectory{lookbibtex}{biblio/bibtex/utils/lookbibtex}
\CTANdirectory{lpic}{macros/latex/contrib/lpic}[lpic]
\CTANdirectory{lsedit}{support/lsedit}
\CTANdirectory{lshort}{info/lshort/english}[lshort-english]
\CTANdirectory*{lshort-parent}{info/lshort}[lshort]
\CTANdirectory{ltx3pub}{info/ltx3pub}[ltx3pub]
\CTANdirectory{ltxindex}{macros/latex/contrib/ltxindex}[ltxindex]
\CTANdirectory{luatex}{systems/luatex}[luatex]
\CTANdirectory{lucida}{fonts/psfonts/bh/lucida}[lucida]
\CTANdirectory{lucida-psnfss}{macros/latex/contrib/psnfssx/lucidabr}[psnfssx-luc]
\CTANdirectory{luximono}{fonts/LuxiMono}[luximono]
\CTANdirectory{lwc}{info/examples/lwc}
\CTANdirectory{ly1}{fonts/psfonts/ly1}[ly1]
\CTANdirectory*{mactex}{systems/mac/mactex}[mactex]
\CTANdirectory{macros2e}{info/macros2e}[macros2e]
\CTANdirectory{mactotex}{graphics/mactotex}
\CTANdirectory{mailing}{macros/latex/contrib/mailing}
\CTANdirectory{make_latex}{support/make_latex}[make-latex]
\CTANdirectory{makeafm.dir}{fonts/utilities/t1tools/makeafm.dir}
\CTANdirectory{makecell}{macros/latex/contrib/makecell}[makecell]
\CTANdirectory{makedtx}{support/makedtx}[makedtx]
\CTANdirectory{makeindex}{indexing/makeindex}[makeindex]
\CTANdirectory{makeinfo}{macros/texinfo/contrib/texinfo-hu/texinfo/makeinfo}
\CTANdirectory{makeprog}{support/makeprog}\CTANdirectory{maketexwork}{info/maketexwork}
\CTANdirectory{malayalam}{language/malayalam}
\CTANdirectory{malvern}{fonts/malvern}
\CTANdirectory{mapleweb}{web/maple/mapleweb}
\CTANdirectory{marvosym-fonts}{fonts/marvosym}
\CTANdirectory{mathabx}{fonts/mathabx}[mathabx]
\CTANdirectory{mathabx-type1}{fonts/ps-type1/mathabx}[mathabx-type1]
\CTANdirectory{mathastext}{macros/latex/contrib/mathastext}[mathastext]
\CTANdirectory*{mathdesign}{fonts/mathdesign}[mathdesign]
\CTANdirectory{mathdots}{macros/generic/mathdots}
\CTANdirectory{mathematica}{macros/mathematica}
\CTANdirectory{mathpazo}{fonts/mathpazo}[mathpazo]
\CTANdirectory{mathsci2bibtex}{biblio/bibtex/utils/mathsci2bibtex}
\CTANdirectory{mathspic}{graphics/mathspic}[mathspic]
\CTANdirectory{mcite}{macros/latex/contrib/mcite}[mcite]
\CTANdirectory{mciteplus}{macros/latex/contrib/mciteplus}[mciteplus]
\CTANdirectory{mdframed}{macros/latex/contrib/mdframed}[mdframed]
\CTANdirectory{mdsymbol}{fonts/mdsymbol}[mdsymbol]
\CTANdirectory{mdwtools}{macros/latex/contrib/mdwtools}[mdwtools]
\CTANdirectory{memdesign}{info/memdesign}[memdesign]
\CTANdirectory{memoir}{macros/latex/contrib/memoir}[memoir]
\CTANdirectory{messtex}{support/messtex}
\CTANdirectory{metalogo}{macros/latex/contrib/metalogo}[metalogo]
\CTANdirectory{metapost}{graphics/metapost}
\CTANdirectory{metatype1}{fonts/utilities/metatype1}[metatype1]
\CTANdirectory{mf2ps}{fonts/utilities/mf2ps}
\CTANdirectory{mf2pt1}{support/mf2pt1}[mf2pt1]
\CTANdirectory{mf_optimized_kerning}{fonts/cm/mf_optimized_kerning}
\CTANdirectory{mfbook}{fonts/cm/utilityfonts/mfbook}
\CTANdirectory{mff-29}{fonts/utilities/mff-29}
\CTANdirectory{mffiles}{language/telugu/mffiles}
\CTANdirectory{mflogo}{macros/latex/contrib/mflogo}[mflogo]
\CTANdirectory{mfnfss}{macros/latex/contrib/mfnfss}
\CTANdirectory{mfpic}{graphics/mfpic}
\CTANdirectory{mfware}{systems/knuth/dist/mfware}
\CTANdirectory{mh}{macros/latex/contrib/mh}[mh]
\CTANdirectory{miktex}{systems/win32/miktex}[miktex]
\CTANdirectory{microtype}{macros/latex/contrib/microtype}[microtype]
\CTANdirectory{midi2tex}{support/midi2tex}
\CTANdirectory{midnight}{macros/generic/midnight}
\CTANdirectory{minionpro}{fonts/minionpro}[minionpro]
\CTANdirectory{minitoc}{macros/latex/contrib/minitoc}[minitoc]
\CTANdirectory{minted}{macros/latex/contrib/minted}[minted]
\CTANdirectory{mkjobtexmf}{support/mkjobtexmf}[mkjobtexmf]
\CTANdirectory{mma2ltx}{graphics/mma2ltx}
\CTANdirectory{mmap}{macros/latex/contrib/mmap}
\CTANdirectory{mnsymbol}{fonts/mnsymbol}[mnsymbol]
\CTANdirectory{mnu}{support/mnu}
\CTANdirectory{models}{macros/text1/models}
\CTANdirectory{moderncv}{macros/latex/contrib/moderncv}[moderncv]
\CTANdirectory{modes}{fonts/modes}
\CTANdirectory{morefloats}{macros/latex/contrib/morefloats}[morefloats]
\CTANdirectory{moreverb}{macros/latex/contrib/moreverb}[moreverb]
\CTANdirectory{morewrites}{macros/latex/contrib/morewrites}[morewrites]
\CTANdirectory{mparhack}{macros/latex/contrib/mparhack}[mparhack]
\CTANdirectory{mpgraphics}{macros/latex/contrib/mpgraphics}[mpgraphics]
\CTANdirectory{mps2eps}{support/mps2eps}
\CTANdirectory{ms}{macros/latex/contrib/ms}[ms]
\CTANdirectory{msdos}{systems/msdos}
\CTANdirectory{msx2msa}{fonts/vf-files/msx2msa}
\CTANdirectory{msym}{fonts/msym}
\CTANdirectory{mtp2lite}{fonts/mtp2lite}[mtp2lite]
\CTANdirectory{m-tx}{support/m-tx}[m-tx]
\CTANdirectory{multenum}{macros/latex/contrib/multenum}[multenum]
\CTANdirectory{multibbl}{macros/latex/contrib/multibbl}[multibbl]
\CTANdirectory{multibib}{macros/latex/contrib/multibib}[multibib]
\CTANdirectory{multido}{macros/generic/multido}[multido]
\CTANdirectory{multirow}{macros/latex/contrib/multirow}[multirow]
\CTANdirectory{musictex}{macros/musictex}[musictex]
\CTANdirectory{musixtex-egler}{obsolete/macros/musixtex/egler}
\CTANdirectory{musixtex-fonts}{fonts/musixtex-fonts}[musixtex-fonts]
\CTANdirectory{musixtex}{macros/musixtex}[musixtex]
\CTANdirectory{mutex}{macros/mtex}
\CTANdirectory{mwe}{macros/latex/contrib/mwe}[mwe]
\CTANdirectory{mxedruli}{fonts/georgian/mxedruli}
\CTANdirectory{nag}{macros/latex/contrib/nag}[nag]
\CTANdirectory{natbib}{macros/latex/contrib/natbib}[natbib]
\CTANdirectory{navigator}{macros/generic/navigator}[navigator]
\CTANdirectory{nawk}{web/spiderweb/src/nawk}
\CTANdirectory{ncctools}{macros/latex/contrib/ncctools}[ncctools]
\CTANdirectory{needspace}{macros/latex/contrib/needspace}[needspace]
\CTANdirectory{newalg}{macros/latex/contrib/newalg}[newalg]
\CTANdirectory{newcommand}{support/newcommand}[newcommand]
\CTANdirectory{newlfm}{macros/latex/contrib/newlfm}[newlfm]
\CTANdirectory{newsletr}{macros/plain/contrib/newsletr}
\CTANdirectory{newpx}{fonts/newpx}[newpx]
\CTANdirectory{newtx}{fonts/newtx}[newtx]
\CTANdirectory{newverbs}{macros/latex/contrib/newverbs}[newverbs]
\CTANdirectory{nedit-latex}{support/NEdit-LaTeX-Extensions}
\CTANdirectory{nonumonpart}{macros/latex/contrib/nonumonpart}[nonumonpart]
\CTANdirectory{nopageno}{macros/latex/contrib/nopageno}[nopageno]
\CTANdirectory{norbib}{biblio/bibtex/contrib/norbib}
\CTANdirectory{notoccite}{macros/latex/contrib/notoccite}[notoccite]
\CTANdirectory{noweb}{web/noweb}[noweb]
\CTANdirectory{ntg}{usergrps/ntg}
\CTANdirectory{ntgclass}{macros/latex/contrib/ntgclass}[ntgclass]
\CTANdirectory{ntheorem}{macros/latex/contrib/ntheorem}[ntheorem]
\CTANdirectory{nts-l}{digests/nts-l}
\CTANdirectory{nts}{systems/nts}
\CTANdirectory{numprint}{macros/latex/contrib/numprint}[numprint]
\CTANdirectory{nuweb}{web/nuweb}
\CTANdirectory{nuweb0.87b}{web/nuweb/nuweb0.87b}
\CTANdirectory{nuweb_ami}{web/nuweb/nuweb_ami}
\CTANdirectory{oberdiek}{macros/latex/contrib/oberdiek}[oberdiek]
\CTANdirectory{objectz}{macros/latex/contrib/objectz}
\CTANdirectory{ocr-a}{fonts/ocr-a}
\CTANdirectory{ocr-b}{fonts/ocr-b}
\CTANdirectory{ofs}{macros/generic/ofs}[ofs]
\CTANdirectory{ogham}{fonts/ogham}
\CTANdirectory{ogonek}{macros/latex/contrib/ogonek}
\CTANdirectory{okuda}{fonts/okuda}
\CTANdirectory{omega}{systems/omega}
\CTANdirectory{optional}{macros/latex/contrib/optional}[optional]
\CTANdirectory{os2}{systems/os2}
\CTANdirectory{osmanian}{fonts/osmanian}
\CTANdirectory{overpic}{macros/latex/contrib/overpic}[overpic]
\CTANdirectory{oztex}{systems/mac/oztex}
\CTANdirectory{page}{support/lametex/page}
\CTANdirectory{palladam}{language/tamil/palladam}
\CTANdirectory{pandora}{fonts/pandora}
\CTANdirectory{paralist}{macros/latex/contrib/paralist}[paralist]
\CTANdirectory{parallel}{macros/latex/contrib/parallel}[parallel]
\CTANdirectory{parskip}{macros/latex/contrib/parskip}[parskip]
\CTANdirectory{passivetex}{macros/xmltex/contrib/passivetex}[passivetex]
\CTANdirectory{patchcmd}{macros/latex/contrib/patchcmd}[patchcmd]
\CTANdirectory{path}{macros/generic/path}[path]
\CTANdirectory{pbox}{macros/latex/contrib/pbox}[pbox]
\CTANdirectory{pcwritex}{support/pcwritex}
\CTANdirectory{pdcmac}{macros/plain/contrib/pdcmac}[pdcmac]
\CTANdirectory{pdfcomment}{macros/latex/contrib/pdfcomment}[pdfcomment]
\CTANdirectory{pdfpages}{macros/latex/contrib/pdfpages}[pdfpages]
\CTANdirectory{pdfrack}{support/pdfrack}[pdfrack]
\CTANdirectory{pdfscreen}{macros/latex/contrib/pdfscreen}[pdfscreen]
\CTANdirectory{pdftex}{systems/pdftex}[pdftex]
\CTANdirectory{pdftex-graphics}{graphics/metapost/contrib/tools/mptopdf}[pdf-mps-supp]
\CTANdirectory{pdftricks}{graphics/pdftricks}[pdftricks]
\CTANdirectory{pdftricks2}{graphics/pdftricks2}[pdftricks2]
\CTANdirectory{pgf}{graphics/pgf/base}[pgf]
\CTANdirectory{phonetic}{fonts/phonetic}
\CTANdirectory{phy-bstyles}{biblio/bibtex/contrib/phy-bstyles}
\CTANdirectory{physe}{macros/physe}
\CTANdirectory{phyzzx}{macros/phyzzx}
\CTANdirectory{picinpar}{macros/latex209/contrib/picinpar}[picinpar]
\CTANdirectory{picins}{macros/latex209/contrib/picins}[picins]
\CTANdirectory{pict2e}{macros/latex/contrib/pict2e}[pict2e]
\CTANdirectory{pictex}{graphics/pictex}[pictex]
\CTANdirectory{pictex-addon}{graphics/pictex/addon}[pictexwd]
\CTANdirectory{pictex-converter}{support/pictex-converter}
\CTANdirectory{pictex-summary}{info/pictex/summary}[pictexsum]
\CTANdirectory{pinlabel}{macros/latex/contrib/pinlabel}[pinlabel]
\CTANdirectory{pkbbox}{fonts/utilities/pkbbox}
\CTANdirectory{pkfix}{support/pkfix}[pkfix]
\CTANdirectory{pkfix-helper}{support/pkfix-helper}[pkfix-helper]
\CTANdirectory{placeins}{macros/latex/contrib/placeins}[placeins]
\CTANdirectory{plain}{macros/plain/base}[plain]
\CTANdirectory*{plastex}{support/plastex}[plastex]
\CTANdirectory{plnfss}{macros/plain/plnfss}[plnfss]
\CTANdirectory{plttopic}{support/plttopic}
\CTANdirectory{pmtex}{systems/os2/pmtex}
\CTANdirectory{pmx}{support/pmx}
\CTANdirectory{l3experimental}{macros/latex/contrib/l3experimental}[l3experimental]
\CTANdirectory{l3kernel}{macros/latex/contrib/l3kernel}[l3kernel]
\CTANdirectory{l3packages}{macros/latex/contrib/l3packages}[l3packages]
\CTANdirectory{polish}{language/polish}
\CTANdirectory{polyglossia}{macros/latex/contrib/polyglossia}[polyglossia]
\CTANdirectory{poorman}{fonts/poorman}
\CTANdirectory{portuguese}{language/portuguese}
\CTANdirectory{poster}{macros/generic/poster}
\CTANdirectory{powerdot}{macros/latex/contrib/powerdot}[powerdot]
\CTANdirectory{pp}{support/pp}
\CTANdirectory{preprint}{macros/latex/contrib/preprint}[preprint]
\CTANdirectory{present}{macros/plain/contrib/present}[present]
\CTANdirectory{preview}{macros/latex/contrib/preview}[preview]
\CTANdirectory{print-fine}{support/print-fine}
\CTANdirectory{printbib}{biblio/bibtex/utils/printbib}
\CTANdirectory{printlen}{macros/latex/contrib/printlen}[printlen]
\CTANdirectory{printsamples}{fonts/utilities/mf2ps/doc/printsamples}
\CTANdirectory{program}{macros/latex/contrib/program}[program]
\CTANdirectory{proofs}{macros/generic/proofs}
\CTANdirectory*{protext}{systems/win32/protext}[protext]
\CTANdirectory{ppower4}{support/ppower4}[ppower4]
\CTANdirectory{ppr-prv}{macros/latex/contrib/ppr-prv}[ppr-prv]
\CTANdirectory{prosper}{macros/latex/contrib/prosper}[prosper]
\CTANdirectory{ps-type3}{fonts/cm/ps-type3}
\CTANdirectory{ps2mf}{fonts/utilities/ps2mf}
\CTANdirectory{ps2pk}{fonts/utilities/ps2pk}[ps2pk]
\CTANdirectory{psbook}{systems/msdos/dviware/psbook}
\CTANdirectory{psbox}{macros/generic/psbox}
\CTANdirectory{pseudocode}{macros/latex/contrib/pseudocode}[pseudocode]
\CTANdirectory{psfig}{graphics/psfig}[psfig]
\CTANdirectory{psfrag}{macros/latex/contrib/psfrag}[psfrag]
\CTANdirectory{psfragx}{macros/latex/contrib/psfragx}[psfragx]
\CTANdirectory{psizzl}{macros/psizzl}
\CTANdirectory{psnfss}{macros/latex/required/psnfss}[psnfss]
\CTANdirectory{psnfss-addons}{macros/latex/contrib/psnfss-addons}
\CTANdirectory{psnfssx-mathtime}{macros/latex/contrib/psnfssx/mathtime}
\CTANdirectory{pspicture}{macros/latex/contrib/pspicture}[pspicture]
\CTANdirectory{psprint}{dviware/psprint}
\CTANdirectory{pst-layout}{graphics/pstricks/contrib/pst-layout}[pst-layout]
\CTANdirectory{pst-pdf}{macros/latex/contrib/pst-pdf}[pst-pdf]
\CTANdirectory{pstoedit}{support/pstoedit}[pstoedit]
\CTANdirectory{pstricks}{graphics/pstricks}[pstricks]
\CTANdirectory{psutils}{support/psutils}
\CTANdirectory{punk}{fonts/punk}
\CTANdirectory{purifyeps}{support/purifyeps}[purifyeps]
\CTANdirectory{pxfonts}{fonts/pxfonts}[pxfonts]
\CTANdirectory{qdtexvpl}{fonts/utilities/qdtexvpl}[qdtexvpl]
\CTANdirectory{qfig}{support/qfig}
\CTANdirectory{quotchap}{macros/latex/contrib/quotchap}[quotchap]
\CTANdirectory{r2bib}{biblio/bibtex/utils/r2bib}[r2bib]
\CTANdirectory{rcs}{macros/latex/contrib/rcs}[rcs]
\CTANdirectory{rcsinfo}{macros/latex/contrib/rcsinfo}[rcsinfo]
\CTANdirectory{realcalc}{macros/generic/realcalc}
\CTANdirectory{refcheck}{macros/latex/contrib/refcheck}[refcheck]
\CTANdirectory{refer-tools}{biblio/bibtex/utils/refer-tools}
\CTANdirectory{refman}{macros/latex/contrib/refman}[refman]
\CTANdirectory{regexpatch}{macros/latex/contrib/regexpatch}[regexpatch]
\CTANdirectory{revtex4-1}{macros/latex/contrib/revtex}[revtex4-1]
\CTANdirectory{rnototex}{support/rnototex}[rnototex]
\CTANdirectory{rotating}{macros/latex/contrib/rotating}[rotating]
\CTANdirectory{rotfloat}{macros/latex/contrib/rotfloat}[rotfloat]
\CTANdirectory{rsfs}{fonts/rsfs}[rsfs]
\CTANdirectory{rsfso}{fonts/rsfso}[rsfso]
\CTANdirectory{rtf2tex}{support/rtf2tex}[rtf2tex]
\CTANdirectory{rtf2html}{support/rtf2html}
\CTANdirectory{rtf2latex}{support/rtf2latex}
\CTANdirectory{rtf2latex2e}{support/rtf2latex2e}[rtf2latex2e]
\CTANdirectory{rtflatex}{support/rtflatex}
\CTANdirectory{rtfutils}{support/tex2rtf/rtfutils}
\CTANdirectory{rumgraph}{support/rumgraph}
\CTANdirectory{sam2p}{graphics/sam2p}[sam2p]
\CTANdirectory{sansmath}{macros/latex/contrib/sansmath}[sansmath]
\CTANdirectory{sauerj}{macros/latex/contrib/sauerj}[sauerj]
\CTANdirectory{savetrees}{macros/latex/contrib/savetrees}[savetrees]
\CTANdirectory{schemeweb}{web/schemeweb}[schemeweb]
\CTANdirectory{sciposter}{macros/latex/contrib/sciposter}[sciposter]
\CTANdirectory{sectsty}{macros/latex/contrib/sectsty}[sectsty]
\CTANdirectory{selectp}{macros/latex/contrib/selectp}[selectp]
\CTANdirectory{seminar}{macros/latex/contrib/seminar}[seminar]
\CTANdirectory{shade}{macros/generic/shade}[shade]
\CTANdirectory{shorttoc}{macros/latex/contrib/shorttoc}[shorttoc]
\CTANdirectory{showexpl}{macros/latex/contrib/showexpl}[showexpl]
\CTANdirectory{showlabels}{macros/latex/contrib/showlabels}[showlabels]
\CTANdirectory{slashbox}{macros/latex/contrib/slashbox}[slashbox]
\CTANdirectory{smallcap}{macros/latex/contrib/smallcap}[smallcap]
\CTANdirectory{smartref}{macros/latex/contrib/smartref}[smartref]
\CTANdirectory{snapshot}{macros/latex/contrib/snapshot}[snapshot]
\CTANdirectory{soul}{macros/latex/contrib/soul}[soul]
\CTANdirectory{spain}{biblio/bibtex/contrib/spain}
\CTANdirectory{spelling}{macros/luatex/generic/spelling}[spelling]
\CTANdirectory{spiderweb}{web/spiderweb}[spiderweb]
\CTANdirectory{splitbib}{macros/latex/contrib/splitbib}[splitbib]
\CTANdirectory{splitindex}{macros/latex/contrib/splitindex}[splitindex]
\CTANdirectory{standalone}{macros/latex/contrib/standalone}[standalone]
\CTANdirectory{stix}{fonts/stix}[stix]
\CTANdirectory{sttools}{macros/latex/contrib/sttools}[sttools]
\CTANdirectory{sty2dtx}{support/sty2dtx}[sty2dtx]
\CTANdirectory{subdepth}{macros/latex/contrib/subdepth}[subdepth]
\CTANdirectory{subfig}{macros/latex/contrib/subfig}[subfig]
\CTANdirectory{subfiles}{macros/latex/contrib/subfiles}[subfiles]
\CTANdirectory{supertabular}{macros/latex/contrib/supertabular}[supertabular]
\CTANdirectory{svn}{macros/latex/contrib/svn}[svn]
\CTANdirectory{svninfo}{macros/latex/contrib/svninfo}[svninfo]
\CTANdirectory{swebib}{biblio/bibtex/contrib/swebib}
\CTANdirectory*{symbols}{info/symbols/comprehensive}[comprehensive]
\CTANdirectory{tablefootnote}{macros/latex/contrib/tablefootnote}[tablefootnote]
\CTANdirectory{tabls}{macros/latex/contrib/tabls}[tabls]
\CTANdirectory{tabulary}{macros/latex/contrib/tabulary}[tabulary]
\CTANdirectory{tagging}{macros/latex/contrib/tagging}
\CTANdirectory{talk}{macros/latex/contrib/talk}[talk]
\CTANdirectory{tcolorbox}{macros/latex/contrib/tcolorbox}[tcolorbox]
\CTANdirectory{tds}{tds}[tds]
\CTANdirectory{ted}{macros/latex/contrib/ted}[ted]
\CTANdirectory*{testflow}{macros/latex/contrib/IEEEtran/testflow}[testflow]
\CTANdirectory*{tetex}{obsolete/systems/unix/teTeX/current/distrib}[tetex]
\CTANdirectory{tex2mail}{support/tex2mail}[tex2mail]
\CTANdirectory{tex2rtf}{support/tex2rtf}[tex2rtf]
\CTANdirectory{texbytopic}{info/texbytopic}[texbytopic]
\CTANdirectory{texcnvfaq}{help/wp-conv}[wp-conv]
\CTANdirectory{texcount}{support/texcount}[texcount]
\CTANdirectory{texdef}{support/texdef}[texdef]
\CTANdirectory{tex-gpc}{systems/unix/tex-gpc}[tex-gpc]
\CTANdirectory{tex-gyre}{fonts/tex-gyre}[tex-gyre]
\CTANdirectory{tex-gyre-math}{fonts/tex-gyre-math}[tex-gyre-math]
\CTANdirectory{tex-overview}{info/tex-overview}[tex-overview]
\CTANdirectory*{texhax}{digests/texhax}[texhax]
\CTANdirectory{texi2html}{support/texi2html}[texi2html]
\CTANdirectory{texindex}{indexing/texindex}[texindex]
\CTANdirectory{texinfo}{macros/texinfo/texinfo}[texinfo]
\CTANdirectory*{texlive}{systems/texlive}[texlive]
\CTANdirectory*{texmacs}{support/TeXmacs}[texmacs]
\CTANdirectory*{texniccenter}{systems/win32/TeXnicCenter}[texniccenter]
\CTANdirectory{texpower}{macros/latex/contrib/texpower}[texpower]
\CTANdirectory{texshell}{systems/msdos/texshell}[texshell]
\CTANdirectory{texsis}{macros/texsis}[texsis]
\CTANdirectory{textcase}{macros/latex/contrib/textcase}[textcase]
\CTANdirectory{textfit}{macros/latex/contrib/textfit}[textfit]
\CTANdirectory{textmerg}{macros/latex/contrib/textmerg}[textmerg]
\CTANdirectory{textpos}{macros/latex/contrib/textpos}[textpos]
\CTANdirectory{textures_figs}{systems/mac/textures_figs}
\CTANdirectory{texutils}{systems/atari/texutils}
\CTANdirectory{tgrind}{support/tgrind}[tgrind]
\CTANdirectory{threeparttable}{macros/latex/contrib/threeparttable}[threeparttable]
\CTANdirectory{threeparttablex}{macros/latex/contrib/threeparttablex}[threeparttablex]
\CTANdirectory{tib}{biblio/tib}[tib]
\CTANdirectory{tiny_c2l}{support/tiny_c2l}[tinyc2l]
\CTANdirectory{tip}{info/examples/tip}
\CTANdirectory{titleref}{macros/latex/contrib/titleref}[titleref]
\CTANdirectory{titlesec}{macros/latex/contrib/titlesec}[titlesec]
\CTANdirectory{titling}{macros/latex/contrib/titling}[titling]
\CTANdirectory{tlc2}{info/examples/tlc2}
\CTANdirectory{tmmath}{fonts/micropress/tmmath}[tmmath]
\CTANdirectory{tocbibind}{macros/latex/contrib/tocbibind}[tocbibind]
\CTANdirectory{tocloft}{macros/latex/contrib/tocloft}[tocloft]
\CTANdirectory{tocvsec2}{macros/latex/contrib/tocvsec2}[tocvsec2]
\CTANdirectory{totpages}{macros/latex/contrib/totpages}[totpages]
\CTANdirectory{tr2latex}{support/tr2latex}[tr2latex]
\CTANdirectory{transfig}{graphics/transfig}[transfig]
\CTANdirectory{try}{support/try}[try]
\CTANdirectory{tt2001}{fonts/ps-type1/tt2001}[tt2001]
\CTANdirectory{tth}{support/tth/dist}[tth]
\CTANdirectory{ttn}{digests/ttn}
\CTANdirectory{tug}{usergrps/tug}
\CTANdirectory{tugboat}{digests/tugboat}
\CTANdirectory{tweb}{web/tweb}[tweb]
\CTANdirectory{txfonts}{fonts/txfonts}[txfonts]
\CTANdirectory{txfontsb}{fonts/txfontsb}[txfontsb]
\CTANdirectory{txtdist}{support/txt}[txt]
\CTANdirectory{type1cm}{macros/latex/contrib/type1cm}[type1cm]
\CTANdirectory{ucharclasses}{macros/xetex/latex/ucharclasses}[ucharclasses]
\CTANdirectory{ucs}{macros/latex/contrib/ucs}[ucs]
\CTANdirectory{ucthesis}{macros/latex/contrib/ucthesis}[ucthesis]
\CTANdirectory{uktex}{digests/uktex}
\CTANdirectory{ulem}{macros/latex/contrib/ulem}[ulem]
\CTANdirectory{umrand}{macros/generic/umrand}
\CTANdirectory{underscore}{macros/latex/contrib/underscore}[underscore]
\CTANdirectory{unicode-math}{macros/latex/contrib/unicode-math}[unicode-math]
\CTANdirectory{unix}{systems/unix}
\CTANdirectory{unpacked}{macros/latex/unpacked}
\CTANdirectory{untex}{support/untex}[untex]
\CTANdirectory{url}{macros/latex/contrib/url}[url]
\CTANdirectory{urlbst}{biblio/bibtex/contrib/urlbst}[urlbst]
\CTANdirectory{urw-base35}{fonts/urw/base35}[urw-base35]
\CTANdirectory{urwchancal}{fonts/urwchancal}[urwchancal]
\CTANdirectory{usebib}{macros/latex/contrib/usebib}[usebib]
\CTANdirectory{utopia}{fonts/utopia}[utopia]
\CTANdirectory{varisize}{macros/plain/contrib/varisize}[varisize]
\CTANdirectory{varwidth}{macros/latex/contrib/varwidth}[varwidth]
\CTANdirectory{vc}{support/vc}[vc]
\CTANdirectory{verbatim}{macros/latex/required/tools}[verbatim]
\CTANdirectory{verbatimbox}{macros/latex/contrib/verbatimbox}[verbatimbox]
\CTANdirectory{verbdef}{macros/latex/contrib/verbdef}[verbdef]
\CTANdirectory{version}{macros/latex/contrib/version}[version]
\CTANdirectory{vertbars}{macros/latex/contrib/vertbars}[vertbars]
\CTANdirectory{vita}{macros/latex/contrib/vita}[vita]
\CTANdirectory{vmargin}{macros/latex/contrib/vmargin}[vmargin]
\CTANdirectory{vmspell}{support/vmspell}[vmspell]
\CTANdirectory{vpp}{support/view_print_ps_pdf}[vpp]
\CTANdirectory{vruler}{macros/latex/contrib/vruler}[vruler]
\CTANdirectory{vtex-common}{systems/vtex/common}
\CTANdirectory{vtex-linux}{systems/vtex/linux}[vtex-free]
\CTANdirectory{vtex-os2}{systems/vtex/os2}[vtex-free]
\CTANdirectory{wallpaper}{macros/latex/contrib/wallpaper}[wallpaper]
\CTANdirectory{was}{macros/latex/contrib/was}[was]
\CTANdirectory{wd2latex}{support/wd2latex}
\CTANdirectory{web}{systems/knuth/dist/web}[web]
\CTANdirectory*{winedt}{systems/win32/winedt}[winedt]
\CTANdirectory{wordcount}{macros/latex/contrib/wordcount}[wordcount]
\CTANdirectory{wp2latex}{support/wp2latex}[wp2latex]
\CTANdirectory{wrapfig}{macros/latex/contrib/wrapfig}[wrapfig]
\CTANdirectory{xargs}{macros/latex/contrib/xargs}[xargs]
\CTANdirectory{xbibfile}{biblio/bibtex/utils/xbibfile}[xbibfile]
\CTANdirectory{xcolor}{macros/latex/contrib/xcolor}[xcolor]
\CTANdirectory{xcomment}{macros/generic/xcomment}[xcomment]
\CTANdirectory*{xdvi}{dviware/xdvi}[xdvi]
\CTANdirectory{xecjk}{macros/xetex/latex/xecjk}[xecjk]
\CTANdirectory{xetexref}{info/xetexref}[xetexref]
\CTANdirectory*{xfig}{graphics/xfig}[xfig]
\CTANdirectory*{xindy}{indexing/xindy}[xindy]
\CTANdirectory{xits}{fonts/xits}[xits]
\CTANdirectory{xkeyval}{macros/latex/contrib/xkeyval}[xkeyval]
\CTANdirectory{xmltex}{macros/xmltex/base}[xmltex]
\CTANdirectory*{xpdf}{support/xpdf}[xpdf]
\CTANdirectory{xtab}{macros/latex/contrib/xtab}[xtab]
\CTANdirectory{xwatermark}{macros/latex/contrib/xwatermark}[xwatermark]
\CTANdirectory{yagusylo}{macros/latex/contrib/yagusylo}[yagusylo]
\CTANdirectory{yhmath}{fonts/yhmath}[yhmath]
\CTANdirectory{zefonts}{fonts/zefonts}[zefonts]
\CTANdirectory{ziffer}{macros/latex/contrib/ziffer}[ziffer]
\CTANdirectory{zoon-mp-eg}{info/metapost/examples}[metapost-examples]
\CTANdirectory{zwpagelayout}{macros/latex/contrib/zwpagelayout}[zwpagelayout]
\endinput

% $Id: filectan.tex,v 1.143 2012/12/07 19:34:33 rf10 Exp rf10 $
%
% protect ourself against being read twice
\csname readCTANfiles\endcsname
\let\readCTANfiles\endinput
%
% interesting/useful individual files to be found on CTAN
\CTANfile{CTAN-sites}{CTAN.sites}
\CTANfile{CTAN-uploads}{README.uploads}% yes, it really is in the root
\CTANfile{Excalibur}{systems/mac/support/excalibur/Excalibur-4.0.2.sit.hqx}[excalibur]
\CTANfile{expl3-doc}{macros/latex/contrib/l3kernel/expl3.pdf}[l3kernel]
\CTANfile{f-byname}{FILES.byname}
\CTANfile{f-last7}{FILES.last07days}
\CTANfile{interface3-doc}{macros/latex/contrib/l3kernel/interface3.pdf}[l3kernel]
\CTANfile{LitProg-FAQ}{help/comp.programming.literate_FAQ}
\CTANfile{OpenVMSTeX}{systems/OpenVMS/TEX97_CTAN.ZIP}
\CTANfile{T1instguide}{info/Type1fonts/fontinstallationguide/fontinstallationguide.pdf}
\CTANfile{TeX-FAQ}{obsolete/help/TeX,_LaTeX,_etc.:_Frequently_Asked_Questions_with_Answers}
\CTANfile{abstract-bst}{biblio/bibtex/utils/bibtools/abstract.bst}
\CTANfile{backgrnd}{macros/generic/misc/backgrnd.tex}[backgrnd]
\CTANfile{bbl2html}{biblio/bibtex/utils/misc/bbl2html.awk}[bbl2html]
\CTANfile{beginlatex-pdf}{info/beginlatex/beginlatex-3.6.pdf}[beginlatex]
\CTANfile{bibtex-faq}{biblio/bibtex/contrib/doc/btxFAQ.pdf}
\CTANfile{bidstobibtex}{biblio/bibtex/utils/bids/bids.to.bibtex}[bidstobibtex]
\CTANfile{btxmactex}{macros/eplain/tex/btxmac.tex}[eplain]
\CTANfile{catalogue}{help/Catalogue/catalogue.html}
\CTANfile{cat-licences}{help/Catalogue/licenses.html}
\CTANfile{clsguide}{macros/latex/doc/clsguide.pdf}[clsguide]
\CTANfile{compactbib}{macros/latex/contrib/compactbib/compactbib.sty}[compactbib]
%\CTANfile{compan-ctan}{info/companion.ctan}
\CTANfile{context-tmf}{macros/context/current/cont-tmf.zip}[context]
\CTANfile{dvitype}{systems/knuth/dist/texware/dvitype.web}[dvitype]
\CTANfile{edmetrics}{systems/mac/textures/utilities/EdMetrics.sea.hqx}[edmetrics]
\CTANfile{epsf}{macros/generic/epsf/epsf.tex}[epsf]
\CTANfile{figsinlatex}{obsolete/info/figsinltx.ps}
\CTANfile{finplain}{biblio/bibtex/contrib/misc/finplain.bst}
\CTANfile{fix-cm}{macros/latex/unpacked/fix-cm.sty}[fix-cm]
\CTANfile{fntguide.pdf}{macros/latex/doc/fntguide.pdf}[fntguide]
\CTANfile{fontdef}{macros/latex/base/fontdef.dtx}
\CTANfile{fontmath}{macros/latex/unpacked/fontmath.ltx}
\CTANfile{gentle}{info/gentle/gentle.pdf}[gentle]
\CTANfile{gkpmac}{systems/knuth/local/lib/gkpmac.tex}[gkpmac]
\CTANfile{knuth-letter}{systems/knuth/local/lib/letter.tex}
\CTANfile{knuth-tds}{macros/latex/contrib/latex-tds/knuth.tds.zip}
\CTANfile{latex209-base}{obsolete/macros/latex209/distribs/latex209.tar.gz}[latex209]
\CTANfile{latex-classes}{macros/latex/base/classes.dtx}
\CTANfile{latex-source}{macros/latex/base/source2e.tex}
\CTANfile{latexcount}{support/latexcount/latexcount.pl}[latexcount]
\CTANfile{latexcheat}{info/latexcheat/latexcheat/latexsheet.pdf}[latexcheat]
\CTANfile{letterspacing}{macros/generic/misc/letterspacing.tex}[letterspacing]
\CTANfile{ltablex}{macros/latex/contrib/ltablex/ltablex.sty}[ltablex]
\CTANfile{ltxguide}{macros/latex/base/ltxguide.cls}
\CTANfile{ltxtable}{macros/latex/contrib/carlisle/ltxtable.tex}[ltxtable]
\CTANfile{lw35nfss-zip}{macros/latex/required/psnfss/lw35nfss.zip}[lw35nfss]
\CTANfile{macmakeindex}{systems/mac/macmakeindex2.12.sea.hqx}
\CTANfile{mathscript}{info/symbols/math/scriptfonts.pdf}
\CTANfile{mathsurvey.html}{info/Free_Math_Font_Survey/en/survey.html}
\CTANfile{mathsurvey.pdf}{info/Free_Math_Font_Survey/en/survey.pdf}
\CTANfile{memoir-man}{macros/latex/contrib/memoir/memman.pdf}
\CTANfile{metafp-pdf}{info/metafont/metafp/metafp.pdf}[metafp]
\CTANfile{mf-beginners}{info/metafont/beginners/metafont-for-beginners.pdf}[metafont-beginners]
\CTANfile{mf-list}{info/metafont-list}
\CTANfile{miktex-portable}{systems/win32/miktex/setup/miktex-portable.exe}
\CTANfile{miktex-setup}{systems/win32/miktex/setup/setup.exe}[miktex]
\CTANfile{mil}{info/mil/mil.pdf}
\CTANfile{mil-short}{info/Math_into_LaTeX-4/Short_Course.pdf}[math-into-latex-4]
\CTANfile{modes-file}{fonts/modes/modes.mf}[modes]
\CTANfile{mtw}{info/makingtexwork/mtw-1.0.1-html.tar.gz}
\CTANfile{multind}{macros/latex209/contrib/misc/multind.sty}[multind]
\CTANfile{nextpage}{macros/latex/contrib/misc/nextpage.sty}[nextpage]
\CTANfile{noTeX}{biblio/bibtex/utils/misc/noTeX.bst}[notex]
\CTANfile{numline}{obsolete/macros/latex/contrib/numline/numline.sty}[numline]
\CTANfile{patch}{macros/generic/misc/patch.doc}[patch]
\CTANfile{picins-summary}{macros/latex209/contrib/picins/picins.txt}
\CTANfile{pk300}{fonts/cm/pk/pk300.zip}
\CTANfile{pk300w}{fonts/cm/pk/pk300w.zip}
\CTANfile{QED}{macros/generic/proofs/taylor/QED.sty}[qed]
\CTANfile{removefr}{macros/latex/contrib/fragments/removefr.tex}[removefr]
\CTANfile{repeat}{macros/generic/eijkhout/repeat.tex}[repeat]
\CTANfile{resume}{obsolete/macros/latex209/contrib/resume/resume.sty}
\CTANfile{savesym}{macros/latex/contrib/savesym/savesym.sty}[savesym]
\CTANfile{setspace}{macros/latex/contrib/setspace/setspace.sty}[setspace]
\CTANfile{simpl-latex}{info/simplified-latex/simplified-intro.pdf}[simplified-latex]
\CTANfile{sober}{macros/latex209/contrib/misc/sober.sty}[sober]
\CTANfile{tex2bib}{biblio/bibtex/utils/tex2bib/tex2bib}[tex2bib]
\CTANfile{tex2bib-doc}{biblio/bibtex/utils/tex2bib/README}
\CTANfile{tex4ht}{obsolete/support/TeX4ht/tex4ht-all.zip}[tex4ht]
\CTANfile{texlive-unix}{systems/texlive/tlnet/install-tl-unx.tar.gz}
\CTANfile{texlive-windows}{systems/texlive/tlnet/install-tl.zip}
\CTANfile{texnames}{info/biblio/texnames.sty}
\CTANfile{texsis-index}{macros/texsis/index/index.tex}
\CTANfile{topcapt}{macros/latex/contrib/misc/topcapt.sty}[topcapt]
\CTANfile{tracking}{macros/latex/contrib/tracking/tracking.sty}[tracking]
\CTANfile{ttb-pdf}{info/bibtex/tamethebeast/ttb_en.pdf}[tamethebeast]
\CTANfile{type1ec}{fonts/ps-type1/cm-super/type1ec.sty}[type1ec]
\CTANfile{ukhyph}{language/hyphenation/ukhyphen.tex}
\CTANfile{upquote}{macros/latex/contrib/upquote/upquote.sty}[upquote]
\CTANfile{faq-a4}{help/uk-tex-faq/newfaq.pdf}
\CTANfile{faq-letter}{help/uk-tex-faq/letterfaq.pdf}
\CTANfile{versions}{macros/latex/contrib/versions/versions.sty}[versions]
\CTANfile{vf-howto}{info/virtualfontshowto/virtualfontshowto.txt}[vf-howto]
\CTANfile{vf-knuth}{info/knuth/virtual-fonts}[vf-knuth]
\CTANfile{visualFAQ}{info/visualFAQ/visualFAQ.pdf}[visualfaq]
\CTANfile{voss-mathmode}{info/math/voss/mathmode/Mathmode.pdf}
\CTANfile{wujastyk-txh}{digests/texhax/txh/wujastyk.txh}
\CTANfile{xampl-bib}{biblio/bibtex/base/xampl.bib}
\CTANfile{xtexcad}{graphics/xtexcad/xtexcad-2.4.1.tar.gz}

%%%%%%%%%%%%%%%%%%%%%%%%%%%%%%%%%%%%%%%%%%%%%%%%%%%%%%%%%%%%%%%%%

\ifsinglecolumn
  \def\faqfileversion{3.28}    \def\faqfiledate{2014-06-10}
%
% The above line defines the file version and date, and must remain
% the first line with any `assignment' in the file, or things will
% blow up in a stupid fashion
%
% get lists of CTAN labels
%
% configuration for the lists, if we're going to need to generate urls
% for the files
\InputIfFileExists{archive.cfg}{}{}
%
% ... directories
% $Id: dirctan.tex,v 1.302 2013/07/24 21:43:10 rf10 Exp rf10 $
%
% protect ourself against being read twice
\csname readCTANdirs\endcsname
\let\readCTANdirs\endinput
%
% declarations of significant directories on CTAN
\CTANdirectory{2etools}{macros/latex/required/tools}[tools]
\CTANdirectory{4spell}{support/4spell}[fourspell]
\CTANdirectory*{Catalogue}{help/Catalogue}
\CTANdirectory*{MathTeX}{support/mathtex}[mathtex]
\CTANdirectory{MimeTeX}{support/mimetex}[mimetex]
\CTANdirectory{Tabbing}{macros/latex/contrib/Tabbing}[tabbing]
\CTANdirectory*{TeXtelmExtel}{systems/msdos/emtex-contrib/TeXtelmExtel}
\CTANdirectory{TftI}{info/impatient}[impatient]
\CTANdirectory{a0poster}{macros/latex/contrib/a0poster}[a0poster]
\CTANdirectory*{a2ping}{graphics/a2ping}[a2ping]
\CTANdirectory{a4}{macros/latex/contrib/a4}[a4]
\CTANdirectory*{abc2mtex}{support/abc2mtex}
\CTANdirectory{abstract}{macros/latex/contrib/abstract}[abstract]
\CTANdirectory{abstyles}{biblio/bibtex/contrib/abstyles}
\CTANdirectory{accents}{support/accents}
\CTANdirectory{acronym}{macros/latex/contrib/acronym}[acronym]
\CTANdirectory*{ada}{web/ada/aweb}
\CTANdirectory{addindex}{indexing/addindex}
\CTANdirectory{addlines}{macros/latex/contrib/addlines}[addlines]
\CTANdirectory{adjkerns}{fonts/utilities/adjkerns}
\CTANdirectory{ae}{fonts/ae}[ae]
\CTANdirectory{aeguill}{macros/latex/contrib/aeguill}[aeguill]
\CTANdirectory{afmtopl}{fonts/utilities/afmtopl}
\CTANdirectory{akletter}{macros/latex/contrib/akletter}
\CTANdirectory{aleph}{systems/aleph}
\CTANdirectory{alg}{macros/latex/contrib/alg}
\CTANdirectory{algorithm2e}{macros/latex/contrib/algorithm2e}[algorithm2e]
\CTANdirectory{algorithmicx}{macros/latex/contrib/algorithmicx}[algorithmicx]
\CTANdirectory{algorithms}{macros/latex/contrib/algorithms}[algorithms]
\CTANdirectory*{alpha}{systems/mac/support/alpha}[alpha]
\CTANdirectory{amiga}{systems/amiga}
\CTANdirectory{amscls}{macros/latex/required/amslatex/amscls}[amscls]
\CTANdirectory{amsfonts}{fonts/amsfonts}[amsfonts]
\CTANdirectory{amslatex}{macros/latex/required/amslatex}[amslatex]
\CTANdirectory{amslatex-primer}{info/amslatex/primer}[amslatex-primer]
\CTANdirectory{amspell}{support/amspell}
\CTANdirectory{amsrefs}{macros/latex/contrib/amsrefs}[amsrefs]
\CTANdirectory{amstex}{macros/amstex}[amstex]
\CTANdirectory{anonchap}{macros/latex/contrib/anonchap}[anonchap]
\CTANdirectory{answers}{macros/latex/contrib/answers}[answers]
\CTANdirectory{ant}{systems/ant}[ant]
\CTANdirectory{anyfontsize}{macros/latex/contrib/anyfontsize}[anyfontsize]
\CTANdirectory{apl}{fonts/apl}
\CTANdirectory{aplweb}{web/apl/aplweb}
\CTANdirectory{appl}{web/reduce/rweb/appl}
\CTANdirectory{appendix}{macros/latex/contrib/appendix}[appendix]
\CTANdirectory{arabtex}{language/arabic/arabtex}
\CTANdirectory{arara}{support/arara}[arara]
\CTANdirectory{aro-bend}{info/challenges/aro-bend}[aro-bend]
\CTANdirectory{asana-math}{fonts/Asana-Math}[asana-math]
\CTANdirectory{asc2tex}{systems/msdos/asc2tex}
\CTANdirectory{ascii}{fonts/ascii}
\CTANdirectory*{aspell}{support/aspell}[aspell]
\CTANdirectory{astro}{fonts/astro}
\CTANdirectory{asymptote}{graphics/asymptote}[asymptote]
\CTANdirectory{asyfig}{macros/latex/contrib/asyfig}[asyfig]
\CTANdirectory{atari}{systems/atari}
\CTANdirectory*{atari-cstex}{systems/atari/cs-tex}[atari-cstex]
\CTANdirectory{attachfile}{macros/latex/contrib/attachfile}
\CTANdirectory*{auctex}{support/auctex}[auctex]
\CTANdirectory{autolatex}{support/autolatex}
\CTANdirectory{auto-pst-pdf}{macros/latex/contrib/auto-pst-pdf}[auto-pst-pdf]
\CTANdirectory{aweb}{web/ada/aweb}
\CTANdirectory*{awk}{web/spiderweb/src/awk}
\CTANdirectory{axodraw}{graphics/axodraw}[axodraw]
\CTANdirectory{babel}{macros/latex/required/babel}[babel]
\CTANdirectory{babelbib}{biblio/bibtex/contrib/babelbib}[babelbib]
\CTANdirectory{badge}{macros/plain/contrib/badge}
\CTANdirectory{bakoma}{fonts/cm/ps-type1/bakoma}[bakoma-fonts]
\CTANdirectory*{bakoma-tex}{systems/win32/bakoma}[bakoma]
\CTANdirectory*{bakoma-texfonts}{systems/win32/bakoma/fonts}
\CTANdirectory*{bard}{fonts/bard}
\CTANdirectory{barr}{macros/generic/diagrams/barr}
\CTANdirectory{bashkirian}{fonts/cyrillic/bashkirian}
\CTANdirectory{basix}{macros/generic/basix}
\CTANdirectory{bbding}{fonts/bbding}
\CTANdirectory{bbfig}{support/bbfig}
\CTANdirectory{bbm}{fonts/cm/bbm}[bbm]
\CTANdirectory{bbm-macros}{macros/latex/contrib/bbm}[bbm-macros]
\CTANdirectory{bbold}{fonts/bbold}[bbold]
\CTANdirectory{bdfchess}{fonts/chess/bdfchess}
\CTANdirectory{beamer}{macros/latex/contrib/beamer}[beamer]
\CTANdirectory{beamerposter}{macros/latex/contrib/beamerposter}[beamerposter]
\CTANdirectory{beebe}{dviware/beebe}
\CTANdirectory{belleek}{fonts/belleek}[belleek]
\CTANdirectory{beton}{macros/latex/contrib/beton}[beton]
\CTANdirectory{bezos}{macros/latex/contrib/bezos}[bezos]
\CTANdirectory{bib-fr}{biblio/bibtex/contrib/bib-fr}
\CTANdirectory{bib2dvi}{biblio/bibtex/utils/bib2dvi}
\CTANdirectory{bib2xhtml}{biblio/bibtex/utils/bib2xhtml}
\CTANdirectory*{bibcard}{biblio/bibtex/utils/bibcard}
\CTANdirectory*{bibclean}{biblio/bibtex/utils/bibclean}
\CTANdirectory*{bibdb}{support/bibdb}
\CTANdirectory{biber}{biblio/biber}[biber]
\CTANdirectory{bibextract}{biblio/bibtex/utils/bibextract}
\CTANdirectory{bibgerm}{biblio/bibtex/contrib/germbib}
\CTANdirectory{bibindex}{biblio/bibtex/utils/bibindex}
\CTANdirectory{biblatex}{macros/latex/contrib/biblatex}[biblatex]
\CTANdirectory*{biblatex-contrib}{macros/latex/contrib/biblatex-contrib}
\CTANdirectory{biblio}{info/biblio}
\CTANdirectory{biblist}{macros/latex209/contrib/biblist}
\CTANdirectory{bibsort}{biblio/bibtex/utils/bibsort}
\CTANdirectory{bibtex}{biblio/bibtex/base}[bibtex]
\CTANdirectory*{bibtex8}{biblio/bibtex/8-bit}[bibtex8bit]
\CTANdirectory{bibtex-doc}{biblio/bibtex/contrib/doc}[bibtex]
\CTANdirectory{bibtool}{biblio/bibtex/utils/bibtool}
\CTANdirectory{bibtools}{biblio/bibtex/utils/bibtools}
\CTANdirectory{bibtopic}{macros/latex/contrib/bibtopic}[bibtopic]
\CTANdirectory{bibunits}{macros/latex/contrib/bibunits}[bibunits]
\CTANdirectory{bibview}{biblio/bibtex/utils/bibview}
\CTANdirectory{bigfoot}{macros/latex/contrib/bigfoot}[bigfoot]
\CTANdirectory{bigstrut}{macros/latex/contrib/multirow}[bigstrut]
\CTANdirectory{bit2spr}{graphics/bit2spr}
\CTANdirectory{black}{fonts/cm/utilityfonts/black}
\CTANdirectory{blackboard}{info/symbols/blackboard}[blackboard]
\CTANdirectory{blackletter}{fonts/blackletter}
\CTANdirectory{blindtext}{macros/latex/contrib/blindtext}[blindtext]
\CTANdirectory{blocks}{macros/text1/blocks}
\CTANdirectory{blu}{macros/blu}
\CTANdirectory{bm2font}{graphics/bm2font}
\CTANdirectory{boites}{macros/latex/contrib/boites}[boites]
\CTANdirectory{bold}{fonts/cm/mf-extra/bold}
\CTANdirectory{bold-extra}{macros/latex/contrib/bold-extra}[bold-extra]
\CTANdirectory{bonus}{systems/msdos/emtex-contrib/bonus}
\CTANdirectory{booktabs}{macros/latex/contrib/booktabs}[booktabs]
\CTANdirectory{boondox}{fonts/boondox}[boondox]
\CTANdirectory{borceux}{macros/generic/diagrams/borceux}
\CTANdirectory{braket}{macros/latex/contrib/braket}[braket]
\CTANdirectory{breakurl}{macros/latex/contrib/breakurl}[breakurl]
\CTANdirectory{bridge}{macros/plain/contrib/bridge}
\CTANdirectory{brief_t}{support/brief_t}
\CTANdirectory{bst}{biblio/bibtex/contrib/germbib/bst}
\CTANdirectory{btable}{macros/plain/contrib/btable}
\CTANdirectory{btex8fmt}{macros/generic/cptex/btex8fmt}
\CTANdirectory{btOOL}{biblio/bibtex/utils/btOOL}
\CTANdirectory{bundledoc}{support/bundledoc}[bundledoc]
\CTANdirectory{c}{web/spiderweb/src/c}
\CTANdirectory{c++}{web/spiderweb/src/c++}
\CTANdirectory{c++2latex}{support/C++2LaTeX-1_1pl1}
\CTANdirectory{c2cweb}{web/c_cpp/c2cweb}
\CTANdirectory{c2latex}{support/c2latex}
\CTANdirectory{c_cpp}{web/c_cpp}
\CTANdirectory{caesar-fonts-generic.dir}{macros/generic/caesarcm/caesar-fonts-generic.dir}
\CTANdirectory{caesarcm}{macros/generic/caesarcm}
\CTANdirectory{caesarcmfonts.dir}{macros/generic/caesarcm/caesarcmfonts.dir}
\CTANdirectory{caesarcmv2.dir}{macros/generic/caesarcm/caesarcmv2.dir}
\CTANdirectory{calendar}{macros/plain/contrib/calendar}
\CTANdirectory{calligra}{fonts/calligra}
\CTANdirectory{calrsfs}{macros/latex/contrib/calrsfs}
\CTANdirectory{cancel}{macros/latex/contrib/cancel}[cancel]
\CTANdirectory{capt-of}{macros/latex/contrib/capt-of}[capt-of]
\CTANdirectory{caption}{macros/latex/contrib/caption}[caption]
\CTANdirectory{carlisle}{macros/latex/contrib/carlisle}[carlisle]
\CTANdirectory{cascover}{macros/plain/contrib/cascover}
\CTANdirectory{casslbl}{macros/plain/contrib/casslbl}
\CTANdirectory{catdvi}{dviware/catdvi}[catdvi]
\CTANdirectory{ccaption}{macros/latex/contrib/ccaption}[ccaption]
\CTANdirectory{ccfonts}{macros/latex/contrib/ccfonts}[ccfonts]
\CTANdirectory{cellular}{macros/plain/contrib/cellular}
\CTANdirectory{cellspace}{macros/latex/contrib/cellspace}[cellspace]
\CTANdirectory{changebar}{macros/latex/contrib/changebar}[changebar]
\CTANdirectory{changepage}{macros/latex/contrib/changepage}[changepage]
\CTANdirectory{changes}{macros/latex/contrib/changes}[changes]
\CTANdirectory{chappg}{macros/latex/contrib/chappg}[chappg]
\CTANdirectory{chapterfolder}{macros/latex/contrib/chapterfolder}[chapterfolder]
\CTANdirectory{charconv}{support/charconv}
\CTANdirectory{charter}{fonts/charter}
\CTANdirectory{chbar}{macros/plain/contrib/chbar}
\CTANdirectory{chbars}{macros/latex209/contrib/chbars}
\CTANdirectory{check}{support/check}
\CTANdirectory{chemstruct}{macros/latex209/contrib/chemstruct}
\CTANdirectory{chemtex}{macros/latex209/contrib/chemtex}
\CTANdirectory{cheq}{fonts/chess/cheq}
\CTANdirectory{cherokee}{fonts/cherokee}
\CTANdirectory{chesstools}{support/chesstools}
\CTANdirectory{chi2tex}{support/chi2tex}
\CTANdirectory{china2e}{macros/latex/contrib/china2e}
\CTANdirectory{chinese}{language/chinese}
\CTANdirectory{chngcntr}{macros/latex/contrib/chngcntr}[chngcntr]
\CTANdirectory{circ}{macros/generic/diagrams/circ}
\CTANdirectory{circuit_macros}{graphics/circuit_macros}
\CTANdirectory{cirth}{fonts/cirth}
\CTANdirectory{cite}{macros/latex/contrib/cite}[cite]
\CTANdirectory{citeref}{macros/latex/contrib/citeref}[citeref]
\CTANdirectory{clark}{fonts/utilities/afmtopl/clark}
\CTANdirectory{classico}{fonts/urw/classico}[classico]
\CTANdirectory{clrscode}{macros/latex/contrib/clrscode}[clrscode]
\CTANdirectory{cm}{fonts/cm}
\CTANdirectory{cm-lgc}{fonts/ps-type1/cm-lgc}[cm-lgc]
\CTANdirectory{cm-super}{fonts/ps-type1/cm-super}[cm-super]
\CTANdirectory{cm-unicode}{fonts/cm-unicode}[cm-unicode]
\CTANdirectory{cmactex}{systems/mac/cmactex}[cmactex]
\CTANdirectory{cmap}{macros/latex/contrib/cmap}
\CTANdirectory{cmastro}{fonts/cmastro}
\CTANdirectory{cmbright}{fonts/cmbright}[cmbright]
\CTANdirectory{cmcyralt}{macros/latex/contrib/cmcyralt}
%[fonts/cmcyralt]
\CTANdirectory{cmfrak}{fonts/gothic/cmfrak}
\CTANdirectory{cmoefont}{fonts/cmoefont}
\CTANdirectory{cmolddig}{fonts/cmolddig}
\CTANdirectory{cmoutlines}{fonts/cm/cmoutlines}[cmoutlines]
\CTANdirectory{cmpica}{fonts/cmpica}
\CTANdirectory{cms_help_files}{macros/text1/cms_help_files}
\CTANdirectory{cmtest}{fonts/cm/cmtest}
\CTANdirectory{cnoweb}{web/c_cpp/cnoweb}
\CTANdirectory{collref}{macros/latex/contrib/collref}
\CTANdirectory{combine}{macros/latex/contrib/combine}[combine]
\CTANdirectory{commado}{macros/generic/commado}[commado]
\CTANdirectory{comment}{macros/latex/contrib/comment}[comment]
\CTANdirectory{committee}{fonts/unsupported/committee}
\CTANdirectory{comp-fonts-FAQ}{help/comp-fonts-FAQ}
\CTANdirectory{components-of-TeX}{info/components-of-TeX}
\CTANdirectory{compugraphics_8600}{macros/text1/compugraphics_8600}
\CTANdirectory{concmath}{macros/latex/contrib/concmath}[concmath]
\CTANdirectory{concmath-f}{fonts/concmath}[concmath-fonts]
\CTANdirectory{concrete}{fonts/concrete}[concrete]
\CTANdirectory{context}{macros/context/current}[context]
\CTANdirectory{context-contrib}{macros/context/contrib}
\CTANdirectory{cprotect}{macros/latex/contrib/cprotect}[cprotect]
\CTANdirectory{cptex}{macros/generic/cptex}
\CTANdirectory{crop}{macros/latex/contrib/crop}[crop]
\CTANdirectory*{crosstex}{biblio/crosstex}[crosstex]
\CTANdirectory{crosswrd}{macros/latex/contrib/crosswrd}
\CTANdirectory{crudetype}{dviware/crudetype}[crudetype]
\CTANdirectory{crw}{macros/plain/contrib/crw}
\CTANdirectory{cs-tex}{systems/atari/cs-tex}
\CTANdirectory{csvsimple}{macros/latex/contrib/csvsimple}[csvsimple]
\CTANdirectory{ctable}{macros/latex/contrib/ctable}[ctable]
\CTANdirectory{ctan}{help/ctan}
\CTANdirectory{cun}{fonts/cun}
\CTANdirectory{currfile}{macros/latex/contrib/currfile}[currfile]
\CTANdirectory{currvita}{macros/latex/contrib/currvita}[currvita]
\CTANdirectory{curve}{macros/latex/contrib/curve}[curve]
\CTANdirectory{curves}{macros/latex/contrib/curves}
\CTANdirectory{custom-bib}{macros/latex/contrib/custom-bib}[custom-bib]
\CTANdirectory{cutwin}{macros/latex/contrib/cutwin}[cutwin]
\CTANdirectory{cweb}{web/c_cpp/cweb}
\CTANdirectory{cweb-p}{web/c_cpp/cweb-p}
\CTANdirectory{cypriote}{fonts/cypriote}
\CTANdirectory{cyrillic}{language/cyrillic}
\CTANdirectory{cyrtug}{language/cyrtug}
\CTANdirectory{dante}{usergrps/dante}
\CTANdirectory{dante-faq}{help/de-tex-faq}
\CTANdirectory{databases}{biblio/bibtex/databases}
\CTANdirectory{datatool}{macros/latex/contrib/datatool}[datatool]
\CTANdirectory{datetime}{macros/latex/contrib/datetime}
\CTANdirectory{db2tex}{support/db2tex}
\CTANdirectory{dblfloatfix}{macros/latex/contrib/dblfloatfix}[dblfloatfix]
\CTANdirectory{dbtex}{support/dbtex}
\CTANdirectory{dc-latex}{language/hyphen-accent/dc-latex}
\CTANdirectory{dc-nfss}{language/hyphen-accent/dc-nfss}
\CTANdirectory{detex}{support/detex}[detex]
\CTANdirectory{devanagari}{language/devanagari}
\CTANdirectory{diagbox}{macros/latex/contrib/diagbox}[diagbox]
\CTANdirectory{diagrams}{macros/generic/diagrams}
\CTANdirectory{dijkstra}{web/spiderweb/src/dijkstra}
\CTANdirectory{dinbrief}{macros/latex/contrib/dinbrief}
\CTANdirectory{dingbat}{fonts/dingbat}
\CTANdirectory{djgpp}{systems/msdos/djgpp}
\CTANdirectory{dk-bib}{biblio/bibtex/contrib/dk-bib}
\CTANdirectory{dktools}{support/dktools}[dktools]
\CTANdirectory{dm-latex}{language/hyphen-accent/dm-latex}
\CTANdirectory{dm-plain}{language/hyphen-accent/dm-plain}
\CTANdirectory{doc2sty}{language/swedish/slatex/doc2sty}
\CTANdirectory{docmfp}{macros/latex/contrib/docmfp}[docmfp]
\CTANdirectory{docmute}{macros/latex/contrib/docmute}[docmute]
\CTANdirectory{docu}{support/makeprog/docu}
\CTANdirectory{document}{biblio/bibtex/contrib/germbib/document}
\CTANdirectory{dos-dc}{systems/msdos/dos-dc}
\CTANdirectory{dos-psfonts}{systems/msdos/emtex-fonts/psfonts}
\CTANdirectory{doublestroke}{fonts/doublestroke}[doublestroke]
\CTANdirectory{dowith}{macros/generic/dowith}[dowith]
\CTANdirectory{dpfloat}{macros/latex/contrib/dpfloat}[dpfloat]
\CTANdirectory{dpmigcc}{systems/msdos/dpmigcc}
\CTANdirectory{draftcopy}{macros/latex/contrib/draftcopy}[draftcopy]
\CTANdirectory{draftwatermark}{macros/latex/contrib/draftwatermark}[draftwatermark]
\CTANdirectory{dratex}{graphics/dratex}[dratex]
\CTANdirectory{drawing}{graphics/drawing}
\CTANdirectory{dropcaps}{macros/latex209/contrib/dropcaps}
\CTANdirectory{dropping}{macros/latex/contrib/dropping}[dropping]
\CTANdirectory{dtl}{dviware/dtl}[dtl]
\CTANdirectory{dtxgen}{support/dtxgen}[dtxgen]
\CTANdirectory{dtxtut}{info/dtxtut}[dtxtut]
\CTANdirectory{duerer}{fonts/duerer}
\CTANdirectory{dvgt}{dviware/dvgt}
\CTANdirectory{dvi-augsburg}{dviware/dvi-augsburg}
\CTANdirectory{dvi2bitmap}{dviware/dvi2bitmap}[dvi2bitmap]
\CTANdirectory{dvi2pcl}{dviware/dvi2pcl}
\CTANdirectory{dvi2tty}{dviware/dvi2tty}[dvi2tty]
\CTANdirectory{dviasm}{dviware/dviasm}[dviasm]
\CTANdirectory{dvibit}{dviware/dvibit}
\CTANdirectory{dvibook}{dviware/dvibook}
\CTANdirectory{dvichk}{dviware/dvichk}
\CTANdirectory{dvicopy}{dviware/dvicopy}
\CTANdirectory{dvidjc}{dviware/dvidjc}
\CTANdirectory{dvidvi}{dviware/dvidvi}
\CTANdirectory{dviimp}{dviware/dviimp}
\CTANdirectory{dviljk}{dviware/dviljk}
\CTANdirectory{dvimerge}{dviware/dvimerge}
\CTANdirectory{dvimfj}{systems/msdos/emtex-contrib/dvimfj}
\CTANdirectory{dvipage}{dviware/dvipage}
\CTANdirectory{dvipaste}{dviware/dvipaste}
\CTANdirectory{dvipdfm}{dviware/dvipdfm}
\CTANdirectory{dvipdfmx}{dviware/dvipdfmx}[dvipdfmx]
\CTANdirectory{dvipj}{dviware/dvipj}
\CTANdirectory{dvipng}{dviware/dvipng}[dvipng]
\CTANdirectory{dvips-pc}{systems/msdos/dviware/dvips}
\CTANdirectory{dvips}{dviware/dvips}[dvips]
\CTANdirectory{dvistd}{dviware/driv-standard}
\CTANdirectory{dvisun}{dviware/dvisun}
\CTANdirectory{dvitty}{dviware/dvitty}
\CTANdirectory{dvivga}{dviware/dvivga}
\CTANdirectory{e4t}{systems/msdos/e4t}
\CTANdirectory*{e-TeX}{systems/e-tex}
\CTANdirectory{easytex}{systems/msdos/easytex}
\CTANdirectory{ebib}{biblio/bibtex/utils/ebib}
\CTANdirectory{ec}{fonts/ec}[ec]
\CTANdirectory{ec-plain}{macros/ec-plain}[ec-plain]
\CTANdirectory{eco}{fonts/eco}[eco]
\CTANdirectory{economic}{biblio/bibtex/contrib/economic}
\CTANdirectory{edmac}{macros/plain/contrib/edmac}[edmac]
\CTANdirectory{ednotes}{macros/latex/contrib/ednotes}[ednotes]
\CTANdirectory{eepic}{macros/latex/contrib/eepic}[eepic]
\CTANdirectory{ega2mf}{fonts/utilities/ega2mf}
\CTANdirectory{egplot}{macros/latex/contrib/egplot}[egplot]
\CTANdirectory{eiad}{fonts/eiad}
\CTANdirectory{elvish}{fonts/elvish}
\CTANdirectory{elwell}{fonts/utilities/afmtopl/elwell}
\CTANdirectory{emp}{macros/latex/contrib/emp}[emp]
\CTANdirectory{emptypage}{macros/latex/contrib/emptypage}[emptypage]
\CTANdirectory{emt2tex}{systems/msdos/emtex-contrib/emt2tex}
\CTANdirectory{emtex}{systems/msdos/emtex}
\CTANdirectory{emtex-contrib}{systems/msdos/emtex-contrib}
\CTANdirectory{emtex-fonts}{systems/msdos/emtex-fonts}
\CTANdirectory{emtextds}{obsolete/systems/os2/emtex-contrib/emtexTDS}
\CTANdirectory{enctex}{systems/enctex}[enctex]
\CTANdirectory{endfloat}{macros/latex/contrib/endfloat}[endfloat]
\CTANdirectory{english}{language/english}
\CTANdirectory{engwar}{fonts/engwar}
\CTANdirectory{enumitem}{macros/latex/contrib/enumitem}[enumitem]
\CTANdirectory{environment}{support/lsedit/environment}
\CTANdirectory{epic}{macros/latex/contrib/epic}[epic]
\CTANdirectory{epigraph}{macros/latex/contrib/epigraph}[epigraph]
\CTANdirectory{eplain}{macros/eplain}[eplain]
\CTANdirectory{epmtex}{systems/os2/epmtex}
\CTANdirectory{epslatex}{info/epslatex}[epslatex]
\CTANdirectory*{epstopdf}{support/epstopdf}[epstopdf]
\CTANdirectory{eqparbox}{macros/latex/contrib/eqparbox}[eqparbox]
\CTANdirectory{ergotex}{systems/msdos/ergotex}
\CTANdirectory{errata}{systems/knuth/dist/errata}
\CTANdirectory{eso-pic}{macros/latex/contrib/eso-pic}[eso-pic]
\CTANdirectory{et}{support/et}
\CTANdirectory{etex}{systems/e-tex}[etex]
\CTANdirectory{etex-pkg}{macros/latex/contrib/etex-pkg}[etex-pkg]
\CTANdirectory{etextools}{macros/latex/contrib/etextools}[etextools]
\CTANdirectory{ethiopia}{language/ethiopia}
\CTANdirectory{ethtex}{language/ethiopia/ethtex}
\CTANdirectory{etoc}{macros/latex/contrib/etoc}[etoc]
\CTANdirectory{etoolbox}{macros/latex/contrib/etoolbox}[etoolbox]
\CTANdirectory{euler-latex}{macros/latex/contrib/euler}[euler]
\CTANdirectory{eulervm}{fonts/eulervm}[eulervm]
\CTANdirectory{euro-ce}{fonts/euro-ce}[euro-ce]
\CTANdirectory{euro-fonts}{fonts/euro}
\CTANdirectory{eurofont}{macros/latex/contrib/eurofont}[eurofont]
\CTANdirectory{europecv}{macros/latex/contrib/europecv}[europecv]
\CTANdirectory{eurosym}{fonts/eurosym}[eurosym]
\CTANdirectory{everypage}{macros/latex/contrib/everypage}[everypage]
\CTANdirectory{excalibur}{systems/mac/support/excalibur}
\CTANdirectory{excel2latex}{support/excel2latex}
\CTANdirectory{excerpt}{web/spiderweb/tools/excerpt}
\CTANdirectory{excludeonly}{macros/latex/contrib/excludeonly}[excludeonly]
\CTANdirectory{expdlist}{macros/latex/contrib/expdlist}
\CTANdirectory{extract}{macros/latex/contrib/extract}[extract]
\CTANdirectory{extsizes}{macros/latex/contrib/extsizes}[extsizes]
% cat links to here
\CTANdirectory{fancyhdr}{macros/latex/contrib/fancyhdr}[fancyhdr]
\CTANdirectory{fancyvrb}{macros/latex/contrib/fancyvrb}[fancyvrb]
\CTANdirectory{faq}{help/uk-tex-faq}[uk-tex-faq]
\CTANdirectory{fc}{fonts/jknappen/fc}
\CTANdirectory{fdsymbol}{fonts/fdsymbol}[fdsymbol]
\CTANdirectory{feyn}{fonts/feyn}
\CTANdirectory{feynman}{macros/latex209/contrib/feynman}
\CTANdirectory{feynmf}{macros/latex/contrib/feynmf}[feynmf]
\CTANdirectory{fig2eng}{graphics/fig2eng}
\CTANdirectory{fig2mf}{graphics/fig2mf}
\CTANdirectory{fig2mfpic}{graphics/fig2mfpic}
\CTANdirectory{figflow}{macros/plain/contrib/figflow}[figflow]
\CTANdirectory{filehook}{macros/latex/contrib/filehook}[filehook]
\CTANdirectory{fink}{macros/latex/contrib/fink}[fink]
\CTANdirectory{first-latex-doc}{info/first-latex-doc}
\CTANdirectory{fix2col}{macros/latex/contrib/fix2col}[fix2col]
\CTANdirectory{fixfoot}{macros/latex/contrib/fixfoot}[fixfoot]
\CTANdirectory{float}{macros/latex/contrib/float}[float]
\CTANdirectory{floatflt}{macros/latex/contrib/floatflt}[floatflt]
\CTANdirectory{flow}{support/flow}
\CTANdirectory{flowfram}{macros/latex/contrib/flowfram}[flowfram]
\CTANdirectory{fltpage}{macros/latex/contrib/fltpage}[fltpage]
\CTANdirectory{fnbreak}{macros/latex/contrib/fnbreak}[fnbreak]
\CTANdirectory{fncychap}{macros/latex/contrib/fncychap}[fncychap]
\CTANdirectory{fncylab}{macros/latex/contrib/fncylab}
\CTANdirectory{foiltex}{macros/latex/contrib/foiltex}[foiltex]
\CTANdirectory{font-change}{macros/plain/contrib/font-change}[font-change]
\CTANdirectory{fontch}{macros/plain/contrib/fontch}[fontch]
\CTANdirectory{fontinst}{fonts/utilities/fontinst}[fontinst]
\CTANdirectory{fontname}{info/fontname}[fontname]
\CTANdirectory{fontspec}{macros/latex/contrib/fontspec}[fontspec]
\CTANdirectory{font_selection}{macros/plain/contrib/font_selection}[font-selection]
\CTANdirectory{footbib}{macros/latex/contrib/footbib}[footbib]
\CTANdirectory{footmisc}{macros/latex/contrib/footmisc}[footmisc]
\CTANdirectory{footnpag}{macros/latex/contrib/footnpag}[footnpag]
\CTANdirectory{forarray}{macros/latex/contrib/forarray}[forarray]
\CTANdirectory{forloop}{macros/latex/contrib/forloop}[forloop]
\CTANdirectory{for_tex}{biblio/bibtex/contrib/germbib/for_tex}
\CTANdirectory{fourier}{fonts/fourier-GUT}[fourier]
\CTANdirectory{fouriernc}{fonts/fouriernc}[fouriernc]
\CTANdirectory{framed}{macros/latex/contrib/framed}[framed]
\CTANdirectory{frankenstein}{macros/latex/contrib/frankenstein}[frankenstein]
\CTANdirectory{french-faq}{help/LaTeX-FAQ-francaise}
\CTANdirectory{funnelweb}{web/funnelweb}
\CTANdirectory{futhark}{fonts/futhark}
\CTANdirectory{futhorc}{fonts/futhorc}
\CTANdirectory{fweb}{web/fweb}[fweb]
\CTANdirectory{garamondx}{fonts/garamondx}[garamondx]
\CTANdirectory{gellmu}{support/gellmu}[gellmu]
\CTANdirectory{genfam}{support/genfam}
\CTANdirectory{geometry}{macros/latex/contrib/geometry}[geometry]
\CTANdirectory{germbib}{biblio/bibtex/contrib/germbib}
\CTANdirectory{getoptk}{macros/plain/contrib/getoptk}[getoptk]
\CTANdirectory{gfs}{info/examples/FirstSteps} % gratzer's
\CTANdirectory{gitinfo}{macros/latex/contrib/gitinfo}[gitinfo]
\CTANdirectory{glo+idxtex}{indexing/glo+idxtex}[idxtex]
\CTANdirectory{gmp}{macros/latex/contrib/gmp}[gmp]
\CTANdirectory{gnuplot}{graphics/gnuplot}
\CTANdirectory{go}{fonts/go}
\CTANdirectory{gothic}{fonts/gothic}
\CTANdirectory{graphbase}{support/graphbase}
\CTANdirectory{graphics}{macros/latex/required/graphics}[graphics]
\CTANdirectory{graphics-plain}{macros/plain/graphics}[graphics-pln]
\CTANdirectory{graphicx-psmin}{macros/latex/contrib/graphicx-psmin}[graphicx-psmin]
\CTANdirectory{gray}{fonts/cm/utilityfonts/gray}
\CTANdirectory{greek}{fonts/greek}
\CTANdirectory{greektex}{fonts/greek/greektex}
\CTANdirectory{gsftopk}{fonts/utilities/gsftopk}[gsftopk]
\CTANdirectory{gut}{usergrps/gut}
\CTANdirectory*{gv}{support/gv}[gv]
\CTANdirectory{ha-prosper}{macros/latex/contrib/ha-prosper}[ha-prosper]
\CTANdirectory{half}{fonts/cm/utilityfonts/half}
\CTANdirectory{halftone}{fonts/halftone}
\CTANdirectory{hands}{fonts/hands}
\CTANdirectory{harvard}{macros/latex/contrib/harvard}
\CTANdirectory{harvmac}{macros/plain/contrib/harvmac}
\CTANdirectory{hebrew}{language/hebrew}
\CTANdirectory{help}{help}
\CTANdirectory{here}{macros/latex/contrib/here}[here]
\CTANdirectory{hershey}{fonts/hershey}
\CTANdirectory{hfbright}{fonts/ps-type1/hfbright}[hfbright]
\CTANdirectory{hge}{fonts/hge}
\CTANdirectory{hieroglyph}{fonts/hieroglyph}
\CTANdirectory{highlight}{support/highlight}
\CTANdirectory{histyle}{macros/plain/contrib/histyle}
\CTANdirectory{hp2pl}{support/hp2pl}
\CTANdirectory{hp2xx}{support/hp2xx}
\CTANdirectory{hpgl2ps}{graphics/hpgl2ps}
\CTANdirectory{hptex}{macros/hptex}
\CTANdirectory{hptomf}{support/hptomf}
\CTANdirectory{html2latex}{support/html2latex}[html2latex]
\CTANdirectory{htmlhelp}{info/htmlhelp}
\CTANdirectory{hvfloat}{macros/latex/contrib/hvfloat}[hvfloat]
\CTANdirectory{hvmath}{fonts/micropress/hvmath}[hvmath-fonts]
\CTANdirectory{hyacc-cm}{macros/generic/hyacc-cm}
\CTANdirectory{hyper}{macros/latex/contrib/hyper}
\CTANdirectory{hyperbibtex}{biblio/bibtex/utils/hyperbibtex}
\CTANdirectory{hypernat}{macros/latex/contrib/hypernat}[hypernat]
\CTANdirectory{hyperref}{macros/latex/contrib/hyperref}[hyperref]
\CTANdirectory{hyphen-accent}{language/hyphen-accent}
\CTANdirectory{hyphenat}{macros/latex/contrib/hyphenat}[hyphenat]
\CTANdirectory{hyphenation}{language/hyphenation}
\CTANdirectory{ibygrk}{fonts/greek/ibygrk}
\CTANdirectory{iching}{fonts/iching}
\CTANdirectory{icons}{support/icons}
\CTANdirectory{ifmtarg}{macros/latex/contrib/ifmtarg}[ifmtarg]
\CTANdirectory{ifmslide}{macros/latex/contrib/ifmslide}[ifmslide]
\CTANdirectory{ifoddpage}{macros/latex/contrib/ifoddpage}[ifoddpage]
\CTANdirectory{ifxetex}{macros/generic/ifxetex}[ifxetex]
\CTANdirectory{imakeidx}{macros/latex/contrib/imakeidx}[imakeidx]
\CTANdirectory{imaketex}{support/imaketex}
\CTANdirectory{impact}{web/systems/mac/impact}
\CTANdirectory{import}{macros/latex/contrib/import}[import]
\CTANdirectory{index}{macros/latex/contrib/index}[index]
\CTANdirectory{indian}{language/indian}
\CTANdirectory{infpic}{macros/generic/infpic}
\CTANdirectory{inlinebib}{biblio/bibtex/contrib/inlinebib}
\CTANdirectory{inrsdoc}{macros/inrstex/inrsdoc}
\CTANdirectory{inrsinputs}{macros/inrstex/inrsinputs}
\CTANdirectory{inrstex}{macros/inrstex}
\CTANdirectory{isi2bibtex}{biblio/bibtex/utils/isi2bibtex}[isi2bibtex]
\CTANdirectory{iso-tex}{support/iso-tex}
\CTANdirectory{isodoc}{macros/latex/contrib/isodoc}[isodoc]
\CTANdirectory{ispell}{support/ispell}[ispell]
\CTANdirectory{ite}{support/ite}[ite]
\CTANdirectory{ivd2dvi}{dviware/ivd2dvi}
\CTANdirectory{jadetex}{macros/jadetex}
\CTANdirectory{jemtex2}{systems/msdos/jemtex2}
\CTANdirectory{jknappen-macros}{macros/latex/contrib/jknappen}
\CTANdirectory{jpeg2ps}{support/jpeg2ps}[jpeg2ps]
\CTANdirectory{jspell}{support/jspell}[jspell]
\CTANdirectory{jurabib}{macros/latex/contrib/jurabib}[jurabib]
\CTANdirectory{kamal}{support/kamal}
\CTANdirectory{kane}{dviware/kane}
\CTANdirectory{karta}{fonts/karta}
\CTANdirectory{kd}{fonts/greek/kd}
\CTANdirectory{kelem}{web/spiderweb/src/kelem}
\CTANdirectory{kelly}{fonts/greek/kelly}
\CTANdirectory{klinz}{fonts/klinz}
\CTANdirectory{knit}{web/knit}
\CTANdirectory{knot}{fonts/knot}
\CTANdirectory{knuth}{systems/knuth}
\CTANdirectory{knuth-dist}{systems/knuth/dist}[knuth-dist]
\CTANdirectory{koma-script}{macros/latex/contrib/koma-script}[koma-script]
\CTANdirectory{korean}{fonts/korean}
\CTANdirectory{kpfonts}{fonts/kpfonts}[kpfonts]
\CTANdirectory{kyocera}{dviware/kyocera}
\CTANdirectory{l2a}{support/l2a}[l2a]
\CTANdirectory{l2sl}{language/swedish/slatex/l2sl}
\CTANdirectory*{l2tabu}{info/l2tabu}
\CTANdirectory{l2x}{support/l2x}
\CTANdirectory{la}{fonts/la}
\CTANdirectory{laan}{macros/generic/laan}
\CTANdirectory{laansort}{macros/generic/laansort}
\CTANdirectory{labelcas}{macros/latex/contrib/labelcas}[labelcas]
\CTANdirectory{labels}{macros/latex/contrib/labels}
\CTANdirectory{labtex}{macros/generic/labtex}
\CTANdirectory{lacheck}{support/lacheck}[lacheck]
\CTANdirectory{lametex}{support/lametex}
\CTANdirectory{lamstex}{macros/lamstex}
\CTANdirectory{lastpage}{macros/latex/contrib/lastpage}[lastpage]
\CTANdirectory{latex}{macros/latex/base}
\CTANdirectory{latex-course}{info/latex-course}[latex-course]
\CTANdirectory{latex-essential}{info/latex-essential}
\CTANdirectory{latex2e-help-texinfo}{info/latex2e-help-texinfo}[latex2e-help-texinfo]
\CTANdirectory{latex4jed}{support/jed}[latex4jed]
\CTANdirectory*{latex-tds}{macros/latex/contrib/latex-tds}[latex-tds]
\CTANdirectory*{latex2html}{support/latex2html}[latex2html]
\CTANdirectory{latex2rtf}{support/latex2rtf}
\CTANdirectory{latexdiff}{support/latexdiff}[latexdiff]
\CTANdirectory{latexdoc}{macros/latex/doc}[latex-doc]
\CTANdirectory{latexhlp}{systems/atari/latexhlp}
\CTANdirectory{latexmake}{support/latexmake}[latexmake]
\CTANdirectory{latex-make}{support/latex-make}[latex-make]
\CTANdirectory{latex_maker}{support/latex_maker}[mk]
\CTANdirectory{latexmk}{support/latexmk}[latexmk]
\CTANdirectory{lecturer}{macros/generic/lecturer}[lecturer]
\CTANdirectory{ledmac}{macros/latex/contrib/ledmac}[ledmac]
\CTANdirectory{lettrine}{macros/latex/contrib/lettrine}[lettrine]
\CTANdirectory{levy}{fonts/greek/levy}
\CTANdirectory{lextex}{macros/plain/contrib/lextex}
\CTANdirectory{lgc}{info/examples/lgc}
\CTANdirectory{lgrind}{support/lgrind}[lgrind]
\CTANdirectory{libertine}{fonts/libertine}[libertine]
\CTANdirectory{libgreek}{macros/latex/contrib/libgreek}[libgreek]
\CTANdirectory{lilyglyphs}{macros/luatex/latex/lilyglyphs}
\CTANdirectory{lineno}{macros/latex/contrib/lineno}[lineno]
\CTANdirectory{lipsum}{macros/latex/contrib/lipsum}[lipsum]
\CTANdirectory{listbib}{macros/latex/contrib/listbib}[listbib]
\CTANdirectory{listings}{macros/latex/contrib/listings}[listings]
\CTANdirectory{lm}{fonts/lm}[lm]
\CTANdirectory{lm-math}{fonts/lm-math}[lm-math]
\CTANdirectory{lollipop}{macros/lollipop}[lollipop]
\CTANdirectory{lookbibtex}{biblio/bibtex/utils/lookbibtex}
\CTANdirectory{lpic}{macros/latex/contrib/lpic}[lpic]
\CTANdirectory{lsedit}{support/lsedit}
\CTANdirectory{lshort}{info/lshort/english}[lshort-english]
\CTANdirectory*{lshort-parent}{info/lshort}[lshort]
\CTANdirectory{ltx3pub}{info/ltx3pub}[ltx3pub]
\CTANdirectory{ltxindex}{macros/latex/contrib/ltxindex}[ltxindex]
\CTANdirectory{luatex}{systems/luatex}[luatex]
\CTANdirectory{lucida}{fonts/psfonts/bh/lucida}[lucida]
\CTANdirectory{lucida-psnfss}{macros/latex/contrib/psnfssx/lucidabr}[psnfssx-luc]
\CTANdirectory{luximono}{fonts/LuxiMono}[luximono]
\CTANdirectory{lwc}{info/examples/lwc}
\CTANdirectory{ly1}{fonts/psfonts/ly1}[ly1]
\CTANdirectory*{mactex}{systems/mac/mactex}[mactex]
\CTANdirectory{macros2e}{info/macros2e}[macros2e]
\CTANdirectory{mactotex}{graphics/mactotex}
\CTANdirectory{mailing}{macros/latex/contrib/mailing}
\CTANdirectory{make_latex}{support/make_latex}[make-latex]
\CTANdirectory{makeafm.dir}{fonts/utilities/t1tools/makeafm.dir}
\CTANdirectory{makecell}{macros/latex/contrib/makecell}[makecell]
\CTANdirectory{makedtx}{support/makedtx}[makedtx]
\CTANdirectory{makeindex}{indexing/makeindex}[makeindex]
\CTANdirectory{makeinfo}{macros/texinfo/contrib/texinfo-hu/texinfo/makeinfo}
\CTANdirectory{makeprog}{support/makeprog}\CTANdirectory{maketexwork}{info/maketexwork}
\CTANdirectory{malayalam}{language/malayalam}
\CTANdirectory{malvern}{fonts/malvern}
\CTANdirectory{mapleweb}{web/maple/mapleweb}
\CTANdirectory{marvosym-fonts}{fonts/marvosym}
\CTANdirectory{mathabx}{fonts/mathabx}[mathabx]
\CTANdirectory{mathabx-type1}{fonts/ps-type1/mathabx}[mathabx-type1]
\CTANdirectory{mathastext}{macros/latex/contrib/mathastext}[mathastext]
\CTANdirectory*{mathdesign}{fonts/mathdesign}[mathdesign]
\CTANdirectory{mathdots}{macros/generic/mathdots}
\CTANdirectory{mathematica}{macros/mathematica}
\CTANdirectory{mathpazo}{fonts/mathpazo}[mathpazo]
\CTANdirectory{mathsci2bibtex}{biblio/bibtex/utils/mathsci2bibtex}
\CTANdirectory{mathspic}{graphics/mathspic}[mathspic]
\CTANdirectory{mcite}{macros/latex/contrib/mcite}[mcite]
\CTANdirectory{mciteplus}{macros/latex/contrib/mciteplus}[mciteplus]
\CTANdirectory{mdframed}{macros/latex/contrib/mdframed}[mdframed]
\CTANdirectory{mdsymbol}{fonts/mdsymbol}[mdsymbol]
\CTANdirectory{mdwtools}{macros/latex/contrib/mdwtools}[mdwtools]
\CTANdirectory{memdesign}{info/memdesign}[memdesign]
\CTANdirectory{memoir}{macros/latex/contrib/memoir}[memoir]
\CTANdirectory{messtex}{support/messtex}
\CTANdirectory{metalogo}{macros/latex/contrib/metalogo}[metalogo]
\CTANdirectory{metapost}{graphics/metapost}
\CTANdirectory{metatype1}{fonts/utilities/metatype1}[metatype1]
\CTANdirectory{mf2ps}{fonts/utilities/mf2ps}
\CTANdirectory{mf2pt1}{support/mf2pt1}[mf2pt1]
\CTANdirectory{mf_optimized_kerning}{fonts/cm/mf_optimized_kerning}
\CTANdirectory{mfbook}{fonts/cm/utilityfonts/mfbook}
\CTANdirectory{mff-29}{fonts/utilities/mff-29}
\CTANdirectory{mffiles}{language/telugu/mffiles}
\CTANdirectory{mflogo}{macros/latex/contrib/mflogo}[mflogo]
\CTANdirectory{mfnfss}{macros/latex/contrib/mfnfss}
\CTANdirectory{mfpic}{graphics/mfpic}
\CTANdirectory{mfware}{systems/knuth/dist/mfware}
\CTANdirectory{mh}{macros/latex/contrib/mh}[mh]
\CTANdirectory{miktex}{systems/win32/miktex}[miktex]
\CTANdirectory{microtype}{macros/latex/contrib/microtype}[microtype]
\CTANdirectory{midi2tex}{support/midi2tex}
\CTANdirectory{midnight}{macros/generic/midnight}
\CTANdirectory{minionpro}{fonts/minionpro}[minionpro]
\CTANdirectory{minitoc}{macros/latex/contrib/minitoc}[minitoc]
\CTANdirectory{minted}{macros/latex/contrib/minted}[minted]
\CTANdirectory{mkjobtexmf}{support/mkjobtexmf}[mkjobtexmf]
\CTANdirectory{mma2ltx}{graphics/mma2ltx}
\CTANdirectory{mmap}{macros/latex/contrib/mmap}
\CTANdirectory{mnsymbol}{fonts/mnsymbol}[mnsymbol]
\CTANdirectory{mnu}{support/mnu}
\CTANdirectory{models}{macros/text1/models}
\CTANdirectory{moderncv}{macros/latex/contrib/moderncv}[moderncv]
\CTANdirectory{modes}{fonts/modes}
\CTANdirectory{morefloats}{macros/latex/contrib/morefloats}[morefloats]
\CTANdirectory{moreverb}{macros/latex/contrib/moreverb}[moreverb]
\CTANdirectory{morewrites}{macros/latex/contrib/morewrites}[morewrites]
\CTANdirectory{mparhack}{macros/latex/contrib/mparhack}[mparhack]
\CTANdirectory{mpgraphics}{macros/latex/contrib/mpgraphics}[mpgraphics]
\CTANdirectory{mps2eps}{support/mps2eps}
\CTANdirectory{ms}{macros/latex/contrib/ms}[ms]
\CTANdirectory{msdos}{systems/msdos}
\CTANdirectory{msx2msa}{fonts/vf-files/msx2msa}
\CTANdirectory{msym}{fonts/msym}
\CTANdirectory{mtp2lite}{fonts/mtp2lite}[mtp2lite]
\CTANdirectory{m-tx}{support/m-tx}[m-tx]
\CTANdirectory{multenum}{macros/latex/contrib/multenum}[multenum]
\CTANdirectory{multibbl}{macros/latex/contrib/multibbl}[multibbl]
\CTANdirectory{multibib}{macros/latex/contrib/multibib}[multibib]
\CTANdirectory{multido}{macros/generic/multido}[multido]
\CTANdirectory{multirow}{macros/latex/contrib/multirow}[multirow]
\CTANdirectory{musictex}{macros/musictex}[musictex]
\CTANdirectory{musixtex-egler}{obsolete/macros/musixtex/egler}
\CTANdirectory{musixtex-fonts}{fonts/musixtex-fonts}[musixtex-fonts]
\CTANdirectory{musixtex}{macros/musixtex}[musixtex]
\CTANdirectory{mutex}{macros/mtex}
\CTANdirectory{mwe}{macros/latex/contrib/mwe}[mwe]
\CTANdirectory{mxedruli}{fonts/georgian/mxedruli}
\CTANdirectory{nag}{macros/latex/contrib/nag}[nag]
\CTANdirectory{natbib}{macros/latex/contrib/natbib}[natbib]
\CTANdirectory{navigator}{macros/generic/navigator}[navigator]
\CTANdirectory{nawk}{web/spiderweb/src/nawk}
\CTANdirectory{ncctools}{macros/latex/contrib/ncctools}[ncctools]
\CTANdirectory{needspace}{macros/latex/contrib/needspace}[needspace]
\CTANdirectory{newalg}{macros/latex/contrib/newalg}[newalg]
\CTANdirectory{newcommand}{support/newcommand}[newcommand]
\CTANdirectory{newlfm}{macros/latex/contrib/newlfm}[newlfm]
\CTANdirectory{newsletr}{macros/plain/contrib/newsletr}
\CTANdirectory{newpx}{fonts/newpx}[newpx]
\CTANdirectory{newtx}{fonts/newtx}[newtx]
\CTANdirectory{newverbs}{macros/latex/contrib/newverbs}[newverbs]
\CTANdirectory{nedit-latex}{support/NEdit-LaTeX-Extensions}
\CTANdirectory{nonumonpart}{macros/latex/contrib/nonumonpart}[nonumonpart]
\CTANdirectory{nopageno}{macros/latex/contrib/nopageno}[nopageno]
\CTANdirectory{norbib}{biblio/bibtex/contrib/norbib}
\CTANdirectory{notoccite}{macros/latex/contrib/notoccite}[notoccite]
\CTANdirectory{noweb}{web/noweb}[noweb]
\CTANdirectory{ntg}{usergrps/ntg}
\CTANdirectory{ntgclass}{macros/latex/contrib/ntgclass}[ntgclass]
\CTANdirectory{ntheorem}{macros/latex/contrib/ntheorem}[ntheorem]
\CTANdirectory{nts-l}{digests/nts-l}
\CTANdirectory{nts}{systems/nts}
\CTANdirectory{numprint}{macros/latex/contrib/numprint}[numprint]
\CTANdirectory{nuweb}{web/nuweb}
\CTANdirectory{nuweb0.87b}{web/nuweb/nuweb0.87b}
\CTANdirectory{nuweb_ami}{web/nuweb/nuweb_ami}
\CTANdirectory{oberdiek}{macros/latex/contrib/oberdiek}[oberdiek]
\CTANdirectory{objectz}{macros/latex/contrib/objectz}
\CTANdirectory{ocr-a}{fonts/ocr-a}
\CTANdirectory{ocr-b}{fonts/ocr-b}
\CTANdirectory{ofs}{macros/generic/ofs}[ofs]
\CTANdirectory{ogham}{fonts/ogham}
\CTANdirectory{ogonek}{macros/latex/contrib/ogonek}
\CTANdirectory{okuda}{fonts/okuda}
\CTANdirectory{omega}{systems/omega}
\CTANdirectory{optional}{macros/latex/contrib/optional}[optional]
\CTANdirectory{os2}{systems/os2}
\CTANdirectory{osmanian}{fonts/osmanian}
\CTANdirectory{overpic}{macros/latex/contrib/overpic}[overpic]
\CTANdirectory{oztex}{systems/mac/oztex}
\CTANdirectory{page}{support/lametex/page}
\CTANdirectory{palladam}{language/tamil/palladam}
\CTANdirectory{pandora}{fonts/pandora}
\CTANdirectory{paralist}{macros/latex/contrib/paralist}[paralist]
\CTANdirectory{parallel}{macros/latex/contrib/parallel}[parallel]
\CTANdirectory{parskip}{macros/latex/contrib/parskip}[parskip]
\CTANdirectory{passivetex}{macros/xmltex/contrib/passivetex}[passivetex]
\CTANdirectory{patchcmd}{macros/latex/contrib/patchcmd}[patchcmd]
\CTANdirectory{path}{macros/generic/path}[path]
\CTANdirectory{pbox}{macros/latex/contrib/pbox}[pbox]
\CTANdirectory{pcwritex}{support/pcwritex}
\CTANdirectory{pdcmac}{macros/plain/contrib/pdcmac}[pdcmac]
\CTANdirectory{pdfcomment}{macros/latex/contrib/pdfcomment}[pdfcomment]
\CTANdirectory{pdfpages}{macros/latex/contrib/pdfpages}[pdfpages]
\CTANdirectory{pdfrack}{support/pdfrack}[pdfrack]
\CTANdirectory{pdfscreen}{macros/latex/contrib/pdfscreen}[pdfscreen]
\CTANdirectory{pdftex}{systems/pdftex}[pdftex]
\CTANdirectory{pdftex-graphics}{graphics/metapost/contrib/tools/mptopdf}[pdf-mps-supp]
\CTANdirectory{pdftricks}{graphics/pdftricks}[pdftricks]
\CTANdirectory{pdftricks2}{graphics/pdftricks2}[pdftricks2]
\CTANdirectory{pgf}{graphics/pgf/base}[pgf]
\CTANdirectory{phonetic}{fonts/phonetic}
\CTANdirectory{phy-bstyles}{biblio/bibtex/contrib/phy-bstyles}
\CTANdirectory{physe}{macros/physe}
\CTANdirectory{phyzzx}{macros/phyzzx}
\CTANdirectory{picinpar}{macros/latex209/contrib/picinpar}[picinpar]
\CTANdirectory{picins}{macros/latex209/contrib/picins}[picins]
\CTANdirectory{pict2e}{macros/latex/contrib/pict2e}[pict2e]
\CTANdirectory{pictex}{graphics/pictex}[pictex]
\CTANdirectory{pictex-addon}{graphics/pictex/addon}[pictexwd]
\CTANdirectory{pictex-converter}{support/pictex-converter}
\CTANdirectory{pictex-summary}{info/pictex/summary}[pictexsum]
\CTANdirectory{pinlabel}{macros/latex/contrib/pinlabel}[pinlabel]
\CTANdirectory{pkbbox}{fonts/utilities/pkbbox}
\CTANdirectory{pkfix}{support/pkfix}[pkfix]
\CTANdirectory{pkfix-helper}{support/pkfix-helper}[pkfix-helper]
\CTANdirectory{placeins}{macros/latex/contrib/placeins}[placeins]
\CTANdirectory{plain}{macros/plain/base}[plain]
\CTANdirectory*{plastex}{support/plastex}[plastex]
\CTANdirectory{plnfss}{macros/plain/plnfss}[plnfss]
\CTANdirectory{plttopic}{support/plttopic}
\CTANdirectory{pmtex}{systems/os2/pmtex}
\CTANdirectory{pmx}{support/pmx}
\CTANdirectory{l3experimental}{macros/latex/contrib/l3experimental}[l3experimental]
\CTANdirectory{l3kernel}{macros/latex/contrib/l3kernel}[l3kernel]
\CTANdirectory{l3packages}{macros/latex/contrib/l3packages}[l3packages]
\CTANdirectory{polish}{language/polish}
\CTANdirectory{polyglossia}{macros/latex/contrib/polyglossia}[polyglossia]
\CTANdirectory{poorman}{fonts/poorman}
\CTANdirectory{portuguese}{language/portuguese}
\CTANdirectory{poster}{macros/generic/poster}
\CTANdirectory{powerdot}{macros/latex/contrib/powerdot}[powerdot]
\CTANdirectory{pp}{support/pp}
\CTANdirectory{preprint}{macros/latex/contrib/preprint}[preprint]
\CTANdirectory{present}{macros/plain/contrib/present}[present]
\CTANdirectory{preview}{macros/latex/contrib/preview}[preview]
\CTANdirectory{print-fine}{support/print-fine}
\CTANdirectory{printbib}{biblio/bibtex/utils/printbib}
\CTANdirectory{printlen}{macros/latex/contrib/printlen}[printlen]
\CTANdirectory{printsamples}{fonts/utilities/mf2ps/doc/printsamples}
\CTANdirectory{program}{macros/latex/contrib/program}[program]
\CTANdirectory{proofs}{macros/generic/proofs}
\CTANdirectory*{protext}{systems/win32/protext}[protext]
\CTANdirectory{ppower4}{support/ppower4}[ppower4]
\CTANdirectory{ppr-prv}{macros/latex/contrib/ppr-prv}[ppr-prv]
\CTANdirectory{prosper}{macros/latex/contrib/prosper}[prosper]
\CTANdirectory{ps-type3}{fonts/cm/ps-type3}
\CTANdirectory{ps2mf}{fonts/utilities/ps2mf}
\CTANdirectory{ps2pk}{fonts/utilities/ps2pk}[ps2pk]
\CTANdirectory{psbook}{systems/msdos/dviware/psbook}
\CTANdirectory{psbox}{macros/generic/psbox}
\CTANdirectory{pseudocode}{macros/latex/contrib/pseudocode}[pseudocode]
\CTANdirectory{psfig}{graphics/psfig}[psfig]
\CTANdirectory{psfrag}{macros/latex/contrib/psfrag}[psfrag]
\CTANdirectory{psfragx}{macros/latex/contrib/psfragx}[psfragx]
\CTANdirectory{psizzl}{macros/psizzl}
\CTANdirectory{psnfss}{macros/latex/required/psnfss}[psnfss]
\CTANdirectory{psnfss-addons}{macros/latex/contrib/psnfss-addons}
\CTANdirectory{psnfssx-mathtime}{macros/latex/contrib/psnfssx/mathtime}
\CTANdirectory{pspicture}{macros/latex/contrib/pspicture}[pspicture]
\CTANdirectory{psprint}{dviware/psprint}
\CTANdirectory{pst-layout}{graphics/pstricks/contrib/pst-layout}[pst-layout]
\CTANdirectory{pst-pdf}{macros/latex/contrib/pst-pdf}[pst-pdf]
\CTANdirectory{pstoedit}{support/pstoedit}[pstoedit]
\CTANdirectory{pstricks}{graphics/pstricks}[pstricks]
\CTANdirectory{psutils}{support/psutils}
\CTANdirectory{punk}{fonts/punk}
\CTANdirectory{purifyeps}{support/purifyeps}[purifyeps]
\CTANdirectory{pxfonts}{fonts/pxfonts}[pxfonts]
\CTANdirectory{qdtexvpl}{fonts/utilities/qdtexvpl}[qdtexvpl]
\CTANdirectory{qfig}{support/qfig}
\CTANdirectory{quotchap}{macros/latex/contrib/quotchap}[quotchap]
\CTANdirectory{r2bib}{biblio/bibtex/utils/r2bib}[r2bib]
\CTANdirectory{rcs}{macros/latex/contrib/rcs}[rcs]
\CTANdirectory{rcsinfo}{macros/latex/contrib/rcsinfo}[rcsinfo]
\CTANdirectory{realcalc}{macros/generic/realcalc}
\CTANdirectory{refcheck}{macros/latex/contrib/refcheck}[refcheck]
\CTANdirectory{refer-tools}{biblio/bibtex/utils/refer-tools}
\CTANdirectory{refman}{macros/latex/contrib/refman}[refman]
\CTANdirectory{regexpatch}{macros/latex/contrib/regexpatch}[regexpatch]
\CTANdirectory{revtex4-1}{macros/latex/contrib/revtex}[revtex4-1]
\CTANdirectory{rnototex}{support/rnototex}[rnototex]
\CTANdirectory{rotating}{macros/latex/contrib/rotating}[rotating]
\CTANdirectory{rotfloat}{macros/latex/contrib/rotfloat}[rotfloat]
\CTANdirectory{rsfs}{fonts/rsfs}[rsfs]
\CTANdirectory{rsfso}{fonts/rsfso}[rsfso]
\CTANdirectory{rtf2tex}{support/rtf2tex}[rtf2tex]
\CTANdirectory{rtf2html}{support/rtf2html}
\CTANdirectory{rtf2latex}{support/rtf2latex}
\CTANdirectory{rtf2latex2e}{support/rtf2latex2e}[rtf2latex2e]
\CTANdirectory{rtflatex}{support/rtflatex}
\CTANdirectory{rtfutils}{support/tex2rtf/rtfutils}
\CTANdirectory{rumgraph}{support/rumgraph}
\CTANdirectory{sam2p}{graphics/sam2p}[sam2p]
\CTANdirectory{sansmath}{macros/latex/contrib/sansmath}[sansmath]
\CTANdirectory{sauerj}{macros/latex/contrib/sauerj}[sauerj]
\CTANdirectory{savetrees}{macros/latex/contrib/savetrees}[savetrees]
\CTANdirectory{schemeweb}{web/schemeweb}[schemeweb]
\CTANdirectory{sciposter}{macros/latex/contrib/sciposter}[sciposter]
\CTANdirectory{sectsty}{macros/latex/contrib/sectsty}[sectsty]
\CTANdirectory{selectp}{macros/latex/contrib/selectp}[selectp]
\CTANdirectory{seminar}{macros/latex/contrib/seminar}[seminar]
\CTANdirectory{shade}{macros/generic/shade}[shade]
\CTANdirectory{shorttoc}{macros/latex/contrib/shorttoc}[shorttoc]
\CTANdirectory{showexpl}{macros/latex/contrib/showexpl}[showexpl]
\CTANdirectory{showlabels}{macros/latex/contrib/showlabels}[showlabels]
\CTANdirectory{slashbox}{macros/latex/contrib/slashbox}[slashbox]
\CTANdirectory{smallcap}{macros/latex/contrib/smallcap}[smallcap]
\CTANdirectory{smartref}{macros/latex/contrib/smartref}[smartref]
\CTANdirectory{snapshot}{macros/latex/contrib/snapshot}[snapshot]
\CTANdirectory{soul}{macros/latex/contrib/soul}[soul]
\CTANdirectory{spain}{biblio/bibtex/contrib/spain}
\CTANdirectory{spelling}{macros/luatex/generic/spelling}[spelling]
\CTANdirectory{spiderweb}{web/spiderweb}[spiderweb]
\CTANdirectory{splitbib}{macros/latex/contrib/splitbib}[splitbib]
\CTANdirectory{splitindex}{macros/latex/contrib/splitindex}[splitindex]
\CTANdirectory{standalone}{macros/latex/contrib/standalone}[standalone]
\CTANdirectory{stix}{fonts/stix}[stix]
\CTANdirectory{sttools}{macros/latex/contrib/sttools}[sttools]
\CTANdirectory{sty2dtx}{support/sty2dtx}[sty2dtx]
\CTANdirectory{subdepth}{macros/latex/contrib/subdepth}[subdepth]
\CTANdirectory{subfig}{macros/latex/contrib/subfig}[subfig]
\CTANdirectory{subfiles}{macros/latex/contrib/subfiles}[subfiles]
\CTANdirectory{supertabular}{macros/latex/contrib/supertabular}[supertabular]
\CTANdirectory{svn}{macros/latex/contrib/svn}[svn]
\CTANdirectory{svninfo}{macros/latex/contrib/svninfo}[svninfo]
\CTANdirectory{swebib}{biblio/bibtex/contrib/swebib}
\CTANdirectory*{symbols}{info/symbols/comprehensive}[comprehensive]
\CTANdirectory{tablefootnote}{macros/latex/contrib/tablefootnote}[tablefootnote]
\CTANdirectory{tabls}{macros/latex/contrib/tabls}[tabls]
\CTANdirectory{tabulary}{macros/latex/contrib/tabulary}[tabulary]
\CTANdirectory{tagging}{macros/latex/contrib/tagging}
\CTANdirectory{talk}{macros/latex/contrib/talk}[talk]
\CTANdirectory{tcolorbox}{macros/latex/contrib/tcolorbox}[tcolorbox]
\CTANdirectory{tds}{tds}[tds]
\CTANdirectory{ted}{macros/latex/contrib/ted}[ted]
\CTANdirectory*{testflow}{macros/latex/contrib/IEEEtran/testflow}[testflow]
\CTANdirectory*{tetex}{obsolete/systems/unix/teTeX/current/distrib}[tetex]
\CTANdirectory{tex2mail}{support/tex2mail}[tex2mail]
\CTANdirectory{tex2rtf}{support/tex2rtf}[tex2rtf]
\CTANdirectory{texbytopic}{info/texbytopic}[texbytopic]
\CTANdirectory{texcnvfaq}{help/wp-conv}[wp-conv]
\CTANdirectory{texcount}{support/texcount}[texcount]
\CTANdirectory{texdef}{support/texdef}[texdef]
\CTANdirectory{tex-gpc}{systems/unix/tex-gpc}[tex-gpc]
\CTANdirectory{tex-gyre}{fonts/tex-gyre}[tex-gyre]
\CTANdirectory{tex-gyre-math}{fonts/tex-gyre-math}[tex-gyre-math]
\CTANdirectory{tex-overview}{info/tex-overview}[tex-overview]
\CTANdirectory*{texhax}{digests/texhax}[texhax]
\CTANdirectory{texi2html}{support/texi2html}[texi2html]
\CTANdirectory{texindex}{indexing/texindex}[texindex]
\CTANdirectory{texinfo}{macros/texinfo/texinfo}[texinfo]
\CTANdirectory*{texlive}{systems/texlive}[texlive]
\CTANdirectory*{texmacs}{support/TeXmacs}[texmacs]
\CTANdirectory*{texniccenter}{systems/win32/TeXnicCenter}[texniccenter]
\CTANdirectory{texpower}{macros/latex/contrib/texpower}[texpower]
\CTANdirectory{texshell}{systems/msdos/texshell}[texshell]
\CTANdirectory{texsis}{macros/texsis}[texsis]
\CTANdirectory{textcase}{macros/latex/contrib/textcase}[textcase]
\CTANdirectory{textfit}{macros/latex/contrib/textfit}[textfit]
\CTANdirectory{textmerg}{macros/latex/contrib/textmerg}[textmerg]
\CTANdirectory{textpos}{macros/latex/contrib/textpos}[textpos]
\CTANdirectory{textures_figs}{systems/mac/textures_figs}
\CTANdirectory{texutils}{systems/atari/texutils}
\CTANdirectory{tgrind}{support/tgrind}[tgrind]
\CTANdirectory{threeparttable}{macros/latex/contrib/threeparttable}[threeparttable]
\CTANdirectory{threeparttablex}{macros/latex/contrib/threeparttablex}[threeparttablex]
\CTANdirectory{tib}{biblio/tib}[tib]
\CTANdirectory{tiny_c2l}{support/tiny_c2l}[tinyc2l]
\CTANdirectory{tip}{info/examples/tip}
\CTANdirectory{titleref}{macros/latex/contrib/titleref}[titleref]
\CTANdirectory{titlesec}{macros/latex/contrib/titlesec}[titlesec]
\CTANdirectory{titling}{macros/latex/contrib/titling}[titling]
\CTANdirectory{tlc2}{info/examples/tlc2}
\CTANdirectory{tmmath}{fonts/micropress/tmmath}[tmmath]
\CTANdirectory{tocbibind}{macros/latex/contrib/tocbibind}[tocbibind]
\CTANdirectory{tocloft}{macros/latex/contrib/tocloft}[tocloft]
\CTANdirectory{tocvsec2}{macros/latex/contrib/tocvsec2}[tocvsec2]
\CTANdirectory{totpages}{macros/latex/contrib/totpages}[totpages]
\CTANdirectory{tr2latex}{support/tr2latex}[tr2latex]
\CTANdirectory{transfig}{graphics/transfig}[transfig]
\CTANdirectory{try}{support/try}[try]
\CTANdirectory{tt2001}{fonts/ps-type1/tt2001}[tt2001]
\CTANdirectory{tth}{support/tth/dist}[tth]
\CTANdirectory{ttn}{digests/ttn}
\CTANdirectory{tug}{usergrps/tug}
\CTANdirectory{tugboat}{digests/tugboat}
\CTANdirectory{tweb}{web/tweb}[tweb]
\CTANdirectory{txfonts}{fonts/txfonts}[txfonts]
\CTANdirectory{txfontsb}{fonts/txfontsb}[txfontsb]
\CTANdirectory{txtdist}{support/txt}[txt]
\CTANdirectory{type1cm}{macros/latex/contrib/type1cm}[type1cm]
\CTANdirectory{ucharclasses}{macros/xetex/latex/ucharclasses}[ucharclasses]
\CTANdirectory{ucs}{macros/latex/contrib/ucs}[ucs]
\CTANdirectory{ucthesis}{macros/latex/contrib/ucthesis}[ucthesis]
\CTANdirectory{uktex}{digests/uktex}
\CTANdirectory{ulem}{macros/latex/contrib/ulem}[ulem]
\CTANdirectory{umrand}{macros/generic/umrand}
\CTANdirectory{underscore}{macros/latex/contrib/underscore}[underscore]
\CTANdirectory{unicode-math}{macros/latex/contrib/unicode-math}[unicode-math]
\CTANdirectory{unix}{systems/unix}
\CTANdirectory{unpacked}{macros/latex/unpacked}
\CTANdirectory{untex}{support/untex}[untex]
\CTANdirectory{url}{macros/latex/contrib/url}[url]
\CTANdirectory{urlbst}{biblio/bibtex/contrib/urlbst}[urlbst]
\CTANdirectory{urw-base35}{fonts/urw/base35}[urw-base35]
\CTANdirectory{urwchancal}{fonts/urwchancal}[urwchancal]
\CTANdirectory{usebib}{macros/latex/contrib/usebib}[usebib]
\CTANdirectory{utopia}{fonts/utopia}[utopia]
\CTANdirectory{varisize}{macros/plain/contrib/varisize}[varisize]
\CTANdirectory{varwidth}{macros/latex/contrib/varwidth}[varwidth]
\CTANdirectory{vc}{support/vc}[vc]
\CTANdirectory{verbatim}{macros/latex/required/tools}[verbatim]
\CTANdirectory{verbatimbox}{macros/latex/contrib/verbatimbox}[verbatimbox]
\CTANdirectory{verbdef}{macros/latex/contrib/verbdef}[verbdef]
\CTANdirectory{version}{macros/latex/contrib/version}[version]
\CTANdirectory{vertbars}{macros/latex/contrib/vertbars}[vertbars]
\CTANdirectory{vita}{macros/latex/contrib/vita}[vita]
\CTANdirectory{vmargin}{macros/latex/contrib/vmargin}[vmargin]
\CTANdirectory{vmspell}{support/vmspell}[vmspell]
\CTANdirectory{vpp}{support/view_print_ps_pdf}[vpp]
\CTANdirectory{vruler}{macros/latex/contrib/vruler}[vruler]
\CTANdirectory{vtex-common}{systems/vtex/common}
\CTANdirectory{vtex-linux}{systems/vtex/linux}[vtex-free]
\CTANdirectory{vtex-os2}{systems/vtex/os2}[vtex-free]
\CTANdirectory{wallpaper}{macros/latex/contrib/wallpaper}[wallpaper]
\CTANdirectory{was}{macros/latex/contrib/was}[was]
\CTANdirectory{wd2latex}{support/wd2latex}
\CTANdirectory{web}{systems/knuth/dist/web}[web]
\CTANdirectory*{winedt}{systems/win32/winedt}[winedt]
\CTANdirectory{wordcount}{macros/latex/contrib/wordcount}[wordcount]
\CTANdirectory{wp2latex}{support/wp2latex}[wp2latex]
\CTANdirectory{wrapfig}{macros/latex/contrib/wrapfig}[wrapfig]
\CTANdirectory{xargs}{macros/latex/contrib/xargs}[xargs]
\CTANdirectory{xbibfile}{biblio/bibtex/utils/xbibfile}[xbibfile]
\CTANdirectory{xcolor}{macros/latex/contrib/xcolor}[xcolor]
\CTANdirectory{xcomment}{macros/generic/xcomment}[xcomment]
\CTANdirectory*{xdvi}{dviware/xdvi}[xdvi]
\CTANdirectory{xecjk}{macros/xetex/latex/xecjk}[xecjk]
\CTANdirectory{xetexref}{info/xetexref}[xetexref]
\CTANdirectory*{xfig}{graphics/xfig}[xfig]
\CTANdirectory*{xindy}{indexing/xindy}[xindy]
\CTANdirectory{xits}{fonts/xits}[xits]
\CTANdirectory{xkeyval}{macros/latex/contrib/xkeyval}[xkeyval]
\CTANdirectory{xmltex}{macros/xmltex/base}[xmltex]
\CTANdirectory*{xpdf}{support/xpdf}[xpdf]
\CTANdirectory{xtab}{macros/latex/contrib/xtab}[xtab]
\CTANdirectory{xwatermark}{macros/latex/contrib/xwatermark}[xwatermark]
\CTANdirectory{yagusylo}{macros/latex/contrib/yagusylo}[yagusylo]
\CTANdirectory{yhmath}{fonts/yhmath}[yhmath]
\CTANdirectory{zefonts}{fonts/zefonts}[zefonts]
\CTANdirectory{ziffer}{macros/latex/contrib/ziffer}[ziffer]
\CTANdirectory{zoon-mp-eg}{info/metapost/examples}[metapost-examples]
\CTANdirectory{zwpagelayout}{macros/latex/contrib/zwpagelayout}[zwpagelayout]
\endinput

%
% ... files
% $Id: filectan.tex,v 1.143 2012/12/07 19:34:33 rf10 Exp rf10 $
%
% protect ourself against being read twice
\csname readCTANfiles\endcsname
\let\readCTANfiles\endinput
%
% interesting/useful individual files to be found on CTAN
\CTANfile{CTAN-sites}{CTAN.sites}
\CTANfile{CTAN-uploads}{README.uploads}% yes, it really is in the root
\CTANfile{Excalibur}{systems/mac/support/excalibur/Excalibur-4.0.2.sit.hqx}[excalibur]
\CTANfile{expl3-doc}{macros/latex/contrib/l3kernel/expl3.pdf}[l3kernel]
\CTANfile{f-byname}{FILES.byname}
\CTANfile{f-last7}{FILES.last07days}
\CTANfile{interface3-doc}{macros/latex/contrib/l3kernel/interface3.pdf}[l3kernel]
\CTANfile{LitProg-FAQ}{help/comp.programming.literate_FAQ}
\CTANfile{OpenVMSTeX}{systems/OpenVMS/TEX97_CTAN.ZIP}
\CTANfile{T1instguide}{info/Type1fonts/fontinstallationguide/fontinstallationguide.pdf}
\CTANfile{TeX-FAQ}{obsolete/help/TeX,_LaTeX,_etc.:_Frequently_Asked_Questions_with_Answers}
\CTANfile{abstract-bst}{biblio/bibtex/utils/bibtools/abstract.bst}
\CTANfile{backgrnd}{macros/generic/misc/backgrnd.tex}[backgrnd]
\CTANfile{bbl2html}{biblio/bibtex/utils/misc/bbl2html.awk}[bbl2html]
\CTANfile{beginlatex-pdf}{info/beginlatex/beginlatex-3.6.pdf}[beginlatex]
\CTANfile{bibtex-faq}{biblio/bibtex/contrib/doc/btxFAQ.pdf}
\CTANfile{bidstobibtex}{biblio/bibtex/utils/bids/bids.to.bibtex}[bidstobibtex]
\CTANfile{btxmactex}{macros/eplain/tex/btxmac.tex}[eplain]
\CTANfile{catalogue}{help/Catalogue/catalogue.html}
\CTANfile{cat-licences}{help/Catalogue/licenses.html}
\CTANfile{clsguide}{macros/latex/doc/clsguide.pdf}[clsguide]
\CTANfile{compactbib}{macros/latex/contrib/compactbib/compactbib.sty}[compactbib]
%\CTANfile{compan-ctan}{info/companion.ctan}
\CTANfile{context-tmf}{macros/context/current/cont-tmf.zip}[context]
\CTANfile{dvitype}{systems/knuth/dist/texware/dvitype.web}[dvitype]
\CTANfile{edmetrics}{systems/mac/textures/utilities/EdMetrics.sea.hqx}[edmetrics]
\CTANfile{epsf}{macros/generic/epsf/epsf.tex}[epsf]
\CTANfile{figsinlatex}{obsolete/info/figsinltx.ps}
\CTANfile{finplain}{biblio/bibtex/contrib/misc/finplain.bst}
\CTANfile{fix-cm}{macros/latex/unpacked/fix-cm.sty}[fix-cm]
\CTANfile{fntguide.pdf}{macros/latex/doc/fntguide.pdf}[fntguide]
\CTANfile{fontdef}{macros/latex/base/fontdef.dtx}
\CTANfile{fontmath}{macros/latex/unpacked/fontmath.ltx}
\CTANfile{gentle}{info/gentle/gentle.pdf}[gentle]
\CTANfile{gkpmac}{systems/knuth/local/lib/gkpmac.tex}[gkpmac]
\CTANfile{knuth-letter}{systems/knuth/local/lib/letter.tex}
\CTANfile{knuth-tds}{macros/latex/contrib/latex-tds/knuth.tds.zip}
\CTANfile{latex209-base}{obsolete/macros/latex209/distribs/latex209.tar.gz}[latex209]
\CTANfile{latex-classes}{macros/latex/base/classes.dtx}
\CTANfile{latex-source}{macros/latex/base/source2e.tex}
\CTANfile{latexcount}{support/latexcount/latexcount.pl}[latexcount]
\CTANfile{latexcheat}{info/latexcheat/latexcheat/latexsheet.pdf}[latexcheat]
\CTANfile{letterspacing}{macros/generic/misc/letterspacing.tex}[letterspacing]
\CTANfile{ltablex}{macros/latex/contrib/ltablex/ltablex.sty}[ltablex]
\CTANfile{ltxguide}{macros/latex/base/ltxguide.cls}
\CTANfile{ltxtable}{macros/latex/contrib/carlisle/ltxtable.tex}[ltxtable]
\CTANfile{lw35nfss-zip}{macros/latex/required/psnfss/lw35nfss.zip}[lw35nfss]
\CTANfile{macmakeindex}{systems/mac/macmakeindex2.12.sea.hqx}
\CTANfile{mathscript}{info/symbols/math/scriptfonts.pdf}
\CTANfile{mathsurvey.html}{info/Free_Math_Font_Survey/en/survey.html}
\CTANfile{mathsurvey.pdf}{info/Free_Math_Font_Survey/en/survey.pdf}
\CTANfile{memoir-man}{macros/latex/contrib/memoir/memman.pdf}
\CTANfile{metafp-pdf}{info/metafont/metafp/metafp.pdf}[metafp]
\CTANfile{mf-beginners}{info/metafont/beginners/metafont-for-beginners.pdf}[metafont-beginners]
\CTANfile{mf-list}{info/metafont-list}
\CTANfile{miktex-portable}{systems/win32/miktex/setup/miktex-portable.exe}
\CTANfile{miktex-setup}{systems/win32/miktex/setup/setup.exe}[miktex]
\CTANfile{mil}{info/mil/mil.pdf}
\CTANfile{mil-short}{info/Math_into_LaTeX-4/Short_Course.pdf}[math-into-latex-4]
\CTANfile{modes-file}{fonts/modes/modes.mf}[modes]
\CTANfile{mtw}{info/makingtexwork/mtw-1.0.1-html.tar.gz}
\CTANfile{multind}{macros/latex209/contrib/misc/multind.sty}[multind]
\CTANfile{nextpage}{macros/latex/contrib/misc/nextpage.sty}[nextpage]
\CTANfile{noTeX}{biblio/bibtex/utils/misc/noTeX.bst}[notex]
\CTANfile{numline}{obsolete/macros/latex/contrib/numline/numline.sty}[numline]
\CTANfile{patch}{macros/generic/misc/patch.doc}[patch]
\CTANfile{picins-summary}{macros/latex209/contrib/picins/picins.txt}
\CTANfile{pk300}{fonts/cm/pk/pk300.zip}
\CTANfile{pk300w}{fonts/cm/pk/pk300w.zip}
\CTANfile{QED}{macros/generic/proofs/taylor/QED.sty}[qed]
\CTANfile{removefr}{macros/latex/contrib/fragments/removefr.tex}[removefr]
\CTANfile{repeat}{macros/generic/eijkhout/repeat.tex}[repeat]
\CTANfile{resume}{obsolete/macros/latex209/contrib/resume/resume.sty}
\CTANfile{savesym}{macros/latex/contrib/savesym/savesym.sty}[savesym]
\CTANfile{setspace}{macros/latex/contrib/setspace/setspace.sty}[setspace]
\CTANfile{simpl-latex}{info/simplified-latex/simplified-intro.pdf}[simplified-latex]
\CTANfile{sober}{macros/latex209/contrib/misc/sober.sty}[sober]
\CTANfile{tex2bib}{biblio/bibtex/utils/tex2bib/tex2bib}[tex2bib]
\CTANfile{tex2bib-doc}{biblio/bibtex/utils/tex2bib/README}
\CTANfile{tex4ht}{obsolete/support/TeX4ht/tex4ht-all.zip}[tex4ht]
\CTANfile{texlive-unix}{systems/texlive/tlnet/install-tl-unx.tar.gz}
\CTANfile{texlive-windows}{systems/texlive/tlnet/install-tl.zip}
\CTANfile{texnames}{info/biblio/texnames.sty}
\CTANfile{texsis-index}{macros/texsis/index/index.tex}
\CTANfile{topcapt}{macros/latex/contrib/misc/topcapt.sty}[topcapt]
\CTANfile{tracking}{macros/latex/contrib/tracking/tracking.sty}[tracking]
\CTANfile{ttb-pdf}{info/bibtex/tamethebeast/ttb_en.pdf}[tamethebeast]
\CTANfile{type1ec}{fonts/ps-type1/cm-super/type1ec.sty}[type1ec]
\CTANfile{ukhyph}{language/hyphenation/ukhyphen.tex}
\CTANfile{upquote}{macros/latex/contrib/upquote/upquote.sty}[upquote]
\CTANfile{faq-a4}{help/uk-tex-faq/newfaq.pdf}
\CTANfile{faq-letter}{help/uk-tex-faq/letterfaq.pdf}
\CTANfile{versions}{macros/latex/contrib/versions/versions.sty}[versions]
\CTANfile{vf-howto}{info/virtualfontshowto/virtualfontshowto.txt}[vf-howto]
\CTANfile{vf-knuth}{info/knuth/virtual-fonts}[vf-knuth]
\CTANfile{visualFAQ}{info/visualFAQ/visualFAQ.pdf}[visualfaq]
\CTANfile{voss-mathmode}{info/math/voss/mathmode/Mathmode.pdf}
\CTANfile{wujastyk-txh}{digests/texhax/txh/wujastyk.txh}
\CTANfile{xampl-bib}{biblio/bibtex/base/xampl.bib}
\CTANfile{xtexcad}{graphics/xtexcad/xtexcad-2.4.1.tar.gz}


% facilitate auto-processing of this stuff; this line is clunkily
% detected (and ignored) in the build-faqbody script
%\let\faqinput\input
\def\faqinput#1{%\message{*** inputting #1; group level \the\currentgrouplevel}
  \input{#1}%
  %\message{*** out of #1; group level \the\currentgrouplevel}
}

% the stuff to print
\faqinput{faq-intro}               % introduction
\faqinput{faq-backgrnd}            % background
\faqinput{faq-docs}                % docs
\faqinput{faq-bits+pieces}         % bits and pieces
\faqinput{faq-getit}               % getting software
\faqinput{faq-texsys}              % TeX systems
\faqinput{faq-dvi}                 % DVI drivers and previewers
\faqinput{faq-support}             % support stuff
\faqinput{faq-litprog}             % programming for literates
\faqinput{faq-fmt-conv}            % format conversions
\faqinput{faq-install}             % installation and so on
\faqinput{faq-fonts}               % fonts and what to do
\faqinput{faq-hyp+pdf}             % hyper-foodle and pdf
\faqinput{faq-graphics}            % graphics
\faqinput{faq-biblio}              % bib stuff
\faqinput{faq-adj-types}           % adjusting typesetting
\faqinput{faq-lab-ref}             % labels and references
\faqinput{faq-how-do-i}            % how to do things
\faqinput{faq-symbols}             % symbols, their natural history and use
\faqinput{faq-mac-prog}            % macro programming
\faqinput{faq-t-g-wr}              % things going wrong
\faqinput{faq-wdidt}               % why does it do that
%*****************************************quote environments up to here
\faqinput{faq-jot-err}             % joy (hem hem) of tex errors
\faqinput{faq-projects}            % current projects
%
% This is the last section, and is to remain the last section...
\faqinput{faq-the-end}             % wrapping it all up

% \end{document} is in calling file (e.g., newfaq.tex)

\else
  \begin{multicols}{2}
  \def\faqfileversion{3.28}    \def\faqfiledate{2014-06-10}
%
% The above line defines the file version and date, and must remain
% the first line with any `assignment' in the file, or things will
% blow up in a stupid fashion
%
% get lists of CTAN labels
%
% configuration for the lists, if we're going to need to generate urls
% for the files
\InputIfFileExists{archive.cfg}{}{}
%
% ... directories
% $Id: dirctan.tex,v 1.302 2013/07/24 21:43:10 rf10 Exp rf10 $
%
% protect ourself against being read twice
\csname readCTANdirs\endcsname
\let\readCTANdirs\endinput
%
% declarations of significant directories on CTAN
\CTANdirectory{2etools}{macros/latex/required/tools}[tools]
\CTANdirectory{4spell}{support/4spell}[fourspell]
\CTANdirectory*{Catalogue}{help/Catalogue}
\CTANdirectory*{MathTeX}{support/mathtex}[mathtex]
\CTANdirectory{MimeTeX}{support/mimetex}[mimetex]
\CTANdirectory{Tabbing}{macros/latex/contrib/Tabbing}[tabbing]
\CTANdirectory*{TeXtelmExtel}{systems/msdos/emtex-contrib/TeXtelmExtel}
\CTANdirectory{TftI}{info/impatient}[impatient]
\CTANdirectory{a0poster}{macros/latex/contrib/a0poster}[a0poster]
\CTANdirectory*{a2ping}{graphics/a2ping}[a2ping]
\CTANdirectory{a4}{macros/latex/contrib/a4}[a4]
\CTANdirectory*{abc2mtex}{support/abc2mtex}
\CTANdirectory{abstract}{macros/latex/contrib/abstract}[abstract]
\CTANdirectory{abstyles}{biblio/bibtex/contrib/abstyles}
\CTANdirectory{accents}{support/accents}
\CTANdirectory{acronym}{macros/latex/contrib/acronym}[acronym]
\CTANdirectory*{ada}{web/ada/aweb}
\CTANdirectory{addindex}{indexing/addindex}
\CTANdirectory{addlines}{macros/latex/contrib/addlines}[addlines]
\CTANdirectory{adjkerns}{fonts/utilities/adjkerns}
\CTANdirectory{ae}{fonts/ae}[ae]
\CTANdirectory{aeguill}{macros/latex/contrib/aeguill}[aeguill]
\CTANdirectory{afmtopl}{fonts/utilities/afmtopl}
\CTANdirectory{akletter}{macros/latex/contrib/akletter}
\CTANdirectory{aleph}{systems/aleph}
\CTANdirectory{alg}{macros/latex/contrib/alg}
\CTANdirectory{algorithm2e}{macros/latex/contrib/algorithm2e}[algorithm2e]
\CTANdirectory{algorithmicx}{macros/latex/contrib/algorithmicx}[algorithmicx]
\CTANdirectory{algorithms}{macros/latex/contrib/algorithms}[algorithms]
\CTANdirectory*{alpha}{systems/mac/support/alpha}[alpha]
\CTANdirectory{amiga}{systems/amiga}
\CTANdirectory{amscls}{macros/latex/required/amslatex/amscls}[amscls]
\CTANdirectory{amsfonts}{fonts/amsfonts}[amsfonts]
\CTANdirectory{amslatex}{macros/latex/required/amslatex}[amslatex]
\CTANdirectory{amslatex-primer}{info/amslatex/primer}[amslatex-primer]
\CTANdirectory{amspell}{support/amspell}
\CTANdirectory{amsrefs}{macros/latex/contrib/amsrefs}[amsrefs]
\CTANdirectory{amstex}{macros/amstex}[amstex]
\CTANdirectory{anonchap}{macros/latex/contrib/anonchap}[anonchap]
\CTANdirectory{answers}{macros/latex/contrib/answers}[answers]
\CTANdirectory{ant}{systems/ant}[ant]
\CTANdirectory{anyfontsize}{macros/latex/contrib/anyfontsize}[anyfontsize]
\CTANdirectory{apl}{fonts/apl}
\CTANdirectory{aplweb}{web/apl/aplweb}
\CTANdirectory{appl}{web/reduce/rweb/appl}
\CTANdirectory{appendix}{macros/latex/contrib/appendix}[appendix]
\CTANdirectory{arabtex}{language/arabic/arabtex}
\CTANdirectory{arara}{support/arara}[arara]
\CTANdirectory{aro-bend}{info/challenges/aro-bend}[aro-bend]
\CTANdirectory{asana-math}{fonts/Asana-Math}[asana-math]
\CTANdirectory{asc2tex}{systems/msdos/asc2tex}
\CTANdirectory{ascii}{fonts/ascii}
\CTANdirectory*{aspell}{support/aspell}[aspell]
\CTANdirectory{astro}{fonts/astro}
\CTANdirectory{asymptote}{graphics/asymptote}[asymptote]
\CTANdirectory{asyfig}{macros/latex/contrib/asyfig}[asyfig]
\CTANdirectory{atari}{systems/atari}
\CTANdirectory*{atari-cstex}{systems/atari/cs-tex}[atari-cstex]
\CTANdirectory{attachfile}{macros/latex/contrib/attachfile}
\CTANdirectory*{auctex}{support/auctex}[auctex]
\CTANdirectory{autolatex}{support/autolatex}
\CTANdirectory{auto-pst-pdf}{macros/latex/contrib/auto-pst-pdf}[auto-pst-pdf]
\CTANdirectory{aweb}{web/ada/aweb}
\CTANdirectory*{awk}{web/spiderweb/src/awk}
\CTANdirectory{axodraw}{graphics/axodraw}[axodraw]
\CTANdirectory{babel}{macros/latex/required/babel}[babel]
\CTANdirectory{babelbib}{biblio/bibtex/contrib/babelbib}[babelbib]
\CTANdirectory{badge}{macros/plain/contrib/badge}
\CTANdirectory{bakoma}{fonts/cm/ps-type1/bakoma}[bakoma-fonts]
\CTANdirectory*{bakoma-tex}{systems/win32/bakoma}[bakoma]
\CTANdirectory*{bakoma-texfonts}{systems/win32/bakoma/fonts}
\CTANdirectory*{bard}{fonts/bard}
\CTANdirectory{barr}{macros/generic/diagrams/barr}
\CTANdirectory{bashkirian}{fonts/cyrillic/bashkirian}
\CTANdirectory{basix}{macros/generic/basix}
\CTANdirectory{bbding}{fonts/bbding}
\CTANdirectory{bbfig}{support/bbfig}
\CTANdirectory{bbm}{fonts/cm/bbm}[bbm]
\CTANdirectory{bbm-macros}{macros/latex/contrib/bbm}[bbm-macros]
\CTANdirectory{bbold}{fonts/bbold}[bbold]
\CTANdirectory{bdfchess}{fonts/chess/bdfchess}
\CTANdirectory{beamer}{macros/latex/contrib/beamer}[beamer]
\CTANdirectory{beamerposter}{macros/latex/contrib/beamerposter}[beamerposter]
\CTANdirectory{beebe}{dviware/beebe}
\CTANdirectory{belleek}{fonts/belleek}[belleek]
\CTANdirectory{beton}{macros/latex/contrib/beton}[beton]
\CTANdirectory{bezos}{macros/latex/contrib/bezos}[bezos]
\CTANdirectory{bib-fr}{biblio/bibtex/contrib/bib-fr}
\CTANdirectory{bib2dvi}{biblio/bibtex/utils/bib2dvi}
\CTANdirectory{bib2xhtml}{biblio/bibtex/utils/bib2xhtml}
\CTANdirectory*{bibcard}{biblio/bibtex/utils/bibcard}
\CTANdirectory*{bibclean}{biblio/bibtex/utils/bibclean}
\CTANdirectory*{bibdb}{support/bibdb}
\CTANdirectory{biber}{biblio/biber}[biber]
\CTANdirectory{bibextract}{biblio/bibtex/utils/bibextract}
\CTANdirectory{bibgerm}{biblio/bibtex/contrib/germbib}
\CTANdirectory{bibindex}{biblio/bibtex/utils/bibindex}
\CTANdirectory{biblatex}{macros/latex/contrib/biblatex}[biblatex]
\CTANdirectory*{biblatex-contrib}{macros/latex/contrib/biblatex-contrib}
\CTANdirectory{biblio}{info/biblio}
\CTANdirectory{biblist}{macros/latex209/contrib/biblist}
\CTANdirectory{bibsort}{biblio/bibtex/utils/bibsort}
\CTANdirectory{bibtex}{biblio/bibtex/base}[bibtex]
\CTANdirectory*{bibtex8}{biblio/bibtex/8-bit}[bibtex8bit]
\CTANdirectory{bibtex-doc}{biblio/bibtex/contrib/doc}[bibtex]
\CTANdirectory{bibtool}{biblio/bibtex/utils/bibtool}
\CTANdirectory{bibtools}{biblio/bibtex/utils/bibtools}
\CTANdirectory{bibtopic}{macros/latex/contrib/bibtopic}[bibtopic]
\CTANdirectory{bibunits}{macros/latex/contrib/bibunits}[bibunits]
\CTANdirectory{bibview}{biblio/bibtex/utils/bibview}
\CTANdirectory{bigfoot}{macros/latex/contrib/bigfoot}[bigfoot]
\CTANdirectory{bigstrut}{macros/latex/contrib/multirow}[bigstrut]
\CTANdirectory{bit2spr}{graphics/bit2spr}
\CTANdirectory{black}{fonts/cm/utilityfonts/black}
\CTANdirectory{blackboard}{info/symbols/blackboard}[blackboard]
\CTANdirectory{blackletter}{fonts/blackletter}
\CTANdirectory{blindtext}{macros/latex/contrib/blindtext}[blindtext]
\CTANdirectory{blocks}{macros/text1/blocks}
\CTANdirectory{blu}{macros/blu}
\CTANdirectory{bm2font}{graphics/bm2font}
\CTANdirectory{boites}{macros/latex/contrib/boites}[boites]
\CTANdirectory{bold}{fonts/cm/mf-extra/bold}
\CTANdirectory{bold-extra}{macros/latex/contrib/bold-extra}[bold-extra]
\CTANdirectory{bonus}{systems/msdos/emtex-contrib/bonus}
\CTANdirectory{booktabs}{macros/latex/contrib/booktabs}[booktabs]
\CTANdirectory{boondox}{fonts/boondox}[boondox]
\CTANdirectory{borceux}{macros/generic/diagrams/borceux}
\CTANdirectory{braket}{macros/latex/contrib/braket}[braket]
\CTANdirectory{breakurl}{macros/latex/contrib/breakurl}[breakurl]
\CTANdirectory{bridge}{macros/plain/contrib/bridge}
\CTANdirectory{brief_t}{support/brief_t}
\CTANdirectory{bst}{biblio/bibtex/contrib/germbib/bst}
\CTANdirectory{btable}{macros/plain/contrib/btable}
\CTANdirectory{btex8fmt}{macros/generic/cptex/btex8fmt}
\CTANdirectory{btOOL}{biblio/bibtex/utils/btOOL}
\CTANdirectory{bundledoc}{support/bundledoc}[bundledoc]
\CTANdirectory{c}{web/spiderweb/src/c}
\CTANdirectory{c++}{web/spiderweb/src/c++}
\CTANdirectory{c++2latex}{support/C++2LaTeX-1_1pl1}
\CTANdirectory{c2cweb}{web/c_cpp/c2cweb}
\CTANdirectory{c2latex}{support/c2latex}
\CTANdirectory{c_cpp}{web/c_cpp}
\CTANdirectory{caesar-fonts-generic.dir}{macros/generic/caesarcm/caesar-fonts-generic.dir}
\CTANdirectory{caesarcm}{macros/generic/caesarcm}
\CTANdirectory{caesarcmfonts.dir}{macros/generic/caesarcm/caesarcmfonts.dir}
\CTANdirectory{caesarcmv2.dir}{macros/generic/caesarcm/caesarcmv2.dir}
\CTANdirectory{calendar}{macros/plain/contrib/calendar}
\CTANdirectory{calligra}{fonts/calligra}
\CTANdirectory{calrsfs}{macros/latex/contrib/calrsfs}
\CTANdirectory{cancel}{macros/latex/contrib/cancel}[cancel]
\CTANdirectory{capt-of}{macros/latex/contrib/capt-of}[capt-of]
\CTANdirectory{caption}{macros/latex/contrib/caption}[caption]
\CTANdirectory{carlisle}{macros/latex/contrib/carlisle}[carlisle]
\CTANdirectory{cascover}{macros/plain/contrib/cascover}
\CTANdirectory{casslbl}{macros/plain/contrib/casslbl}
\CTANdirectory{catdvi}{dviware/catdvi}[catdvi]
\CTANdirectory{ccaption}{macros/latex/contrib/ccaption}[ccaption]
\CTANdirectory{ccfonts}{macros/latex/contrib/ccfonts}[ccfonts]
\CTANdirectory{cellular}{macros/plain/contrib/cellular}
\CTANdirectory{cellspace}{macros/latex/contrib/cellspace}[cellspace]
\CTANdirectory{changebar}{macros/latex/contrib/changebar}[changebar]
\CTANdirectory{changepage}{macros/latex/contrib/changepage}[changepage]
\CTANdirectory{changes}{macros/latex/contrib/changes}[changes]
\CTANdirectory{chappg}{macros/latex/contrib/chappg}[chappg]
\CTANdirectory{chapterfolder}{macros/latex/contrib/chapterfolder}[chapterfolder]
\CTANdirectory{charconv}{support/charconv}
\CTANdirectory{charter}{fonts/charter}
\CTANdirectory{chbar}{macros/plain/contrib/chbar}
\CTANdirectory{chbars}{macros/latex209/contrib/chbars}
\CTANdirectory{check}{support/check}
\CTANdirectory{chemstruct}{macros/latex209/contrib/chemstruct}
\CTANdirectory{chemtex}{macros/latex209/contrib/chemtex}
\CTANdirectory{cheq}{fonts/chess/cheq}
\CTANdirectory{cherokee}{fonts/cherokee}
\CTANdirectory{chesstools}{support/chesstools}
\CTANdirectory{chi2tex}{support/chi2tex}
\CTANdirectory{china2e}{macros/latex/contrib/china2e}
\CTANdirectory{chinese}{language/chinese}
\CTANdirectory{chngcntr}{macros/latex/contrib/chngcntr}[chngcntr]
\CTANdirectory{circ}{macros/generic/diagrams/circ}
\CTANdirectory{circuit_macros}{graphics/circuit_macros}
\CTANdirectory{cirth}{fonts/cirth}
\CTANdirectory{cite}{macros/latex/contrib/cite}[cite]
\CTANdirectory{citeref}{macros/latex/contrib/citeref}[citeref]
\CTANdirectory{clark}{fonts/utilities/afmtopl/clark}
\CTANdirectory{classico}{fonts/urw/classico}[classico]
\CTANdirectory{clrscode}{macros/latex/contrib/clrscode}[clrscode]
\CTANdirectory{cm}{fonts/cm}
\CTANdirectory{cm-lgc}{fonts/ps-type1/cm-lgc}[cm-lgc]
\CTANdirectory{cm-super}{fonts/ps-type1/cm-super}[cm-super]
\CTANdirectory{cm-unicode}{fonts/cm-unicode}[cm-unicode]
\CTANdirectory{cmactex}{systems/mac/cmactex}[cmactex]
\CTANdirectory{cmap}{macros/latex/contrib/cmap}
\CTANdirectory{cmastro}{fonts/cmastro}
\CTANdirectory{cmbright}{fonts/cmbright}[cmbright]
\CTANdirectory{cmcyralt}{macros/latex/contrib/cmcyralt}
%[fonts/cmcyralt]
\CTANdirectory{cmfrak}{fonts/gothic/cmfrak}
\CTANdirectory{cmoefont}{fonts/cmoefont}
\CTANdirectory{cmolddig}{fonts/cmolddig}
\CTANdirectory{cmoutlines}{fonts/cm/cmoutlines}[cmoutlines]
\CTANdirectory{cmpica}{fonts/cmpica}
\CTANdirectory{cms_help_files}{macros/text1/cms_help_files}
\CTANdirectory{cmtest}{fonts/cm/cmtest}
\CTANdirectory{cnoweb}{web/c_cpp/cnoweb}
\CTANdirectory{collref}{macros/latex/contrib/collref}
\CTANdirectory{combine}{macros/latex/contrib/combine}[combine]
\CTANdirectory{commado}{macros/generic/commado}[commado]
\CTANdirectory{comment}{macros/latex/contrib/comment}[comment]
\CTANdirectory{committee}{fonts/unsupported/committee}
\CTANdirectory{comp-fonts-FAQ}{help/comp-fonts-FAQ}
\CTANdirectory{components-of-TeX}{info/components-of-TeX}
\CTANdirectory{compugraphics_8600}{macros/text1/compugraphics_8600}
\CTANdirectory{concmath}{macros/latex/contrib/concmath}[concmath]
\CTANdirectory{concmath-f}{fonts/concmath}[concmath-fonts]
\CTANdirectory{concrete}{fonts/concrete}[concrete]
\CTANdirectory{context}{macros/context/current}[context]
\CTANdirectory{context-contrib}{macros/context/contrib}
\CTANdirectory{cprotect}{macros/latex/contrib/cprotect}[cprotect]
\CTANdirectory{cptex}{macros/generic/cptex}
\CTANdirectory{crop}{macros/latex/contrib/crop}[crop]
\CTANdirectory*{crosstex}{biblio/crosstex}[crosstex]
\CTANdirectory{crosswrd}{macros/latex/contrib/crosswrd}
\CTANdirectory{crudetype}{dviware/crudetype}[crudetype]
\CTANdirectory{crw}{macros/plain/contrib/crw}
\CTANdirectory{cs-tex}{systems/atari/cs-tex}
\CTANdirectory{csvsimple}{macros/latex/contrib/csvsimple}[csvsimple]
\CTANdirectory{ctable}{macros/latex/contrib/ctable}[ctable]
\CTANdirectory{ctan}{help/ctan}
\CTANdirectory{cun}{fonts/cun}
\CTANdirectory{currfile}{macros/latex/contrib/currfile}[currfile]
\CTANdirectory{currvita}{macros/latex/contrib/currvita}[currvita]
\CTANdirectory{curve}{macros/latex/contrib/curve}[curve]
\CTANdirectory{curves}{macros/latex/contrib/curves}
\CTANdirectory{custom-bib}{macros/latex/contrib/custom-bib}[custom-bib]
\CTANdirectory{cutwin}{macros/latex/contrib/cutwin}[cutwin]
\CTANdirectory{cweb}{web/c_cpp/cweb}
\CTANdirectory{cweb-p}{web/c_cpp/cweb-p}
\CTANdirectory{cypriote}{fonts/cypriote}
\CTANdirectory{cyrillic}{language/cyrillic}
\CTANdirectory{cyrtug}{language/cyrtug}
\CTANdirectory{dante}{usergrps/dante}
\CTANdirectory{dante-faq}{help/de-tex-faq}
\CTANdirectory{databases}{biblio/bibtex/databases}
\CTANdirectory{datatool}{macros/latex/contrib/datatool}[datatool]
\CTANdirectory{datetime}{macros/latex/contrib/datetime}
\CTANdirectory{db2tex}{support/db2tex}
\CTANdirectory{dblfloatfix}{macros/latex/contrib/dblfloatfix}[dblfloatfix]
\CTANdirectory{dbtex}{support/dbtex}
\CTANdirectory{dc-latex}{language/hyphen-accent/dc-latex}
\CTANdirectory{dc-nfss}{language/hyphen-accent/dc-nfss}
\CTANdirectory{detex}{support/detex}[detex]
\CTANdirectory{devanagari}{language/devanagari}
\CTANdirectory{diagbox}{macros/latex/contrib/diagbox}[diagbox]
\CTANdirectory{diagrams}{macros/generic/diagrams}
\CTANdirectory{dijkstra}{web/spiderweb/src/dijkstra}
\CTANdirectory{dinbrief}{macros/latex/contrib/dinbrief}
\CTANdirectory{dingbat}{fonts/dingbat}
\CTANdirectory{djgpp}{systems/msdos/djgpp}
\CTANdirectory{dk-bib}{biblio/bibtex/contrib/dk-bib}
\CTANdirectory{dktools}{support/dktools}[dktools]
\CTANdirectory{dm-latex}{language/hyphen-accent/dm-latex}
\CTANdirectory{dm-plain}{language/hyphen-accent/dm-plain}
\CTANdirectory{doc2sty}{language/swedish/slatex/doc2sty}
\CTANdirectory{docmfp}{macros/latex/contrib/docmfp}[docmfp]
\CTANdirectory{docmute}{macros/latex/contrib/docmute}[docmute]
\CTANdirectory{docu}{support/makeprog/docu}
\CTANdirectory{document}{biblio/bibtex/contrib/germbib/document}
\CTANdirectory{dos-dc}{systems/msdos/dos-dc}
\CTANdirectory{dos-psfonts}{systems/msdos/emtex-fonts/psfonts}
\CTANdirectory{doublestroke}{fonts/doublestroke}[doublestroke]
\CTANdirectory{dowith}{macros/generic/dowith}[dowith]
\CTANdirectory{dpfloat}{macros/latex/contrib/dpfloat}[dpfloat]
\CTANdirectory{dpmigcc}{systems/msdos/dpmigcc}
\CTANdirectory{draftcopy}{macros/latex/contrib/draftcopy}[draftcopy]
\CTANdirectory{draftwatermark}{macros/latex/contrib/draftwatermark}[draftwatermark]
\CTANdirectory{dratex}{graphics/dratex}[dratex]
\CTANdirectory{drawing}{graphics/drawing}
\CTANdirectory{dropcaps}{macros/latex209/contrib/dropcaps}
\CTANdirectory{dropping}{macros/latex/contrib/dropping}[dropping]
\CTANdirectory{dtl}{dviware/dtl}[dtl]
\CTANdirectory{dtxgen}{support/dtxgen}[dtxgen]
\CTANdirectory{dtxtut}{info/dtxtut}[dtxtut]
\CTANdirectory{duerer}{fonts/duerer}
\CTANdirectory{dvgt}{dviware/dvgt}
\CTANdirectory{dvi-augsburg}{dviware/dvi-augsburg}
\CTANdirectory{dvi2bitmap}{dviware/dvi2bitmap}[dvi2bitmap]
\CTANdirectory{dvi2pcl}{dviware/dvi2pcl}
\CTANdirectory{dvi2tty}{dviware/dvi2tty}[dvi2tty]
\CTANdirectory{dviasm}{dviware/dviasm}[dviasm]
\CTANdirectory{dvibit}{dviware/dvibit}
\CTANdirectory{dvibook}{dviware/dvibook}
\CTANdirectory{dvichk}{dviware/dvichk}
\CTANdirectory{dvicopy}{dviware/dvicopy}
\CTANdirectory{dvidjc}{dviware/dvidjc}
\CTANdirectory{dvidvi}{dviware/dvidvi}
\CTANdirectory{dviimp}{dviware/dviimp}
\CTANdirectory{dviljk}{dviware/dviljk}
\CTANdirectory{dvimerge}{dviware/dvimerge}
\CTANdirectory{dvimfj}{systems/msdos/emtex-contrib/dvimfj}
\CTANdirectory{dvipage}{dviware/dvipage}
\CTANdirectory{dvipaste}{dviware/dvipaste}
\CTANdirectory{dvipdfm}{dviware/dvipdfm}
\CTANdirectory{dvipdfmx}{dviware/dvipdfmx}[dvipdfmx]
\CTANdirectory{dvipj}{dviware/dvipj}
\CTANdirectory{dvipng}{dviware/dvipng}[dvipng]
\CTANdirectory{dvips-pc}{systems/msdos/dviware/dvips}
\CTANdirectory{dvips}{dviware/dvips}[dvips]
\CTANdirectory{dvistd}{dviware/driv-standard}
\CTANdirectory{dvisun}{dviware/dvisun}
\CTANdirectory{dvitty}{dviware/dvitty}
\CTANdirectory{dvivga}{dviware/dvivga}
\CTANdirectory{e4t}{systems/msdos/e4t}
\CTANdirectory*{e-TeX}{systems/e-tex}
\CTANdirectory{easytex}{systems/msdos/easytex}
\CTANdirectory{ebib}{biblio/bibtex/utils/ebib}
\CTANdirectory{ec}{fonts/ec}[ec]
\CTANdirectory{ec-plain}{macros/ec-plain}[ec-plain]
\CTANdirectory{eco}{fonts/eco}[eco]
\CTANdirectory{economic}{biblio/bibtex/contrib/economic}
\CTANdirectory{edmac}{macros/plain/contrib/edmac}[edmac]
\CTANdirectory{ednotes}{macros/latex/contrib/ednotes}[ednotes]
\CTANdirectory{eepic}{macros/latex/contrib/eepic}[eepic]
\CTANdirectory{ega2mf}{fonts/utilities/ega2mf}
\CTANdirectory{egplot}{macros/latex/contrib/egplot}[egplot]
\CTANdirectory{eiad}{fonts/eiad}
\CTANdirectory{elvish}{fonts/elvish}
\CTANdirectory{elwell}{fonts/utilities/afmtopl/elwell}
\CTANdirectory{emp}{macros/latex/contrib/emp}[emp]
\CTANdirectory{emptypage}{macros/latex/contrib/emptypage}[emptypage]
\CTANdirectory{emt2tex}{systems/msdos/emtex-contrib/emt2tex}
\CTANdirectory{emtex}{systems/msdos/emtex}
\CTANdirectory{emtex-contrib}{systems/msdos/emtex-contrib}
\CTANdirectory{emtex-fonts}{systems/msdos/emtex-fonts}
\CTANdirectory{emtextds}{obsolete/systems/os2/emtex-contrib/emtexTDS}
\CTANdirectory{enctex}{systems/enctex}[enctex]
\CTANdirectory{endfloat}{macros/latex/contrib/endfloat}[endfloat]
\CTANdirectory{english}{language/english}
\CTANdirectory{engwar}{fonts/engwar}
\CTANdirectory{enumitem}{macros/latex/contrib/enumitem}[enumitem]
\CTANdirectory{environment}{support/lsedit/environment}
\CTANdirectory{epic}{macros/latex/contrib/epic}[epic]
\CTANdirectory{epigraph}{macros/latex/contrib/epigraph}[epigraph]
\CTANdirectory{eplain}{macros/eplain}[eplain]
\CTANdirectory{epmtex}{systems/os2/epmtex}
\CTANdirectory{epslatex}{info/epslatex}[epslatex]
\CTANdirectory*{epstopdf}{support/epstopdf}[epstopdf]
\CTANdirectory{eqparbox}{macros/latex/contrib/eqparbox}[eqparbox]
\CTANdirectory{ergotex}{systems/msdos/ergotex}
\CTANdirectory{errata}{systems/knuth/dist/errata}
\CTANdirectory{eso-pic}{macros/latex/contrib/eso-pic}[eso-pic]
\CTANdirectory{et}{support/et}
\CTANdirectory{etex}{systems/e-tex}[etex]
\CTANdirectory{etex-pkg}{macros/latex/contrib/etex-pkg}[etex-pkg]
\CTANdirectory{etextools}{macros/latex/contrib/etextools}[etextools]
\CTANdirectory{ethiopia}{language/ethiopia}
\CTANdirectory{ethtex}{language/ethiopia/ethtex}
\CTANdirectory{etoc}{macros/latex/contrib/etoc}[etoc]
\CTANdirectory{etoolbox}{macros/latex/contrib/etoolbox}[etoolbox]
\CTANdirectory{euler-latex}{macros/latex/contrib/euler}[euler]
\CTANdirectory{eulervm}{fonts/eulervm}[eulervm]
\CTANdirectory{euro-ce}{fonts/euro-ce}[euro-ce]
\CTANdirectory{euro-fonts}{fonts/euro}
\CTANdirectory{eurofont}{macros/latex/contrib/eurofont}[eurofont]
\CTANdirectory{europecv}{macros/latex/contrib/europecv}[europecv]
\CTANdirectory{eurosym}{fonts/eurosym}[eurosym]
\CTANdirectory{everypage}{macros/latex/contrib/everypage}[everypage]
\CTANdirectory{excalibur}{systems/mac/support/excalibur}
\CTANdirectory{excel2latex}{support/excel2latex}
\CTANdirectory{excerpt}{web/spiderweb/tools/excerpt}
\CTANdirectory{excludeonly}{macros/latex/contrib/excludeonly}[excludeonly]
\CTANdirectory{expdlist}{macros/latex/contrib/expdlist}
\CTANdirectory{extract}{macros/latex/contrib/extract}[extract]
\CTANdirectory{extsizes}{macros/latex/contrib/extsizes}[extsizes]
% cat links to here
\CTANdirectory{fancyhdr}{macros/latex/contrib/fancyhdr}[fancyhdr]
\CTANdirectory{fancyvrb}{macros/latex/contrib/fancyvrb}[fancyvrb]
\CTANdirectory{faq}{help/uk-tex-faq}[uk-tex-faq]
\CTANdirectory{fc}{fonts/jknappen/fc}
\CTANdirectory{fdsymbol}{fonts/fdsymbol}[fdsymbol]
\CTANdirectory{feyn}{fonts/feyn}
\CTANdirectory{feynman}{macros/latex209/contrib/feynman}
\CTANdirectory{feynmf}{macros/latex/contrib/feynmf}[feynmf]
\CTANdirectory{fig2eng}{graphics/fig2eng}
\CTANdirectory{fig2mf}{graphics/fig2mf}
\CTANdirectory{fig2mfpic}{graphics/fig2mfpic}
\CTANdirectory{figflow}{macros/plain/contrib/figflow}[figflow]
\CTANdirectory{filehook}{macros/latex/contrib/filehook}[filehook]
\CTANdirectory{fink}{macros/latex/contrib/fink}[fink]
\CTANdirectory{first-latex-doc}{info/first-latex-doc}
\CTANdirectory{fix2col}{macros/latex/contrib/fix2col}[fix2col]
\CTANdirectory{fixfoot}{macros/latex/contrib/fixfoot}[fixfoot]
\CTANdirectory{float}{macros/latex/contrib/float}[float]
\CTANdirectory{floatflt}{macros/latex/contrib/floatflt}[floatflt]
\CTANdirectory{flow}{support/flow}
\CTANdirectory{flowfram}{macros/latex/contrib/flowfram}[flowfram]
\CTANdirectory{fltpage}{macros/latex/contrib/fltpage}[fltpage]
\CTANdirectory{fnbreak}{macros/latex/contrib/fnbreak}[fnbreak]
\CTANdirectory{fncychap}{macros/latex/contrib/fncychap}[fncychap]
\CTANdirectory{fncylab}{macros/latex/contrib/fncylab}
\CTANdirectory{foiltex}{macros/latex/contrib/foiltex}[foiltex]
\CTANdirectory{font-change}{macros/plain/contrib/font-change}[font-change]
\CTANdirectory{fontch}{macros/plain/contrib/fontch}[fontch]
\CTANdirectory{fontinst}{fonts/utilities/fontinst}[fontinst]
\CTANdirectory{fontname}{info/fontname}[fontname]
\CTANdirectory{fontspec}{macros/latex/contrib/fontspec}[fontspec]
\CTANdirectory{font_selection}{macros/plain/contrib/font_selection}[font-selection]
\CTANdirectory{footbib}{macros/latex/contrib/footbib}[footbib]
\CTANdirectory{footmisc}{macros/latex/contrib/footmisc}[footmisc]
\CTANdirectory{footnpag}{macros/latex/contrib/footnpag}[footnpag]
\CTANdirectory{forarray}{macros/latex/contrib/forarray}[forarray]
\CTANdirectory{forloop}{macros/latex/contrib/forloop}[forloop]
\CTANdirectory{for_tex}{biblio/bibtex/contrib/germbib/for_tex}
\CTANdirectory{fourier}{fonts/fourier-GUT}[fourier]
\CTANdirectory{fouriernc}{fonts/fouriernc}[fouriernc]
\CTANdirectory{framed}{macros/latex/contrib/framed}[framed]
\CTANdirectory{frankenstein}{macros/latex/contrib/frankenstein}[frankenstein]
\CTANdirectory{french-faq}{help/LaTeX-FAQ-francaise}
\CTANdirectory{funnelweb}{web/funnelweb}
\CTANdirectory{futhark}{fonts/futhark}
\CTANdirectory{futhorc}{fonts/futhorc}
\CTANdirectory{fweb}{web/fweb}[fweb]
\CTANdirectory{garamondx}{fonts/garamondx}[garamondx]
\CTANdirectory{gellmu}{support/gellmu}[gellmu]
\CTANdirectory{genfam}{support/genfam}
\CTANdirectory{geometry}{macros/latex/contrib/geometry}[geometry]
\CTANdirectory{germbib}{biblio/bibtex/contrib/germbib}
\CTANdirectory{getoptk}{macros/plain/contrib/getoptk}[getoptk]
\CTANdirectory{gfs}{info/examples/FirstSteps} % gratzer's
\CTANdirectory{gitinfo}{macros/latex/contrib/gitinfo}[gitinfo]
\CTANdirectory{glo+idxtex}{indexing/glo+idxtex}[idxtex]
\CTANdirectory{gmp}{macros/latex/contrib/gmp}[gmp]
\CTANdirectory{gnuplot}{graphics/gnuplot}
\CTANdirectory{go}{fonts/go}
\CTANdirectory{gothic}{fonts/gothic}
\CTANdirectory{graphbase}{support/graphbase}
\CTANdirectory{graphics}{macros/latex/required/graphics}[graphics]
\CTANdirectory{graphics-plain}{macros/plain/graphics}[graphics-pln]
\CTANdirectory{graphicx-psmin}{macros/latex/contrib/graphicx-psmin}[graphicx-psmin]
\CTANdirectory{gray}{fonts/cm/utilityfonts/gray}
\CTANdirectory{greek}{fonts/greek}
\CTANdirectory{greektex}{fonts/greek/greektex}
\CTANdirectory{gsftopk}{fonts/utilities/gsftopk}[gsftopk]
\CTANdirectory{gut}{usergrps/gut}
\CTANdirectory*{gv}{support/gv}[gv]
\CTANdirectory{ha-prosper}{macros/latex/contrib/ha-prosper}[ha-prosper]
\CTANdirectory{half}{fonts/cm/utilityfonts/half}
\CTANdirectory{halftone}{fonts/halftone}
\CTANdirectory{hands}{fonts/hands}
\CTANdirectory{harvard}{macros/latex/contrib/harvard}
\CTANdirectory{harvmac}{macros/plain/contrib/harvmac}
\CTANdirectory{hebrew}{language/hebrew}
\CTANdirectory{help}{help}
\CTANdirectory{here}{macros/latex/contrib/here}[here]
\CTANdirectory{hershey}{fonts/hershey}
\CTANdirectory{hfbright}{fonts/ps-type1/hfbright}[hfbright]
\CTANdirectory{hge}{fonts/hge}
\CTANdirectory{hieroglyph}{fonts/hieroglyph}
\CTANdirectory{highlight}{support/highlight}
\CTANdirectory{histyle}{macros/plain/contrib/histyle}
\CTANdirectory{hp2pl}{support/hp2pl}
\CTANdirectory{hp2xx}{support/hp2xx}
\CTANdirectory{hpgl2ps}{graphics/hpgl2ps}
\CTANdirectory{hptex}{macros/hptex}
\CTANdirectory{hptomf}{support/hptomf}
\CTANdirectory{html2latex}{support/html2latex}[html2latex]
\CTANdirectory{htmlhelp}{info/htmlhelp}
\CTANdirectory{hvfloat}{macros/latex/contrib/hvfloat}[hvfloat]
\CTANdirectory{hvmath}{fonts/micropress/hvmath}[hvmath-fonts]
\CTANdirectory{hyacc-cm}{macros/generic/hyacc-cm}
\CTANdirectory{hyper}{macros/latex/contrib/hyper}
\CTANdirectory{hyperbibtex}{biblio/bibtex/utils/hyperbibtex}
\CTANdirectory{hypernat}{macros/latex/contrib/hypernat}[hypernat]
\CTANdirectory{hyperref}{macros/latex/contrib/hyperref}[hyperref]
\CTANdirectory{hyphen-accent}{language/hyphen-accent}
\CTANdirectory{hyphenat}{macros/latex/contrib/hyphenat}[hyphenat]
\CTANdirectory{hyphenation}{language/hyphenation}
\CTANdirectory{ibygrk}{fonts/greek/ibygrk}
\CTANdirectory{iching}{fonts/iching}
\CTANdirectory{icons}{support/icons}
\CTANdirectory{ifmtarg}{macros/latex/contrib/ifmtarg}[ifmtarg]
\CTANdirectory{ifmslide}{macros/latex/contrib/ifmslide}[ifmslide]
\CTANdirectory{ifoddpage}{macros/latex/contrib/ifoddpage}[ifoddpage]
\CTANdirectory{ifxetex}{macros/generic/ifxetex}[ifxetex]
\CTANdirectory{imakeidx}{macros/latex/contrib/imakeidx}[imakeidx]
\CTANdirectory{imaketex}{support/imaketex}
\CTANdirectory{impact}{web/systems/mac/impact}
\CTANdirectory{import}{macros/latex/contrib/import}[import]
\CTANdirectory{index}{macros/latex/contrib/index}[index]
\CTANdirectory{indian}{language/indian}
\CTANdirectory{infpic}{macros/generic/infpic}
\CTANdirectory{inlinebib}{biblio/bibtex/contrib/inlinebib}
\CTANdirectory{inrsdoc}{macros/inrstex/inrsdoc}
\CTANdirectory{inrsinputs}{macros/inrstex/inrsinputs}
\CTANdirectory{inrstex}{macros/inrstex}
\CTANdirectory{isi2bibtex}{biblio/bibtex/utils/isi2bibtex}[isi2bibtex]
\CTANdirectory{iso-tex}{support/iso-tex}
\CTANdirectory{isodoc}{macros/latex/contrib/isodoc}[isodoc]
\CTANdirectory{ispell}{support/ispell}[ispell]
\CTANdirectory{ite}{support/ite}[ite]
\CTANdirectory{ivd2dvi}{dviware/ivd2dvi}
\CTANdirectory{jadetex}{macros/jadetex}
\CTANdirectory{jemtex2}{systems/msdos/jemtex2}
\CTANdirectory{jknappen-macros}{macros/latex/contrib/jknappen}
\CTANdirectory{jpeg2ps}{support/jpeg2ps}[jpeg2ps]
\CTANdirectory{jspell}{support/jspell}[jspell]
\CTANdirectory{jurabib}{macros/latex/contrib/jurabib}[jurabib]
\CTANdirectory{kamal}{support/kamal}
\CTANdirectory{kane}{dviware/kane}
\CTANdirectory{karta}{fonts/karta}
\CTANdirectory{kd}{fonts/greek/kd}
\CTANdirectory{kelem}{web/spiderweb/src/kelem}
\CTANdirectory{kelly}{fonts/greek/kelly}
\CTANdirectory{klinz}{fonts/klinz}
\CTANdirectory{knit}{web/knit}
\CTANdirectory{knot}{fonts/knot}
\CTANdirectory{knuth}{systems/knuth}
\CTANdirectory{knuth-dist}{systems/knuth/dist}[knuth-dist]
\CTANdirectory{koma-script}{macros/latex/contrib/koma-script}[koma-script]
\CTANdirectory{korean}{fonts/korean}
\CTANdirectory{kpfonts}{fonts/kpfonts}[kpfonts]
\CTANdirectory{kyocera}{dviware/kyocera}
\CTANdirectory{l2a}{support/l2a}[l2a]
\CTANdirectory{l2sl}{language/swedish/slatex/l2sl}
\CTANdirectory*{l2tabu}{info/l2tabu}
\CTANdirectory{l2x}{support/l2x}
\CTANdirectory{la}{fonts/la}
\CTANdirectory{laan}{macros/generic/laan}
\CTANdirectory{laansort}{macros/generic/laansort}
\CTANdirectory{labelcas}{macros/latex/contrib/labelcas}[labelcas]
\CTANdirectory{labels}{macros/latex/contrib/labels}
\CTANdirectory{labtex}{macros/generic/labtex}
\CTANdirectory{lacheck}{support/lacheck}[lacheck]
\CTANdirectory{lametex}{support/lametex}
\CTANdirectory{lamstex}{macros/lamstex}
\CTANdirectory{lastpage}{macros/latex/contrib/lastpage}[lastpage]
\CTANdirectory{latex}{macros/latex/base}
\CTANdirectory{latex-course}{info/latex-course}[latex-course]
\CTANdirectory{latex-essential}{info/latex-essential}
\CTANdirectory{latex2e-help-texinfo}{info/latex2e-help-texinfo}[latex2e-help-texinfo]
\CTANdirectory{latex4jed}{support/jed}[latex4jed]
\CTANdirectory*{latex-tds}{macros/latex/contrib/latex-tds}[latex-tds]
\CTANdirectory*{latex2html}{support/latex2html}[latex2html]
\CTANdirectory{latex2rtf}{support/latex2rtf}
\CTANdirectory{latexdiff}{support/latexdiff}[latexdiff]
\CTANdirectory{latexdoc}{macros/latex/doc}[latex-doc]
\CTANdirectory{latexhlp}{systems/atari/latexhlp}
\CTANdirectory{latexmake}{support/latexmake}[latexmake]
\CTANdirectory{latex-make}{support/latex-make}[latex-make]
\CTANdirectory{latex_maker}{support/latex_maker}[mk]
\CTANdirectory{latexmk}{support/latexmk}[latexmk]
\CTANdirectory{lecturer}{macros/generic/lecturer}[lecturer]
\CTANdirectory{ledmac}{macros/latex/contrib/ledmac}[ledmac]
\CTANdirectory{lettrine}{macros/latex/contrib/lettrine}[lettrine]
\CTANdirectory{levy}{fonts/greek/levy}
\CTANdirectory{lextex}{macros/plain/contrib/lextex}
\CTANdirectory{lgc}{info/examples/lgc}
\CTANdirectory{lgrind}{support/lgrind}[lgrind]
\CTANdirectory{libertine}{fonts/libertine}[libertine]
\CTANdirectory{libgreek}{macros/latex/contrib/libgreek}[libgreek]
\CTANdirectory{lilyglyphs}{macros/luatex/latex/lilyglyphs}
\CTANdirectory{lineno}{macros/latex/contrib/lineno}[lineno]
\CTANdirectory{lipsum}{macros/latex/contrib/lipsum}[lipsum]
\CTANdirectory{listbib}{macros/latex/contrib/listbib}[listbib]
\CTANdirectory{listings}{macros/latex/contrib/listings}[listings]
\CTANdirectory{lm}{fonts/lm}[lm]
\CTANdirectory{lm-math}{fonts/lm-math}[lm-math]
\CTANdirectory{lollipop}{macros/lollipop}[lollipop]
\CTANdirectory{lookbibtex}{biblio/bibtex/utils/lookbibtex}
\CTANdirectory{lpic}{macros/latex/contrib/lpic}[lpic]
\CTANdirectory{lsedit}{support/lsedit}
\CTANdirectory{lshort}{info/lshort/english}[lshort-english]
\CTANdirectory*{lshort-parent}{info/lshort}[lshort]
\CTANdirectory{ltx3pub}{info/ltx3pub}[ltx3pub]
\CTANdirectory{ltxindex}{macros/latex/contrib/ltxindex}[ltxindex]
\CTANdirectory{luatex}{systems/luatex}[luatex]
\CTANdirectory{lucida}{fonts/psfonts/bh/lucida}[lucida]
\CTANdirectory{lucida-psnfss}{macros/latex/contrib/psnfssx/lucidabr}[psnfssx-luc]
\CTANdirectory{luximono}{fonts/LuxiMono}[luximono]
\CTANdirectory{lwc}{info/examples/lwc}
\CTANdirectory{ly1}{fonts/psfonts/ly1}[ly1]
\CTANdirectory*{mactex}{systems/mac/mactex}[mactex]
\CTANdirectory{macros2e}{info/macros2e}[macros2e]
\CTANdirectory{mactotex}{graphics/mactotex}
\CTANdirectory{mailing}{macros/latex/contrib/mailing}
\CTANdirectory{make_latex}{support/make_latex}[make-latex]
\CTANdirectory{makeafm.dir}{fonts/utilities/t1tools/makeafm.dir}
\CTANdirectory{makecell}{macros/latex/contrib/makecell}[makecell]
\CTANdirectory{makedtx}{support/makedtx}[makedtx]
\CTANdirectory{makeindex}{indexing/makeindex}[makeindex]
\CTANdirectory{makeinfo}{macros/texinfo/contrib/texinfo-hu/texinfo/makeinfo}
\CTANdirectory{makeprog}{support/makeprog}\CTANdirectory{maketexwork}{info/maketexwork}
\CTANdirectory{malayalam}{language/malayalam}
\CTANdirectory{malvern}{fonts/malvern}
\CTANdirectory{mapleweb}{web/maple/mapleweb}
\CTANdirectory{marvosym-fonts}{fonts/marvosym}
\CTANdirectory{mathabx}{fonts/mathabx}[mathabx]
\CTANdirectory{mathabx-type1}{fonts/ps-type1/mathabx}[mathabx-type1]
\CTANdirectory{mathastext}{macros/latex/contrib/mathastext}[mathastext]
\CTANdirectory*{mathdesign}{fonts/mathdesign}[mathdesign]
\CTANdirectory{mathdots}{macros/generic/mathdots}
\CTANdirectory{mathematica}{macros/mathematica}
\CTANdirectory{mathpazo}{fonts/mathpazo}[mathpazo]
\CTANdirectory{mathsci2bibtex}{biblio/bibtex/utils/mathsci2bibtex}
\CTANdirectory{mathspic}{graphics/mathspic}[mathspic]
\CTANdirectory{mcite}{macros/latex/contrib/mcite}[mcite]
\CTANdirectory{mciteplus}{macros/latex/contrib/mciteplus}[mciteplus]
\CTANdirectory{mdframed}{macros/latex/contrib/mdframed}[mdframed]
\CTANdirectory{mdsymbol}{fonts/mdsymbol}[mdsymbol]
\CTANdirectory{mdwtools}{macros/latex/contrib/mdwtools}[mdwtools]
\CTANdirectory{memdesign}{info/memdesign}[memdesign]
\CTANdirectory{memoir}{macros/latex/contrib/memoir}[memoir]
\CTANdirectory{messtex}{support/messtex}
\CTANdirectory{metalogo}{macros/latex/contrib/metalogo}[metalogo]
\CTANdirectory{metapost}{graphics/metapost}
\CTANdirectory{metatype1}{fonts/utilities/metatype1}[metatype1]
\CTANdirectory{mf2ps}{fonts/utilities/mf2ps}
\CTANdirectory{mf2pt1}{support/mf2pt1}[mf2pt1]
\CTANdirectory{mf_optimized_kerning}{fonts/cm/mf_optimized_kerning}
\CTANdirectory{mfbook}{fonts/cm/utilityfonts/mfbook}
\CTANdirectory{mff-29}{fonts/utilities/mff-29}
\CTANdirectory{mffiles}{language/telugu/mffiles}
\CTANdirectory{mflogo}{macros/latex/contrib/mflogo}[mflogo]
\CTANdirectory{mfnfss}{macros/latex/contrib/mfnfss}
\CTANdirectory{mfpic}{graphics/mfpic}
\CTANdirectory{mfware}{systems/knuth/dist/mfware}
\CTANdirectory{mh}{macros/latex/contrib/mh}[mh]
\CTANdirectory{miktex}{systems/win32/miktex}[miktex]
\CTANdirectory{microtype}{macros/latex/contrib/microtype}[microtype]
\CTANdirectory{midi2tex}{support/midi2tex}
\CTANdirectory{midnight}{macros/generic/midnight}
\CTANdirectory{minionpro}{fonts/minionpro}[minionpro]
\CTANdirectory{minitoc}{macros/latex/contrib/minitoc}[minitoc]
\CTANdirectory{minted}{macros/latex/contrib/minted}[minted]
\CTANdirectory{mkjobtexmf}{support/mkjobtexmf}[mkjobtexmf]
\CTANdirectory{mma2ltx}{graphics/mma2ltx}
\CTANdirectory{mmap}{macros/latex/contrib/mmap}
\CTANdirectory{mnsymbol}{fonts/mnsymbol}[mnsymbol]
\CTANdirectory{mnu}{support/mnu}
\CTANdirectory{models}{macros/text1/models}
\CTANdirectory{moderncv}{macros/latex/contrib/moderncv}[moderncv]
\CTANdirectory{modes}{fonts/modes}
\CTANdirectory{morefloats}{macros/latex/contrib/morefloats}[morefloats]
\CTANdirectory{moreverb}{macros/latex/contrib/moreverb}[moreverb]
\CTANdirectory{morewrites}{macros/latex/contrib/morewrites}[morewrites]
\CTANdirectory{mparhack}{macros/latex/contrib/mparhack}[mparhack]
\CTANdirectory{mpgraphics}{macros/latex/contrib/mpgraphics}[mpgraphics]
\CTANdirectory{mps2eps}{support/mps2eps}
\CTANdirectory{ms}{macros/latex/contrib/ms}[ms]
\CTANdirectory{msdos}{systems/msdos}
\CTANdirectory{msx2msa}{fonts/vf-files/msx2msa}
\CTANdirectory{msym}{fonts/msym}
\CTANdirectory{mtp2lite}{fonts/mtp2lite}[mtp2lite]
\CTANdirectory{m-tx}{support/m-tx}[m-tx]
\CTANdirectory{multenum}{macros/latex/contrib/multenum}[multenum]
\CTANdirectory{multibbl}{macros/latex/contrib/multibbl}[multibbl]
\CTANdirectory{multibib}{macros/latex/contrib/multibib}[multibib]
\CTANdirectory{multido}{macros/generic/multido}[multido]
\CTANdirectory{multirow}{macros/latex/contrib/multirow}[multirow]
\CTANdirectory{musictex}{macros/musictex}[musictex]
\CTANdirectory{musixtex-egler}{obsolete/macros/musixtex/egler}
\CTANdirectory{musixtex-fonts}{fonts/musixtex-fonts}[musixtex-fonts]
\CTANdirectory{musixtex}{macros/musixtex}[musixtex]
\CTANdirectory{mutex}{macros/mtex}
\CTANdirectory{mwe}{macros/latex/contrib/mwe}[mwe]
\CTANdirectory{mxedruli}{fonts/georgian/mxedruli}
\CTANdirectory{nag}{macros/latex/contrib/nag}[nag]
\CTANdirectory{natbib}{macros/latex/contrib/natbib}[natbib]
\CTANdirectory{navigator}{macros/generic/navigator}[navigator]
\CTANdirectory{nawk}{web/spiderweb/src/nawk}
\CTANdirectory{ncctools}{macros/latex/contrib/ncctools}[ncctools]
\CTANdirectory{needspace}{macros/latex/contrib/needspace}[needspace]
\CTANdirectory{newalg}{macros/latex/contrib/newalg}[newalg]
\CTANdirectory{newcommand}{support/newcommand}[newcommand]
\CTANdirectory{newlfm}{macros/latex/contrib/newlfm}[newlfm]
\CTANdirectory{newsletr}{macros/plain/contrib/newsletr}
\CTANdirectory{newpx}{fonts/newpx}[newpx]
\CTANdirectory{newtx}{fonts/newtx}[newtx]
\CTANdirectory{newverbs}{macros/latex/contrib/newverbs}[newverbs]
\CTANdirectory{nedit-latex}{support/NEdit-LaTeX-Extensions}
\CTANdirectory{nonumonpart}{macros/latex/contrib/nonumonpart}[nonumonpart]
\CTANdirectory{nopageno}{macros/latex/contrib/nopageno}[nopageno]
\CTANdirectory{norbib}{biblio/bibtex/contrib/norbib}
\CTANdirectory{notoccite}{macros/latex/contrib/notoccite}[notoccite]
\CTANdirectory{noweb}{web/noweb}[noweb]
\CTANdirectory{ntg}{usergrps/ntg}
\CTANdirectory{ntgclass}{macros/latex/contrib/ntgclass}[ntgclass]
\CTANdirectory{ntheorem}{macros/latex/contrib/ntheorem}[ntheorem]
\CTANdirectory{nts-l}{digests/nts-l}
\CTANdirectory{nts}{systems/nts}
\CTANdirectory{numprint}{macros/latex/contrib/numprint}[numprint]
\CTANdirectory{nuweb}{web/nuweb}
\CTANdirectory{nuweb0.87b}{web/nuweb/nuweb0.87b}
\CTANdirectory{nuweb_ami}{web/nuweb/nuweb_ami}
\CTANdirectory{oberdiek}{macros/latex/contrib/oberdiek}[oberdiek]
\CTANdirectory{objectz}{macros/latex/contrib/objectz}
\CTANdirectory{ocr-a}{fonts/ocr-a}
\CTANdirectory{ocr-b}{fonts/ocr-b}
\CTANdirectory{ofs}{macros/generic/ofs}[ofs]
\CTANdirectory{ogham}{fonts/ogham}
\CTANdirectory{ogonek}{macros/latex/contrib/ogonek}
\CTANdirectory{okuda}{fonts/okuda}
\CTANdirectory{omega}{systems/omega}
\CTANdirectory{optional}{macros/latex/contrib/optional}[optional]
\CTANdirectory{os2}{systems/os2}
\CTANdirectory{osmanian}{fonts/osmanian}
\CTANdirectory{overpic}{macros/latex/contrib/overpic}[overpic]
\CTANdirectory{oztex}{systems/mac/oztex}
\CTANdirectory{page}{support/lametex/page}
\CTANdirectory{palladam}{language/tamil/palladam}
\CTANdirectory{pandora}{fonts/pandora}
\CTANdirectory{paralist}{macros/latex/contrib/paralist}[paralist]
\CTANdirectory{parallel}{macros/latex/contrib/parallel}[parallel]
\CTANdirectory{parskip}{macros/latex/contrib/parskip}[parskip]
\CTANdirectory{passivetex}{macros/xmltex/contrib/passivetex}[passivetex]
\CTANdirectory{patchcmd}{macros/latex/contrib/patchcmd}[patchcmd]
\CTANdirectory{path}{macros/generic/path}[path]
\CTANdirectory{pbox}{macros/latex/contrib/pbox}[pbox]
\CTANdirectory{pcwritex}{support/pcwritex}
\CTANdirectory{pdcmac}{macros/plain/contrib/pdcmac}[pdcmac]
\CTANdirectory{pdfcomment}{macros/latex/contrib/pdfcomment}[pdfcomment]
\CTANdirectory{pdfpages}{macros/latex/contrib/pdfpages}[pdfpages]
\CTANdirectory{pdfrack}{support/pdfrack}[pdfrack]
\CTANdirectory{pdfscreen}{macros/latex/contrib/pdfscreen}[pdfscreen]
\CTANdirectory{pdftex}{systems/pdftex}[pdftex]
\CTANdirectory{pdftex-graphics}{graphics/metapost/contrib/tools/mptopdf}[pdf-mps-supp]
\CTANdirectory{pdftricks}{graphics/pdftricks}[pdftricks]
\CTANdirectory{pdftricks2}{graphics/pdftricks2}[pdftricks2]
\CTANdirectory{pgf}{graphics/pgf/base}[pgf]
\CTANdirectory{phonetic}{fonts/phonetic}
\CTANdirectory{phy-bstyles}{biblio/bibtex/contrib/phy-bstyles}
\CTANdirectory{physe}{macros/physe}
\CTANdirectory{phyzzx}{macros/phyzzx}
\CTANdirectory{picinpar}{macros/latex209/contrib/picinpar}[picinpar]
\CTANdirectory{picins}{macros/latex209/contrib/picins}[picins]
\CTANdirectory{pict2e}{macros/latex/contrib/pict2e}[pict2e]
\CTANdirectory{pictex}{graphics/pictex}[pictex]
\CTANdirectory{pictex-addon}{graphics/pictex/addon}[pictexwd]
\CTANdirectory{pictex-converter}{support/pictex-converter}
\CTANdirectory{pictex-summary}{info/pictex/summary}[pictexsum]
\CTANdirectory{pinlabel}{macros/latex/contrib/pinlabel}[pinlabel]
\CTANdirectory{pkbbox}{fonts/utilities/pkbbox}
\CTANdirectory{pkfix}{support/pkfix}[pkfix]
\CTANdirectory{pkfix-helper}{support/pkfix-helper}[pkfix-helper]
\CTANdirectory{placeins}{macros/latex/contrib/placeins}[placeins]
\CTANdirectory{plain}{macros/plain/base}[plain]
\CTANdirectory*{plastex}{support/plastex}[plastex]
\CTANdirectory{plnfss}{macros/plain/plnfss}[plnfss]
\CTANdirectory{plttopic}{support/plttopic}
\CTANdirectory{pmtex}{systems/os2/pmtex}
\CTANdirectory{pmx}{support/pmx}
\CTANdirectory{l3experimental}{macros/latex/contrib/l3experimental}[l3experimental]
\CTANdirectory{l3kernel}{macros/latex/contrib/l3kernel}[l3kernel]
\CTANdirectory{l3packages}{macros/latex/contrib/l3packages}[l3packages]
\CTANdirectory{polish}{language/polish}
\CTANdirectory{polyglossia}{macros/latex/contrib/polyglossia}[polyglossia]
\CTANdirectory{poorman}{fonts/poorman}
\CTANdirectory{portuguese}{language/portuguese}
\CTANdirectory{poster}{macros/generic/poster}
\CTANdirectory{powerdot}{macros/latex/contrib/powerdot}[powerdot]
\CTANdirectory{pp}{support/pp}
\CTANdirectory{preprint}{macros/latex/contrib/preprint}[preprint]
\CTANdirectory{present}{macros/plain/contrib/present}[present]
\CTANdirectory{preview}{macros/latex/contrib/preview}[preview]
\CTANdirectory{print-fine}{support/print-fine}
\CTANdirectory{printbib}{biblio/bibtex/utils/printbib}
\CTANdirectory{printlen}{macros/latex/contrib/printlen}[printlen]
\CTANdirectory{printsamples}{fonts/utilities/mf2ps/doc/printsamples}
\CTANdirectory{program}{macros/latex/contrib/program}[program]
\CTANdirectory{proofs}{macros/generic/proofs}
\CTANdirectory*{protext}{systems/win32/protext}[protext]
\CTANdirectory{ppower4}{support/ppower4}[ppower4]
\CTANdirectory{ppr-prv}{macros/latex/contrib/ppr-prv}[ppr-prv]
\CTANdirectory{prosper}{macros/latex/contrib/prosper}[prosper]
\CTANdirectory{ps-type3}{fonts/cm/ps-type3}
\CTANdirectory{ps2mf}{fonts/utilities/ps2mf}
\CTANdirectory{ps2pk}{fonts/utilities/ps2pk}[ps2pk]
\CTANdirectory{psbook}{systems/msdos/dviware/psbook}
\CTANdirectory{psbox}{macros/generic/psbox}
\CTANdirectory{pseudocode}{macros/latex/contrib/pseudocode}[pseudocode]
\CTANdirectory{psfig}{graphics/psfig}[psfig]
\CTANdirectory{psfrag}{macros/latex/contrib/psfrag}[psfrag]
\CTANdirectory{psfragx}{macros/latex/contrib/psfragx}[psfragx]
\CTANdirectory{psizzl}{macros/psizzl}
\CTANdirectory{psnfss}{macros/latex/required/psnfss}[psnfss]
\CTANdirectory{psnfss-addons}{macros/latex/contrib/psnfss-addons}
\CTANdirectory{psnfssx-mathtime}{macros/latex/contrib/psnfssx/mathtime}
\CTANdirectory{pspicture}{macros/latex/contrib/pspicture}[pspicture]
\CTANdirectory{psprint}{dviware/psprint}
\CTANdirectory{pst-layout}{graphics/pstricks/contrib/pst-layout}[pst-layout]
\CTANdirectory{pst-pdf}{macros/latex/contrib/pst-pdf}[pst-pdf]
\CTANdirectory{pstoedit}{support/pstoedit}[pstoedit]
\CTANdirectory{pstricks}{graphics/pstricks}[pstricks]
\CTANdirectory{psutils}{support/psutils}
\CTANdirectory{punk}{fonts/punk}
\CTANdirectory{purifyeps}{support/purifyeps}[purifyeps]
\CTANdirectory{pxfonts}{fonts/pxfonts}[pxfonts]
\CTANdirectory{qdtexvpl}{fonts/utilities/qdtexvpl}[qdtexvpl]
\CTANdirectory{qfig}{support/qfig}
\CTANdirectory{quotchap}{macros/latex/contrib/quotchap}[quotchap]
\CTANdirectory{r2bib}{biblio/bibtex/utils/r2bib}[r2bib]
\CTANdirectory{rcs}{macros/latex/contrib/rcs}[rcs]
\CTANdirectory{rcsinfo}{macros/latex/contrib/rcsinfo}[rcsinfo]
\CTANdirectory{realcalc}{macros/generic/realcalc}
\CTANdirectory{refcheck}{macros/latex/contrib/refcheck}[refcheck]
\CTANdirectory{refer-tools}{biblio/bibtex/utils/refer-tools}
\CTANdirectory{refman}{macros/latex/contrib/refman}[refman]
\CTANdirectory{regexpatch}{macros/latex/contrib/regexpatch}[regexpatch]
\CTANdirectory{revtex4-1}{macros/latex/contrib/revtex}[revtex4-1]
\CTANdirectory{rnototex}{support/rnototex}[rnototex]
\CTANdirectory{rotating}{macros/latex/contrib/rotating}[rotating]
\CTANdirectory{rotfloat}{macros/latex/contrib/rotfloat}[rotfloat]
\CTANdirectory{rsfs}{fonts/rsfs}[rsfs]
\CTANdirectory{rsfso}{fonts/rsfso}[rsfso]
\CTANdirectory{rtf2tex}{support/rtf2tex}[rtf2tex]
\CTANdirectory{rtf2html}{support/rtf2html}
\CTANdirectory{rtf2latex}{support/rtf2latex}
\CTANdirectory{rtf2latex2e}{support/rtf2latex2e}[rtf2latex2e]
\CTANdirectory{rtflatex}{support/rtflatex}
\CTANdirectory{rtfutils}{support/tex2rtf/rtfutils}
\CTANdirectory{rumgraph}{support/rumgraph}
\CTANdirectory{sam2p}{graphics/sam2p}[sam2p]
\CTANdirectory{sansmath}{macros/latex/contrib/sansmath}[sansmath]
\CTANdirectory{sauerj}{macros/latex/contrib/sauerj}[sauerj]
\CTANdirectory{savetrees}{macros/latex/contrib/savetrees}[savetrees]
\CTANdirectory{schemeweb}{web/schemeweb}[schemeweb]
\CTANdirectory{sciposter}{macros/latex/contrib/sciposter}[sciposter]
\CTANdirectory{sectsty}{macros/latex/contrib/sectsty}[sectsty]
\CTANdirectory{selectp}{macros/latex/contrib/selectp}[selectp]
\CTANdirectory{seminar}{macros/latex/contrib/seminar}[seminar]
\CTANdirectory{shade}{macros/generic/shade}[shade]
\CTANdirectory{shorttoc}{macros/latex/contrib/shorttoc}[shorttoc]
\CTANdirectory{showexpl}{macros/latex/contrib/showexpl}[showexpl]
\CTANdirectory{showlabels}{macros/latex/contrib/showlabels}[showlabels]
\CTANdirectory{slashbox}{macros/latex/contrib/slashbox}[slashbox]
\CTANdirectory{smallcap}{macros/latex/contrib/smallcap}[smallcap]
\CTANdirectory{smartref}{macros/latex/contrib/smartref}[smartref]
\CTANdirectory{snapshot}{macros/latex/contrib/snapshot}[snapshot]
\CTANdirectory{soul}{macros/latex/contrib/soul}[soul]
\CTANdirectory{spain}{biblio/bibtex/contrib/spain}
\CTANdirectory{spelling}{macros/luatex/generic/spelling}[spelling]
\CTANdirectory{spiderweb}{web/spiderweb}[spiderweb]
\CTANdirectory{splitbib}{macros/latex/contrib/splitbib}[splitbib]
\CTANdirectory{splitindex}{macros/latex/contrib/splitindex}[splitindex]
\CTANdirectory{standalone}{macros/latex/contrib/standalone}[standalone]
\CTANdirectory{stix}{fonts/stix}[stix]
\CTANdirectory{sttools}{macros/latex/contrib/sttools}[sttools]
\CTANdirectory{sty2dtx}{support/sty2dtx}[sty2dtx]
\CTANdirectory{subdepth}{macros/latex/contrib/subdepth}[subdepth]
\CTANdirectory{subfig}{macros/latex/contrib/subfig}[subfig]
\CTANdirectory{subfiles}{macros/latex/contrib/subfiles}[subfiles]
\CTANdirectory{supertabular}{macros/latex/contrib/supertabular}[supertabular]
\CTANdirectory{svn}{macros/latex/contrib/svn}[svn]
\CTANdirectory{svninfo}{macros/latex/contrib/svninfo}[svninfo]
\CTANdirectory{swebib}{biblio/bibtex/contrib/swebib}
\CTANdirectory*{symbols}{info/symbols/comprehensive}[comprehensive]
\CTANdirectory{tablefootnote}{macros/latex/contrib/tablefootnote}[tablefootnote]
\CTANdirectory{tabls}{macros/latex/contrib/tabls}[tabls]
\CTANdirectory{tabulary}{macros/latex/contrib/tabulary}[tabulary]
\CTANdirectory{tagging}{macros/latex/contrib/tagging}
\CTANdirectory{talk}{macros/latex/contrib/talk}[talk]
\CTANdirectory{tcolorbox}{macros/latex/contrib/tcolorbox}[tcolorbox]
\CTANdirectory{tds}{tds}[tds]
\CTANdirectory{ted}{macros/latex/contrib/ted}[ted]
\CTANdirectory*{testflow}{macros/latex/contrib/IEEEtran/testflow}[testflow]
\CTANdirectory*{tetex}{obsolete/systems/unix/teTeX/current/distrib}[tetex]
\CTANdirectory{tex2mail}{support/tex2mail}[tex2mail]
\CTANdirectory{tex2rtf}{support/tex2rtf}[tex2rtf]
\CTANdirectory{texbytopic}{info/texbytopic}[texbytopic]
\CTANdirectory{texcnvfaq}{help/wp-conv}[wp-conv]
\CTANdirectory{texcount}{support/texcount}[texcount]
\CTANdirectory{texdef}{support/texdef}[texdef]
\CTANdirectory{tex-gpc}{systems/unix/tex-gpc}[tex-gpc]
\CTANdirectory{tex-gyre}{fonts/tex-gyre}[tex-gyre]
\CTANdirectory{tex-gyre-math}{fonts/tex-gyre-math}[tex-gyre-math]
\CTANdirectory{tex-overview}{info/tex-overview}[tex-overview]
\CTANdirectory*{texhax}{digests/texhax}[texhax]
\CTANdirectory{texi2html}{support/texi2html}[texi2html]
\CTANdirectory{texindex}{indexing/texindex}[texindex]
\CTANdirectory{texinfo}{macros/texinfo/texinfo}[texinfo]
\CTANdirectory*{texlive}{systems/texlive}[texlive]
\CTANdirectory*{texmacs}{support/TeXmacs}[texmacs]
\CTANdirectory*{texniccenter}{systems/win32/TeXnicCenter}[texniccenter]
\CTANdirectory{texpower}{macros/latex/contrib/texpower}[texpower]
\CTANdirectory{texshell}{systems/msdos/texshell}[texshell]
\CTANdirectory{texsis}{macros/texsis}[texsis]
\CTANdirectory{textcase}{macros/latex/contrib/textcase}[textcase]
\CTANdirectory{textfit}{macros/latex/contrib/textfit}[textfit]
\CTANdirectory{textmerg}{macros/latex/contrib/textmerg}[textmerg]
\CTANdirectory{textpos}{macros/latex/contrib/textpos}[textpos]
\CTANdirectory{textures_figs}{systems/mac/textures_figs}
\CTANdirectory{texutils}{systems/atari/texutils}
\CTANdirectory{tgrind}{support/tgrind}[tgrind]
\CTANdirectory{threeparttable}{macros/latex/contrib/threeparttable}[threeparttable]
\CTANdirectory{threeparttablex}{macros/latex/contrib/threeparttablex}[threeparttablex]
\CTANdirectory{tib}{biblio/tib}[tib]
\CTANdirectory{tiny_c2l}{support/tiny_c2l}[tinyc2l]
\CTANdirectory{tip}{info/examples/tip}
\CTANdirectory{titleref}{macros/latex/contrib/titleref}[titleref]
\CTANdirectory{titlesec}{macros/latex/contrib/titlesec}[titlesec]
\CTANdirectory{titling}{macros/latex/contrib/titling}[titling]
\CTANdirectory{tlc2}{info/examples/tlc2}
\CTANdirectory{tmmath}{fonts/micropress/tmmath}[tmmath]
\CTANdirectory{tocbibind}{macros/latex/contrib/tocbibind}[tocbibind]
\CTANdirectory{tocloft}{macros/latex/contrib/tocloft}[tocloft]
\CTANdirectory{tocvsec2}{macros/latex/contrib/tocvsec2}[tocvsec2]
\CTANdirectory{totpages}{macros/latex/contrib/totpages}[totpages]
\CTANdirectory{tr2latex}{support/tr2latex}[tr2latex]
\CTANdirectory{transfig}{graphics/transfig}[transfig]
\CTANdirectory{try}{support/try}[try]
\CTANdirectory{tt2001}{fonts/ps-type1/tt2001}[tt2001]
\CTANdirectory{tth}{support/tth/dist}[tth]
\CTANdirectory{ttn}{digests/ttn}
\CTANdirectory{tug}{usergrps/tug}
\CTANdirectory{tugboat}{digests/tugboat}
\CTANdirectory{tweb}{web/tweb}[tweb]
\CTANdirectory{txfonts}{fonts/txfonts}[txfonts]
\CTANdirectory{txfontsb}{fonts/txfontsb}[txfontsb]
\CTANdirectory{txtdist}{support/txt}[txt]
\CTANdirectory{type1cm}{macros/latex/contrib/type1cm}[type1cm]
\CTANdirectory{ucharclasses}{macros/xetex/latex/ucharclasses}[ucharclasses]
\CTANdirectory{ucs}{macros/latex/contrib/ucs}[ucs]
\CTANdirectory{ucthesis}{macros/latex/contrib/ucthesis}[ucthesis]
\CTANdirectory{uktex}{digests/uktex}
\CTANdirectory{ulem}{macros/latex/contrib/ulem}[ulem]
\CTANdirectory{umrand}{macros/generic/umrand}
\CTANdirectory{underscore}{macros/latex/contrib/underscore}[underscore]
\CTANdirectory{unicode-math}{macros/latex/contrib/unicode-math}[unicode-math]
\CTANdirectory{unix}{systems/unix}
\CTANdirectory{unpacked}{macros/latex/unpacked}
\CTANdirectory{untex}{support/untex}[untex]
\CTANdirectory{url}{macros/latex/contrib/url}[url]
\CTANdirectory{urlbst}{biblio/bibtex/contrib/urlbst}[urlbst]
\CTANdirectory{urw-base35}{fonts/urw/base35}[urw-base35]
\CTANdirectory{urwchancal}{fonts/urwchancal}[urwchancal]
\CTANdirectory{usebib}{macros/latex/contrib/usebib}[usebib]
\CTANdirectory{utopia}{fonts/utopia}[utopia]
\CTANdirectory{varisize}{macros/plain/contrib/varisize}[varisize]
\CTANdirectory{varwidth}{macros/latex/contrib/varwidth}[varwidth]
\CTANdirectory{vc}{support/vc}[vc]
\CTANdirectory{verbatim}{macros/latex/required/tools}[verbatim]
\CTANdirectory{verbatimbox}{macros/latex/contrib/verbatimbox}[verbatimbox]
\CTANdirectory{verbdef}{macros/latex/contrib/verbdef}[verbdef]
\CTANdirectory{version}{macros/latex/contrib/version}[version]
\CTANdirectory{vertbars}{macros/latex/contrib/vertbars}[vertbars]
\CTANdirectory{vita}{macros/latex/contrib/vita}[vita]
\CTANdirectory{vmargin}{macros/latex/contrib/vmargin}[vmargin]
\CTANdirectory{vmspell}{support/vmspell}[vmspell]
\CTANdirectory{vpp}{support/view_print_ps_pdf}[vpp]
\CTANdirectory{vruler}{macros/latex/contrib/vruler}[vruler]
\CTANdirectory{vtex-common}{systems/vtex/common}
\CTANdirectory{vtex-linux}{systems/vtex/linux}[vtex-free]
\CTANdirectory{vtex-os2}{systems/vtex/os2}[vtex-free]
\CTANdirectory{wallpaper}{macros/latex/contrib/wallpaper}[wallpaper]
\CTANdirectory{was}{macros/latex/contrib/was}[was]
\CTANdirectory{wd2latex}{support/wd2latex}
\CTANdirectory{web}{systems/knuth/dist/web}[web]
\CTANdirectory*{winedt}{systems/win32/winedt}[winedt]
\CTANdirectory{wordcount}{macros/latex/contrib/wordcount}[wordcount]
\CTANdirectory{wp2latex}{support/wp2latex}[wp2latex]
\CTANdirectory{wrapfig}{macros/latex/contrib/wrapfig}[wrapfig]
\CTANdirectory{xargs}{macros/latex/contrib/xargs}[xargs]
\CTANdirectory{xbibfile}{biblio/bibtex/utils/xbibfile}[xbibfile]
\CTANdirectory{xcolor}{macros/latex/contrib/xcolor}[xcolor]
\CTANdirectory{xcomment}{macros/generic/xcomment}[xcomment]
\CTANdirectory*{xdvi}{dviware/xdvi}[xdvi]
\CTANdirectory{xecjk}{macros/xetex/latex/xecjk}[xecjk]
\CTANdirectory{xetexref}{info/xetexref}[xetexref]
\CTANdirectory*{xfig}{graphics/xfig}[xfig]
\CTANdirectory*{xindy}{indexing/xindy}[xindy]
\CTANdirectory{xits}{fonts/xits}[xits]
\CTANdirectory{xkeyval}{macros/latex/contrib/xkeyval}[xkeyval]
\CTANdirectory{xmltex}{macros/xmltex/base}[xmltex]
\CTANdirectory*{xpdf}{support/xpdf}[xpdf]
\CTANdirectory{xtab}{macros/latex/contrib/xtab}[xtab]
\CTANdirectory{xwatermark}{macros/latex/contrib/xwatermark}[xwatermark]
\CTANdirectory{yagusylo}{macros/latex/contrib/yagusylo}[yagusylo]
\CTANdirectory{yhmath}{fonts/yhmath}[yhmath]
\CTANdirectory{zefonts}{fonts/zefonts}[zefonts]
\CTANdirectory{ziffer}{macros/latex/contrib/ziffer}[ziffer]
\CTANdirectory{zoon-mp-eg}{info/metapost/examples}[metapost-examples]
\CTANdirectory{zwpagelayout}{macros/latex/contrib/zwpagelayout}[zwpagelayout]
\endinput

%
% ... files
% $Id: filectan.tex,v 1.143 2012/12/07 19:34:33 rf10 Exp rf10 $
%
% protect ourself against being read twice
\csname readCTANfiles\endcsname
\let\readCTANfiles\endinput
%
% interesting/useful individual files to be found on CTAN
\CTANfile{CTAN-sites}{CTAN.sites}
\CTANfile{CTAN-uploads}{README.uploads}% yes, it really is in the root
\CTANfile{Excalibur}{systems/mac/support/excalibur/Excalibur-4.0.2.sit.hqx}[excalibur]
\CTANfile{expl3-doc}{macros/latex/contrib/l3kernel/expl3.pdf}[l3kernel]
\CTANfile{f-byname}{FILES.byname}
\CTANfile{f-last7}{FILES.last07days}
\CTANfile{interface3-doc}{macros/latex/contrib/l3kernel/interface3.pdf}[l3kernel]
\CTANfile{LitProg-FAQ}{help/comp.programming.literate_FAQ}
\CTANfile{OpenVMSTeX}{systems/OpenVMS/TEX97_CTAN.ZIP}
\CTANfile{T1instguide}{info/Type1fonts/fontinstallationguide/fontinstallationguide.pdf}
\CTANfile{TeX-FAQ}{obsolete/help/TeX,_LaTeX,_etc.:_Frequently_Asked_Questions_with_Answers}
\CTANfile{abstract-bst}{biblio/bibtex/utils/bibtools/abstract.bst}
\CTANfile{backgrnd}{macros/generic/misc/backgrnd.tex}[backgrnd]
\CTANfile{bbl2html}{biblio/bibtex/utils/misc/bbl2html.awk}[bbl2html]
\CTANfile{beginlatex-pdf}{info/beginlatex/beginlatex-3.6.pdf}[beginlatex]
\CTANfile{bibtex-faq}{biblio/bibtex/contrib/doc/btxFAQ.pdf}
\CTANfile{bidstobibtex}{biblio/bibtex/utils/bids/bids.to.bibtex}[bidstobibtex]
\CTANfile{btxmactex}{macros/eplain/tex/btxmac.tex}[eplain]
\CTANfile{catalogue}{help/Catalogue/catalogue.html}
\CTANfile{cat-licences}{help/Catalogue/licenses.html}
\CTANfile{clsguide}{macros/latex/doc/clsguide.pdf}[clsguide]
\CTANfile{compactbib}{macros/latex/contrib/compactbib/compactbib.sty}[compactbib]
%\CTANfile{compan-ctan}{info/companion.ctan}
\CTANfile{context-tmf}{macros/context/current/cont-tmf.zip}[context]
\CTANfile{dvitype}{systems/knuth/dist/texware/dvitype.web}[dvitype]
\CTANfile{edmetrics}{systems/mac/textures/utilities/EdMetrics.sea.hqx}[edmetrics]
\CTANfile{epsf}{macros/generic/epsf/epsf.tex}[epsf]
\CTANfile{figsinlatex}{obsolete/info/figsinltx.ps}
\CTANfile{finplain}{biblio/bibtex/contrib/misc/finplain.bst}
\CTANfile{fix-cm}{macros/latex/unpacked/fix-cm.sty}[fix-cm]
\CTANfile{fntguide.pdf}{macros/latex/doc/fntguide.pdf}[fntguide]
\CTANfile{fontdef}{macros/latex/base/fontdef.dtx}
\CTANfile{fontmath}{macros/latex/unpacked/fontmath.ltx}
\CTANfile{gentle}{info/gentle/gentle.pdf}[gentle]
\CTANfile{gkpmac}{systems/knuth/local/lib/gkpmac.tex}[gkpmac]
\CTANfile{knuth-letter}{systems/knuth/local/lib/letter.tex}
\CTANfile{knuth-tds}{macros/latex/contrib/latex-tds/knuth.tds.zip}
\CTANfile{latex209-base}{obsolete/macros/latex209/distribs/latex209.tar.gz}[latex209]
\CTANfile{latex-classes}{macros/latex/base/classes.dtx}
\CTANfile{latex-source}{macros/latex/base/source2e.tex}
\CTANfile{latexcount}{support/latexcount/latexcount.pl}[latexcount]
\CTANfile{latexcheat}{info/latexcheat/latexcheat/latexsheet.pdf}[latexcheat]
\CTANfile{letterspacing}{macros/generic/misc/letterspacing.tex}[letterspacing]
\CTANfile{ltablex}{macros/latex/contrib/ltablex/ltablex.sty}[ltablex]
\CTANfile{ltxguide}{macros/latex/base/ltxguide.cls}
\CTANfile{ltxtable}{macros/latex/contrib/carlisle/ltxtable.tex}[ltxtable]
\CTANfile{lw35nfss-zip}{macros/latex/required/psnfss/lw35nfss.zip}[lw35nfss]
\CTANfile{macmakeindex}{systems/mac/macmakeindex2.12.sea.hqx}
\CTANfile{mathscript}{info/symbols/math/scriptfonts.pdf}
\CTANfile{mathsurvey.html}{info/Free_Math_Font_Survey/en/survey.html}
\CTANfile{mathsurvey.pdf}{info/Free_Math_Font_Survey/en/survey.pdf}
\CTANfile{memoir-man}{macros/latex/contrib/memoir/memman.pdf}
\CTANfile{metafp-pdf}{info/metafont/metafp/metafp.pdf}[metafp]
\CTANfile{mf-beginners}{info/metafont/beginners/metafont-for-beginners.pdf}[metafont-beginners]
\CTANfile{mf-list}{info/metafont-list}
\CTANfile{miktex-portable}{systems/win32/miktex/setup/miktex-portable.exe}
\CTANfile{miktex-setup}{systems/win32/miktex/setup/setup.exe}[miktex]
\CTANfile{mil}{info/mil/mil.pdf}
\CTANfile{mil-short}{info/Math_into_LaTeX-4/Short_Course.pdf}[math-into-latex-4]
\CTANfile{modes-file}{fonts/modes/modes.mf}[modes]
\CTANfile{mtw}{info/makingtexwork/mtw-1.0.1-html.tar.gz}
\CTANfile{multind}{macros/latex209/contrib/misc/multind.sty}[multind]
\CTANfile{nextpage}{macros/latex/contrib/misc/nextpage.sty}[nextpage]
\CTANfile{noTeX}{biblio/bibtex/utils/misc/noTeX.bst}[notex]
\CTANfile{numline}{obsolete/macros/latex/contrib/numline/numline.sty}[numline]
\CTANfile{patch}{macros/generic/misc/patch.doc}[patch]
\CTANfile{picins-summary}{macros/latex209/contrib/picins/picins.txt}
\CTANfile{pk300}{fonts/cm/pk/pk300.zip}
\CTANfile{pk300w}{fonts/cm/pk/pk300w.zip}
\CTANfile{QED}{macros/generic/proofs/taylor/QED.sty}[qed]
\CTANfile{removefr}{macros/latex/contrib/fragments/removefr.tex}[removefr]
\CTANfile{repeat}{macros/generic/eijkhout/repeat.tex}[repeat]
\CTANfile{resume}{obsolete/macros/latex209/contrib/resume/resume.sty}
\CTANfile{savesym}{macros/latex/contrib/savesym/savesym.sty}[savesym]
\CTANfile{setspace}{macros/latex/contrib/setspace/setspace.sty}[setspace]
\CTANfile{simpl-latex}{info/simplified-latex/simplified-intro.pdf}[simplified-latex]
\CTANfile{sober}{macros/latex209/contrib/misc/sober.sty}[sober]
\CTANfile{tex2bib}{biblio/bibtex/utils/tex2bib/tex2bib}[tex2bib]
\CTANfile{tex2bib-doc}{biblio/bibtex/utils/tex2bib/README}
\CTANfile{tex4ht}{obsolete/support/TeX4ht/tex4ht-all.zip}[tex4ht]
\CTANfile{texlive-unix}{systems/texlive/tlnet/install-tl-unx.tar.gz}
\CTANfile{texlive-windows}{systems/texlive/tlnet/install-tl.zip}
\CTANfile{texnames}{info/biblio/texnames.sty}
\CTANfile{texsis-index}{macros/texsis/index/index.tex}
\CTANfile{topcapt}{macros/latex/contrib/misc/topcapt.sty}[topcapt]
\CTANfile{tracking}{macros/latex/contrib/tracking/tracking.sty}[tracking]
\CTANfile{ttb-pdf}{info/bibtex/tamethebeast/ttb_en.pdf}[tamethebeast]
\CTANfile{type1ec}{fonts/ps-type1/cm-super/type1ec.sty}[type1ec]
\CTANfile{ukhyph}{language/hyphenation/ukhyphen.tex}
\CTANfile{upquote}{macros/latex/contrib/upquote/upquote.sty}[upquote]
\CTANfile{faq-a4}{help/uk-tex-faq/newfaq.pdf}
\CTANfile{faq-letter}{help/uk-tex-faq/letterfaq.pdf}
\CTANfile{versions}{macros/latex/contrib/versions/versions.sty}[versions]
\CTANfile{vf-howto}{info/virtualfontshowto/virtualfontshowto.txt}[vf-howto]
\CTANfile{vf-knuth}{info/knuth/virtual-fonts}[vf-knuth]
\CTANfile{visualFAQ}{info/visualFAQ/visualFAQ.pdf}[visualfaq]
\CTANfile{voss-mathmode}{info/math/voss/mathmode/Mathmode.pdf}
\CTANfile{wujastyk-txh}{digests/texhax/txh/wujastyk.txh}
\CTANfile{xampl-bib}{biblio/bibtex/base/xampl.bib}
\CTANfile{xtexcad}{graphics/xtexcad/xtexcad-2.4.1.tar.gz}


% facilitate auto-processing of this stuff; this line is clunkily
% detected (and ignored) in the build-faqbody script
%\let\faqinput\input
\def\faqinput#1{%\message{*** inputting #1; group level \the\currentgrouplevel}
  \input{#1}%
  %\message{*** out of #1; group level \the\currentgrouplevel}
}

% the stuff to print
\faqinput{faq-intro}               % introduction
\faqinput{faq-backgrnd}            % background
\faqinput{faq-docs}                % docs
\faqinput{faq-bits+pieces}         % bits and pieces
\faqinput{faq-getit}               % getting software
\faqinput{faq-texsys}              % TeX systems
\faqinput{faq-dvi}                 % DVI drivers and previewers
\faqinput{faq-support}             % support stuff
\faqinput{faq-litprog}             % programming for literates
\faqinput{faq-fmt-conv}            % format conversions
\faqinput{faq-install}             % installation and so on
\faqinput{faq-fonts}               % fonts and what to do
\faqinput{faq-hyp+pdf}             % hyper-foodle and pdf
\faqinput{faq-graphics}            % graphics
\faqinput{faq-biblio}              % bib stuff
\faqinput{faq-adj-types}           % adjusting typesetting
\faqinput{faq-lab-ref}             % labels and references
\faqinput{faq-how-do-i}            % how to do things
\faqinput{faq-symbols}             % symbols, their natural history and use
\faqinput{faq-mac-prog}            % macro programming
\faqinput{faq-t-g-wr}              % things going wrong
\faqinput{faq-wdidt}               % why does it do that
%*****************************************quote environments up to here
\faqinput{faq-jot-err}             % joy (hem hem) of tex errors
\faqinput{faq-projects}            % current projects
%
% This is the last section, and is to remain the last section...
\faqinput{faq-the-end}             % wrapping it all up

% \end{document} is in calling file (e.g., newfaq.tex)

  \end{multicols}
\fi

\typeout{*** That makes \thequestion\space questions ***}
\end{document}
