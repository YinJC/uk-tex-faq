% $Id: faq-support.tex,v 1.9 2014/01/28 18:17:36 rf10 Exp rf10 $

\section{Support Packages for \TeX{}}

\Question[Q-xfigetc]{\AllTeX{}-friendly drawing packages}
\AliasQuestion{Q-xfig}
\AliasQuestion{Q-figetc}

\ProgName{(X)Fig} is a menu driven tool that allows you to
draw objects on the screen of an \acro{X} workstation; \ProgName{transfig}
is a set of tools which translate the code \ProgName{fig}.  The list
of export formats is very long, and includes \MF{} and \MP{},
Encapsulated \PS{} and \acro{PDF}, as well as combinations that wrap
a graphics format in a \LaTeX{} import file, which may include
\LaTeX{} commands to place text (compiled by \LaTeX{} itself) as
labels, etc., in the figures.

There's no explicit port of \ProgName{xfig} to windows (although it is
believed to work under \ProgName{cygwin} with its X-windows system).
However, the program % ! line break
\href{http://tams-www.informatik.uni-hamburg.de/applets/jfig/}{\ProgName{jfig}}
is thought by many to be an acceptable substitute, written in Java.

\ProgName{Asymptote} is a widely-praised development of the \MP{}
language, which can draw 2D or 3D diagrams, and can also label
diagrams with \LaTeX{} text; copious documentation is available via
\href{http://asymptote.sourceforge.net}{\ProgName{asymptote}'s web site}.
\begin{ctanrefs}
\item[asymptote]\CTANref{asymptote}
\item[xfig]\CTANref{xfig}
\item[transfig]\CTANref{transfig}
\end{ctanrefs}
\LastEdit{2011-05-02}

\Question[Q-texcad]{\TeX{}\acro{CAD}, a drawing package for \LaTeX{}}

\TeX{}\acro{CAD} is a program for the \acro{PC} which enables the user to draw diagrams
on screen using a mouse or arrow keys, with an on-screen menu of available 
picture-elements. Its output is code for the \LaTeX{}
\environment{picture} environment. 
Optionally, it can be set to include lines at all angles using 
the em\TeX{} driver-family
\htmlignore
(\Qref*{\csx{special}s}{Q-specials}).
\endhtmlignore
\begin{htmlversion}
\Qref{\csx{special}s}{Q-specials}.
\end{htmlversion}
\TeX{}\acro{CAD} is part of the em\TeX{} distribution.

A Unix port of the program (\ProgName{xtexcad}) has been made.
\begin{ctanrefs}
\item[emtex]\CTANref{emtex}
\item[xtexcad]\CTANref{xtexcad}
\end{ctanrefs}

\Question[Q-spell]{Spelling checkers for work with \TeX{}}

`Traditional' approaches to the problem (of checking your spelling) were
designed to work with a plain text file; in our case, we have an \alltex{}
source.  For the user, this is a simple-to-understand way to do the
job; but for the spell-checker programmer, it requires heuristic (and
hence fallible) analysis of \alltex{} macros and so on.  The
alternative, of viewing the text \emph{after} \alltex{} has processed
the results, is covered below.

The user of an \Qref*{shell/editor}{Q-editors} will usually find it
embeds a spelling checker.  For command-line use, there are several
choices, depending on the system you're using.

For Unix, \ProgName{ispell} was long the program of choice; it is well
integrated with \ProgName{emacs}, and deals with some \TeX{} syntax.
However, it has more-or-less been replaced everywhere, by
\ProgName{aspell}, which was designed as a successor, and certainly
performs better on most metrics; there remains some question as to its
performance with \AllTeX{} sources.  The most recent offering (which
is widely used in other open-source software projects) is
\begin{hyperversion}
  \href{http://hunspell.sourceforge.net/}{\ProgName{Hunspell}}.
\end{hyperversion}
\begin{flatversion}
  Hunspell~(see \URL{http://hunspell.sourceforge.net/}).
\end{flatversion}
\ProgName{Hunspell} is available for other architectures, too; a web
search shows versions available for Windows, at least.

For the Macintosh, \ProgName{Excalibur} has long been used; its
distribution comes with dictionaries for several languages.
\ProgName{Hunspell} (see above) is actually part of OS X from version 10.6.

The \acro{VMS} Pascal program \ProgName{spell} makes special cases of
some important features of \LaTeX{} syntax.

For \MSDOS{}, there are several programs.  \ProgName{Amspell} can be
called from within an editor, and \ProgName{jspell} is an extended
version of \ProgName{ispell}.

An alternative approach takes \alltex{} output, and checks that.  A
straightforward approach is to produce \acro{PDF} output, and process
it with \ProgName{pdftotext}, using any plain text checker on the
result (the checkers listed above all work in this r\^ole).  For this
to work reasonably well, the user should disable hyphenation before
making the \acro{PDF} output.

The (experimental) \LuaTeX{}/\LaTeX{} package \Package{spelling} goes
one step further: it uses \ProgName{lua} code to extract words
\emph{while typesetting is going on}, but before hyphenation is
applied.  Each word is looked up in a list of known bad spellings, and
the word highlighted if it appears there.  In parallel, a text file is
created, which can be processed by a `normal' spelling checker to
produce a revised ``bad spelling'' list.  (The package documentation
shows the end result; it includes words such as `spellling', which are
duly highlighted.)
\begin{ctanrefs}
\item[4spell]\CTANref{4spell}
\item[amspell]\CTANref{amspell}
\item[aspell]Browse \CTANref{aspell}~--- choose just those language
  dictionaries (under subdirectory \File{dict/}) that you need.
\item[excalibur]\CTANref{Excalibur}
\item[ispell]\CTANref{ispell}
\item[jspell]\CTANref{jspell}
\item[spelling.sty]\CTANref{spelling}
\item[\nothtml{\bgroup\rmfamily}\acro{VMS}\nothtml{\egroup} spell]%
  \CTANref{vmspell}
\item[winedt]\CTANref{winedt}
\end{ctanrefs}
\LastEdit{2013-10-02}

\Question[Q-wordcount]{How many words have you written?}

One often has to submit a document (e.g., a paper or a dissertation)
under some sort of constraint about its size.  Sensible people set a
constraint in terms of numbers of pages, but there are some that
persist in limiting the numbers of words you type.

A simple solution to the requirement can be achieved following a
simple observation: the powers that be are unlikely to count all the
words of a document submitted to them.  Therefore, a statistical
method can be employed: find how many words there are on a full page;
find how many full pages there are in the document (allowing for
displays of various sorts, this number will probably not be an
integer); multiply the two.  However, if the document to be submitted
is to determine the success of the rest of one's life, it takes a
brave person to thumb their nose at authority quite so
comprehensively\dots{}

The simplest method is to strip out the \AllTeX{} markup, and to count
what's left.  On a  Unix-like system, this may be done using
\ProgName{detex} and the built-in \ProgName{wc}:
\begin{quote}
\begin{verbatim}
detex <filename> | wc -w
\end{verbatim}
\end{quote}
The technique is beguilingly simple, but it's not terribly accurate

The \ProgName{latexcount} script does the same sort of job, in one
``step''; being a \ProgName{perl} script, it is in principle rather
easily configured (see documentation inside the script).
Several \Qref*{editors and shells}{Q-editors} offer something similar.

\ProgName{TeXcount} goes a long way with heuristics for counting,
starting from a \LaTeX{} file; the documentation is comprehensive, and
you may try the script on-line via the % ! line break
\begin{hyperversion}
  \href{http://folk.uio.no/einarro/Comp/texwordcount.html}{package home page}.
\end{hyperversion}
\begin{flatversion}
  package home page~(see
  \URL{http://folk.uio.no/einarro/Comp/texwordcount.html}).
\end{flatversion}

However, even quite sophisticated stripping of \AllTeX{} markup can
never be entirely reliable: markup itself may contribute typeset
words, or even consume words that appear in the text.

The \Package{wordcount} package
contains a Bourne shell (i.e., typically Unix) script for running a
\LaTeX{} file with a special piece of supporting \TeX{} code, and then
counting word indications in the log file.  This is probably as
accurate automatic counting as you can get, if it works for you.
\begin{ctanrefs}
\item[detex]\CTANref{detex}
\item[latexcount.pl]\CTANref{latexcount}
\item[TeXcount]\CTANref{texcount}
\item[wordcount]\CTANref{wordcount}
\end{ctanrefs}

