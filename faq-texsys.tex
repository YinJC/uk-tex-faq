% $Id: faq-texsys.tex,v 1.17 2013/07/24 21:50:56 rf10 Exp $

\section{\TeX{} Systems}

\Question[Q-TeXsystems]{\AllTeX{} for different machines}

We list here the free or shareware packages;
\htmlignore
a later \Qref{question}{Q-commercial} discusses commercial \TeX{} products.
\endhtmlignore
\begin{htmlversion}
  another question addresses
  \Qref{commercial \TeX{} vendors'}{Q-commercial} products.
\end{htmlversion}

The list is provided in four answers:
\begin{itemize}
\item \TeX{} systems for use with
  \Qref*{Unix and \acro{GNU} Linux systems}{Q-sysunix}
\item \TeX{} systems for use with % ! line break
  \Qref*{Modern Windows systems}{Q-syswin32}
\item \TeX{} systems for use with % ! line break
  \Qref*{Macintosh systems}{Q-sysmac}
\item \TeX{} systems for % ! line break
  \Qref*{Other sorts of systems}{Q-sysother}
\end{itemize}

\Question[Q-sysunix]{Unix and \acro{GNU} Linux systems}

\begin{quote}
  Note that \macosx{}, though it is also a Unix-based system, has
  different options; users should refer to the information in
  \Qref[question]{Mac systems}{Q-sysmac}.
\end{quote}

The \TeX{} distribution of choice, for Unix systems (including
\acro{GNU}/Linux and most other free Unix-like systems) is \texlive{},
which is distributed as part of the \Qref*{\TeX{} collection}{Q-CD}.

\texlive{} may also be installed ``over the network''; a network
installer is provided, and once you have a system (whether installed
from the network or installed off-line from a disc) a manager
(\ProgName{tlmgr}) can both keep your installation up-to-date and add
packages you didn't install at first.

\texlive{} may be run with no installation at all; the web page % ! line break
\href{http://www.tug.org/texlive/portable.html}{\texlive{} portable usage}
describes the options for installing \texlive{} on a memory stick for
use on another computer, or for using the \texlive{} \acro{DVD} with
no installation at all.

\TeX{}-gpc is a ``back-to-basics'' distribution of \TeX{} utilities,
\emph{only} (unlike \texlive{}, no `tailored' package bundles are
provided).  It is distributed as source, and compiles with \acro{GNU}
Pascal, thereby coming as close as you're likely to get to Knuth's original
distribution.  It is known to work well, but the omission of \eTeX{}
and \pdftex{} will rule it out of many users' choices.
\begin{ctanrefs}
\item[tex-gpc]\CTANref{tex-gpc}
\item[texlive]Browse \CTANref{texlive}
\item[texlive installer (Unix)]\CTANref{texlive-unix}
\end{ctanrefs}
\LastEdit{2013-07-18}

\Question[Q-syswin32]{(Modern) Windows systems}

Windows users nowadays have a real choice, between two excellent
distributions, \miktex{} and \texlive{}.  \texlive{} on windows has
only in recent years been a real challenger to the long-established
\miktex{}, and even now \miktex{} has features that \texlive{} lacks.
Both are comprehensive
distributions, offering all the established \TeX{} variants (\TeX{},
\pdftex{}~--- both with \eTeX{} variants~--- as well as \xetex{} and
\LuaTeX{}), together with a wide range of support tools.
  
Both \miktex{} and \texlive{} offer management tools, including the
means of keeping an installation ``up-to-date'', by reinstalling
packages that have been updated on \acro{CTAN} (the delay between a
package update appearing, and it being available to the distribution
users) can be as short as a day (and is never very long).

\miktex{}, by Christian Schenk, is the longer-established of the pair,
and has a large audience of satisfied users; \texlive{} is the
dominant distribution in use in the world of Unix-like systems, and so
its Windows version may be expected to appeal to those who use both
Unix-like and Windows systems.  The latest release of \miktex{}~---
version 2.9~--- requires Windows~XP, or later (so it does not work on
Windows~2000 or earlier).

Both distributions may be used in a configuration which involves no
installation at all.  \miktex{}'s ``portable'' distribution may be
unpacked on a memory stick, and used on any windows computer without
making any direct use of the hard drive.  The web page % ! line break
\href{http://www.tug.org/texlive/portable.html}{\texlive{} portable usage}
describes the options for installing \texlive{} on a memory stick, or
for using the \texlive{} \acro{DVD} with no installation at all.
  
Both \miktex{} and \texlive{} may be downloaded and installed, package
by package, over the net.  This is a mammoth undertaking, only to be
undertaken by those with a good network connection (and a patient
disposition!).

A ready-to-run copy of the \miktex{} distribution,
on \acro{DVD} may be bought via the % ! line break
\href{http://www.miktex.org/cd/}{\miktex{} web site}.  \miktex{} may
also be installed using Pro\TeX{}t, on the % ! line break
\Qref*{\TeX{} Collection \acro{DVD}}{Q-CD}.

The \Qref*{\TeX{} Collection \acro{DVD}}{Q-CD} also provides an
offline installer for \texlive{}.

%  can't find any evidence this is still available
%
%% \href{http://foundry.supelec.fr/projects/xemtex/}{XEm\TeX{}}, by
%% Fabrice Popineau (he who created the excellent, but now defunct,
%% fp\TeX{} distribution), is another integrated distribution of \TeX{},
%% \LaTeX{}, \CONTeXT{}, \ProgName{XEmacs} and other friends for Windows.
%% All programs have been compiled natively to take the best advantage of
%% the Windows environment.  Configuration is provided so that the
%% resulting set of programs runs out-of-the-box, but the distribution is
%% not actively promoted.
%
% this isn't right, but i'm not sure how to deal with it...
%
%%   The (Japanese) \acro{W}32\acro{TEX} distribution was motivated by
%%   the needs of Japanese users (Japanese won't fit in a ``simple''
%%   character set like \acro{ASCII}, but \TeX{} is based on a version of
%%   \acro{ASCII}).  Despite its origins, \acro{W}32\acro{TEX} is said to
%%   be a good bet for Western users, notably those whose disks are short
%%   of space: the minimum documented download is as small as
%%   95\,MBytes.  Investigate the distribution at
%%   \URL{http://www.fsci.fuk.kindai.ac.jp/kakuto/win32-ptex/web2c75-e.html}
%
% the url is now http://w32tex.org/current/
  
A further (free) option is available thanks to the
\href{http://www.cygwin.com}{CygWin bundle}, which presents a
Unix-like environment in Windows systems (and also provides an
X-windows server).  The (now obsolete) te\TeX{} distribution is
provided as part of the CygWin distribution, but there is a CygWin
build of \texlive{} so you can have a current \TeX{} system.  \TeX{}
under CygWin is reputedly somewhat slower than native Win32
implementations such as \miktex{}, and of course the \TeX{}
applications behave like Unix-system applications.

BaKoMa \TeX{}, by Basil Malyshev, is a comprehensive (shareware)
distribution, which focuses on support of Acrobat.  The distribution
comes with a bunch of Type~1 fonts packaged to work with BaKoMa
\TeX{}, which further the focus.
\begin{ctanrefs}
\item[bakoma]\CTANref{bakoma-tex}
\item[miktex]\CTANref{miktex}; % ! line break so only one \CTANref on line
  acquire \CTANref{miktex-setup} (also available from the \miktex{}
  web site), and read installation instructions from % ! line break
  \href{http://www.miktex.org/2.9/setup}{the \miktex{} installation page}
\item[\nothtml{\bgroup\rmfamily}Portable\nothtml{\egroup} miktex]\CTANref{miktex-portable} 
\item[protext.exe]\CTANref{protext}
\item[texlive]Browse \CTANref{texlive}
\item[texlive installer (Windows)]\CTANref{texlive-windows}
\end{ctanrefs}
\LastEdit{2013-04-11}

\Question[Q-sysmac]{Macintosh systems}

The \Qref*{\TeX{} collection}{Q-CD} \acro{DVD} includes Mac\TeX{},
which is a Mac-tailored version of \texlive{}; details may be found on
the \href{http://tug.org/mactex}{\acro{TUG} web site}.  If you don't
have the disc, you can download the distribution from \acro{CTAN} (but
note that it's pretty big).  Mac\TeX{} is an instance of \texlive{},
and has a Mac-tailored graphical \texlive{} manager, so that you can
keep your distribution up-to-date.

Note that installing Mac\TeX{} requires \texttt{root} privilege.  This
is a pity, since it offers several extras that aren't available via a
standard \texlive{}, which aren't therefore available to ``ordinary
folk'' at work.  For those who don't have \texttt{root} privilege, the
option is to install using the \texlive{} \texttt{tlinstall} utility.

\href{http://www.trevorrow.com/oztex/}{Oz\TeX{}}, by Andrew Trevorrow,
is a shareware version of \TeX{} for the Macintosh.  A \acro{DVI}
previewer and \PS{} driver are also included.
Oz\TeX{} is a Carbon app, so will run under \macosx{} (see
\url{http://www.trevorrow.com/oztex/ozosx.html} for details), but it
is \emph{not} a current version: it doesn't even offer \pdftex{}.  A
mailing list is provided by \acro{TUG}: sign up via
\url{http://tug.org/mailman/listinfo/oztex}

Another partly shareware program is
\href{http://www.kiffe.com/cmactex.html}{CMac\TeX{}}, put together by
Tom Kiffe.  CMac\TeX{} is much closer than Oz\TeX{} to the Unix \TeX{}
model of things (it uses \ProgName{dvips}, for instance).  CMac\TeX{}
runs natively under \macosx{}; it includes a port of a version of
\Qref*{Omega}{Q-omegaleph}.

\begin{flatversion}
  A wiki for Mac users is to be found at
  \URL{http://mactex-wiki.tug.org/}
\end{flatversion}
\begin{hyperversion}
  Further information may be available in the % ! line break
  \href{http://mactex-wiki.tug.org/}{MacTeX wiki}.
\end{hyperversion}
The Mac\TeX{}-on-OS~X mailing list is another useful resource for
users; subscribe via the
\href{http://mactex-wiki.tug.org/wiki/index.php?title=TeX_on_Mac_OS_X_mailing_list}{list home page}
\begin{ctanrefs}
\item[cmactex]\CTANref{cmactex}
\item[mactex]\CTANref{mactex}
\item[oztex]\CTANref{oztex}
\end{ctanrefs}

\Question[Q-sysother]{Other systems' \TeX{} availability}

For \acro{PC}s, running \MSDOS{} or \acro{OS/}2, Em\TeX{} (by Eberhard
Mattes) offers \LaTeX{}, \BibTeX{}, previewers, and drivers.  It is
available as a series of zip archives, with documentation in both
German and English.  Appropriate memory managers for using em\TeX{}
with 386 (and better) processors and under pre-'9x Windows, are
included in the distribution.  (Em\TeX{} \emph{can} be made to operate
under Windows, but even back when it was ``current'', such use wasn't
very actively encouraged.)

A version of em\TeX{}, packaged to use a % ! line break
\Qref*{TDS directory structure}{Q-tds}, is separately available as an
em\TeX{} `contribution'.  Note that neither em\TeX{} itself, nor
em\TeX{}-\acro{TDS}, is maintained.  Users of Microsoft operating
systems, who want an up-to-date \AllTeX{} system, need Win32-based
systems.

For \acro{PC}s, running \MSDOS{}, a further option is a port of the
Web2C~7.0 implementation, using the \acro{GNU} \ProgName{djgpp}
compiler.  While this package is more recent than em\TeX{}, it
nevertheless also offers a rather old instance of \AllTeX{}.

For \acro{VAX} systems running Open\acro{VMS}, a \TeX{} distribution
is available \acro{CTAN}, but is almost certainly not the latest (it
is more than 10 years old).  Whether a version is even available for
current \acro{VMS} (which typically runs on Intel 64-bit processors)
is not clear, but it seems unlikely.

For the Atari \acro{ST} and \acro{TT}, \acro{CS}-\TeX{} is available
from \acro{CTAN}; it's offered as a set of \acro{ZOO} archives.

Amiga users have the option of a full implementations of \TeX{} 3.1
(Pas\TeX{}) and \MF{} 2.7.

It's less likely that hobbyists would be running \acro{TOPS}-20
machines, but since \TeX{} was originally written on a \acro{DEC}-10
under \acro{WAITS}, the \acro{TOPS}-20 port is another near approach
to Knuth's original environment. Sources are available by anonymous
\texttt{ftp} from \url{ftp://ftp.math.utah.edu/pub/tex/pub/web}
\begin{ctanrefs}
\item[\nothtml{\rmfamily}Atari TeX]\CTANref{atari-cstex}
\item[djgpp]\CTANref{djgpp}
\item[emtex]\CTANref{emtex}
\item[emtexTDS]\CTANref{emtextds}
\item[OpenVMS]\CTANref{OpenVMSTeX}
\item[PasTeX]\CTANref{amiga}
\end{ctanrefs}
\LastEdit{2013-07-18}

\Question[Q-editors]{\TeX{}-friendly editors and shells}

There are good \TeX{}-writing environments and editors for most
operating systems; some are described below, but this is only a
personal selection:
\begin{description}
\item[Unix] The commonest choices are \ProgName{[X]Emacs} or
  \ProgName{vim}, though several others are available.
  
  \href{http://www.gnu.org/software/emacs/emacs.html}{\acro{GNU}\nobreakspace\ProgName{emacs}}
  and \href{http://www.xemacs.org/}{\ProgName{XEmacs}} are supported
  by the \acro{AUC}-\TeX{}
  bundle (available from \acro{CTAN}).  \acro{AUC}-\TeX{} provides menu
  items and control sequences for common constructs, checks syntax,
  lays out markup nicely, lets you call \TeX{} and drivers from
  within the editor, and everything else like this that you can think
  of.  Complex, but very powerful.

  \href{http://vim.sourceforge.net}{\ProgName{Vim}} is also highly
  configurable (also available for Windows and Macintosh systems).
  Many plugins are available to support the needs of the \AllTeX{} user,
  including syntax highlighting, calling \TeX{} programs,
  auto-insertion and -completion of common \AllTeX{} structures, and
  browsing \LaTeX{} help.  The scripts \File{auctex.vim} and
  \File{bibtex.vim} seem to be the most common recommendations.

  The editor \href{http://nedit.org/}{\ProgName{NEdit}} is also free
  and programmable, and is available for Unix systems.  An
  \acro{AUC}-\TeX{}-alike set of extensions for \ProgName{NEdit} is available
  from \acro{CTAN}.

  \ProgName{LaTeX4Jed} provides much enhanced \LaTeX{} support for the
  \href{http://www.jedsoft.org/jed/}{\ProgName{jed}} editor.
  \ProgName{LaTeX4Jed} is similar to \acro{AUC}-\TeX{}: menus,
  shortcuts, templates, syntax highlighting, document outline,
  integrated debugging, symbol completion, full integration with
  external programs, and more. It was designed with both the beginner
  and the advanced LaTeX user in mind.

  The \ProgName{Kile} editor that is provided with the \acro{KDE}
  window manager provides \acro{GUI} ``shell-like'' facilities, in a
  similar way to the widely-praised \ProgName{Winedt} (see below);
  details (and downloads) are available from the project's
  \href{http://kile.sourceforge.net/}{home on SourceForge}.

  \acro{TUG} is sponsoring the development of a cross-platform editor
  and shell, modelled on the excellent \texshop{} for the Macintosh.
  The result,
  \href{http://www.tug.org/texworks/}{\texworks{}}, is recommended: if
  you're looking for a
  \AllTeX{} development environment, it may be for you.  (It is
  distributed with both \texlive{} and \miktex{}.)

  A possible alternative is
  \href{http://texstudio.sourceforge.net/}{TeXstudio}
\item[Windows-32]\ProgName{TeXnicCenter} is a (free)
  \TeX{}-oriented development system, uniting a powerful platform for
  executing \AllTeX{} and friends with a configurable editor.

  \texworks{} (see above) is also available for Windows systems.
  
  \ProgName{Winedt}, a shareware package, is also highly spoken of.
  It too provides a shell for the use of \TeX{} and related programs,
  as well as a powerful and well-configured editor.  The editor can
  generate its output in \acro{UTF}-8 (to some extent), which is
  useful when working with \Qref*{\xetex{}}{Q-xetex} (and other
  ``next-generation'' \AllTeX{} applications).

  Both \ProgName{emacs} and \ProgName{vim} are available in versions
  for Windows systems.
\item[Macintosh] For \macosx{} users, the free tool of choice appears to be
  \href{http://pages.uoregon.edu/koch/texshop/index.html}{\texshop{}}, which
  combines an editor and a shell with a coherent philosophy of dealing
  with \AllTeX{} in the OS~X environment.  TeXShop is distributed as
  part of the MacTeX system, and will therefore be available out of
  the box on machines on which MacTeX has been installed.

  \ProgName{Vim} is also available for use on Macintosh systems.

  The commercial Textures provides an excellent integrated Macintosh
  environment with its own editor.  More powerful still (as an editor)
  is the shareware \ProgName{Alpha} which is extensible enough to let
  you perform almost any \TeX{}-related job. It also works well with
  Oz\TeX{}.  From release 2.2.0 (at least), Textures works under \macosx{}.
\item[\acro{OS/}2] \ProgName{epmtex} offers an \acro{OS/}2-specific shell.
\end{description}
Atari, Amiga and \acro{N}e\acro{XT} users also have nice
environments. \LaTeX{} users looking for \ProgName{make}-like
facilities should review the answer on
\Qref*{Makefiles for \LaTeX{} documents}{Q-make}.

While many \AllTeX{}-oriented editors can support work on \BibTeX{}
files, there are many systems that provide specific ``database-like''
access to your \BibTeX{} files~---
\begin{hyperversion}
  see ``\Qref{creating a bibliography file}{Q-buildbib}''.
\end{hyperversion}
\begin{flatversion}
  these are discussed in % ! line break
  ``\Qref*{creating a bibliography file}{Q-buildbib}''.
\end{flatversion}
\begin{ctanrefs}
\item[alpha]\CTANref{alpha}
\item[auctex]\CTANref{auctex}
\item[epmtex]\CTANref{epmtex}
\item[LaTeX4Jed]\CTANref{latex4jed}
\item[Nedit LaTeX support]\CTANref{nedit-latex}
\item[TeXnicCenter]\CTANref{texniccenter}
\item[TeXshell]\CTANref{texshell}
\item[TeXtelmExtel]\CTANref{TeXtelmExtel}
\item[winedt]\CTANref{winedt}
\end{ctanrefs}
\LastEdit{2013-05-20}

\Question[Q-commercial]{Commercial \TeX{} implementations}
\keywords{windows,macintosh,commercial}

There are many commercial implementations of \TeX{}. The first
appeared not long after \TeX{} itself appeared.

What follows is probably an incomplete list.  Naturally, no warranty or
fitness for purpose is implied by the inclusion of any vendor in this
list.  The source of the information is given to provide some clues to
its currency.

In general, a commercial implementation will come `complete', that is,
with suitable previewers and printer drivers.  They normally also have
extensive documentation (\emph{i.e}., not just the \TeX{}book!) and some
sort of support service.  In some cases this is a toll free number
(probably applicable only within the \acro{USA} and or Canada), but others
also have email, and normal telephone and fax support.
\begin{description}
\item[\acro{PC}; True\TeX{}] Runs on all versions of Windows.
  \begin{quote}
    Richard J. Kinch\\
    TrueTeX Software\\
    7890 Pebble Beach Court\\
    Lake Worth \acro{FL} 33467\\
    \acro{USA}\\[.25\baselineskip]
    Tel: +1 561-966-8400\\
    Email: \mailto{kinch@truetex.com}\\
    Web: \URL{http://www.truetex.com/}
  \end{quote}
  Source: Mail from Richard Kinch, August 2004.
\item[pc\TeX{}] Long-established: pc\TeX{}32 is a Windows implementation.
  \begin{quote}
    Personal \TeX{} Inc\\
    725 Greenwich Street, Suite 210 \\
    San Francisco, \acro{CA} 94133\\
    \acro{USA}\\[.25\baselineskip]
    Tel: 800-808-7906 (within the \acro{USA})\\
    Tel: +1 415-296-7550\\
    Fax: +1 415-296-7501\\
    Email: \mailto{sales@pctex.com}\\
    Web: \URL{http://www.pctex.com/}
  \end{quote}
  Source: Personal \TeX{} Inc web site, December 2004
\item[\acro{PC}; Scientific Word] Scientific Word and Scientific Workplace
  offer a mechanism for near-\WYSIWYG{} input of \LaTeX{} documents; they
  ship with True\TeX{} from Kinch (see above).  Queries within the \acro{UK}
  and Ireland should be addressed to Scientific Word Ltd., others should be
  addressed directly to the publisher, MacKichan Software Inc.
  \begin{quote}
    Dr Christopher Mabb\\
    Scientific Word Ltd.\\
    990 Anlaby Road,\\
    Hull,\\
    East Yorkshire,\\
    \acro{HU}4 6\acro{AT}\\
    \acro{UK}\\[0.25\baselineskip]
    Tel: 0845 766\,0340 (within the \acro{UK}) \\
    Fax: 0845 603\,9443 (within the \acro{UK}) \\
    Email: \mailto{christopher@sciword.demon.co.uk} \\
    Web: \URL{http://www.sciword.demon.co.uk}
  \end{quote}
  \begin{quote}
    MacKichan Software Inc.\\
    19307 8th Avenue, Suite C\\
    Poulsbo \acro{WA} 98370-7370\\
    \acro{USA}\\[0.25\baselineskip]
    Tel: +1 360 394\,6033\\
    Tel: 877 724\,9673 (within the \acro{USA})
    Fax: +1  360 394\,6039\\
    Email: \mailto{info@mackichan.com}\\
    Web: \URL{http://www.mackichan.com}
  \end{quote}
  Source: Mail from Christopher Mabb, August 2007
%% \item[Macintosh; Textures] ``A \TeX{} system `for the rest of
%%   us'\,''.  A beta release of Textures for \macosx{} is
%%   available~--- see \URL{http://www.bluesky.com/news/news_frames.html}
%%
%%   (Blue Sky also gives away a \MF{} implementation and some
%%   font manipulation tools for Macs.)
%%   \begin{quote}
%%     Blue Sky Research\\
%%     PO Box 80424\\
%%     Portland, \acro{OR} 97280\\
%%     \acro{USA}\\[.25\baselineskip]
%%     Tel: 800-622-8398 (within the \acro{USA})\\
%%     Tel/Fax: +1 503-222-9571\\
%%     Fax: +1 503-246-4574\\
%%     Email: \mailto{sales@bluesky.com}\\
%%     Web: \URL{http://www.bluesky.com/}
%%   \end{quote}
%%   Source: Mail from Gordon Lee, April 2007
\item[Amiga\TeX{}] A full implementation for the Commodore Amiga,
  including full, on-screen and printing support for all \PS{}
  graphics and fonts, IFF raster graphics, automatic font generation,
  and all of the standard macros and utilities.
  \begin{quote}
    Radical Eye Software\\
    \acro{PO} Box 2081\\
    Stanford, \acro{CA} 94309\\
    \acro{USA}
  \end{quote}
  Source: Mail from Tom Rokicki, November 1994
\end{description}
\checked{mc}{1994/11/09}%
\checked{RF}{1994/11/24}%

Note that the company \YandY{} has gone out of business, and \YandY{}
\TeX{} (and support for it) is therefore no longer available.  Users
of \YandY{} systems may care to use the self-help
\href*{http://tug.org/pipermail/yandytex/}{mailing list}
that was established in 2003; the remaining usable content of
\YandY{}'s web site is available at \URL{http://www.tug.org/yandy/}

