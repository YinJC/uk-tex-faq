% $Id: faq-fonts.tex,v 1.21 2014/01/22 17:29:03 rf10 Exp $

\section{Fonts}

\subsection{Adobe Type~1 (``\PS{}'') fonts}

\Question[Q-usepsfont]{Using Adobe Type 1 fonts with \protect\TeX{}}

In order to use any font, \TeX{} needs a
\emph{metric} file (\acro{TFM} file).  Several sets of metrics for
common Adobe Type 1 fonts are
available from the archives; for mechanisms for generating new ones,
see \Qref[question]{metrics for \PS{} fonts}{Q-metrics}.  You
also need the fonts themselves; \PS{} printers come with a set of
fonts built in, but to extend your repertoire you usually
need to buy from one of the many commercial font vendors (see, for
example, \Qref[question]{``choice of fonts''}{Q-psfchoice}).

If you use \LaTeXe{}, access to your printer's fonts is offered by the
\acro{PSNFSS} package; the \LaTeX{}3 project team declare that
\acro{PSNFSS} is a ``required'' part of a \LaTeX{} distribution, and
bug reports may be submitted via the % ! line break
\Qref*{\LaTeX{} bugs system}{Q-latexbug}. 
\acro{PSNFSS} gives you a set of packages for changing the default
roman, sans-serif and typewriter fonts; \emph{e.g}., the
\Package{mathptmx} package will set up \FontName{Times}
\FontName{Roman} as the main text font (and introduces mechanisms to
typeset mathematics using \FontName{Times} and various more-or-less
matching fonts),  while package \Package{avant} changes the sans-serif
family to \FontName{AvantGarde}, and \Package{courier} changes the
typewriter font to \FontName{Courier}.  To go with these
packages, you need the font metric files
and font description (\File{.fd}) files for each font family you
want to use.  For convenience,
metrics for the `common 35' \PS{} fonts found in most \PS{} printers
are provided with \acro{PSNFSS}, packaged as the ``Laserwriter set''.

For older versions of \LaTeX{} there are various schemes, of which the
simplest to use is probably the \acro{PS}\LaTeX{} macros distributed with
\ProgName{dvips}.

For \plaintex{}, you load whatever fonts you like; if the encoding of
the fonts is not the same as Computer Modern it will be up to you to
redefine various macros and accents, or you can use the font
re-encoding mechanisms available in many drivers and in
\ProgName{ps2pk} and \ProgName{afm2tfm}.

Some common problems encountered are discussed elsewhere
(see \Qref[question]{problems with \acro{PS} fonts}{Q-psfontprob}).
\begin{ctanrefs}
\item[Metrics for the `Laserwriter' set of 35 fonts]\CTANref{lw35nfss-zip}
\item[psnfss]\CTANref{psnfss}
\end{ctanrefs}

\Question[Q-PSpreview]{Previewing files using Type 1 fonts}

Originally, free \TeX{} previewers were only capable of displaying
bitmap (\acro{PK}) fonts, but free Type~1 font rendering software has
been available for some time, and many previewers now use such
facilities.

The alternative, for previewers, is automatic generation of the
requisite \acro{PK} files (using \ProgName{gsftopk}, or similar,
behind the scenes).

In the unlikely event that your previewer isn't capable of either, you
have a couple options:
\begin{itemize}
\item Convert the \acro{DVI} file to \PS{} and use a
  \PS{} previewer.  Some systems offer this capability as
  standard, but most people will need to use a separate previewer such
  as \href{http://www.ghostscript.com/}{\ProgName{ghostscript}} or
  \ProgName{ghostscript}-based viewers
  such as (free) \ProgName{gv} or (shareware)
  \href{http://www.ghostgum.com.au/}{\ProgName{gsview}}.
% \item Under Windows on a \acro{PC}, or on a Macintosh, let Adobe Type Manager
%   display the fonts (textures, on the Macintosh, works like this).
% %%   , and under
% %%   Windows you can use \YandY{}'s \ProgName{dviwindo} for bitmap-free
% %%   previewing.
% \begin{narrowversion} % really hyper
%   (See \Qref[question]{}{Q-commercial} for details of these
%   suppliers.)
% \end{narrowversion}
% %\typeout{after narrowversion, before wideversion}
% \begin{wideversion}
%   (See \Qref{commercial suppliers}{Q-commercial} for details.)
% \end{wideversion}
% %\typeout{after wideversion}
\item If you have the \PS{} fonts in Type~1 format, use
  \ProgName{ps2pk} or \ProgName{gsftopk} (designed for use with the
  \href{http://www.ghostscript.com/}{\ProgName{ghostscript}} fonts) to
  make \acro{PK} bitmap fonts which
  your previewer will understand (a process similar to the way some
  browsers fo the job `automatically') This can produce adequate results,
  also suitable for printing with non-\PS{} devices.  (Note: if you
  purchased the fonts, it is advisable to check that their licence
  permits you to convert them, for private use, in this way.)
\end{itemize}
\begin{ctanrefs}
\item[gsftopk]\CTANref{gsftopk}
\item[gv]Browse \CTANref{gv}
\item[ps2pk]\CTANref{ps2pk}
\end{ctanrefs}

\Question[Q-metrics]{\TeX{} font metric files for Type 1 fonts}

Reputable font vendors such as Adobe supply metric files for each
font, in \acro{AFM} (Adobe Font Metric) form; these can be converted
to \acro{TFM} (\TeX{} Font Metric) form. Most modern distributions have
prebuilt metrics which will be more than enough for many people; but you may
need to do the conversion yourself if you have special needs or
acquire a new font. One important question is the \emph{encoding} of
(Latin character) fonts; while we all more or less agree about the
position of about 96 characters in fonts (the basic \acro{ASCII} set), the
rest of the (typically) 256 vary. The most obvious problems are with
floating accents and special characters such as the `pounds sterling'
sign. There are three ways of dealing with this: either you change the
\TeX{} macros which reference the characters (not much fun, and
error-prone); or you change the encoding of the font (easier than you
might think); or you use \Qref*{virtual fonts}{Q-virtualfonts} to
\emph{pretend} to \TeX{} that the encoding is the same as it is used to.
\LaTeXe{} has facilities for dealing with fonts in different
encodings; read the \Qref*{\emph{\LaTeX{} Companion}}{Q-latex-books} for
more details.  In practice, if you do much non-English (but Latin
script) typesetting, you are strongly recommended to use the
\Package{fontenc} package with option `\pkgoption{T1}' to select
\Qref*{`Cork'}{Q-ECfonts} encoding.

An alternative favoured by some is \YandY{}'s ``private'' \acro{LY}1 encoding,
which is designed to sit well with ``Adobe standard'' encoded fonts.
Basic macro support of \acro{LY}1 is available: note that the
``relation with Adobe's encoding'' means that the \acro{LY}1 user
needs no virtual fonts.

Alan Jeffrey's \ProgName{fontinst} package is an \acro{AFM} to
\acro{TFM} converter written in \TeX{}; it is used to generate the
files used by \LaTeXe{}'s \acro{PSNFSS} package to support use of
\PS{} fonts. It is a sophisticated package, not for the faint-hearted,
but is powerful enough to cope with most needs.  Much of its power
relies on the use of \Qref*{virtual fonts}{Q-virtualfonts}.

For slightly simpler problems, Rokicki's \ProgName{afm2tfm},
distributed with \ProgName{dvips}, is fast and
efficient; note that the metrics and styles that come with
\ProgName{dvips} are \emph{not} currently \LaTeXe{} compatible.

For the Macintosh (classic), there is a program called
\ProgName{EdMetrics} which does the job (and more).
\ProgName{EdMetrics} comes with the (commercial)
\Qref*{Textures}{Q-commercial} distribution, but is itself free
software, and is available on \acro{CTAN}.
%% \par
%% Windows users can buy \Qref*{\YandY{}'s}{Q-commercial}
%% Font Manipulation Tools package which includes a powerful
%% \ProgName{afmtotfm} program among many other goodies.
\begin{ctanrefs}
\item[dvips]\CTANref{dvips}
\item[EdMetrics]\CTANref{edmetrics}
\item[fontinst]\CTANref{fontinst}
\item[LY1 support]\CTANref{ly1}
\end{ctanrefs}
\LastEdit{2011-06-01}

\Question[Q-psfontprob]{Deploying Type 1 fonts}

For the \LaTeX{} user trying to use the
\Qref*{\acro{PSNFSS}}{Q-usepsfont} package, three questions may arise.

First, you have to declare to the \acro{DVI} driver that you are using
\PS{} fonts; in the case of \ProgName{dvips}, this means adding
lines to the \File{psfonts.map} file, so that \ProgName{dvips} will know
where the proper fonts are, and won't try to find \acro{PK} 
files.  If the font isn't built into the printer, you have to acquire
it (which may mean that you need to purchase the font files).

Second, your previewer must know what to do with the fonts: see
\Qref[question]{previewing type 1 fonts}{Q-PSpreview}.

Third, the stretch and shrink between words is a function of the
font metric; it is not specified in \acro{AFM} files, so different converters
choose different values. The \PS{} metrics that come with \acro{PSNFSS} 
used to produce quite tight setting, but they were revised in mid 1995
to produce a compromise between American and European practice. 
Sophisticated users may not find even the new the values to their taste, and
want to override them. Even the casual user may find  more
hyphenation or overfull boxes than Computer Modern produces; but \acro{CM}
is extremely generous. 

\Question[Q-psfchoice]{Choice of Type~1 fonts for typesetting Maths}

If you are interested in text alone, you can in principle use any of
the huge numbers of text fonts in Adobe Type~1, TrueType or OpenType
formats.  The constraint is, of course, that your previewer and
printer driver should support such fonts (\TeX{} itself \emph{only}
cares about metrics, not the actual character programs).

If you also need mathematics, then your choice is more limited, in
particular  by the
demands that \TeX{} makes of maths fonts (for details, see the papers
by
\begin{wideversion}
% !! line break !!
  \href{http://tug.org/TUGboat/Articles/tb14-3/tb40horn.pdf}{B.K.P. Horn in \TUGboat{} 14(3)},
  or by % ! line break
  \href{http://tug.org/TUGboat/Articles/tb19-2/tb59bouc.pdf}{Thierry Bouche in \TUGboat{} 19(2)}).
\end{wideversion}
\begin{narrowversion}
  B.K.P. Horn in \TUGboat{} 14(3) or by Thierry Bouche in \TUGboat{}
  19(2), available as
  \URL{http://tug.org/TUGboat/Articles/tb14-3/tb40horn.pdf} or
  \URL{http://tug.org/TUGboat/Articles/tb19-2/tb59bouc.pdf}, respectively).
\end{narrowversion}
There are several options available, which are
based on Knuth's original designs.  Others complement other
commercial and free text font designs; one set (MicroPress's `informal math')
stands alone.

Users should also consider the possibilities of typesetting
\Qref*{maths using OpenType fonts}{Q-otf-maths}.

``Free'' font families that will support \TeX{} mathematics include:
\begin{booklist}
\item[Computer Modern](75 fonts~--- optical scaling) Donald E. Knuth\\
  The \acro{CM} fonts were originally designed in \MF{}, but are also
  now available in scalable outline form.  There are commercial as
  well as public domain versions, and there are both Adobe Type~1 and
  TrueType versions.  A set of outline versions of the fonts was
  developed as a commercial venture by \YandY{} and Blue Sky Research;
  they have since assigned the copyright to the \acro{AMS}, and the
  fonts are now freely available from \acro{CTAN}.  Their quality is
  such that they have become the \emph{de facto} standard for Type~1
  versions of the fonts.

\item[\acro{AMS} fonts](52 fonts, optical scaling) The \acro{AMS}\\
  This set of fonts offers adjuncts to the \acro{CM} set, including
  two sets of symbol fonts (\FontName{msam} and \FontName{msbm}) and
  Euler text fonts.
  These are not a self-standing family, but merit discussion here (not
  least because several other families mimic the symbol fonts).
  Freely-available Type~1 versions of the fonts are available on
  \acro{CTAN}.  The \Package{eulervm} package permits use
  of the Euler maths alphabet in conjunction with text fonts that do
  not provide maths alphabets of their own (for instance, Adobe
  Palatino or Minion).

\item[Mathpazo version 1.003](5 fonts) Diego Puga\\
  The Pazo Math fonts are a family of type~1 fonts suitable for
  typesetting maths in combination with the Palatino family of text
  fonts.  Four of the five fonts of the distribution are maths
  alphabets, in upright and italic shapes, medium and bold weights;
  the fifth font contains a small selection of ``blackboard bold''
  characters (chosen for their mathematical significance).  Support
  under \LaTeXe{} is available in 
  \Qref*{\acro{PSNFSS}}{Q-usepsfont}; the fonts are
  licensed under the \acro{GPL}, with legalese permitting the use of
  the fonts in published documents.

\item[Fourier/Utopia](15 fonts) Michel Bovani\\
  \FontName{Fourier} is a family built on Adobe \FontName{Utopia}
  (which has been released for usage free of charge by Adobe).  The
  fonts provide the basic Computer Modern set of mathematical symbols,
  and add many of the \acro{AMS} mathematical symbols (though you are
  expected to use some from the \acro{AMS} fonts themselves).  There
  are also several other mathematical and decorative symbols.  The
  fonts come with a \Package{fourier} package for use with \LaTeX{};
  text support of \acro{OT}1 encoding is not provided~--- you are
  expected to use \acro{T}1.

  For a sample, see \url{http://www.tug.dk/FontCatalogue/utopia/}

\item[Fourier/New Century Schoolbook]Michael Zedler\\
  \FontName{Fouriernc} is a configuration using the Fourier fonts in
  the context of New Century Schoolbook text fonts.

  For a sample, see \url{http://www.tug.dk/FontCatalogue/newcent/}

\item[KP-fonts]The Johannes Kepler project\\
  The \FontName{kp-fonts} family provides a comprehensive set of text
  and maths fonts.  The set includes replacement fixed-width and sans
  fonts (though some reports have suggested that these are less
  successful, and their use may be suppressed when loading the fonts'
  \Package{kpfonts} \LaTeX{} support package).

  For an example, see \url{http://www.tug.dk/FontCatalogue/kpserif/}

\item[MathDesign](3 free families, 3 commercial-based
  families\dots{}so far) Paul Pichaureau\\
  This set so far offers mathematics fonts to match the free fonts
  Adobe Utopia, URW Garamond and Bitstream Charter (the text versions
  of all of which are separately available, on \acro{CTAN}, in Type~1
  format), and Adobe Garamond Pro, Adobe UtopiaStd and ITC Charter
  (which are commercial fonts, all available for purchase on the web).
  There has been a little comment on these fonts, but none
  from actual users posted to the public forums.  Users, particularly
  those who are willing to discuss their experiences, would obviously
  be welcome.  Browse the \acro{CTAN} directory and see which you
  want: there is a wealth of documentation and examples.

  For samples of the free variants, see
  \url{http://www.tug.dk/FontCatalogue/charter/} for URW Charter,
  \url{http://www.tug.dk/FontCatalogue/garamond/} for URW Garamond and
  \url{http://www.tug.dk/FontCatalogue/utopia-md/} for Adobe Utopia.

\item[Belleek](3 fonts) Richard Kinch\\
  Belleek is the upshot of Kinch's thoughts on how \MF{} might be used
  in the future: they were published simultaneously as \MF{} source,
  as Type~1 fonts, and as TrueType fonts.  The fonts act as ``drop-in''
  replacements for the basic MathTime set (as an example of ``what might
  be done'').

  The paper outlining Kinch's thoughts, proceeding from considerations
  of the `intellectual' superiority of \MF{} to evaluations of why its
  adoption is so limited and what might be done about the problem, is
  to be found at \URL{http://truetex.com/belleek.pdf}

\item[MTPro2 Lite]Pubish or Perish (Michael Spivak)\\
  A (functional) subset of the MathTime Pro 2 font set, that is made
  available, free, for general use.  While it does not offer the full
  power of the commercial product (see below), it is nevertheless a
  desirable font set.

\item[Mathptmx]Alan Jeffrey, Walter Schmidt and others.\\
  This set contains maths italic, symbol, extension, and roman virtual
  fonts, built from Adobe Times, Symbol, Zapf Chancery, and the
  Computer Modern fonts.  The resulting mixture is not  entirely
  acceptable, but can pass in many circumstances.  The real advantage
  is that the mathptm fonts are (effectively) free, and the resulting
  \PS{} files can be freely exchanged.  Support under \LaTeXe{}
  is available in \Qref*{\acro{PSNFSS}}{Q-usepsfont}.

  For a sample, see \url{http://www.tug.dk/FontCatalogue/times/}

\item[Computer Modern Bright]Free scalable outline versions of these
  fonts do exist; they are covered below together with their
  commercial parallels.

\item[URW Classico](4 fonts)  LaTeX support by Bob Tennent\\
  These are clones of Zapf's Optima available from CTAN (for
  non-commercial use only).  Mathematics support can be provided by
  using packages \Package{eulervm} or \Package{sansmath}. As a
  sans-serif font family, Optima is especially suitable for
  presentations.
\end{booklist}
The excellent \emph{font catalogue} keeps an % ! line break
\href{http://www.tug.dk/FontCatalogue/mathfonts.html}{up-to-date list}
which describes the fonts by giving names and short examples, only.
(At the time of writing~--- June 2008~--- the list has several that
are only scheduled for inclusion here.

Another useful document is Stephen Hartke's ``Free maths font
survey'', which is available on \acro{CTAN} in both \acro{PDF} and
\acro{HTML} formats.  The survey covers most of the fonts mentioned in
the font catalogue, but also mentions some (such as \Package{Belleek}
that the catalogue omits.

Fonts capable of setting \TeX{} mathematics, that are available
commercially, include:
\begin{booklist}
\item[BA Math](13 fonts) MicroPress Inc.\\
  BA Math is a family of serif fonts, inspired by the elegant
  and graphically perfect font design of John Baskerville.  BA
  Math comprises the fonts necessary for mathematical typesetting
  (maths italic, math symbols and extensions) in normal and bold
  weights.  The family also includes all \acro{OT}1 and \acro{T}1
  encoded text fonts of various shapes, as well as fonts with most
  useful glyphs of the \acro{TS}1 encoding.  Macros for using the
  fonts with \plaintex{}, \LaTeXo{} and current \LaTeX{} are
  provided.

  \nothtml{\bgroup\raggedwithindent}
  For further details (including samples) see\\
  \URL{http://www.micropress-inc.com/fonts/bamath/bamain.htm}
  \nothtml{\par\egroup}
\item[CH Math](15 fonts) MicroPress Inc.\\
  CH Math is a family of slab serif fonts, designed as a maths
  companion for Bitstream Charter.  (The distribution includes
  four free Bitstream text fonts, in addition to the 15 hand-hinted
  MicroPress fonts.)
  \nothtml{\bgroup\raggedwithindent}
  For further details (including samples) see\\
  \URL{http://www.micropress-inc.com/fonts/chmath/chmain.htm}
  \nothtml{\par\egroup}

\item[Computer Modern Bright](62 fonts~--- optical scaling) Walter
  Schmidt\\
  \acro{CM} Bright is a family of sans serif fonts, based on Knuth's
  \acro{CM} fonts.  It comprises the fonts necessary for mathematical
  typesetting, including \acro{AMS} symbols, as well as text and text
  symbol fonts of various shapes.  The collection comes with its own
  set of files for use with \LaTeX{}.  The \acro{CM} Bright fonts are
  supplied in Type~1 format by MicroPress, Inc.   The
  \Package{hfbright} bundle offers free Type~1 fonts for text using
  the \acro{OT}1 encoding~--- the \Package{cm-super} fonts provide the
  fonts in \acro{T}1 text encoding but don't support \acro{CM} bright
  mathematics.

  \nothtml{\bgroup\raggedwithindent}
  For further details of Micropress' offering (including samples) see\\
  \URL{http://www.micropress-inc.com/fonts/brmath/brmain.htm}
  \nothtml{\par\egroup}

\item[Concrete Math](25 fonts~--- optical scaling) Ulrik Vieth\\
  The Concrete Math font set was derived from the Concrete Roman
  typefaces designed by Knuth.  The set provides a collection of math
  italics, math symbol, and math extension fonts, and fonts of
  \acro{AMS} symbols that fit with the Concrete set, so that Concrete
  may be used as a complete replacement for Computer Modern.  Since
  Concrete is considerably darker than \acro{CM}, the family may
  particularly attractive for use in low-resolution printing or in
  applications such as posters or transparencies.  Concrete Math
  fonts, as well as Concrete Roman fonts, are supplied in Type 1
  format by MicroPress, Inc.

  \nothtml{\bgroup\raggedwithindent}
  For further information (including samples) see\\
  \URL{http://www.micropress-inc.com/fonts/ccmath/ccmain.htm}
  \nothtml{\par\egroup}

\item[HV Math](14 fonts) MicroPress Inc.\\
  HV Math is a family of sans serif fonts, inspired by the
  Helvetica (\acro{TM}) typeface.  HV Math comprises the fonts
  necessary for mathematical typesetting (maths italic, maths symbols
  and extensions) in normal and bold weights.  The family also
  includes all \acro{OT}1 and \acro{T}1 encoded text fonts of various
  shapes, as well as fonts with most useful glyphs of the \acro{TS}1
  encoding.  Macros for using the fonts with \plaintex{}, \LaTeXo{}
  and current \LaTeX{} are provided.  Bitmapped copies of the fonts
  are available free, on \acro{CTAN}.

  \nothtml{\bgroup\raggedwithindent}
  For further details (and samples) see\\
  \URL{http://www.micropress-inc.com/fonts/hvmath/hvmain.htm}
  \nothtml{\par\egroup}

\item[Informal Math](7 outline fonts) MicroPress Inc.\\
  Informal Math is a family of fanciful fonts loosely based on the
  Adobe's Tekton (\acro{TM}) family, fonts which imitate handwritten
  text.  Informal Math comprises the fonts necessary for
  mathematical typesetting (maths italic, maths symbols and extensions)
  in normal weight, as well as \acro{OT}1 encoded text fonts in
  upright and oblique shapes.  Macros for using the fonts with
  \plaintex{}, \LaTeXo{} and current \LaTeX{} are provided.

  \nothtml{\bgroup\raggedwithindent}
  For further details (including samples) see\\
  \URL{http://www.micropress-inc.com/fonts/ifmath/ifmain.htm}
  \nothtml{\par\egroup}

\item[Lucida Bright \emph{with} Lucida New Math](25 fonts) Chuck Bigelow and
  Kris Holmes\\
  Lucida is a family of related fonts including seriffed, sans serif,
  sans serif fixed width, calligraphic, blackletter, fax, Kris Holmes'
  connected handwriting font, \emph{etc}; they're not as `spindly' as
  Computer Modern, with a large x-height, and include a larger set of
  maths symbols, operators, relations and delimiters than \acro{CM}
  (over 800 instead of 384: among others, it also includes the
  \acro{AMS} \FontName{msam} and \FontName{msbm} symbol sets).  `Lucida
  Bright Expert' 
  (14 fonts) adds seriffed fixed width, another handwriting font,
  smallcaps, bold maths, upright `maths italic', \emph{etc}., to the
  set.  Support under \LaTeX{} is available under the auspices of the
  \acro{PSNFSS}, and pre-built metrics are also provided.

  \nothtml{\bgroup\raggedwithindent}
  \acro{TUG} has the right to distribute these fonts; the web site
\begin{narrowversion}
    \href{http://tug.org/lucida/}{``Lucida and TUG''}
\end{narrowversion}
\begin{wideversion}
    ``\href{http://tug.org/lucida/}{Lucida and TUG}''
\end{wideversion}
  has details.
  \nothtml{\par\egroup}

\item[Adobe Lucida, LucidaSans \emph{and} LucidaMath](12 fonts)\\
  Lucida and LucidaMath are generally considered to be a bit heavy.
  The three maths fonts contain only the glyphs in the \acro{CM} maths
  italic, symbol, and extension fonts.  Support for using LucidaMath
  with \TeX{} is not very good; you will need to do some work
  reencoding fonts \emph{etc}.  (In some sense this set is the
  ancestor of the LucidaBright plus LucidaNewMath font set, which are
  not currently available.)

\item[MathTime Pro2]Publish or Perish (Michael Spivak)\\
  This latest instance of the MathTime family covers all the weights
  (medium, bold and heavy) and symbols of previous versions of
  MathTime.  In addition it has a much extended range of symbols, and
  many typographic improvements that make for high-quality documents.
  The fonts are supported under both \plaintex{} and \LaTeXe{}, and
  are exclusively available for purchase from
  \Qref*{Personal \TeX{} Inc}{Q-commercial}.
  \par{}
  \nothtml{\bgroup\raggedwithindent}
  For further details and samples and fliers, see
  \URL{http://www.pctex.com/mtpro2.html}
  \nothtml{\par\egroup}

\item[Minion Pro and MnSymbol]Adobe, \LaTeX{} support and packaging by
  Achim Blumensath \emph{et al}.\\
  \FontName{Minion Pro} derives from the widely-available commercial
  OpenType font of the same name by Adobe; scripts are provided to
  convert relevant parts of it to Adobe Type~1 format.  The
  \Package{MinionPro} package will set up text and maths support using
  \FontName{Minion Pro}, but a separate (free) font set
  \FontName{MnSymbol} greatly extends the symbol coverage.

\item[PA Math] PA~Math is a family of serif fonts
  loosely based on the Palatino (\acro{TM}) typeface.  PA~Math
  comprises the fonts necessary for mathematical typesetting (maths
  italics, maths, calligraphic and oldstyle symbols, and extensions)
  in normal and bold weights. The family also includes all \acro{OT}1,
  \acro{T}1 encoded text fonts of various shapes, as well as fonts
  with the most useful glyphs of the \acro{TS}1 encoding.  Macros for
  using the fonts with \plaintex{}, \LaTeXo{} and current \LaTeX{}
  are provided.

  \nothtml{\bgroup\raggedwithindent}
  For further details (and samples) see\\
  \URL{http://www.micropress-inc.com/fonts/pamath/pamain.htm} 
  \nothtml{\par\egroup}

\item[TM Math](14 fonts) MicroPress Inc.\\
  TM Math is a family of serif fonts, inspired by the Times
  (\acro{TM}) typeface.  TM Math comprises the fonts necessary for
  mathematical typesetting (maths italic, maths symbols and extensions)
  in normal and bold weights. The family also includes all \acro{OT}1
  and \acro{T}1 encoded text fonts of various shapes, as well as fonts
  with most useful glyphs of the \acro{TS}1 encoding.  Macros for
  using the fonts with \plaintex{}, \LaTeXo{} and current \LaTeX{}
  are provided.  Bitmapped copies of the fonts are available free, on
  \acro{CTAN}.

  \nothtml{\bgroup\raggedwithindent}
  For further details (and samples) see\\
  \URL{http://www.micropress-inc.com/fonts/tmmath/tmmain.htm} 
  \nothtml{\par\egroup}
\end{booklist}
Two other font sets should be mentioned, even though they don't
currently produce satisfactory output~--- their author is no longer
working on them, and several problems have been identified:
\begin{booklist}
\item[Pxfonts set version 1.0](26 fonts) by Young Ryu\\
  The \FontName{pxfonts} set consists of
  \begin{itemize}
  \item virtual text fonts using \FontName{Adobe} \FontName{Palatino}
    (or its \acro{URW} replacement, \FontName{Palladio})
    with modified plus, equal and slash symbols;
  \item maths alphabets using \FontName{Palatino} (or \FontName{Palladio};
  \item maths fonts of all symbols in the computer modern maths fonts
    (\FontName{cmsy}, \FontName{cmmi}, \FontName{cmex} and the Greek
    letters of \FontName{cmr}) 
  \item maths fonts of all symbols corresponding to the \acro{AMS}
    fonts (\FontName{msam} and \FontName{msbm});
  \item additional maths fonts of various symbols.
  \end{itemize}
  The text fonts are available in \acro{OT}1, \acro{T}1 and \acro{LY}1
  encodings, and \acro{TS} encoded symbols are also available.  The
  sans serif and monospaced fonts supplied with the \FontName{txfonts}
  set (see below) may be used with \FontName{pxfonts}; the
  \FontName{txfonts} set should be installed whenever \FontName{pxfonts}
  are.  \LaTeX{}, \ProgName{dvips} and \PDFTeX{} support files are
  included.

  The fonts are not perfect; the widths assigned to the characters in
  the \extension{tfm} file are wrong for some glyphs; this can cause
  sequences of characters to ``look wrong'', or in some cases even to
  overlap; the \FontName{newpx} fonts (noted above) aim to reduce
  these problems.

  The fonts are licensed under the \acro{GPL}; use in published
  documents is permitted.

\item[Newpx]by Michael Sharpe from Young Ryu's \ProgName{pxfonts}\\
  This collection is derived from \ProgName{pxfonts}; the maths fonts
  metrics have been adjusted so that the output is less cramped than
  when \ProgName{pxfonts} is used; the appearance of the output is
  much improved.  Two packages are provided, \Package{newpxtext} for
  using the associated text fonts, and \Package{newpxmath} for
  mathematics.

\item[Txfonts set version 3.1](42 fonts) by Young Ryu\\
  The \FontName{txfonts} set consists of
  \begin{itemize}
  \item virtual text fonts using \FontName{Adobe} \FontName{Times} (or
    the \acro{URW} \FontName{Nimbus Roman No9 L} font that
    substitutes for Times, which is distributed as part of the
    \acro{URW} ``basic 35'' collection) with
    modified plus, equal and slash symbols;
  \item matching sets of sans serif and monospace (`typewriter')
    fonts (the sans serif set is based on \FontName{Adobe} \FontName{Helvetica});
  \item maths alphabets using Adobe \FontName{Times}, or the URW
    equivalent \FontName{NimbusRomanNo9};
  \item maths fonts of all symbols in the computer modern maths fonts
    (\FontName{cmsy}, \FontName{cmmi}, \FontName{cmex} and the Greek
    letters of \FontName{cmr})
  \item maths fonts of all symbols corresponding to the \acro{AMS}
    fonts (\FontName{msam} and \FontName{msbm});
  \item additional maths fonts of various symbols.
  \end{itemize}
  The text fonts are available in \acro{OT}1, \acro{T}1 and \acro{LY}1
  encodings, and \acro{TS} encoded symbols are also available.

  The fonts are not perfect; the widths assigned to the characters in
  the \extension{tfm} file are wrong for some glyphs; this can cause
  sequences of characters to ``look wrong'', or in some cases even to
  overlap; the \FontName{newtx} fonts (noted above) aim to reduce
  these problems.

  The fonts are licensed under the \acro{GPL}; use in published
  documents is permitted.
\item[Txfontsb set version 1.00] by Young Ryu and Antonis Tsolomitis\\
  The \ProgName{txfontsb} bundles \ProgName{txfonts}, extended to
  provide a Small Caps set, Old-Style numbers and Greek text (from the
  GNU Freefont set).
\begin{flatversion}
  Documentation
  (\URL{http://mirrors.ctan.org/fonts/txfontsb/doc/txfontsb.pdf})
\end{flatversion}
\begin{hyperversion}
  \href{http://mirrors.ctan.org/fonts/txfontsb/doc/txfontsb.pdf}{Documentation}
\end{hyperversion}
  is available for this variant, too.

\item[Newtx]by Michael Sharpe from Young Ryu's \ProgName{txfonts}\\
  This collection is derived from \ProgName{txfonts}; the maths fonts
  metrics have been adjusted so that the output is less cramped than
  when \ProgName{txfonts} is used; the appearance of the output is
  much improved.  Two packages are provided, \Package{newtxtext} for
  using the associated text fonts, and \Package{newtxmath} for
  mathematics.  Options are provided to substitute
  letters and symbols from the \FontName{Libertine} set, and from the
  Garamond extension font \FontName{garamondx} (but note that
  \FontName{garamondx}, which is an adaptation of \acro{URW} Garamond,
  is not available via \texlive{}).
\end{booklist}
Finally, one must not forget:
\begin{booklist}
\item[Proprietary fonts]Various sources.\\
  Since having a high quality font set in scalable outline form that
  works with \TeX{} can give a publisher a real competitive advantage,
  there are some publishers that have paid (a lot) to have such font
  sets made for them.  Unfortunately, these sets are not available on
  the open market, despite the likelihood that they're more complete
  than those that are.
\end{booklist}
We observe a very limited selection of commercial maths font sets; a
Type~1 maths font has to be explicitly designed for use with \TeX{},
which is an expensive business, and is of little appeal in other
markets.  Furthermore, the \TeX{} market for commercial fonts is
minute by comparison with the huge sales of other font sets.

Text fonts in Type~1 format are available from many vendors including
Adobe, Monotype and Bitstream.  However, be careful with cheap font
``collections''; many of them dodge copyright restrictions by removing
(or crippling) parts of the font programs such as hinting.  Such
behaviour is both unethical and bad for the consumer.
The fonts may not render well (or at all, under \acro{ATM}), may not have the
`standard' complement of 228 glyphs, or may not include metric files
(which you need to make \acro{TFM} files).

\ttype{} was for a long time the ``native'' format for Windows, but
MicroSoft joined the development of the \otype{} specification, and
`modern' windows will work happily with fonts in either format.  Some \TeX{}
implementations such as \Qref*{True\TeX{}}{Q-commercial} use TrueType
versions of Computer Modern and Times Maths fonts to render \TeX{}
documents in Windows without the need for additional system software like
\acro{ATM}.  (When used on a system running Windows~XP or later,
True\TeX{} can also use Adobe Type~1 fonts.)

When choosing fonts, your own system environment may not be the only one of
interest.  If you will be sending your finished documents to others for
further use, you should consider whether a given font format will introduce
compatibility problems.  Publishers may require TrueType exclusively because
their systems are Windows-based, or Type~1 exclusively, because their systems
are based on the early popularity of that format in the publishing industry.
Many service bureaus don't care as long as you present them with a finished
print file (\PS{} or \acro{PDF}) for their output device.
\begin{ctanrefs}
\item[\nothtml{\normalfont}CM family collection]%
  Distributed as part of \CTANref{amsfonts}
\item[\nothtml{\normalfont}AMS font collection]%
  Distributed as part of \CTANref{amsfonts}
\item[Belleek \nothtml{\normalfont}fonts]%
  \CTANref{belleek}
\item[\nothtml{\normalfont} URW Classico fonts]\CTANref{classico}
\item[CM-super \nothtml{\normalfont}collection]\CTANref{cm-super}
\item[eulervm.sty \nothtml{\normalfont}and supporting metrics]%
  \CTANref{eulervm}
\item[fourier \nothtml{\normalfont}(including metrics and other support for \nothtml{\itshape}utopia]%
  \CTANref{fourier}
\item[fouriernc]\CTANref{fouriernc}
\item[garamondx]\CTANref{garamondx}
\item[hfbright \nothtml{\normalfont}collection]\CTANref{hfbright}
\item[hvmath \nothtml{\normalfont\itshape}(free bitmapped version)]%
  \CTANref{hvmath}
\item[kpfonts \nothtml{\normalfont}family]\CTANref{kpfonts}
\item[Libertine \nothtml{\normalfont}family]\CTANref{libertine}
\item[Lucida Bright/Math metrics]\CTANref{lucida}
\item[Lucida PSNFSS support]\CTANref{lucida-psnfss}
\item[MathDesign \nothtml{\normalfont}collection]\CTANref{mathdesign}
\item[\nothtml{\normalfont}Maths Font Survey]\CTANref{mathsurvey.pdf}
  (\acro{PDF}) or \CTANref{mathsurvey.html} (\acro{HTML})
\item[\nothtml{\normalfont}MathTime Pro 2 Lite]\CTANref{mtp2lite}
\item[\nothtml{\normalfont}Minion Pro support]\CTANref{minionpro}
\item[MnSymbol \nothtml{\normalfont}family]\CTANref{mnsymbol}
\item[Newtx \nothtml{\normalfont}fonts]\CTANref{newtx}
\item[NimbusRomanNo9 fonts]distributed in \CTANref{urw-base35}
\item[Palladio fonts]distributed in \CTANref{urw-base35}
\item[pxfonts]\CTANref{pxfonts}
\item[sansmath.sty]\CTANref{sansmath}
\item[tmmath \nothtml{\normalfont\itshape}(free bitmapped version)]%
  \CTANref{tmmath}
\item[txfonts]\CTANref{txfonts}
\item[\nothtml{\normalfont}URW ``35 fonts'' collection]\CTANref{urw-base35}
\item[utopia \nothtml{\normalfont}fonts]\CTANref{utopia}
\end{ctanrefs}
\LastEdit{2014-01-22}

\Question[Q-otf-maths]{Unicode Maths using OpenType fonts}

The \acro{ISO} standard Universal Coding Scheme (UCS), which is
commonly known as Unicode, was adopted early by the designers of
TrueType (\acro{TTF}) and OpenType (\acro{OTF}) fonts.  The
flexibility of the fonts offers hope, for the first time, of a uniform
method for typesetting essentially any language.

\TeX{} users have been eagerly adopting the fonts, for some time,
using \xetex{} (now a rather stable system) and \LuaTeX{} (which is,
at the time of writing, still being developed).

While \TeX{} users were investigating the use of these text fonts,
\acro{ISO} was extending Unicode to provide a means of expressing
mathematics.  As this work proceeded, MicroSoft and (separately) a
consortium of publishing companies were developing OpenType maths
fonts.  (Microsoft contributed on the development of the concepts,
within the \acro{ISO} process.)  MicroSoft's OpenType Maths font,
\FontName{Cambria Math} has been available for purchase for some time.

The first free OpenType Maths font to appear was % !line break
\FontName{Asana Math}, which was eventually followed by
the publishers' consortium's offer of an interim version
of their font, \FontName{STIX}, which has been
redeveloped to provide a more usable whole, \FontName{XITS}, by a group
of \TeX{} users.

Other fonts are appearing, including % ! line break
\FontName{TeX Gyre Termes Math} (based on Times-like fonts) and
\FontName{Tex Gyre Pagella Math} (based on Palatino-like fonts),
and \FontName{LM Math} extending the OpenType version of the
\FontName{Latin Modern} font family.

Actually using a unicode maths font is quite a complicated business,
but the \LaTeX{} package \Package{unicode-math} (supported
by the \Package{fontspec} package) does the essential groundwork.
\begin{ctanrefs}
\item[Asana-Math \nothtml{\rmfamily}font]\CTANref{asana-math}[asana-math]
\item[fontspec.sty]\CTANref{fontspec}
\item[lm-math \nothtml{\rmfamily}fonts]\CTANref{lm-math}
\item[STIX \nothtml{\rmfamily}fonts]\CTANref{stix}
\item[unicode-math.sty]\CTANref{unicode-math}
\item[tex-gyre-math-pagella \nothtml{\rmfamily}font]distributed as
  part of \CTANref{tex-gyre-math}[tex-gyre-math-pagella]
\item[tex-gyre-math-termes \nothtml{\rmfamily}font]distributed as part
  of \CTANref{tex-gyre-math}[tex-gyre-math-termes]
\item[XITS \nothtml{\rmfamily}fonts]\CTANref{xits}
\end{ctanrefs}
\LastEdit*{2014-01-22}
% more to come...

\Question[Q-charshift]{Weird characters in \ProgName{dvips} output}
\keywords{fi ligature pound pounds sterling elephants}

You've innocently generated output, using \ProgName{dvips}, and there
are weird transpositions in it: for example, the \texttt{fi} ligature has
appeared as a \textsterling{} symbol.
This is an unwanted side-effect of the precautions
outlined in \Qref*{generating \PS{} for PDF}{Q-dvips-pdf}.
The \texttt{-G1} switch discussed in that question is appropriate for
Knuth's text fonts, but doesn't work with text fonts that don't follow
Knuth's patterns (such as fonts supplied by Adobe).

If the problem arises, suppress the \texttt{-G1} switch: if you were using it
explicitly, \emph{don't}; if you were using \texttt{-Ppdf}, add \texttt{-G0} to
suppress the implicit switch in the pseudo-printer file.

The problem has been corrected in \ProgName{dvips} v~5.90 (and later
versions); it is unlikely ever to recur\dots{}
\LastEdit{2012-10-24}

\subsection{Macros for using fonts}

\Question[Q-fonts-pln]{Using non-standard fonts in \plaintex{}}

\plaintex{} (in accordance with its description) doesn't do anything
fancy with fonts: it sets up the fonts that Knuth found he needed when
writing the package, and leaves you to do the rest.

To use something other than Knuth's default, you can use Knuth's
mechanism, the \csx{font} primitive:
\begin{quote}
\begin{verbatim}
\font\foo=nonstdfont
...
\foo
Text set using nonstdfont ...
\end{verbatim}
\end{quote}
The name you use (\texttt{nonstdfont}, above) is the name of the
\extension{tfm} file for the font you want.

If you want to use an italic version of \csx{foo}, you need to use
\csx{font} again:
\begin{quote}
\begin{verbatim}
\font\fooi=nonstdfont-italic
...
\fooi
Text set using nonstdfont italic...
\end{verbatim}
\end{quote}
This is all very elementary stuff, and serves for simple use of fonts.
However, there are wrinkles, the most important of which is the matter
of \Qref*{font encodings}{Q-whatenc}.  One almost never sees new fonts
that use Knuth's eccentric font encodings~--- but those encodings are
built into \plaintex{}, so that some macros of \plaintex{} need to be
changed to use the fonts.  \LaTeX{} gets around all these problems by
using a ``font selection scheme''~--- this `\acro{NFSS}' (`\acro{N}'
for `new', as opposed to what \LaTeXo{} had) carries around with it
separate information about the fonts you use, so the changes to
encoding-specific commands happen automagically.

If you only want to use the \Qref*{\acro{EC} fonts}{Q-ECfonts}, you
can in principle use the \Package{ec-plain} bundle, which gives you a version
of \plaintex{} which you can run in the same way that you run
\plaintex{} using the original \acro{CM} fonts, by invoking
\ProgName{tex}.  (\Package{Ec-plain} also extends the \acro{EC} fonts,
for reasons which aren't immediately clear, but which might cause
problems if you're hoping to use Type 1 versions of the fonts.)

The \Package{font_selection} package provides a sort of halfway house:
it provides font face and size, but not family selection.  This gives
you considerable freedom, but leaves you stuck with the original
\acro{CM} fonts.  It's a compact solution, within its restrictions.

Other \plaintex{} approaches to the problem (packages
\Package{plnfss}, \Package{fontch} and \Package{ofs}) break out of the
\plaintex{} model, towards the sort of font selection provided by
\CONTeXT{} and \LaTeX{}~--- font selection that allows you to change
family, as well as size and face.  The remaining packages all make
provision for using encodings other than Knuth's \acro{OT}1.

\Package{Plnfss} has a rather basic set of font family details;
however, it is capable of using font description (\extension{fd}) files
created for \LaTeX{}.  (This is useful, since most modern mechanisms
for integrating outline fonts with \TeX{} generate \extension{fd} files
in their process.)

\Package{Fontch} has special provision for \acro{T}1 and \acro{TS}1
encodings, which you select by arcane commands, such as:
\begin{quote}
\begin{verbatim}
\let\LMTone\relax
\input fontch.tex
\end{verbatim}
\end{quote}
for \acro{T}1.

\Package{Ofs} seems to be the most thoroughly thought-through of the
alternatives, and can select more than one encoding: as well as
\acro{T}1 it covers the encoding \acro{IL}2, which is favoured in the
Czech Republic and Slovakia.  \Package{Ofs} also covers mathematical fonts,
allowing you the dubious pleasure of using fonts such as the % ! line break
\Qref*{\FontName{pxfonts} and \FontName{txfonts}}{Q-psfchoice}.

The \Package{pdcmac} \plaintex{} macro package aims to be a complete
document preparation environment, like \Qref*{\Eplain{}}{Q-eplain}.  One
of its components is a font selection scheme, \Package{pdcfsel}, which
is rather simple but adequately powerful for many uses.  The package
doesn't preload fonts: the user is required to declare the fonts the
document is going to use, and the package provides commands to select
fonts as they're needed.  The distribution includes a configuration to
use Adobe `standard' fonts for typesetting text.  (\Eplain{} itself
seems not to offer a font selection scheme.)

The \Package{font-change} collection takes a rather different
approach~--- it supplies what are (in effect) a series of templates
that may be included in a document to change font usage.  The
package's documentation shows the effect rather well.

Simply to change font \emph{size} in a document (i.e., not changing
the default font itself), may be done using the rather straightforward
\Package{varisize}, which offers font sizes ranging from 7~points to
20~points (nominal sizes, all).  Font size commands are generated when
any of the package files is loaded, so the \File{11pt.tex} defines a
command \csx{elevenpoint}; each of the files ensures there's a ``way
back'', by defining a \csx{tenpoint} command.
\begin{ctanrefs}
\item[ec-plain]\CTANref{ec-plain}
\item[font-change]\CTANref{font-change}
\item[fontch]\CTANref{fontch}
\item[font_selection]\CTANref{font_selection}
\item[ofs]\CTANref{ofs}
\item[pdcmac]\CTANref{pdcmac}
\item[plnfss]\CTANref{plnfss}
\item[varisize]\CTANref{varisize}
\end{ctanrefs}

\subsection{Particular font families}

\Question[Q-concrete]{Using the ``Concrete'' fonts}

The Concrete Roman fonts were designed by Don Knuth for a book called
``Concrete Mathematics'', which he wrote with Graham and Patashnik
(\emph{the} Patashnik, of \BibTeX{} fame).  Knuth
only designed text fonts, since the book used the Euler fonts for
mathematics.  The book was typeset using \plaintex{}, of course, with
additional macros that may be viewed in a file \File{gkpmac.tex},
which is available on \acro{CTAN}.

The packages \Package{beton}, \Package{concmath}, and
\Package{ccfonts} are \LaTeX{} packages that change the default text
fonts from Computer Modern to Concrete.  Packages \Package{beton} and
\Package{ccfonts} also slightly increase the default value of
\csx{baselineskip} to account for the rather heavier weight of the
Concrete fonts.  If you wish to use the \FontName{Euler} fonts for
mathematics, as Knuth did, there's the \Package{euler} package which
has been developed from Knuth's own \plaintex{}-based set: these
macros are currently deprecated (they clash with many things, including
\AMSLaTeX{}).  The independently-developed \Package{eulervm}
bundle is therefore preferred to the \Package{euler} package.  (Note
that installing the \Package{eulervm} bundle involves installing a
series of virtual fonts.  While most modern distributions seem to have
the requisite files installed by default, you may find you have to
install them.  If so, see the file \File{readme} in the
\Package{eulervm} distribution.)

A few years after Knuth's original design, Ulrik Vieth
designed the Concrete Math fonts.  Packages
\Package{concmath}, and \Package{ccfonts} also change the default math
fonts from Computer Modern to Concrete and use the Concrete versions
of the \acro{AMS} fonts (this last behaviour is optional in the case
of the \Package{concmath} package).

There are no bold Concrete fonts, but it is generally accepted that
the Computer Modern Sans Serif demibold condensed fonts are an
adequate substitute.  If you are using \Package{concmath} or
\Package{ccfonts} and you want to follow this suggestion, then use the
package with \pkgoption{boldsans} class option (in spite of the fact
that the \Package{concmath} documentation calls it
\pkgoption{sansbold} class option). If you are using \Package{beton},
add
\begin{quote}
  \cmdinvoke{renewcommand}{\csx{bfdefault}}{sbc}
\end{quote}
to the preamble of your document.

Type~1 versions of the fonts are available.  For \acro{OT}1 encoding,
they are available from \Qref*{MicroPress}{Q-psfchoice}.  The
\Qref*{CM-Super fonts}{Q-textrace} contain Type~1 versions
of the Concrete fonts in \acro{T}1 encoding.
\begin{ctanrefs}
\item[beton.sty]\CTANref{beton}
\item[ccfonts.sty]\CTANref{ccfonts}
\item[\nothtml{\rmfamily}CM-Super fonts]\CTANref{cm-super}
\item[concmath.sty]\CTANref{concmath}
\item[\nothtml{\rmfamily}Concmath fonts]\CTANref{concmath-f}
\item[\nothtml{\rmfamily}Concrete fonts]\CTANref{concrete}
\item[euler.sty]\CTANref{euler-latex}
\item[eulervm \nothtml{\rmfamily}bundle]\CTANref{eulervm}
\item[gkpmac.tex]\CTANref{gkpmac}
\end{ctanrefs}

\Question[Q-uselmfonts]{Using the Latin Modern fonts}

The \Package{lm} fonts are an exciting addition to
the armoury of the \AllTeX{} user: high quality outlines of fonts that
were until recently difficult to obtain, all in a free and
relatively compact package.  However, the spartan information file
that comes with the fonts remarks ``It is presumed that a potential
user knows what to do with all these files''.  This answer aims to
fill in the requirements: the job is really not terribly difficult.

Note that te\TeX{} distributions, from version~3.0, already have the
\Package{lm} fonts: all you need do is use them.  The fonts may also
be installed via the package manager, in a current \miktex{} system.
The remainder of this answer, then, is for people who don't use such
systems.

The font (and related) files appear on \acro{CTAN} as a set of
single-entry \Qref*{\acro{TDS} trees}{Q-tds}~---
\File{fonts}, \File{dvips}, \File{tex} and \File{doc}.  The \File{doc}
subtree really need not be copied (it's really a pair of sample
files), but copy the other three into your existing Local
\texttt{\$TEXMF} tree, and
% beware line break
\Qref*{update the filename database}{Q-inst-wlcf}.

Now, incorporate the fonts in the set searched by \PDFLaTeX{},
\ProgName{dvips}, \ProgName{dvipdfm}/\ProgName{dvipdfmx}, your
previewers and Type~1-to-\acro{PK} conversion programs, by
\begin{itemize}
\item On a te\TeX{} system earlier than version~2.0, edit the file
  \File{$TEXMF/dvips/config/updmap} %$
  and insert an absolute path for the \File{lm.map} just after the
  line that starts \texttt{extra\_modules="} (and before the closing
  quotes).
\item On a te\TeX{} version~2.0 (or later), execute the command
\begin{quote}
\begin{verbatim}
updmap --enable Map lm.map
\end{verbatim}
\end{quote}
\item On a \miktex{} system earlier than version~2.2, the ``Refresh
  filename database'' operation, which you performed after installing
  files, also updates the system's ``\PS{} resources database''.
\item On a \miktex{} system, version~2.2 or later, update
  \File{updmap.cfg} as described in the \miktex{} % beware line break
  \href{http://docs.miktex.org/manual/psfonts.html#chgupdmapcfg}{online documentation}.
  Then execute the command \texttt{initexmf -\relax-mkmaps}, and the
  job is done.
\end{itemize}

To use the fonts in a \LaTeX{} document, you should
\begin{quote}
  \cmdinvoke{usepackage}{lmodern}
\end{quote}
this will make the fonts the default
for all three \LaTeX{} font families (``roman'', ``sans-serif'' and
``typewriter'').  You also need 
\begin{quote}
  \cmdinvoke{usepackage}[T1]{fontenc}
\end{quote}
for text, and 
\begin{quote}
  \cmdinvoke{usepackage}{textcomp}
\end{quote}
if you want to use any of the \acro{TS}1-encoding symbols.  There is
no support for using fonts according to the \acro{OT}1 encoding.
\begin{ctanrefs}
\item[\nothtml{\rmfamily}Latin Modern fonts]\CTANref{lm}
\end{ctanrefs}

\Question[Q-getnff]{Getting `free' fonts not in your distribution}

Some fonts are free to use, but may not be sold.  This creates a
dilemma for distributions: users may want the fonts, but since the
distribution is also available on a \acro{DVD} for sale, the fonts may
not be in the distribtution.

The \acro{CTAN} archives hold such fonts, together with all the
necessary support files, but even with the support files ready-made,
installing a font is a tedious business.

For \texlive{} users, this dilemma is solved by the
\ProgName{getnonfreefonts} script.  Download the script installer from
\url{http://tug.org/fonts/getnonfreefonts/}; the web page tells you
how to run the installer to get the script, and what fonts are
currently available

Once the script is installed, you can ask it what it has available by
saying:
\begin{quote}
\begin{verbatim}
getnonfreefonts -l
\end{verbatim}
\end{quote}
and you can ask it to install a font (in your local texmf tree) by:
\begin{quote}
\begin{verbatim}
getnonfreefonts luximono
\end{verbatim}
\end{quote}
(for example; the printed version of the \acro{FAQ} uses luximono, so
that the example was to hand\dots{}).

(System adminstrators may use \ProgName{getnonfreefonts-sys}, which
will install the font in the `public' \texttt{texmf} tree, so that all
users of the system may use the new font.)

The script will download the relevant font files from \acro{CTAN},
extract them from their \extension{zip} file, install them and update
the font maps.  It even goes so far as to apologise for how long it's
taking!
\LastEdit*{2013-10-22}

\subsection{\MF{} fonts}

\Question[Q-useMF]{Getting \MF{} to do what you want}

\MF{} allows you to create your own fonts, and most \TeX{} users
will never need to use it~--- modern \AllTeX{} systems contain
rather few \MF{} fonts of any significance, and when \MF{} output is
needed the font generation is done, automatically, ``on the fly''.

If you find you have some special requirement that the system doesn't
satisfy, you need to know about \MF{} in rather more detail.  \MF{},
unlike \TeX{}, requires customisation for each output device: such
customisation is conventionally held in a ``mode'' associated with the
device.  Modes are commonly defined using the \texttt{mode\_def}
convention described on page~94 of \emph{The \MF{}book} % ! line break
(see \Qref[question]{\TeX{}-related books}{Q-other-books}).  Your
distribution should provide
a file, conventionally called \File{local.mf}, containing all the
\texttt{mode\_def}s you will be using. In the unlikely event that
\File{local.mf} doesn't already exist, Karl Berry's collection of
modes (\File{modes.mf}) is a good starting point 
(it can be used as a `\File{local.mf}' without modification in a
modern implementation of \MF{}).  
Settings for new output devices are added to \File{modes.mf} as they
become available.

Now create
a \texttt{plain} base file using \ProgName{mf} (in ``initialisation''
mode), \texttt{plain.mf}, and \texttt{local.mf}: 
\begin{quote}
\begin{verbatim}
% mf -ini
This is METAFONT...
**plain # you type plain
(output)
*input local # you type this
(output)
*dump # you type this
Beginning to dump on file plain...
(output)
\end{verbatim}
\end{quote}
This will create a base file named \File{plain.base} (or something
similar; for example, it will be \File{PLAIN.BAS} on \MSDOS{}
systems).  Move the file to the directory containing the base files on
your system, and run \ProgName{texhash} as necessary.

Now you need to make sure \MF{} loads this new base when it starts up. If
\MF{} loads the \texttt{plain} base by default on your system, then you're
ready to go. Under Unix (using the default \texlive{} (and earlier)
distributions this does indeed happen, but we could for instance
define a command \ProgName{plainmf}%
\begin{footnoteenv}
  On the grounds that a command \ProgName{plain} could be misconstrued
  as a reference to \plaintex{}
\end{footnoteenv}
which executes `\texttt{mf -base=plain}' (or, in more traditional
style `\texttt{mf \&plain}') which loads the \texttt{plain} base
file.

The usual way to create a font with \MF{} (with an appropriate base
file loaded) is to start \MF{}'s input with the
\begin{narrowversion}
  lines
  \begin{quote}
    \csx{mode}\texttt{=<mode name>;}\\
    \texttt{mag=<magnification>;}\\
    \texttt{input <font file name>}
  \end{quote}
\end{narrowversion}
\begin{wideversion}
line
\begin{quote}
\begin{verbatim}
\mode=<mode name>; mag=<magnification>; input <font file name>
\end{verbatim}
\end{quote}
\end{wideversion}
in response to the `\texttt{**}' prompt or on the \MF{} command line. (If
\texttt{<mode name>} is unknown or omitted, the mode defaults to
`proof' mode and \MF{} will produce an output file called % ! line break
\File{<font file name>.2602gf})
The \texttt{<magnification>} is a floating point number or a
`magstep' (magsteps define sizes by stating how many times you need to
multiply a base size by \texttt{1.2}, so for a base size of
\texttt{10}, \texttt{magstep 1} is \texttt{12}, \texttt{magstep 2} is
\texttt{14.4}
If \texttt{mag=<magnification>} is omitted, then the default
is \texttt{1}~(\texttt{magstep 0}).  For example, to generate
\FontName{cmr10} at \texttt{12pt} for an Epson,
printer you might type
\begin{quote}
\begin{verbatim}
mf \mode=epson; mag=magstep 1; input cmr10
\end{verbatim}
\end{quote}
Note that under Unix the \texttt{\textbackslash } and \texttt{;}
characters must usually be quoted or escaped, so this would typically
look something like
\begin{quote}
\begin{verbatim}
 mf '\mode=epson; mag=magstep 1; input cmr10'
\end{verbatim}
\end{quote}
If you need a special mode that isn't in the base, you can put its
commands in a file (\emph{e.g.}, \File{ln03.mf}) and invoke it on the
fly with the \csx{smode} command.  For example, to create
\File{cmr10.300gf} for an \acro{LN}03 printer, using the file
\begin{quote}
\begin{verbatim}
% This is ln03.mf as of 1990/02/27
% mode_def courtesy of John Sauter
proofing:=0;
fontmaking:=1;
tracingtitles:=0;
pixels_per_inch:=300;
blacker:=0.65;
fillin:=-0.1;
o_correction:=.5;
\end{verbatim}
\end{quote}
(note the absence of the \texttt{mode\_def} and \texttt{enddef}
commands), you would type
\begin{quote}
\begin{verbatim}
mf \smode="ln03"; input cmr10
\end{verbatim}
\end{quote}
This technique isn't one you should regularly use, but it may
prove useful if you acquire a new printer and want to experiment with
parameters, or for some other reason are regularly editing the
parameters you're using.  Once you've settled on an appropriate set of
parameters, you should use them to rebuild the base file that you use.

Other sources of help are discussed in our list of % ! line break
\Qref*{\MF{} and \MP{} Tutorials}{Q-mfptutorials}.
\begin{ctanrefs}
\item[modes.mf]\CTANref{modes-file}
\end{ctanrefs}
\LastEdit{2011-09-04}

\Question[Q-keepfonts]{Which font files should be kept}

\MF{} produces from its run three files, a metrics (\acro{TFM}) file, a
generic font (\acro{GF}) file, and a log file; all of these files have the
same base name as does the input (\emph{e.g.}, if the input file was
\File{cmr10.mf}, the outputs will be \File{cmr10.tfm},
\File{cmr10.nnngf} (the file name may be mangled if you are using an
operating system which doesn't permit long file names)
and \File{cmr10.log}).

For \TeX{} to use the font, you need a \acro{TFM} file, so you need
to keep that.  However, you are likely to generate the same font
at more than one magnification, and each time you do so you'll
(incidentally) generate another \acro{TFM} file; these files are
all the same, so you only need to keep one of them.

To preview or to produce printed output, the \acro{DVI} processor will need a
font raster file; this is what the \acro{GF} file provides.  However, while
there used (once upon a time) to be \acro{DVI} processors that could use 
\acro{GF} files, modern processors  use
packed raster (\acro{PK}) files (incidentally, \pdftex{} also uses
\acro{PK} files if nothing ``better'' is available, but
see \Qref[]{fuzzy fonts in \acro{PDF}}{Q-fuzzy-type3}).
Therefore, you need to generate a \acro{PK} file from the \acro{GF} file; the
program \ProgName{gftopk} does this for you, and once you've done that you
may throw the \acro{GF} file away.

The log file should never be needed again, unless there was some sort
of problem in the \MF{} run, and need not therefore be kept.
\LastEdit{2011-09-04}

\Question[Q-getbitmap]{Acquiring bitmap fonts}

When \acro{CTAN} was young, most people would start using \TeX{}
with a 300 dots-per-inch (dpi) laser printer, and sets of Computer
Modern bitmap fonts for this resolution are available on \acro{CTAN}.
(There are separate sets for write-black and write-white printers, as
well as sets at 120~dpi and 240~dpi.)

There used to regular requests that \acro{CTAN} should hold
a wider range of resolutions, but they were resisted for two reasons:
\begin{itemize}
\item The need to decide which printers to generate fonts for.  The
  broad-brush approach taken for 300~dpi printers was (more or less)
  justified back then, given the dominance of certain printer
  `engines', but nowadays one could not make any such assumption.
\item Given the above, it has been near-impossible to justify the
  space that would be required by a huge array of bitmap fonts.
\end{itemize}
Fortunately, \AllTeX{} distribution technology has put a stop to these
arguments: most (if not all) current distributions generate bitmap
fonts as needed, and cache them for later re-use.  The impatient
user, who is determined that all bitmap fonts should be created once
and for all, may be supported by scripts such as \ProgName{allcm}
(distributed with \texlive{}, at least; otherwise such a
person should consult ``\Qref*{the use of \MF{}}{Q-useMF})''.

If your output is to a \PS{}-capable device, or if your output is
destined to be converted to \acro{PDF}, you should switch to
using Type~1 versions of the \acro{CM} fonts.  Two free
sets are available; the older (\Package{bakoma}) is
somewhat less well produced than the \Package{bluesky} fonts, which were
originally professionally produced and sold, but were then released
for general public use by their originators \YandY{} and Bluesky Research,
in association with the \acro{AMS} and other scientific publishers
(they are nowadays available under the SIL's Open Fonts Licence).  The
two sets contain slightly different ranges of fonts, but you are
advised to use the \Package{bluesky} set except when \Package{bakoma}
is for some reason absolutely unavoidable.  In recent years, several
other `\MF{}' fonts have been converted to Type~1 format; it's
uncommon ever to need to generate bitmap fonts for any purpose other
than previewing~--- see % ! line wrap
\Qref*{``previewing documents with Type~1 fonts''}{Q-PSpreview}~---
if even then.

More modern fonts may be used in place of the Computer Modern set.  The
\Qref*{\acro{EC} fonts}{Q-ECfonts} and the % ! line break
\Qref*{Latin Modern fonts}{Q-uselmfonts} are both close relatives with
wider ranges of glyphs to offer.
\begin{ctanrefs}
\item[\nothtml{\rmfamily}BaKoMa fonts]\CTANref{bakoma}
\item[\nothtml{\rmfamily}Bluesky fonts]Distributed as part of
  \CTANref{amsfonts}
\item[\nothtml{\rmfamily}CM fonts (write-black printers)]\CTANref{pk300}
\item[\nothtml{\rmfamily}CM fonts (write-white printers)]\CTANref{pk300w}
\item[\nothtml{\rmfamily}EC fonts (Type 1 format)]\CTANref{cm-super}
\item[\nothtml{\rmfamily}Latin Modern fonts]\CTANref{lm}
\end{ctanrefs}
\LastEdit{2011-09-4}
