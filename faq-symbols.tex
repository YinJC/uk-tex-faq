% $Id: faq-symbols.tex,v 1.15 2014/01/28 18:17:36 rf10 Exp rf10 $

\section{Symbols, etc.}

\Question[Q-usesymb]{Using symbols}

Most symbol font sets come with a package that defines commands for
every symbol in the font.  While this is convenient, it can lead to
difficulties, particularly with name clashes when you load packages
that cover fonts which duplicate symbols~--- an issue which is
discussed in
\begin{narrowversion}
  \Qref[question]{}{Q-alreadydef}.
\end{narrowversion}
\begin{wideversion}
  ``\Qref{symbol already defined}{Q-alreadydef}''.
\end{wideversion}
Some font sets (for example the related set: \FontName{FdSymbol},
\FontName{MdSymbol} and \FontName{MnSymbol}) are huge, and the
accompanying macros cover so many symbols that name clashes are surely
a serious problem.

The \Package{pifont} package (originally designed to use the Adobe
\FontName{Zapf Dingbats} font) avoids this sort of problem: it requires
you to know the font position of any symbol you want to use (the
documentation provides font tables).  The basic command is
\cmdinvoke*{ding}{number} for a single symbol; there are commands for
other fancier uses.  \Package{Pifont} also allows you to select other
fonts, for similar use.

The \Package{yagusylo} describes itself as ``an extended version of
\Package{pifont}, gone technicolor''.  It provides all the facilities
of \Package{pifont}, but allows you to create your own mnemonic names
for symbols.  Thus, while you can say % ! line break
\cmdinvoke*{yagding}[family]{symbol number}[colour], you can also
define symbol names with \csx{defdingname}, and then use them
with \cmdinvoke*{yagding*}{symbol name} (the defined name carries the
font family and colour specified in the arguments of
\csx{defdingname}).

\Package{Yagusylo} is somewhat complicated, but its documentation is
clear; it is probably the best tool to use for picking and choosing
symbols from a variety of font families.
\begin{ctanrefs}
\item[FdSymbol \nothtml{\rmfamily}fonts]\CTANref{fdsymbol}
\item[MdSymbol \nothtml{\rmfamily}fonts]\CTANref{mdsymbol}
\item[MnSymbol \nothtml{\rmfamily}fonts]\CTANref{mnsymbol}
\item[pifont.sty]Distributed as part of \CTANref{psnfss}
\item[yagusylo.sty]\CTANref{yagusylo}
\end{ctanrefs}

\Question[Q-numbersets]{Symbols for the number sets}

Mathematicians commonly use special lettering for the real numbers and
other standard number sets. Traditionally these were typeset in bold.
In the ordinary course of events, but mathematicians do not have
access to bold chalk, so they invented special symbols that are now
often used for the number sets.  Such symbols are known as
``blackboard bold'' (or double-stroked) letters; in place of the heavier
strokes of a bold font, (some) strokes of the letters are doubled.
The minimum useful set is upper-case letters `I', `N', `R', `Q' and
`Z'; some fonts offer a figure `1' (for a unit matrix~--- not a number
set at all).

A set of blackboard bold capitals is available in the \acro{AMS}
\FontName{msbm} fonts (\FontName{msbm} is available at a range of
design sizes, with names such as \FontName{msbm10}).  The \acro{AMS}
actually provides a pair of font
families (the other is called \FontName{msam}), which offer a large number of
mathematical symbols to supplement those provided in Knuth's fonts.
The fonts are available in Type~1 format in
modern distributions.   Support for using the fonts under
\LaTeX{} is available in packages \Package{amssymb} and
\Package{amsfonts}.  The font shape is a rather
austere sans, which many people don't like (though it captures the
essence of quickly-chalked writing rather well).

The \FontName{bbold} family is set of blackboard bold fonts written in
\MF{}.  This set offers blackboard bold forms of lower-case letters;
the font source directory also contains sources for a \LaTeX{} package
that enables use of the fonts.  The fonts are not available in Type~1 format.

The \FontName{bbm} family claims to provide
`blackboard' versions of most of the \FontName{cm} fonts~\dots{} including
the bold and bold-extended series.  Again, the fonts are designed in
\MF{} and are not available in Type~1 format.  \LaTeX{} macro support
comes from a package by Torsten Hilbrich.

The \FontName{doublestroke} family comes in just roman
and sans shapes, at a single weight, and is available both as \MF{}
sources and as Type~1; the font covers the uppercase latin letters,
lowercase `h' and 'k', and the digit `1'.

A document that shows the \FontName{bbm}, \FontName{bbold},
\FontName{doublestroke} and \FontName{msbm} fonts, so that you can get
a feel for their appearance, is available (\acro{CTAN} package
\Package{blackboard}).

The \Package{boondox} font set consists of Type~1 versions of the
\acro{STIX} mathematics set (the originals are distributed in
\acro{OTF} format).  The set contains a font
`BOONDOXDoubleStruck-Regular' (blackboard bold) (as well as a `bold'
version of that.

An alternative source of Type~1 fonts with blackboard bold characters
may be found in the steadily increasing set of complete families, both
commercial and free, that have been prepared for use with \AllTeX{}
(see % beware line break
\htmlonly{``}\Qref[question]{choice of outline fonts}{Q-psfchoice}\htmlonly{''}).
Of the free sets, the \FontName{txfonts} and \FontName{pxfonts} families
both come with replicas of \FontName{msam} and \FontName{msbm}, but
(as noted elsewhere, there are other reasons not to use these fonts);
revised versions of the fonts, \FontName{newtx} and \FontName{newpx}
are better adjusted.  The \FontName{mathpazo} family includes a
``mathematically significant'' choice of blackboard bold characters, and the
\FontName{fourier} fonts contain blackboard bold upper-case letters,
the digit `1', and lower-case `k'.

The ``lazy person's'' blackboard bold macros:
\begin{quote}
\begin{verbatim}
\newcommand{\R}{{\textsf{R}\hspace*{-0.9ex}%
  \rule{0.15ex}{1.5ex}\hspace*{0.9ex}}}
\newcommand{\N}{{\textsf{N}\hspace*{-1.0ex}%
  \rule{0.15ex}{1.3ex}\hspace*{1.0ex}}}
\newcommand{\Q}{{\textsf{Q}\hspace*{-1.1ex}%
  \rule{0.15ex}{1.5ex}\hspace*{1.1ex}}}
\newcommand{\C}{{\textsf{C}\hspace*{-0.9ex}%
  \rule{0.15ex}{1.3ex}\hspace*{0.9ex}}}
\end{verbatim}
\end{quote}
are almost acceptable at normal size if the surrounding text is
\FontName{cmr10} (the position of the vertical bar can be affected by
the surrounding font).  However, they are not part of a proper maths font,
and do not work in sub- and superscripts.  As we've seen, there are
plenty of alternatives: that mythical ``lazy'' person can inevitably
do better than the macros, or anything similar using capital `I'
(which looks even worse!).  Voluntary  \AllTeX{} effort has redefined
the meaning of laziness (in this respect!).
\begin{ctanrefs}
\item[AMS support files]Distributed as part of \CTANref{amsfonts}
\item[AMS symbol fonts]Distributed as part of \CTANref{amsfonts}
\item[bbm fonts]\CTANref{bbm}
\item[bbm macros]\CTANref{bbm-macros}
\item[bbold fonts]\CTANref{bbold}
\item[blackboard \nothtml{\rmfamily}evaluation set]\CTANref{blackboard}
\item[doublestroke fonts]\CTANref{doublestroke}
\item[fourier fonts]\CTANref{fourier}
\item[mathpazo fonts]\CTANref{mathpazo}
\item[newpx]\CTANref{newpx}
\item[newtx]\CTANref{newtx}
\item[pxfonts]\CTANref{pxfonts}
\item[txfonts]\CTANref{txfonts}
\end{ctanrefs}
\LastEdit{2013-12-04}

\Question[Q-scriptfonts]{Better script fonts for maths}

The font selected by \csx{mathcal} is the only script font `built
in'. However, there are other useful calligraphic fonts included with
modern \TeX{} distributions.
\begin{description}
\item[Euler] The \Package{eucal} package (part of most sensible \TeX{}
  distributions; the fonts are part of the \acro{AMS} font set) gives
  a slightly curlier font than the default. The package changes the
  font that is selected by \csx{mathcal}.
  
  Type 1 versions of the fonts are available in the \acro{AMS} fonts
  distribution.
\item[mathabx] The \FontName{mathabx} bundle provides calligraphic
  letters (in both upper and lower case); the fonts were developed in
  MetaFont, but a version in Adobe Type 1 format is available.  The
  bundle's documentation offers a series of comparisons of its
  calligraphic set with Computer Modern's (both regular mathematical
  and calligraphic letters); the difference are not large.
\item[mnsymbol] The \FontName{mnsymbol} bundle provides (among many
  other symbols) a set of calligraphic letters, though (again) they're
  rather similar to the default Computer Modern set.
\item[RSFS] The \Package{mathrsfs} package uses a really fancy script
  font (the name stands for ``Ralph Smith's Formal Script'') which is
  already part of most modern \TeX{} distributions (Type~1 versions of
  the font are also provided, courtesy of Taco Hoekwater).  The package
  creates a new command \csx{mathscr}.
\item[RSFSO] The bundle \Package{rsfso} provides a less dramatically
  oblique version of the \acro{RSFS} fonts; the result proves quite
  pleasing~--- similar to the effect of the the (commercial) script
  font in the Adobe Mathematical Pi collection.
\item[Zapf Chancery] is the standard \PS{} calligraphic font.  There
  is no package but you can easily make it available by means of the
  command
\begin{quote}
\begin{narrowversion}
\begin{verbatim}
\DeclareMathAlphabet{\mathscr}{OT1}{pzc}%
                                 {m}{it}
\end{verbatim}
\end{narrowversion}
\begin{wideversion}
\begin{verbatim}
\DeclareMathAlphabet{\mathscr}{OT1}{pzc}{m}{it} 
\end{verbatim}
\end{wideversion}
\end{quote}
  in your preamble.  You may find the font rather too big; if so, you
  can use a scaled version of it like this:
\begin{quote}
\begin{narrowversion}
\begin{verbatim}
\DeclareFontFamily{OT1}{pzc}{}
\DeclareFontShape{OT1}{pzc}{m}{it}%
             {<-> s * [0.900] pzcmi7t}{}
\DeclareMathAlphabet{\mathscr}{OT1}{pzc}%
                                 {m}{it}
\end{verbatim}
\end{narrowversion}
\begin{wideversion}
\begin{verbatim}
\DeclareFontFamily{OT1}{pzc}{}
\DeclareFontShape{OT1}{pzc}{m}{it}{<-> s * [0.900] pzcmi7t}{}
\DeclareMathAlphabet{\mathscr}{OT1}{pzc}{m}{it}
\end{verbatim}
\end{wideversion}
\end{quote}
  Adobe Zapf Chancery (which the above examples use) is distributed in
  any but the most basic \PS{} printers.  A substantially identical
  font (to the extent that the same metrics may be used) is
  available from \acro{URW}, called URW Chancery L: it is distributed
  as part of the ``URW base35'' bundle; the
  \Package{urwchancal} package (which includes virtual fonts to tweak
  appearance) provides for its use as a calligraphic font.

  The TeX Gyre font family also includes a Chancery replacement,
  \FontName{Chorus}; use it with \Package{tgchorus} (and ignore the
  complaints about needing to change font shape).
\end{description}
Examples of the available styles are linked from the packages'
catalogue entries.
\begin{ctanrefs}
\item[eucal.sty]Distributed as part of \CTANref{amsfonts}
\item[euler fonts]Distributed as part of \CTANref{amsfonts}
\item[mathabx \nothtml{\rmfamily}as \MF{}]\CTANref{mathabx}
\item[mathabx \nothtml{\rmfamily}in Type 1 format]\CTANref{mathabx-type1}
\item[mathrsfs.sty]Distributed as part of \CTANref{jknappen-macros}[mathrsfs]
\item[mnsymbol \nothtml{\rmfamily}fonts]\CTANref{mnsymbol}
\item[rsfs \nothtml{\rmfamily}fonts]\CTANref{rsfs}
\item[rsfso \nothtml{\rmfamily}fonts]\CTANref{rsfso}
\item[Script font examples]\CTANref{mathscript}
\item[TeX Gyre Chorus font family]Distributed as part of \CTANref{tex-gyre}[tex-gyre-chorus]
\item[urwchancal]\CTANref{urwchancal}[urwchancal]
\item[URW Chancery L]Distributed as part of \CTANref{urw-base35}
\end{ctanrefs}
\LastEdit{2011-08-17}

\Question[Q-boldgreek]{Setting bold Greek letters in \LaTeX{} maths}

The issue here is complicated by the fact that \csx{mathbf} (the
command for setting bold \emph{text} in \TeX{} maths) affects a select
few mathematical `symbols' (the uppercase Greek letters).

In the default configuration, lower-case Greek letters behave
differently from upper-case Greek letters (the lower-case greek
letters are in the maths fonts, while the upper-case letters are in
the original (OT1-encoded) text fonts).

The \plaintex{} solution \emph{does} work, in a limited way; you set a
maths \emph{style}, before you start an equation; thus
\begin{quote}
\begin{verbatim}
{\boldmath$\theta$}
\end{verbatim}
\end{quote}
does the job, but \csx{boldmath} may not be used in maths mode.  As a
result, this solution requires that you embed single bold characters
in a text box:
\begin{quote}
\begin{verbatim}
$... \mbox{\boldmath$\theta$} ...$
\end{verbatim}
\end{quote}
which then causes problems in superscripts, etc.  With
\Package{amsmath} loaded,
\begin{quote}
\begin{verbatim}
$... \text{\boldmath$\theta$} ...$
\end{verbatim}
\end{quote}
does the trick (and is less bad in regard to superscripts, etc), but
is an unsatisfactory solution, too.

These problems may be addressed by using a bold mathematics package.
\begin{itemize}
\item The \Package{bm} package, which is part of the \LaTeX{} tools
  distribution, defines a command \csx{bm} which may be used anywhere
  in maths mode.
\item The \Package{amsbsy} package (which is part of \AMSLaTeX{})
  defines a command \csx{boldsymbol}, which (though slightly less
  comprehensive than \csx{bm}) covers almost all common cases.
\end{itemize}

All these solutions apply to all mathematical symbols, not merely
Greek letters.
\begin{ctanrefs}
\item[bm.sty]Distributed as part of \CTANref{2etools}[bm]
\item[amsbsy.sty]Distributed as part of \AMSLaTeX{} \CTANref{amslatex}[amsbsy]
\item[amsmath.sty]Distributed as part of \AMSLaTeX{}
  \CTANref{amslatex}[amsmath]
\end{ctanrefs}
\LastEdit{2013-06-12}

\Question[Q-prinvalint]{The Principal Value Integral symbol}

This symbol (an integral sign, `crossed') does not appear in any of
the fonts ordinarily available to \AllTeX{} users, but it can be
created by use of the following macros:
\begin{quote}
\begin{wideversion}
\begin{verbatim}
\def\Xint#1{\mathchoice
   {\XXint\displaystyle\textstyle{#1}}%
   {\XXint\textstyle\scriptstyle{#1}}%
   {\XXint\scriptstyle\scriptscriptstyle{#1}}%
   {\XXint\scriptscriptstyle\scriptscriptstyle{#1}}%
   \!\int}
\def\XXint#1#2#3{{\setbox0=\hbox{$#1{#2#3}{\int}$}
     \vcenter{\hbox{$#2#3$}}\kern-.5\wd0}}
\def\ddashint{\Xint=}
\def\dashint{\Xint-}
\end{verbatim}
\end{wideversion}
\begin{narrowversion}
\begin{verbatim}
\def\Xint#1{\mathchoice
   {\XXint\displaystyle\textstyle{#1}}%
   {\XXint\textstyle\scriptstyle{#1}}%
   {\XXint\scriptstyle\scriptscriptstyle{#1}}%
   {\XXint\scriptscriptstyle
                      \scriptscriptstyle{#1}}%
   \!\int}
\def\XXint#1#2#3{{%
     \setbox0=\hbox{$#1{#2#3}{\int}$}
     \vcenter{\hbox{$#2#3$}}\kern-.5\wd0}}
\def\ddashint{\Xint=}
\def\dashint{\Xint-}
\end{verbatim}
\end{narrowversion}
\end{quote}
\csx{dashint} gives a single-dashed integral sign, \csx{ddashint} a
double-dashed one.

\Question[Q-underscore]{How to typeset an underscore character}

The underscore character `\texttt{\_}' is ordinarily used in \TeX{} to
indicate a subscript in maths mode; if you type \texttt{\_}, on its
own, in the course of ordinary text, \TeX{} will complain.  The
``proper'' \latex{} command for underscore is \csx{textunderscore},
but the \latexo{} command \csx{\_} is an established alias.  Even so,
if you're writing a document which will contain a large number of
underscore characters, the prospect of typing \csx{\_} for every one
of them will daunt most ordinary people.

Moderately skilled macro programmers can readily generate a quick hack
to permit typing `\texttt{\_}' to mean `text underscore' (the answer in
\begin{hyperversion}
  ``\Qref{defining characters as macros}{Q-activechars}''
\end{hyperversion}
\begin{flatversion}
  \Qref[question]{}{Q-activechars}
\end{flatversion}
uses this example to illustrate its techniques).
However, the code \emph{is} somewhat tricky, and more importantly
there are significant points where it's easy to get it wrong.  There
is therefore a package \Package{underscore} which provides a general
solution to this requirement.

There is a problem, though: \acro{OT}1 text fonts don't contain an
underscore character, unless they're in the typewriter version of the
encoding (used by fixed-width fonts such as \texttt{cmtt}).  In place
of such a character, \latex{} (in \acro{OT}1 encoding) uses a short rule
for the command \csx{textunderscore}, but this poses problems
for systems that interpret \acro{PDF}~--- for example those
\acro{PDF}-to-voice systems used by those who find reading difficult.

So either you must ensure that your underscore characters only occur
in text set in a typewriter font, or you must use a more modern
encoding, such as \acro{T}1, which has the same layout for every font,
and thus an underscore in every font.

A stable procedure to achieve this is:
\begin{quote}
\begin{verbatim}
% (1) choose a font that is available as T1
% for example:
\usepackage{lmodern}

% (2) specify encoding
\usepackage[T1]{fontenc}

% (3) load symbol definitions
\usepackage{textcomp}
\end{verbatim}
\end{quote}
which will provide a command \csx{textunderscore} which robustly
selects the right character.  The \Package{underscore} package,
mentioned above, will use this command.
\begin{ctanrefs}
\item[underscore.sty]\CTANref{underscore}
\end{ctanrefs}
\LastEdit{2011-08-17}

\Question[Q-atsign]{How to type an `@' sign?}

Long ago, some packages used to use the `@' sign as a special
character, so that special measures were needed to type it.  While
those packages are still in principle available, few people use them,
and those that do use them have ready access to rather good
documentation.

Ordinary people (such as the author of this \acro{FAQ}) need only type
`@'.
\LastEdit{2011-08-16}

\Question[Q-euro]{Typesetting the Euro sign}

The European currency ``Euro'' (\texteuro {}) is represented by a symbol
of somewhat dubious design, but it's an important currency and
\AllTeX{} users need to typeset it.  When the currency first appeared,
typesetting it was a serious problem for \AllTeX{} users; things are
easier now (most fonts have some way of providing a Euro sign), but
this answer provides a summary of methods ``just in case''.

Note that the Commission of the European Community at first deemed
that the Euro symbol should always be set in a sans-serif font;
fortunately, this eccentric ruling has now been rescinded, and one may
apply best typesetting efforts to making it appear at least slightly
``respectable'' (typographically).

The \acro{TS}1-encoded \acro{TC} fonts provided as part of the \acro{EC} font
distribution provide Euro glyphs.  The fonts are called Text Companion
(\acro{TC}) fonts, and offer the same range
of faces as do the \acro{EC} fonts themselves.  The
\Package{textcomp} package provides a \csx{texteuro} command for
accessing the symbol, which selects a symbol to match the surrounding
text.  The design of the symbol in the \acro{TC} fonts is not
universally loved\dots{}
Nevertheless, use the \acro{TC} font version of the symbol if you are
producing documents using Knuth's Computer Modern Fonts.

The each of the \Package{latin9} and \Package{latin10} input encoding
definitions for the \Package{inputenc} package has a euro character
defined (character position 164, occupied in other \acro{ISO} Latin
character sets by the ``currency symbol'' \textcurrency {}, which
ordinary people seldom see except in character-set listings\dots{}).
The \acro{TC} encoding file offers the command \csx{texteuro} for the
character; that command is (probably) \emph{only} available from the
\Package{textcomp} package.

Use of the \acro{TC} encoding character may therefore made via
\csx{texteuro} or via the Latin-9 or Latin-10 character in ordinary
text.

Note that there is a Microsoft code page position (128), too, and that has
been added to \Package{inputenc} tables for \acro{CP}1252 and
\acro{CP}1257.  (There's another position in \acro{CP}858, which has
it in place of ``dotless i'' in \acro{CP850}; the standardisation of
these things remains within Microsoft, so one can never tell what will
come next\dots{})

Outline fonts which contain nothing but Euro symbols are available
(free) from
\href{ftp://ftp.adobe.com/pub/adobe/type/win/all/eurofont.exe}{Adobe}\nobreakspace---
the file is packaged as a \ProgName{Windows} self-extracting
executable, but it may be decoded as a \extension{zip} format archive
on other operating systems.
The \Package{euro} bundle contains metrics, \ProgName{dvips} map
files, and macros (for \plaintex{} and \LaTeX{}), for using these
fonts in documents.  \LaTeX{} users will find two packages in the
bundle: \Package{eurosans} only offers the sans-serif version (to
conform with the obsolete ruling about sans-serif-only symbols; the
package provides the
command \csx{euro}), whereas \Package{europs} matches the Euro symbol
with the surrounding text (providing the command \csx{EUR}).  To use
either package
with the \Package{latin9} encoding, you need to define \csx{texteuro}
as an alias for the euro command the package defines.

The Adobe fonts are probably the best bet for use in non-Computer
Modern environments.  They are apparently designed to fit with Adobe
Times, Helvetica and Courier, but can probably fit with a wider range
of modern fonts.

The \Package{eurofont} package provides a compendious analysis of the
``problem of the euro symbol'' in its documentation, and offers macros
for configuring the source of the glyphs to be used; however, it seems
rather large for everyday use.

The \Package{euro-ce} bundle is a rather pleasing \MF{}-only design
providing Euro symbols in a number of shapes.  The file
\File{euro-ce.tex}, in the distribution, offers hints as to how a
\plaintex{} user might make use of the fonts.

Euro symbols are found in several other places, which we list here for
completeness.

The \Package{marvosym} font contains a Euro symbol (in a number of
typographic styles), among many other good things; the font is
available in both Adobe Type 1 and TrueType formats.

Other \MF{}-based bundles containing Euro symbols are to be found in
\Package{china2e} (whose primary aim is Chinese dates and suchlike
matters) and the \Package{eurosym} fonts.
\begin{ctanrefs}
\item[china2e bundle]\CTANref{china2e}
\item[EC fonts]\CTANref{ec}
\item[euro fonts]\CTANref{euro-fonts}
\item[euro-ce fonts]\CTANref{euro-ce}
\item[eurofont.sty]\CTANref{eurofont}
\item[eurosym fonts]\CTANref{eurosym}
\item[marvosym fonts]\CTANref{marvosym-fonts}
\item[textcomp.sty]Part of the \LaTeX{} distribution.
\end{ctanrefs}
\LastEdit{2011-11-21}

\Question[Q-tradesyms]{How to get copyright, trademark, etc.}

The ``\nothtml{Comprehensive symbol list'' (}% beware line break
\Qref{Comprehensive symbol list}{Q-symbols}\latexhtml{)}{''}, lists
the symbol commands \csx{textcopyright},
\csx{textregistered} and \csx{texttrademark}, which are available in
\acro{TS}1-encoded fonts, and which are enabled using the
\Package{textcomp} package.

In fact, all three commands are enabled in default \LaTeX{}, but the
glyphs you get aren't terribly beautiful.  In particular,
\csx{textregistered} behaves oddly when included in bold text (for
example, in a section heading), since it is composed of a small-caps
letter, which typically degrades to a regular shape letter when asked
to set in a bold font.  This means that the glyph becomes a circled
``r'', whereas the proper symbol is a circled ``R''.

This effect is of course avoided by use of \Package{textcomp}.

Another problem arises if you want \csx{textregistered} in a
superscript position (to look similar to \csx{texttrademark}).
Using a maths-mode superscript to do this provokes lots of pointless
errors: you \emph{must} use
\begin{quote}
\begin{verbatim}
\textsuperscript{\textregistered}
\end{verbatim}
\end{quote}

\Question[Q-osf]{Using ``old-style'' figures}

These numbers are also called medieval or lowercase figures and their
use is mostly font-specific.  Terminology is confusing since the
lining figures (which are now the default) are a relatively recent
development (19th century) and before they arrived, oldstyle figures
were the norm, even when setting mathematics.  (An example is Thomas
Harriot's \emph{Artis Analyticae Praxis} published in 1631).  In a
typical old style 3, 4, 5, 7 and 9 have descenders and 6 and 8 ascend
above the x-height; sometimes 2 will also ascend (this last seems to
be a variation associated with French typography).

\LaTeX{} provides a command \cmdinvoke*{oldstylenums}{digits}, which
by default uses an old-style set embedded in Knuth's `math italic'
font.  The command is only sensitive to `bold' of the font style of
surrounding text: glyphs (for this command) are only available to
match the normal medium and bold (i.e., bold-extended) weights of the
Computer Modern Roman fonts.

The \Package{textcomp} package changes \csx{oldstylenums} to use the
glyphs in the Text Companion fonts (\LaTeX{} \acro{TS}1 encoding) when
in text mode, and also makes them available using the macros of the
form \csx{text<number>oldstyle}, e.g., \csx{textzerooldstyle}.
(Of course, not all font families can provide this facility.)

Some font packages (e.g., \Package{mathpazo}) make old-style figures
available and provide explicit support for making them the default:
\cmdinvoke{usepackage}[osf]{mathpazo} selects a form where digits are
always old-style in text.  The \Package{fontinst} package will
automatically generate ``old-style versions'' of commercial Adobe Type
1 font families for which ``expert'' sets are available.

It's also possible to make virtual fonts, that offer old-style digits,
from existing font packages.  The \FontName{cmolddig} bundle provides
a such a virtual version of Knuth's originals, and the \FontName{eco}
or \FontName{hfoldsty} bundles both provide versions of the \acro{EC}
fonts.  The \FontName{lm} family offers old-style figures to OpenType
users (see below), but we have no stable mapping for \FontName{lm}
with old-style digits from the Adobe Type 1 versions of the fonts.

Originally, oldstyle figures were only to be found the expert sets of
commercial fonts, but now they are increasingly widely available.  An
example is Matthew Carter's Georgia font, which has old-style figures
as its normal form (the font was created for inclusion with certain
Microsoft products and is intended for on-screen viewing).

OpenType fonts have a pair of axes for number variations~---
proportional/tabular and lining/oldstyle selections are commonly
available.  ``Full feature access'' to OpenType fonts, making such
options available to the \AllTeX{} user, is already supported by
\Qref*{\xetex{}}{Q-xetex} using, for example, the \Package{fontspec}
package.  Similar support is also in the works for
\Qref*{\LuaTeX{}}{Q-luatex}.
\begin{ctanrefs}
\item[boondox \nothtml{\rmfamily}fonts]\CTANref{boondox}
\item[cmolddig \nothtml{\rmfamily}fonts]\CTANref{cmolddig}
\item[eco \nothtml{\rmfamily}fonts]\CTANref{eco}
\item[fontinst]\CTANref{fontinst}
\item[fontspec.sty]\CTANref{fontspec}
\item[mathpazo \nothtml{\rmfamily}fonts]\CTANref{mathpazo}
\end{ctanrefs}
\LastEdit{2013-12-04}
