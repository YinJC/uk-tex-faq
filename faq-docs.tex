% $Id: faq-docs.tex,v 1.46 2014/03/04 15:42:45 rf10 Exp rf10 $

\section{Documentation and Help}

\Question[Q-book-lists]{Books relevant to \tex{} and friends}

There are too many books for them all to appear in a single list, so
the following answers aim to cover ``related'' books, with subject
matter as follows:
\begin{itemize}
\item \Qref*{\tex{} itself and \plaintex{}}{Q-tex-books}
\item \Qref*{\latex{}}{Q-latex-books}
\item \Qref*{Books on other \tex{}-related matters}{Q-other-books}
\item \Qref*{Books on Type}{Q-type-books}
\end{itemize}

These lists only cover books in English: notices of new books, or
warnings that books are now out of print are always welcome.  However,
these \acro{FAQ}s do \emph{not} carry reviews of current published
material.
\LastEdit*{2011-06-01}

\Question[Q-tex-books]{Books on \TeX{}, \plaintex{} and relations}
\AliasQuestion{Q-books}

While Knuth's book is the definitive reference for both \TeX{} and
\plaintex{}, there are many books covering these topics:
\begin{booklist}
\item[The \TeX{}book]by Donald Knuth (Addison-Wesley, 1984,
  \ISBN{0-201-13447-0}, paperback \ISBN{0-201-13448-9})
\item[A Beginner's Book of \TeX{}]by Raymond Seroul and Silvio Levy,
  (Springer Verlag, 1992, \ISBN{0-387-97562-4})
\item[\TeX{} by Example: A Beginner's Guide]by Arvind Borde 
  (Academic Press, 1992, \ISBN{0-12-117650-9}~--- now out of print)
\item[Introduction to \TeX{}]by Norbert Schwarz (Addison-Wesley,
  1989, \ISBN{0-201-51141-X}~--- now out of print)
\item[A \plaintex{} Primer]by Malcolm Clark (Oxford University
  Press, 1993, ISBNs~0-198-53724-7 (hardback) and~0-198-53784-0
  (paperback))
\item[A \TeX{} Primer for Scientists]by Stanley Sawyer and Steven
  Krantz (CRC Press, 1994, \ISBN{0-849-37159-7})
\item[\TeX{} by Topic]by Victor Eijkhout (Addison-Wesley, 1992,
  \ISBN{0-201-56882-9}~--- now out of print, but see
  \Qref[question]{online books}{Q-ol-books}; you can also now buy a copy
  printed, on demand, by Lulu~--- see
  \url{http://www.lulu.com/content/2555607})
\item[\TeX{} for the Beginner]by Wynter Snow (Addison-Wesley, 1992,
  \ISBN{0-201-54799-6})
\item[\TeX{} for the Impatient]by Paul W.~Abrahams, Karl Berry and
  Kathryn A.~Hargreaves (Addison-Wesley, 1990,
  \ISBN{0-201-51375-7}~--- now out of print, but see
  \Qref[question]{online books}{Q-ol-books})
\item[\TeX{} in Practice]by Stephan von Bechtolsheim (Springer
  Verlag, 1993, 4 volumes, \ISBN{3-540-97296-X} for the set, or
% nos in comments are for German distribution (Springer Verlag, Berlin)
  Vol.~1: \ISBN{0-387-97595-0}, % (3-540-97595-0)
  Vol.~2: \ISBN{0-387-97596-9}, % (3-540-97596-9)
  Vol.~3: \ISBN{0-387-97597-7}, and % (3-540-97597-7)
  Vol.~4: \ISBN{0-387-97598-5})% (3-540-97598-5)
\begin{typesetversion}
\item[\TeX{}: Starting from \sqfbox{1}\,\footnotemark]%
\footnotetext{That's `Starting from Square One'}%
\end{typesetversion}
\begin{htmlversion}
\item[\TeX{}: Starting from Square One]
\end{htmlversion}
  by Michael Doob (Springer
  Verlag, 1993, \ISBN{3-540-56441-1}~--- now out of print)
\item[The Joy of \TeX{}]by Michael D.~Spivak (second edition,
  \acro{AMS}, 1990, \ISBN{0-821-82997-1})
\item[The Advanced \TeX{}book]by David Salomon (Springer Verlag, 1995,
  \ISBN{0-387-94556-3})
\end{booklist}
A collection of Knuth's publications about typography is also available:
\begin{booklist}
\item[Digital Typography]by Donald Knuth (CSLI and Cambridge
  University Press, 1999, \ISBN{1-57586-011-2}, paperback
  \ISBN{1-57586-010-4}).
\end{booklist}
\nothtml{\noindent}and in late 2000, a ``Millennium Boxed Set'' of all
5 volumes of Knuth's ``Computers and Typesetting'' series (about
\TeX{} and \MF{}) was published by Addison Wesley:
\begin{booklist}
\item[Computers \& Typesetting, Volumes A--E Boxed Set]by Donald Knuth
  (Addison-Wesley, 2001, \ISBN{0-201-73416-8}).
\end{booklist}
\checked{rf}{2000/02/12 -- http://cseng.aw.com/book/0,3828,0201734168,00.html}
\LastEdit*{2011-06-01}

\Question[Q-latex-books]{Books on \latex{}}

\begin{booklist}
\item[\LaTeX{}, a Document Preparation System]by Leslie Lamport
  (second edition, Addison Wesley, 1994, \ISBN{0-201-52983-1})
\item[Guide to \LaTeX{}]Helmut Kopka and Patrick W.~Daly (fourth
  edition, Addison-Wesley, 2004, \ISBN{0-321-17385-6})
\item[\latex{} Beginner's Guide]by Stefan Kottwitz (Packt Publishing,
  2011, \ISBN*{1847199860}{978-1847199867})
\item[The \LaTeX{} Companion]by Frank Mittelbach, Michel Goossens,
  Johannes Braams, David Carlisle and Chris Rowley (second edition,
  Addison-Wesley, 2004, \ISBN*{0-201-36299-6}{978-0-201-36299-2}); the
  book as also available as a digital download (in \acro{EPUB},
  \acro{MOBI} and \acro{PDF} formats) from
  \url{http://www.informit.com/store/latex-companion-9780133387667}
\item[The \LaTeX{} Graphics Companion:]%
  \emph{Illustrating documents with \TeX{} and \PS{}} by Michel
  Goossens, Sebastian Rahtz, Frank Mittelbach, Denis Roegel and
  Herbert Vo\ss {} (second edition, Addison-Wesley, 2007,
  \ISBN*{0-321-50892-0}{978-0-321-50892-8})
\item[The \LaTeX{} Web Companion:]%
  \emph{Integrating \TeX{}, \acro{HTML} and \acro{XML}} by Michel
  Goossens and Sebastian Rahtz (Addison-Wesley, 1999, \ISBN{0-201-43311-7})
\item[\TeX{} Unbound:]%
  \emph{\LaTeX{} and \TeX{} strategies for fonts, graphics, and more}
  by Alan Hoenig (Oxford University Press, 1998, \ISBN{0-19-509685-1}
  hardback, \ISBN{0-19-509686-X} paperback)
% \item[Math into \LaTeX{}:]\emph{An Introduction to \LaTeX{} and \AMSLaTeX{}}
%   by George Gr\"atzer (third edition Birkh\"auser and Springer Verlag,
%   2000, \ISBN{0-8176-4431-9}, \ISBN{3-7643-4131-9})
% \checked{RF}{2001/01/16}
\item[More Math into \latex{}:]\emph{An Introduction to} \LaTeX{}
  \emph{and} \AMSLaTeX{} by George Gr\"atzer (fourth edition Springer Verlag,
  2007, \ISBN{978-0-387-32289-6}
%% gr\"atzer's home page
%% http://www.maths.umanitoba.ca/homepages/gratzer.html/LaTeXBooks.html
\item[Digital Typography Using \LaTeX{}]Incorporating some
  multilingual aspects, and use of \Qref*{Omega}{Q-omegaleph}, by
  Apostolos Syropoulos, Antonis Tsolomitis and Nick Sofroniou
  (Springer, 2003, \ISBN{0-387-95217-9}).
% A list of errata for the first printing of Digital Typography Using
% \LaTeX{} is available from:
% \URL{http://www.springer-ny.com/catalog/np/jan99np/0-387-98708-8.html}
% (not any longer)
\item[First Steps in \LaTeX{}]by George Gr\"atzer (Birkh\"auser, 1999,
  \ISBN{0-8176-4132-7}) 
\item[\LaTeX{}: Line by Line:]%
  \emph{Tips and Techniques for Document Processing}
  by Antoni Diller (second edition, John Wiley \& Sons,
  1999, \ISBN{0-471-97918-X})
\item[\LaTeX{} for Linux:]\emph{A Vade Mecum}
  by Bernice Sacks Lipkin (Springer-Verlag, 1999,
  \ISBN{0-387-98708-8}, second printing)
\item[Typesetting Mathematics with \latex{}]by Herbert Vo\ss {} (UIT
  Cambridge, 2010, \ISBN{978-1-906-86017-2})
% checked 2011-09-09 http://www.uit.co.uk/BK-TMWL/HomePage
\item[Typesetting Tables with \latex{}]by Herbert Vo\ss {}, (UIT
  Cambridge, 2011, \ISBN{978-1-906-86025-7})
% checked 2011-09-09 http://www.uit.co.uk/BK-TTWL/HomePage
\item[PSTricks: Graphics and PostScript for \tex{} and \latex{}]by
  Herbert Vo\ss {}, (UIT Cambridge, 2011, \ISBN{978-1-906-86013-4})
\end{booklist}
A sample of George Gr\"atzer's ``Math into \LaTeX{}'', in Adobe
Acrobat format, and example files
for the three \LaTeX{} Companions, and for
Gr\"atzer's ``First Steps in \LaTeX{}'', are all available on
\acro{CTAN}.
\begin{ctanrefs}
\item[\nothtml{\rmfamily}Examples for \nothtml{\upshape}First Steps in \LaTeX{}]\CTANref{gfs}
\item[\nothtml{\rmfamily}Examples for \nothtml{\upshape}\LaTeX{} Companion]\CTANref{tlc2}
\item[\nothtml{\rmfamily}Examples for \nothtml{\upshape}\LaTeX{} Graphics Companion]\CTANref{lgc}
\item[\nothtml{\rmfamily}Examples for \nothtml{\upshape}\LaTeX{} Web Companion]\CTANref{lwc}
%\item[\nothtml{\rmfamily}Examples for \nothtml{\upshape}\TeX{} in Practice]\CTANref{tip}
\item[\nothtml{\rmfamily}Sample of \nothtml{\upshape}Math into \LaTeX{}]\CTANref{mil}
\end{ctanrefs}
\LastEdit{2014-03-14}

\Question[Q-other-books]{Books on other \tex{}-related matters}

There's a nicely-presented list of of ``recommended books'' to be had
on the web: \URL{http://www.macrotex.net/texbooks/}

The list of \MF{} books is rather short:
\begin{booklist}
\item[The \MF{}book]by Donald Knuth (Addison Wesley, 1986,
  \ISBN{0-201-13445-4}, \ISBN{0-201-52983-1} paperback)
\end{booklist}
Alan Hoenig's `\textsl{\TeX{} Unbound}' includes some discussion and
examples of using \MF{}.

A book covering a wide range of topics (including installation and
maintenance) is:
\begin{booklist}
\item[Making \TeX{} Work]by Norman Walsh (O'Reilly and Associates,
  Inc, 1994, \ISBN{1-56592-051-1})
\end{booklist}
The book is decidedly dated, and is now out of print, but a copy is
available via \texttt{sourceforge} and on \acro{CTAN}, 
and we list it under ``\Qref*{online books}{Q-ol-books}''.
\LastEdit*{2011-06-01}

\Question[Q-type-books]{Books on Type}

    The following is a partial listing of books on typography in general.
Of these, Bringhurst seems to be the one most often recommended.
\begin{booklist}
\item[The Elements of Typographic Style]by Robert Bringhurst
  (Hartley \& Marks, 1992, \ISBN{0-88179-033-8})
\item[Finer Points in the Spacing \& Arrangement of Type]by Geoffrey Dowding
  (Hartley \& Marks, 1996, \ISBN{0-88179-119-9})
\item[The Thames \& Hudson Manual of Typography]by Ruari McLean
  (Thames \& Hudson, 1980, \ISBN{0-500-68022-1})
\item[The Form of the Book]by Jan Tschichold
  (Lund Humphries, 1991, \ISBN{0-85331-623-6})
\item[Type \& Layout]by Colin Wheildon
  (Strathmore Press, 2006, \ISBN{1-875750-22-3})
\item[The Design of Books]by Adrian Wilson
  (Chronicle Books, 1993, \ISBN{0-8118-0304-X})
\item[Optical Letter Spacing]by David Kindersley and Lida Cardozo Kindersley 
  % ! line break
  (\href{http://www.kindersleyworkshop.co.uk/}{The Cardozo Kindersley Workshop}
  2001, \ISBN{1-874426-139})
\end{booklist}

    There are many catalogues of type specimens but the following books provide
a more interesting overall view of types in general and some of their history.
\begin{booklist}
\item[Alphabets Old \& New]by Lewis F.~Day
  (Senate, 1995, \ISBN{1-85958-160-9})
\item[An Introduction to the History of Printing Types]by Geoffrey Dowding
  (British Library, 1998, UK \ISBN{0-7123-4563-9}; USA \ISBN{1-884718-44-2})
\item[The Alphabet Abecedarium]by Richard A.~Firmage
  (David R.~Godine, 1993, \ISBN{0-87923-998-0})
\item[The Alphabet and Elements of Lettering]by Frederick Goudy
  (Dover, 1963, \ISBN{0-486-20792-7})
\item[Anatomy of a Typeface]by Alexander Lawson
  (David R.~Godine, 1990, \ISBN{0-87923-338-8})
\item[A Tally of Types]by Stanley Morison
  (David R.~Godine, 1999, \ISBN{1-56792-004-7})
\item[Counterpunch]by Fred Smeijers
  (Hyphen, 1996, \ISBN{0-907259-06-5})
\item[Treasury of Alphabets and Lettering]by Jan Tschichold
  (W.~W.~Norton, 1992, \ISBN{0-393-70197-2})
\item[A Short History of the Printed Word]by Warren Chappell and
  Robert Bringhurst (Hartley \& Marks, 1999, \ISBN{0-88179-154-7})
\end{booklist}

    The above lists are limited to books published in English. Typographic 
styles are somewhat language-dependent, and similarly the `interesting' fonts
depend on the particular writing system involved.
\LastEdit{2011-06-01}

\Question[Q-whereFAQ]{Where to find \acro{FAQ}s}

Bobby Bodenheimer's article, from which this \acro{FAQ} was developed, used
to be posted (nominally monthly) to newsgroup
\Newsgroup{comp.text.tex}. The (long
obsolete) last posted copy of that article is kept on \acro{CTAN} for
auld lang syne.

\begin{pdfversion}
A version of the \href{http://www.tex.ac.uk/faq}{present \acro{FAQ}}
may be browsed via the World-Wide Web, and its sources
\end{pdfversion}
\begin{dviversion}
A version of the present \acro{FAQ} may be browsed via the World-Wide Web, at
\acro{URL} \URL{http://www.tex.ac.uk/faq}, and its sources
\end{dviversion}
\begin{htmlversion}
The sources of the present \acro{FAQ}
\end{htmlversion}
are available from \acro{CTAN}.

This \acro{FAQ} and others are regularly mentioned, on
\Newsgroup{comp.text.tex} and elsewhere, in a ``pointer \acro{FAQ}''
which is also saved at \URL{http://tug.org/tex-ptr-faq}

A 2006 innovation from Scott Pakin is the ``visual'' \LaTeX{} \acro{FAQ}.
This is a document with (mostly rubbish) text formatted so as to
highlight things we discuss here, and providing Acrobat hyper-links to
the relevant answers in this \acro{FAQ} on the Web.  The visual
\acro{FAQ} is provided in \acro{PDF} format, on \acro{CTAN}; it works
best using Adobe Acrobat Reader 7 (or later); some features are
missing with other readers, or with earlier versions of Acrobat Reader

Another excellent information source, available in English, is the
\href{http://tex.loria.fr}{\AllTeX{} navigator}.

Both the Francophone \TeX{} user group Gutenberg and the Czech/Slovak
user group CS-TUG have published translations of this \acro{FAQ}, with
extensions appropriate to their languages.

% Herbert Vo\ss {}'s excellent % beware line break
% \href{http://texnik.de/}{\LaTeX{} tips and tricks}
% provides excellent advice on most topics one might imagine (though
% it's not strictly a \acro{FAQ})~--- highly recommended for most
% ordinary mortals' use.

The Open Directory Project (\acro{ODP}) maintains a list of sources of
\AllTeX{} help, including \acro{FAQ}s.  View the \TeX{} area at
\URL{http://dmoz.org/Computers/Software/Typesetting/TeX/}

Other non-English \acro{FAQ}s are available (off-\acro{CTAN}):
\begin{booklist}
\item[German]Posted regularly to \Newsgroup{de.comp.tex}, and archived
  on \acro{CTAN}; the \acro{FAQ} also appears at
  \URL{http://www.dante.de/faq/de-tex-faq/}
\item[French]%
%%   An interactive (full-screen!) FAQ may be found at
%%   \URL{http://www.frenchpro6.com/screen.pdf/FAQscreen.pdf}, and a copy
%%   for printing at \URL{http://frenchle.free.fr/FAQ.pdf};
  A FAQ used to be posted regularly to
  \Newsgroup{fr.comp.text.tex}, and is archived on \acro{CTAN}~---
  sadly, that effort seems to have fallen by the wayside.
%% \item[Spanish]See \URL{http://apolo.us.es/CervanTeX/FAQ/}
\item[Czech]See \URL{http://www.fi.muni.cz/cstug/csfaq/}
\end{booklist}

Resources available on \acro{CTAN} are:
\begin{ctanrefs}
\item[\nothtml{\rmfamily}Dante \acro{FAQ}]\CTANref{dante-faq}
\item[\nothtml{\rmfamily}French \acro{FAQ}]\CTANref{french-faq}
\item[\nothtml{\rmfamily}Sources of this \acro{FAQ}]\CTANref{faq}
\item[\nothtml{\rmfamily}Obsolete \texttt{comp.text.tex} \acro{FAQ}]%
  \CTANref{TeX-FAQ}
\item[\nothtml{\rmfamily}The visual \acro{FAQ}]\CTANref{visualFAQ}
\end{ctanrefs}
\LastEdit{2013-07-02}

\Question[Q-gethelp]{Getting help online}

We assume, here, that you have looked at all relevant
\Qref*{\acro{FAQ} answers}{Q-whereFAQ} you can find, you've looked in
any \Qref*{books}{Q-book-lists} you have, and scanned relevant
\Qref*{tutorials}{Q-tutorials*}\dots{} and still you don't know what
to do.

There are two more steps you can take before formulating a question to
the \tex{} world at large.

First, (if you are seeking a particular package or program), start by
looking on your own system: you might already have what you seek~---
the better \TeX{} distributions provide a wide range of supporting
material.  The \Qref*{\acro{CTAN} Catalogue}{Q-catalogue} can also
identify packages that might help: you can % ! line break
\href{http://www.tex.ac.uk/search}{search it}, or you can browse it
\begin{hyperversion}
% !!!! line break
``\href{http://mirrors.ctan.org/help/Catalogue/bytopic.html}{by topic}''.
\end{hyperversion}
\begin{flatversion}
``by topic'' at
\url{http://mirrors.ctan.org/help/Catalogue/bytopic.html} 
\end{flatversion}
Each catalogue entry has a brief description of the package, and links to
known documentation on the net.  In fact, a large proportion of
\acro{CTAN} package directories now include documentation, so it's
often worth looking at the catalogue entry for a package you're considering
using (where possible, each package link in the main body of these
\acro{FAQ}s \hyperflat{has a link to}{shows the \acro{URL} of}
the relevant catalogue entry).

Failing that, look to see if anyone has solved the problem before;
places where people ask are:
\begin{enumerate}
\item newsgroup \Newsgroup{comp.text.tex}, whose ``historical posts''
  are accessible via
  \href{http://groups.google.com/group/comp.text.tex}{Google groups},
  and
\item the mailing list \texttt{texhax} via its
  \href{http://tug.org/pipermail/texhax/}{archive}, or via the `Gmane'
  newsgroup \Newsgroup{gmane.comp.tex.texhax}, which holds a
  \emph{very} long history of the list.  A long shot would be to
  search the archives of the mailing list's ancient posts on
  \acro{CTAN}, which go back to the days when it was a digest: in
  those days, a question asked in one issue would only ever be
  answered in the next one.
\end{enumerate}
If the ``back question'' searches fail, you must ask the world at
large.

So, how do you like to ask questions?~--- the three available
mechanisms are:
\begin{enumerate}
\item Mailing lists: there are various specialist mailing lists, but
  the place for `general' \alltex{} queries is the \texttt{texhax}
  mailing list.  Mail to \mailto{texhax@tug.org} to ask a question,
  but it's probably better to subscribe to the list
  (via \URL{http://tug.org/mailman/listinfo/texhax}) % ! no ~ allowed
  first~--- not everyone will answer to you as well as to the list.
\item Newsgroup: to ask a question on \Newsgroup{comp.text.tex}, you
  can use your own news client (if you have one), or use the ``+ new
  post'' button on
  \URL{http://groups.google.com/group/comp.text.tex}.
\item Web forum: alternatives are: the % ! line break
  \href{http://www.latex-community.org/}{``\LaTeX{} community'' site},
  which offers a variety of `categories' to place your query, and the
  % the next line will tend to break if you add _anything_ to it!
  \href{http://tex.stackexchange.com/}{\TeX{}, \LaTeX{} and friends Q\&A site}
  (``StackExchange'').

  StackExchange has a scheme for voting on the quality of answers (and
  hence of those who offer support).  This arrangement is supposed to
  enable you to rank any answers that are posted.

  StackExchange offers
  \href{http://meta.tex.stackexchange.com/questions/1436/welcome-to-tex-sx}{hints about ``good behaviour''},
  which any user should at least scan before asking for help there.
  (The hints' principal aim is to maximise the chance that you get useful
  advice from the first answer; for example, it suggests that you supply a
  \Qref*{minimal example of your problem}{Q-askquestion}, just as
  these \acro{FAQ}s do.  There are people on the site who can be abrasive
  to those asking questions, who seem not to be following the
  guidelines for good behaviour)
\end{enumerate}
Do \textbf{not} try mailing the \LaTeX{} project team, the maintainers
of the \texlive{} or \miktex{} distributions or the maintainers of
these \acro{FAQ}s for help; while all these addresses reach
experienced \AllTeX{} users, no small group can possibly have
expertise in every area of usage so that the ``big'' lists and forums
are a far better bet.
\begin{ctanrefs}
\item[texhax \nothtml{\rmfamily}`back copies']\CTANref{texhax}
\end{ctanrefs}
\LastEdit{2014-01-28}

\Question[Q-maillists*]{Specialist mailing lists}

The previous question, ``\Qref*{getting help}{Q-gethelp}'', talked of
the two major forums in which \AllTeX{}, \MF{} and \MP{} are
discussed; however, these aren't the only ones available.

The \acro{TUG} web site offers many mailing lists other than just
\texttt{texhax} via its % ! line break
\href{http://tug.org/mailman/listinfo}{mail list management page}.

The French national \TeX{} user group, Gutenberg, offers a \MF{} (and,
de facto, \MP{}) mailing list, \mailto{metafont@ens.fr}: subscribe to
it by sending a message
\begin{quote}
\begin{verbatim}
subscribe metafont
\end{verbatim}
\end{quote}
to \mailto{sympa@ens.fr}

(Note that there's also a \MP{}-specific mailing list available via the
\acro{TUG} list server; in fact there's little danger of becoming confused
by subscribing to both.)

Announcements of \TeX{}-related installations on the \acro{CTAN}
archives are sent to the mailing list \texttt{ctan-ann}.  Subscribe
to the list via its \ProgName{MailMan} web-site
\URL{https://lists.dante.de/mailman/listinfo/ctan-ann}; list archives
are available at the same address.  The list archives may also be
browsed via \URL{http://www.mail-archive.com/ctan-ann@dante.de/}, and
an \acro{RSS} feed is also available:
\URL{http://www.mail-archive.com/ctan-ann@dante.de/maillist.xml}

\Question[Q-askquestion]{How to ask a question}

You want help from the community at large; you've decided where you're
going to \Qref*{ask your question}{Q-gethelp}, but how do you
phrase it?

Excellent ``general'' advice (how to ask questions of anyone) is
contained in
%beware line break
\href{http://catb.org/~esr/faqs/smart-questions.html}{Eric Raymond's article on the topic}.
Eric's an extremely self-confident person, and this comes through in
his advice; but his guidelines are very good, even for us in the
un-self-confident majority.  It's important to remember that you don't
have a right to advice from the world, but that if you express
yourself well, you will usually find someone who will be pleased to
help.

So how do you express yourself in the \AllTeX{} world?  There aren't
any comprehensive rules, but a few guidelines may help in the
application of your own common sense.
\begin{itemize}
\item Make sure you're asking the right people.  Don't ask in a \TeX{}
  forum about printer device drivers for the \ProgName{Foobar}
  operating system.  Yes, \TeX{} users need printers, but no, \TeX{}
  users will typically \emph{not} be \ProgName{Foobar} systems
  managers.

  Similarly, avoid posing a question in a language that the majority
  of the group don't use: post in Ruritanian to
  \Newsgroup{de.comp.text.tex} and you may have a long wait before a
  German- and Ruritanian-speaking \TeX{} expert notices your
  question.
\item If your question is (or may be) \TeX{}-system-specific, report
  what system you're using, or intend to use: ``I can't install
  \TeX{}'' is as good as useless, whereas ``I'm trying to install the
  \ProgName{mumbleTeX} distribution on the \ProgName{Grobble}
  operating system'' gives all the context a potential respondent
  might need.  Another common situation where this information is
  important is when you're having trouble installing something new in
  your system: ``I want to add the \Package{glugtheory} package to my
  \ProgName{mumbleTeX v12.0} distribution on the \ProgName{Grobble 2024}
  operating system''.
\item If you need to know how to do something, make clear what your
  environment is: ``I want to do \emph{x} in \plaintex{}'', or ``I
  want to do \emph{y} in \LaTeX{} running the \Class{boggle}
  class''.  If you thought you knew how, but your attempts are
  failing, tell us what you've tried: ``I've already tried installing
  the \Package{elephant} in the \Package{minicar} directory, and it
  didn't work, even after refreshing the filename database''.
\item If something's going wrong within \AllTeX{}, pretend you're
  \Qref*{submitting a \LaTeX{} bug report}{Q-latexbug},
  and try to generate a \Qref*{minimum failing example}{Q-minxampl}.
  If your example 
  needs your local \Class{xyzthesis} class, or some other resource
  not generally available, be sure to include a pointer to how the
  resource can be obtained.
\item Figures are special, of course.  Sometimes the figure itself is
  \emph{really} needed, but most problems may be demonstrated with a
  ``figure substitute'' in the form of a
  \cmdinvoke*{rule}{width}{height} command, for some value of
  \meta{width} and \meta{height}.  If the (real) figure is needed,
  don't try posting it: far better to put it on the web somewhere.
\item Be as succinct as possible.  Your helpers don't usually need to
  know \emph{why} you're doing something, just \emph{what} you're
  doing and where the problem is.
\end{itemize}

\Question[Q-minxampl]{How to make a ``minimum example''}

\Qref[Question]{Our advice on asking questions}{Q-askquestion}
suggests that you prepare a ``minimum example'' (also commonly known
as a ``\emph{minimal} example'') of failing behaviour,
as a sample to post with your question.  If you have a problem in a
two hundred page document, it may be unclear how to proceed from this
problem to a succinct demonstration of your problem.

There are two valid approaches to this task: building up, and hacking
down.

% ! line break
\latexhtml{The ``building up'' process}{\textbf{\emph{Building up}}} starts
with a basic document structure
(for \LaTeX{}, this would have \csx{documentclass},
\cmdinvoke{begin}{document}, \cmdinvoke{end}{document}) and adds
things.  First to add is a paragraph or so around the actual point
where the problem occurs.  (It may prove difficult to find the actual
line that's provoking the problem.  If the original problem is an
error, reviewing % ! line break
\Qref[the answer to question]{``the structure of \TeX{} errors''}{Q-errstruct}
may help.)

Note that there are things that can go wrong in one part of the
document as a result of something in another part: the commonest is
problems in the table of contents (from something in a section title,
or whatever), or the list of \meta{something} (from something in a
\csx{caption}).  In such a case, include the section title or caption
(the caption probably needs the \environment{figure} or
\environment{table} environment around it, but it \emph{doesn't} need
the figure or table itself).

If this file you've built up shows the problem already, then you're done.
Otherwise, try adding packages; the optimum is a file with only one
package in it, but you may find that the guilty package won't even load
properly unless another package has been loaded.  (Another common case
is that package \Package{A} only fails when package \Package{B} has been
loaded.)

% ! line break
\latexhtml{The ``hacking down'' route}{\textbf{\emph{Hacking down}}} starts
with your entire document, and
removes bits until the file no longer fails (and then of course
restores the last thing removed).  Don't forget to hack out any
unnecessary packages, but mostly, the difficulty is choosing what to
hack out of the body of the document; this is the mirror of the
problem above, in the ``building up'' route.

If you've added a package (or more than one), add \csx{listfiles} to
the preamble too: that way, \LaTeX{} will produce a list of the
packages you've used and their version numbers.  This information may
be useful evidence for people trying to help you.

The process of `building up', and to some extent that of `hacking
down', can be helped by stuff available on \acro{CTAN}:
\begin{itemize}
\item the \Class{minimal} class (part of the \latex{} distribution)
  does what its name says: it provides nothing more than what is
  needed to get \latex{} code going, and
\item the \Package{mwe} bundle provides a number of images in formats
  that \alltex{} documents can use, and a small package \Package{mwe}
  which loads other useful packages (such as \Package{blindtext} and
  \Package{lipsum}, both capable of producing dummy text in a
  document).
\end{itemize}

What if none of of these cut-down derivatives of your document will
show your error?  Whatever you do, don't post the whole of the document: if
you can, it may be useful to make a copy available on the web
somewhere: people will probably understand if it's impossible~\dots{}\ 
or inadvisable, in the case of something confidential.

If the whole document is indeed necessary, it could be that your
error is an overflow of some sort; the best you can do is to post the
code ``around'' the error, and (of course) the full text of the error.

It may seem that all this work is rather excessive for preparing a
simple post.  There are two responses to that, both based on the
relative inefficiency of asking a question on the internet.

First, preparing a minimum document very often leads \emph{you} to the
answer, without all the fuss of posting and looking for responses.

Second, your prime aim is to get an answer as quickly as possible; a
well-prepared example stands a good chance of attracting an answer
``in a single pass'': if the person replying to your post finds she
needs more information, you have to find that request, post again, and
wait for your benefactor to produce a second response.

All things considered, a good example file can save you a day, for
perhaps half an hour's effort invested.

Much of the above advice, differently phrased, may also be read on the
web at \URL{http://www.minimalbeispiel.de/mini-en.html}; source of
that article may be found at \URL{http://www.minimalbeispiel.de/}, in
both German and English.
\begin{ctanrefs}
\item[blindtext.sty]\CTANref{blindtext}
\item[lipsum.sty]\CTANref{lipsum}
\item[minimal.cls]Distributed as part of \CTANref{latex}
\item[mwe.sty]\CTANref{mwe}
\end{ctanrefs}
\LastEdit{2013-01-09}

\Question[Q-tutorials*]{Tutorials, etc., for \tex{}-based systems}

From a situation where every \AllTeX{} user \emph{had} to buy a book
if she was not to find herself groping blindly along, we now have what
almost amounts to an embarrassment of riches of online documentation.
The following answers attempt to create lists of sources ``by topic''.

First we have introductory manuals, for
\Qref*{\plaintex{}}{Q-man-tex} and \Qref*{\LaTeX{}}{Q-man-latex}.

Next comes a selection of
\Qref*{``specialised'' \AllTeX{} tutorials}{Q-tutbitslatex},
each of which concentrates on a single \LaTeX{} topic.

A small selection of reference documents (it would be good to have
more) are listed in an \Qref*{answer of their own}{Q-ref-doc}.

Next comes the (somewhat newer) field of % ! line break
\Qref*{\TeX{}-related WIKIs}{Q-doc-wiki}.

A rather short list gives us a % ! line break
\Qref*{Typography style tutorial}{Q-typo-style}.
\LastEdit{2011-09-26}

\Question[Q-man-tex]{Online introductions: \plaintex{}}

Michael Doob's splendid `Gentle Introduction' to \plaintex{}
(available on \acro{CTAN}) has been stable for a very long time.

Another recommendable document is D. R.~Wilkins' `Getting started with \TeX{}',
available on the web at \URL{http://www.ntg.nl/doc/wilkins/pllong.pdf}
\begin{ctanrefs}
\item[\nothtml{\rmfamily}Gentle Introduction]\CTANref{gentle}
\end{ctanrefs}

\Question[Q-man-latex]{Online introductions: \LaTeX{}}

A pleasing little document, ``Getting something out of \LaTeX{}'' is
designed to give a feel of \LaTeX{} to someone who's never used it at
all.  It's not a tutorial, merely helps the user to decide whether to
go on to a tutorial, and thence to `real' use of \LaTeX{}.

Tobias Oetiker's `(Not so) Short Introduction to \LaTeXe{}', is
regularly updated, as people suggest better ways of explaining things,
etc.  The introduction is available on \acro{CTAN}, together with
translations into a rather large set of languages.

Peter Flynn's ``Beginner's \LaTeX{}'' (which started life as course
material) is a pleasing read.  A complete copy may be found on
\acro{CTAN}, but it may also be browsed over the web
(\URL{http://mirrors.ctan.org/info/beginlatex/html/}).

Harvey Greenberg's `Simplified Introduction to \LaTeX{}' was written
for a lecture course, and is also available on \acro{CTAN} (in \PS{}
only, unfortunately).

The fourth edition of George Gr\"atzer's book ``Math into \LaTeX{}''
contains a ``short course'' in \LaTeX{} itself, and that course has
been made publicly available on \acro{CTAN}.

Philip Hirschhorn's ``Getting up and running with \AMSLaTeX{}'' has a
brief introduction to \LaTeX{} itself, followed by a substantial
introduction to the use of the \acro{AMS} classes and the
\Package{amsmath} package and other things that are potentially of
interest to those writing documents containing mathematics.

Edith Hodgen's % beware line break
\href{http://www.mcs.vuw.ac.nz/~david/latex/notes.pdf}{\LaTeX{}, a Braindump}
starts you from the ground up~--- giving a basic tutorial in the use
of \ProgName{Linux} to get you going (rather a large file\dots{}).
Its parent site, David Friggens' % ! line break
\href{http://www.mcs.vuw.ac.nz/~david/latex/}{documentation page} is a
useful collection of links in itself.

% ! line break
\href{http://www.andy-roberts.net/misc/latex/}{Andy Roberts' introductory material}
is a pleasing short introduction to the use of \AllTeX{}; some of the
slides for \emph{actual} tutorials are to be found on the page, as
well.

D. R.~Wilkins' % ! line break
\href{http://www.maths.tcd.ie/~dwilkins/LaTeXPrimer/}{`Getting started with \latex{}'}
also looks good (it appears shorter~--- more of a primer~--- than some
of the other offerings).

Chris Harrison's % ! line break
\href{http://xoph.co/20111024/latex-tutorial/}{LaTeX tutorial}
presents basic \LaTeX{} in a rather pleasing and straightforward way.

Nicola Talbot's % ! line break
\href{http://www.dickimaw-books.com/latex/novices/}{\LaTeX{} for complete novices}
does what it claims: the author teaches \LaTeX{} at the University of
East Anglia.  The ``Novices'' tutorial is one of several % ! line break
\href{http://www.dickimaw-books.com/latex/}{introductory tutorials},
which include exercises (with solutions).  Other tutorials include
those for % ! line break
\href{http://www.dickimaw-books.com/latex/thesis/}{writing theses/dissertations with \LaTeX{}}, and for % ! line break
\href{http://www.dickimaw-books.com/latex/admin/}{using \LaTeX{} in administrative work}

Engelbert Buxbaum provides the `slides' for his \latex{} course `The
\latex{} document preparation system'; this seems to be a departmental
course at his university.

Mark van Dongen's % line break
\href{"http://csweb.ucc.ie/~dongen/LaTeX-and-Friends.pdf}{`\latex and friends'}
appeared as he was writing his book on the subject (soon to be published).

An interesting (and practical) tutorial about what \emph{not} to do is
\Package{l2tabu}, or ``A list of sins of \LaTeXe{} users'' by Mark
Trettin, translated into English by J\"urgen Fenn.  The
tutorial is available from \acro{CTAN} as a \acro{PDF} file (though
the source is also available).
\begin{ctanrefs}
\item[\nothtml{\rmfamily}Beginner's \LaTeX{}]\CTANref{beginlatex-pdf}
% ! line break
\item[\nothtml{\rmfamily}Getting something out of \latex{}]\CTANref{first-latex-doc}
% ! line break
\item[\nothtml{\rmfamily}Getting up and running with \AMSLaTeX{}]\CTANref{amslatex-primer}
\item[\nothtml{\rmfamily}Slides for \latex{} course]\CTANref{latex-course}
\item[\nothtml{\rmfamily}Not so Short Introduction]\CTANref{lshort}
  (in English, you may browse for sources and other language versions at
  \CTANref{lshort-parent})
\item[\nothtml{\rmfamily}Simplified \LaTeX{}]\CTANref{simpl-latex}
\item[\nothtml{\rmfamily}Short Course in \LaTeX{}]\CTANref{mil-short}
% ! line break
\item[\nothtml{\rmfamily}The sins of \LaTeX{} users]Browse
  \CTANref{l2tabu} for a copy of the document in a language that is
  convenient for you
\end{ctanrefs}
\LastEdit{2012-11-29}

\Question[Q-tutbitslatex]{\AllTeX{} tutorials}
\AliasQuestion{Q-doc-dirs}

The \acro{AMS} publishes a ``Short Math Guide for \LaTeX{}'', which is
available (in several formats) via
\URL{http://www.ams.org/tex/amslatex.html} (the ``Additional
Documentation'' about half-way down the page.

Herbert Vo\ss {} has written an extensive guide to mathematics in
\LaTeX{}, and a development of it has been % ! line break
\Qref*{published as a book}{Q-latex-books}.

Two documents written more than ten years apart about font usage in
\TeX{} are worth reading: % ! line break
\href{http://www.tug.org/TUGboat/Articles/tb14-2/tb39rahtz-nfss.pdf}{Essential NFSS}
by Sebastian Rahtz, and % ! line break
\href{http://tug.org/pracjourn/2006-1/schmidt/schmidt.pdf}{Font selection in LaTeX},
cast in the form of an \acro{FAQ}, by Walter Schmidt.  A general
compendium of font information (including the two above) may be found
on the \href{http://www.tug.org/fonts/}{TUG web site}.

T\acro{UG} India is developing a series of online \LaTeX{} tutorials
which can be strongly recommended: select single chapters at a time
from \URL{http://www.tug.org/tutorials/tugindia}\nobreakspace--- there
are 17~chapters in the series, two of which are mostly introductory.

Peter Smith's
\begin{narrowversion}
  ``\LaTeX{} for Logicians''
  (\URL{http://www.logicmatters.net/latex-for-logicians/})
\end{narrowversion}
\begin{wideversion}
  % ! line break
  ``\href{http://www.logicmatters.net/latex-for-logicians/}{\LaTeX{} for Logicians}''
\end{wideversion}
page covers a rather smaller subject area, but is similarly comprehensive
(mostly by links to documents on relevant topics, rather than as a
monolithic document).

Keith Reckdahl's ``Using Imported Graphics in \LaTeXe{}''
(\Package{epslatex}) is an
excellent introduction to graphics use.  It's available on
\acro{CTAN}, but not in the \texlive{} or \miktex{} distributions, for
lack of sources.

Stefan Kottwitz manages a web site devoted to the use of the drawing
packages % ! line break
\Qref*{\acro{PGF} and \acro{T}ik\acro{Z}}{Q-drawing}, % ! line break
\url{http://www.texample.net/}

Included is % ! line break
 \href{http://www.texample.net/tikz/examples/}{examples catalogue}
includes examples (with output) from the package documentation as well
as code written by the original site maintainer (Kjell Magne Fauske)
and others.

The compendious \acro{PGF}/\acro{T}ik\acro{Z} manual is clear, but is
bewildering for some beginners.  The % ! line break
\href{http://cremeronline.com/LaTeX/minimaltikz.pdf}{`minimal' introduction}
has helped at least the present author.

Vincent Zoonekynd provides a set of excellent (and graphic) tutorials
on the programming of % !line breaks, ...
\href{http://zoonek.free.fr/LaTeX/LaTeX_samples_title/0.html}{title page styles},
\href{http://zoonek.free.fr/LaTeX/LaTeX_samples_chapter/0.html}{chapter heading styles}
and
\href{http://zoonek.free.fr/LaTeX/LaTeX_samples_section/0.html}{section heading styles}.
In each file, there is a selection of graphics representing an output
style, and for each style, the code that produces it is shown.

An invaluable step-by-step setup guide for establishing a ``work
flow'' through your \AllTeX{} system, so that output appears at the
correct size and position on standard-sized paper, and that the print
quality is satisfactory, is Mike Shell's \Package{testflow}.  The
tutorial consists of a large plain text document, and there is a
supporting \LaTeX{} file together with correct output, both in \PS{} and
\acro{PDF}, for each of \acro{A}4 and ``letter'' paper sizes.  The
complete kit is available on \acro{CTAN} (distributed with the
author's macros for papers submitted for \acro{IEEE} publications).
The issues are also covered in a later % ! line break
\Qref{\acro{FAQ} answer}{Q-dvips-pdf}.

Documentation of Japanese \ensuremath{\Omega{}} use appears in
Haruhiko Okumura's page
% ! line break
\href{http://oku.edu.mie-u.ac.jp/~okumura/texfaq/japanese/}{typesetting Japanese with Omega}
(the parent page is in Japanese, so out of the scope of this
\acro{FAQ} list).

Some university departments make their local documentation available
on the web.  Most straightforwardly, there's the simple translation of
existing documentation into \acro{HTML}, for example the \acro{INFO}
documentation of the \AllTeX{} installation, of which a sample is the
\LaTeX{} documentation available at
\URL{http://www.tac.dk/cgi-bin/info2www?(latex)}

More ambitiously, some university departments have enthusiastic
documenters who 
make public record of their \AllTeX{} support.  For example, Tim Love
(of Cambridge University Engineering Department) maintains his
department's pages at
\URL{http://www-h.eng.cam.ac.uk/help/tpl/textprocessing/}
%%  and Mimi
%% Burbank (of the School of Computer Science \& Information Technology
%% at the University of Florida) manages her department's at
%% \URL{http://www.csit.fsu.edu/~mimi/tex/}\nobreakspace--- both sets are fine
%% examples of good practice.
\begin{ctanrefs}
\item[\nothtml{\rmfamily}Graphics in \LaTeXe{}]\CTANref{epslatex}
\item[testflow]\CTANref{testflow}
\item[\nothtml{\rmfamily}Herbert Vo\ss {}'s Maths tutorial]\CTANref{voss-mathmode}
\end{ctanrefs}

\Question[Q-ref-doc]{Reference documents}

For \TeX{} primitive commands a rather nice % ! line break
\href{http://www.nmt.edu/tcc/help/pubs/texcrib.pdf}{quick reference booklet},
by John W.~Shipman, is available; it's arranged in the same way as the
\TeX{}book.  By contrast, you can view David Bausum's % ! line break
\href{http://www.tug.org/utilities/plain/cseq.html}{list of \TeX{} primitives}
alphabetically or arranged by ``family''.  Either way, the list has a
link for each control sequence, that leads you to a detailed
description, which includes page references to the \TeX{}book.

There doesn't seem to be a reference that takes in \plaintex{} as
well as the primitive commands.

An interesting \LaTeX{} ``cheat sheet'' is available from \acro{CTAN}:
it's a list of (more or less) everything you `ought to' remember, for
basic \LaTeX{} use. % line break
(It's laid out very compactly for printing on N.\ American `letter';
printed on \acro{ISO} \acro{A}4, using Adobe Acrobat's ``shrink to
fit'', it strains aged eyes\dots{}) 

For command organised references to \LaTeX{},  Karl Berry (et
al)'s % !line break
\href{http://home.gna.org/latexrefman}{LaTeX reference manual} is (to
an extent) work in progress, but is generally reliable (source is
available on the.archive as well).

Martin Scharrer's ``List of internal \latex{} macros'' is a help to
those aiming to write a class or package.

The reference provided by the Emerson Center of Emory
University), % ! line break
\href{http://www.emerson.emory.edu/services/latex/latex2e/latex2e_toc.html}{LaTeXe help}
also looks good.
\begin{ctanrefs}
\item[\nothtml{\rmfamily}Cheat sheet]\CTANref{latexcheat}
\item[\nothtml{\rmfamily}LaTeX reference manual]\CTANref{latex2e-help-texinfo}
\item[\nothtml{\rmfamily}LaTeX internal macros]\CTANref{macros2e}
\end{ctanrefs}
\LastEdit{2012-02-16}

\Question[Q-doc-wiki]{\acro{WIKI} books for \TeX{} and friends}

The \emph{\acro{WIKI}} concept can be a boon to everyone, if used sensibly.
The ``general'' \acro{WIKI} allows \emph{anyone} to add stuff, or to edit
stuff that someone else has added: while there is obvious potential
for chaos, there is evidence that a strong user community can keep a
\acro{WIKI} under control.

Following the encouraging performance of the % ! line break
\href{http://contextgarden.net/}{\CONTeXT{} \acro{WIKI}}, valiant
efforts have been made generating ``\acro{WIKI} books'' for \AllTeX{}
users.  Thus we have % ! line break
\href{http://en.wikibooks.org/wiki/TeX}{(Plain) \TeX{} \acro{WIKI} book} and 
\href{http://en.wikibooks.org/wiki/LaTeX}{\LaTeX{} \acro{WIKI} book}~---
both well established.  Both are highly rated as reference sources,
and even as introductory texts.
\LastEdit{2012-07-25}

\Question[Q-typo-style]{Typography tutorials}

Peter Wilson's article \Package{memdesign} has a lengthy introductory
section on typographic considerations, which is a fine tutorial,
written by someone who is aware of the issues as they apply to
\AllTeX{} users.  (\Package{Memdesign} now distributed separately from
the manual for his \Class{memoir} class, but was originally part of
that manual)

There's also (at least one) typographic style tutorial available on
the Web, the excellent % ! line break
``\href*{http://www.nbcs.rutgers.edu/~hedrick/typography/typography.janson-syntax.107514.pdf}{Guidelines for Typography in NBCS}''.
In fact, its % !!
\href*{http://www.nbcs.rutgers.edu/~hedrick/typography/index.html}{parent page}
is also worth a read: among other things, it provides copies of the
``guidelines'' document in a wide variety of primary fonts, for
comparison purposes.  The author is careful to explain that he has no
ambition to supplant such excellent books as
\Qref*{Bringhurst's}{Q-type-books}, but the document (though it does
contain its Rutgers-local matter) is a fine introduction to the issues
of producing readable documents.
\begin{ctanrefs}
\item[memdesign]\CTANref{memdesign}
\end{ctanrefs}

\Question[Q-ol-books]{Freely available \AllTeX{} books}

People have long argued for \AllTeX{} books to be made available on
the web, and until relatively recently this demand went un-answered.

The first to appear was Victor Eijkhout's excellent ``\TeX{} by
Topic'' in 2001 (it had been published by Addison-Wesley, but was long
out of print).  The book is now available on \acro{CTAN}; it's not a
beginner's tutorial but it's a fine reference.  It's also available,
as a printed copy, via the on-line publishers
\href{http://www.lulu.com/content/2555607/}{Lulu} (not quite free, of
course, but not a \emph{bad} deal\dots{}).

Addison-Wesley have also released the copyright of ``\TeX{} for the
Impatient'' by Paul W.~Abrahams, Karl Berry and Kathryn A.~Hargreaves,
another book whose unavailability many have lamented.  The authors
have re-released the book under the \acro{GNU} Free Documentation
Licence, and it is available from \acro{CTAN}.

Norm Walsh's ``Making \TeX{} Work'' (originally published by O'Reilly)
is also available (free) on the Web, at
\URL{http://makingtexwork.sourceforge.net/mtw/}; the sources of the
Web page are on \acro{CTAN}.  The book was an excellent resource in
its day, but while it is now somewhat dated, it still has its uses,
and is a welcome addition to the list of on-line resources.  A project
to update it is believed to be under way.
\begin{ctanrefs}
\item[\nothtml{\rmfamily}Making \TeX{} Work]\CTANref{mtw}
\item[\nothtml{\rmfamily}\TeX{} by Topic]\CTANref{texbytopic}
\item[\nothtml{\rmfamily}\TeX{} for the Impatient]\CTANref{TftI}
\end{ctanrefs}
\LastEdit{2011-06-29}

\Question[Q-pkgdoc]{Documentation of packages}

These \acro{FAQ}s regularly suggest packages that will ``solve''
particular problems.  In some cases, the answer provides a recipe for
the job.  In other cases, or when the solution needs elaborating, how
is the poor user to find out what to do?

If you're lucky, the package you need is already in your installation.
If you're particularly lucky, you're using a distribution that gives
access to package documentation and the documentation is available in
a form that can easily be shown.

On \texlive{}-based distributions, help should be available from the
\ProgName{texdoc} command, as in:
\begin{quote}
\begin{verbatim}
texdoc footmisc
\end{verbatim}
\end{quote}
which opens a window showing documentation of the \Package{footmisc}
package.  (The window is tailored to the file type, in the way normal
for the system.)

If \ProgName{texdoc} can't find any documentation, it may launch a Web
browser to look at the package's entry in the \acro{CTAN} catalogue.
The catalogue has an entry for package documentation, and most authors
respond to the \acro{CTAN} team's request for documentation of
packages, you will more often than not find documentation that way.

On \miktex{} systems, the same function is provided by the
\ProgName{mthelp}.

Note that the site \url{texdoc.net} provides access to the
documentation you \emph{would} have if you had a \emph{full}
installation of \texlive{}; on the site you can simply ask for a
package (as you would ask \ProgName{texdoc}, or you can use the site's
index of documentation to find what you want.  (This is helpful for
some of us: many people don't have a full \alltex{} installation on
their mobile phone~\dots{} yet.)

If your luck (as defined above) doesn't hold out, you've got to find
documentation by other means.  That is, you have to find the
documentation for yourself.  The rest of this answer offers a range of
possible techniques.

The commonest form of documentation of \LaTeX{} add-ons is within the
\extension{dtx} file in which the code is distributed (see
\Qref[answer]{documented \LaTeX{} sources}{Q-dtx}).  Such files
are supposedly processable by \LaTeX{} itself, but there are
occasional hiccups on the way to readable documentation.  Common
problems are that the package itself is needed to process its own
documentation (so must be unpacked before processing), and that the
\extension{dtx} file will \emph{not} in fact process with \LaTeX{}.  In the
latter case, the \extension{ins} file will usually produce a
\extension{drv} (or similarly-named) file, which you process with
\LaTeX{} instead.  (Sometimes the package author even thinks to
mention this wrinkle in a package \texttt{README} file.)

Another common form is the separate documentation file; particularly
if a package is ``conceptually large'' (and therefore needs a lot of
documentation), the documentation would prove a cumbersome extension
to the \extension{dtx} file.  Examples of such cases are the \Class{memoir}
class, the \Class{KOMA-script} bundle
(whose developers take the trouble to produce detailed documentation
in both German and English), the \Package{pgf} documentation (which
would make a substantial book in its own right)
and the \Package{fancyhdr} package (whose
documentation derives from a definitive tutorial in a mathematical
journal).  Even if the documentation is not separately identified in a
\texttt{README} file, it should not be too difficult to recognise its
existence.

Documentation within the package itself is the third common form.
Such documentation ordinarily appears in comments at the head of the
file, though at least one eminent author regularly places it after the
\csx{endinput} command in the package.  (This is desirable, since
\csx{endinput} is a `logical' end-of-file, and \AllTeX{} doesn't read
beyond it: thus such documentation does not `cost' any package loading time.)

The above suggestions cover most possible ways of finding
documentation.  If, despite your best efforts, you can't find
it in any of the above places, there's the awful possibility that the
author didn't bother to document his package (on the ``if it was hard
to write, it should be hard to use'' philosophy).  Most ordinary
mortals will seek support from some more experienced user at this
stage, though it \emph{is} possible to proceed in the way that the original
author apparently expected\dots{}by reading his code.
\LastEdit{2012-11-09}

\Question[Q-writecls]{Learning to write \LaTeX{} classes and packages}

There's nothing particularly magic about the commands you use when
writing a package, so you can simply bundle up a set of \LaTeX{}
\csx{(re)newcommand} and \csx{(re)newenvironment} commands, put them in
a file \File{package.sty} and you have a package.

However, any but the most trivial package will require rather more
sophistication.  Some details of \LaTeX{} commands for the job are to
be found in `\LaTeXe{} for class and package writers'
(\File{clsguide}, part of the \LaTeX{} documentation distribution).
Beyond this, a good knowledge of \TeX{} itself is valuable: thus books
such as the \Qref*{\TeX{}book}{Q-tex-books} or % ! line break
\Qref*{\TeX{} by topic}{Q-ol-books} are relevant.  With good \TeX{}
knowledge it is possible to use the documented source of \LaTeX{} as
reference material (dedicated authors will acquaint themselves with the
source as a matter of course).  A complete set of the documented
source of \LaTeX{} may be prepared by processing the file
\File{source2e.tex} in the \LaTeX{} distribution.  Such processing is
noticeably tedious, but Heiko Oberdiek has prepared a well-linked
\acro{PDF} version, which is in the file \File{base.tds.zip} of his
\ProgName{latex-tds} distribution.  Individual files in the \LaTeX{}
distribution may be processed separately with \LaTeX{}, like any
well-constructed \Qref*{\extension{dtx} file}{Q-dtx}.

Writing good classes is not easy; it's a good idea to read some
established ones (\File{classes.dtx}, for example, is the documented
source of the standard classes other than \Class{Letter}, and may
itself be formatted with \LaTeX{}).  Classes that are not part of the
distribution are commonly based on ones that are, and start by loading
the standard class with \csx{LoadClass}~--- an example of this
technique may be seen in \Package{ltxguide.cls}

An % !! line break
\href{http://tug.org/TUGboat/Articles/tb28-1/tb88flynn.pdf}{annotated version of \Class{article}},
as it appears in \File{classes.dtx}, was published in
\TUGboat{} 28(1).  The article, by Peter Flynn, is a good guide to
understanding \File{classes.dtx}
\begin{ctanrefs}
\item[classes.dtx]\CTANref{latex-classes}
\item[clsguide.pdf]\CTANref{clsguide}
\item[latex-tds \nothtml{rmfamily}collection]\CTANref{latex-tds}
\item[ltxguide.cls]\CTANref{ltxguide}
\item[\nothtml{\rmfamily}\LaTeX{} documentation]\CTANref{latexdoc}
\item[source2e.tex]\CTANref{latex-source}
\end{ctanrefs}
\LastEdit{2011-07-19}

\Question[Q-latex3-prog]{\latex{}3 programming}
As yet, there is no book ``Programming in \latex{}3'', and even if
such a thing existed there would be lots of gaps.  So there is no
\emph{comprehensive} support.

However, there are some `resources':
\begin{itemize}
\item The ``introduction to \latex{}3 ideas'' in the \Package{expl3}
  documentation gives a broad-brush overview of the concepts.
\item Joseph Wright has written a short series of % !line break
  \href{http://www.texdev.net/index.php?s=programming+latex3}{blog posts},
  which may help.
\item There is also a complete command reference in the
  \Package{interface3} document.
\end{itemize}

The documents are still subject to development; some of the broader
design issues are discussed on the \latex{} mailing list
\texttt{latex-l}~--- you may subscribe to that list by sending a
message by sending a message
\begin{quote}
  `\texttt{subscribe latex-l <\emph{your name}>}'
\end{quote}
to \mailto{listserv@urz.Uni-Heidelberg.de}
\begin{ctanrefs}
\item[expl3.pdf]\CTANref{expl3-doc}
\item[interface3.pdf]\CTANref{interface3-doc}
\end{ctanrefs}
\LastEdit*{2012-11-29}

\Question[Q-mfptutorials]{\MF{} and \MP{} Tutorials}

Apart from Knuth's book, there seems to be only one publicly-available
\href{http://metafont.tutorial.free.fr/}{tutorial for \MF{}}, by
Christophe Grandsire (a copy in \acro{PDF} form may be downloaded).
Geoffrey Tobin's \textit{\MF{} for Beginners} % !! line break
(see \Qref[question]{using \MF{}}{Q-useMF}) describes how the \MF{}
system works and how to avoid some of the potential pitfalls.

Peter Wilson's experience of running both \MF{} and \MP{} (the
programs), \textit{Some Experiences in Running \MF{} and \MP{}}
(available on \ctan{}) offers the benefit of Peter's experience (he
has designed a 
number of `historical' fonts using \MF{}).  For \MF{} the article is
geared towards testing and installing new \MF{} fonts, while its \MP{}
section describes how to use \MP{} illustrations in \LaTeX{} and
\PDFLaTeX{} documents, with an emphasis on how to use appropriate
fonts for any text or mathematics.

Hans Hagen (of \CONTeXT{} fame) offers a \MP{} tutorial called
MetaFun (which admittedly concentrates on the use of \MP{} within
\CONTeXT{}).  It may be found on his company's % ! line break
\href{http://www.pragma-ade.com/general/manuals/metafun-p.pdf}{`manuals' page}.

Another \MP{} tutorial in English is: % ! line breaks
\url{http://www.tlhiv.org/MetaPost/tutorial/} by Urs Oswald.
One in French (listed here because it's clearly enough written
that even this author understands it),
\url{http://pauillac.inria.fr/~cheno/metapost/metapost.pdf}
by Laurent Ch\'eno.

Urs Oswald's tutorial uses Troy Henderson's tool
(\URL{http://www.tlhiv.org/mppreview}) for testing little bits of
\MP{}; it is an invaluable aid to the learner:
\URL{http://www.tlhiv.org/mppreview}

A three-part introduction, by Mari Voipio, was published in
\href{http://tug.org/TUGboat/intromp/tb106voipio-grid.pdf}{\TUGboat34(1) (Entry-level \MP{}: On the grid)},
\TUGboat34(2)
\href{http://tug.org/TUGboat/intromp/tb107voipio-moveit.pdf}{(Entry-level \MP{}: Move it!\@)}, and
\href{http://tug.org/TUGboat/intromp/tb108voipio-color.pdf}{\TUGboat34(2) (Entry-level \MP{}: Color)}.

Vincent Zoonekynd's massive set of example \MP{} files is available on
\acro{CTAN}; the set includes a \ProgName{Perl} script to convert the
set to html, and the set may be % beware line break
\href{http://zoonek.free.fr/LaTeX/Metapost/metapost.html}{viewed on the web}.
While these examples don't exactly constitute a ``tutorial'', they're
most certainly valuable learning material.  Urs Oswald presents a
\href{http://www.ursoswald.ch/metapost/tutorial.pdf}{similar document},
written more as a document, and presented in \acro{PDF}.
\begin{ctanrefs}
\item[\nothtml{\rmfamily}Beginners' guide]\CTANref{mf-beginners}
\item[\nothtml{\rmfamily}Peter Wilson's ``experiences'']\CTANref{metafp-pdf}
\item[\nothtml{\rmfamily}Vincent Zoonekynd's examples]\CTANref{zoon-mp-eg}
\end{ctanrefs}
\LastEdit{2014-02-18}

\Question[Q-BibTeXing]{\bibtex{} Documentation}

\bibtex{}, a program originally designed to produce bibliographies in
conjunction with \LaTeX{}, is explained in Section~4.3 and Appendix~B
of Leslie Lamport's \LaTeX{} manual.
The document ``\bibtex{}ing'', in the \bibtex{} distribution (look for
\File{btxdoc}),
expands on the chapter in Lamport's book.  \emph{The \LaTeX{} Companion}
also has information on \bibtex{} and writing \bibtex{} style files.
(See \Qref[question]{\latex{} books}{Q-latex-books} for details of both
books.)

The web site ``\href{http://www.bibtex.org}{Your \bibtex{} resource}''
offers a solid introduction, but doesn't go into very great detail.

The document ``Designing \bibtex{} Styles'', also in the \bibtex{}
distribution (look for
\File{btxhak}), explains the postfix stack-based language used to
write \bibtex{} styles (\File{.bst} files). The file\File{btxbst.doc},
also in the \bibtex{} distribution,
is the template for the four standard styles (\Package{plain},
\Package{abbrv}, \Package{alpha}, and \Package{unsrt}); it also
contains their documentation.

A useful tutorial of the whole process of using \bibtex{} is Nicolas
Markey's ``\emph{Tame the BeaST (The B to X of \bibtex{})}'', which
may also be found on \acro{CTAN}.  A summary and \acro{FAQ} by Michael
Shell and David Hoadley, is also to be recommended.
\begin{ctanrefs}
\item[\nothtml{\rmfamily}\bibtex{} distribution]\CTANref{bibtex}
\item[\nothtml{\rmfamily}Shell and Hoadley's FAQ]\CTANref{bibtex-faq}
\item[\nothtml{\rmfamily}Tame the BeaST]\CTANref{ttb-pdf}
\end{ctanrefs}
\LastEdit{2013-10-15}

\Question[Q-symbols]{Where can I find the symbol for\,\dots{}}
\keywords{assignment circular integral degrees diagonal dots Fourier transform}
\keywords{Laplace greater less complex integer natural real}
\keywords{cent euro}

There is a wide range of symbols available for use with \TeX{}, most
of which are not shown (or even mentioned) in \AllTeX{} books.
\emph{The Comprehensive \LaTeX{} Symbol List} (by Scott Pakin
% beware line wrap
\emph{et al.}\@) illustrates over 2000 symbols, and details the
commands and the \LaTeX{} packages needed to produce them.

However, while the symbol list is a wonderful resource, it is never
easy to find a particular symbol there.  A graphical % ! line break
\href{http://detexify.kirelabs.org/classify.html}{symbol search} is
available on the web.  The site provides you a scratch area on which
you draw the symbol you're thinking of, with your mouse; when you've
finished drawing, the classifier tries to match your sketch with
symbols it knows about.  The matching process is pretty good, even for
the sketches of a \emph{really} poor draughtsman (such as the present
author), and it's often worth trying more than once.  `Detexify apps'
are available for both Android and iPhone devices, you can use them to
draw the symbol with your fingertip~--- a less challenging procedure
than using your workstation's mouse, by all accounts!

If you are using Unicode maths in \xetex{} or \luatex{}, your own
distribution ought to provide the Unicode maths symbol table
\File{unimath-symbols.pdf}; this lists the things available in the
commonly-used mathematics fonts.  (If the file isn't already available
on your system, you can download it from \acro{CTAN}, where it live
with the \Package{unicode-math} package.

Other questions in this \acro{FAQ} offer specific help on kinds of
symbols:
\begin{itemize}
\item \Qref*{Script fonts for mathematics}{Q-scriptfonts}
\item \Qref*{Fonts for the number sets}{Q-numbersets}
\item \Qref*{Typesetting the principal value integral}{Q-prinvalint}
\end{itemize}
\begin{ctanrefs}
\item[\nothtml{\rmfamily}Symbol List]Browse \CTANref{symbols}; there
  are processed versions \acro{PDF} form for both A4 and letter paper.
\item[\nothtml{\rmfamily}Unicode maths symbols]Distributed as part of
  \CTANref{unicode-math}
\end{ctanrefs}
\LastEdit{2012-09-03}

\Question[Q-docpictex]{The \pictex{} manual}

\pictex{} is a set of macros by Michael Wichura for drawing diagrams
and pictures. The
macros are freely available; however, the
\pictex{} manual itself is not free.
Unfortunately, \acro{TUG} is no longer able to supply copies of the
manual (as it once did), and it is now available only through Personal
\TeX{} Inc, the vendors of PC\TeX{} (\URL{http://www.pctex.com/}).  The
manual is \emph{not} available electronically.

However, there \emph{is} a summary of \pictex{} commands available on
\acro{CTAN}, which is a great aide-memoire for those who basically
know the package to start with.
\begin{ctanrefs}
\item[\pictex{}]\CTANref{pictex}
\item[\pictex{} summary]\CTANref{pictex-summary}
\end{ctanrefs}
